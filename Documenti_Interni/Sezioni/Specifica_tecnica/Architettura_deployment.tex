\section{Architettura di deployment}
%dubbio che sia corretto
\begin{comment}
    L'architettura di deployment, detta anche "architettura di rilascio", rappresenta la struttura e la configurazione di un sistema software in fase di esecuzione. Essa definisce come i componenti software, i dati e le risorse di rete sono distribuiti e interconnessi nell'ambiente di produzione.
\end{comment}


\subsection{Architettura a microservizi nel IoT}
l modello di architettura dei microservizi è emerso come alternativa alle architetture e alle applicazioni monolitiche, difficili da mantenere ed evolvere a causa dell'elevato accoppiamento tra i loro componenti. Un'architettura a microservizi si basa sul concetto di costruire un'applicazione come un insieme di piccoli servizi interconnessi, che comunicano attraverso protocolli leggeri. Ogni servizio è progettato per svolgere una funzione specifica e può essere sviluppato, testato, deployato e scalato in modo indipendente dagli altri servizi. Questo approccio consente di ottenere una maggiore flessibilità, scalabilità e resilienza rispetto alle architetture monolitiche, consentendo inoltre di adattarsi meglio ai cambiamenti e alle esigenze del business.
Tutti i microservizi di un’applicazione risultano essere completamente separati
tra di loro, e possono essere compilati ed implementati in modo indipendente.
gli altri.
Poiché eseguito in modo del tutto indipendente, ciascun microservizio può essere
aggiornato, distribuito e ridimensionato per rispondere alla richiesta di funzioni
specifiche di un’applicazione. Così, man mano che la domanda per determinati
servizi aumenta, i microservizi possono essere distribuiti su più server e infrastrutture, in base alle esigenze aziendali. In questo modo si possono creare più repliche
solo di quei microservizi che sono soggetti ad un carico lavorativo maggiore, senza
andare a creare una nuova istanza dell’intera applicazione.
Un altro vantaggio viene garantito dall’indipendenza dalla macchina su cui
il microservizio è eseguito. Tale aspetto permette di assegnare diverse potenze
computazionali ai vari microservizi dell’applicazione, andando a bilanciare le risorse
tra i diversi componenti.


Tuttavia, l’architettura a microservizi introduce anche nuove difficoltà. Essendo un sistema composto da più componenti è necessario scegliere e configurare
attentamente le connessioni tra i vari microservizi, talvolta anche introducendo
meccanismi di sicurezza. Inoltre, la fase di testing diventa molto più complicata
rispetto a quella nelle applicazioni monolitiche. Occorre ricordare che un errore in
una parte dell’architettura potrebbe causare un errore in un componente a vari
passi di distanza, a seconda del modo in cui i servizi sono strutturati per supportarsi
a vicenda.
Anche la fase iniziale di installazione dell’applicazione risulta essere più critica.
Per semplificare il deployment, occorre innanzitutto investire notevolmente in
soluzioni di automazione. La complessità dei microservizi renderebbe il deployment
manuale estremamente difficile.
È proprio la natura distribuita di queste applicazioni che rende più complessa
la loro gestione e analisi. In questo tipo di architettura, ad una singola richiesta
corrisponde un lungo percorso all’interno della rete dei microservizi. Se si dovesse
riscontare un eventuale errore o bug, risulta quindi complicato andare ad identificare
quale tra le varie componenti coinvolte ha riportato dei malfunzionamenti. Per
questo motivo è diventato sempre più importante lo sviluppo di piattaforme di
monitoraggio, che sono in grado di tracciare tutte le comunicazioni effettuate e
controllare il corretto funzionamento dei singoli microservizi.
\paragraph*{Perchè l'architettura a microservizi?}
La decisione di adottare un'architettura a microservizi è stata motivata dalla necessità di creare una struttura modulare e scalabile. L'applicazione è stata suddivisa in una suite di microservizi, ciascuno dei quali può essere sviluppato, modificato, deployato e scalato indipendentemente dagli altri.

L'architettura a microservizi si rivela una scelta solida nell'ambito dell'IoT, poiché si prevede che le diverse parti del sistema evolveranno in maniera indipendente nel tempo e poiché la scalabilità e l'isolamento dei guasti sono un aspetto critico. Questa scelta garantisce maggiore flessibilità e prestazioni ottimizzate mediante un utilizzo accurato delle risorse disponibili.

Infatti, nel contesto dell'architettura a microservizi, si presenta l'opportunità di assegnare risorse specifiche a ciascun servizio, il che permette loro di scalare in modo differenziato in base alle necessità. Questo è particolarmente vantaggioso in scenari in cui i servizi potrebbero essere sottoscritti a specifici topic e argomenti all'interno di un sistema di streaming come Kafka.

L'allocazione di risorse individualizzate consente ai servizi di adattarsi dinamicamente alla loro attività e al volume di dati con cui devono interagire. Ad esempio, un servizio che riceve un alto flusso di dati da un particolare topic potrebbe richiedere una maggiore capacità di calcolo e di memorizzazione rispetto a un altro servizio che gestisce un carico meno intenso.

Inoltre, questa flessibilità nell'allocazione delle risorse consente di ottimizzare l'efficienza complessiva del sistema, garantendo che le risorse siano allocate in modo proporzionale alla richiesta effettiva dei servizi. Ciò contribuisce a migliorare le prestazioni complessive del sistema e a garantire una gestione ottimale delle risorse disponibili.

\subsection{Messaggistica nei scenari IoT}
Nelle città intelligenti, dove vengono ricevuti grandi quantità di eventi, abbiamo bisogno di una componente software che faciliti le comunicazioni e la gestione dei dati di input. In tale ambito è utile utilizzare un broker di messaggi. I broker di messaggi consentono di mantenere completamente disaccoppiati l'origine e la destinazione dei messaggi attraverso l'implementazione di un meccanismo di comunicazione asincrono. Consentono inoltre di archiviare i messaggi nel broker finché non possono essere elaborati dal componente di destinazione, nonché da altre funzionalità di gestione. Ogni broker, infatti, offre una serie di funzionalità per la comunicazione, sia attraverso code di messaggi standard, sia attraverso un meccanismo di pubblicazione/sottoscrizione che fa uso di topic; questi sono i due modelli più utilizzati. Tra i vantaggi derivanti dall'utilizzo delle code di messaggi possiamo citare la loro capacità di implementare un algoritmo di bilanciamento del carico. Nel caso degli argomenti dei messaggi, ogni messaggio pubblicato può essere elaborato da tutti i consumatori iscritti a quell'argomento.

\paragraph{Apache Kafka e Schema Registry}
Come gia esposto in precedenza, per la gestione dei dati in streaming è stato scelto Apache Kafka, un sistema di messaggistica distribuito che offre un'architettura a bassa latenza e ad alta affidabilità. Kafka è stato progettato per gestire flussi di dati in tempo reale e per garantire la scalabilità e la tolleranza ai guasti. Inoltre, offre un'API di alto livello per la gestione dei flussi di dati, consentendo di scrivere applicazioni che possono elaborare i dati in tempo reale. Queste caratteristiche lo rendono particolarmente adatto per l'elaborazione dei dati in tempo reale nei contesti IoT, in cui è necessario gestire grandi quantità di dati provenienti da diverse fonti e garantire una bassa latenza e un'elevata affidabilità.

Inoltre l'utlizzo dello Schema Registry permette la definizioni di contratti su cui fanno affidamento produttori e consumatori di dati, garantendo la compatibilità tra le diverse versioni dei messaggi e la corretta interpretazione dei dati da parte dei consumatori. Questo è particolarmente utile in scenari in cui i dati vengono prodotti e consumati da diverse applicazioni e componenti, garantendo che i dati siano correttamente interpretati e che le applicazioni siano in grado di elaborarli in modo coerente.


\subsection{Il container deployment}
Con la tecnica della virtualizzazione leggera, viene implementato un livello di
isolamento a livello software offerto dal sistema
operativo stesso. Nascono così i container, delle unità standard di software che includo il
codice e tutte le sue dipendenze cosicché l’applicazione possa girare velocemente e
in maniera affidabile da un computer ad un altro.
Un progetto completamente open source che ancora oggi continua a perfezionare
la struttura dei container è Docker. Docker implementa un nuovo livello di
astrazione all’interno del sistemo operativo, introducendo una semplicissima gestione
dei container.
Docker ha anche sviluppato quelle che sono chiamate Docker Images, ovvero dei
pacchetti eseguibili di software, leggeri ed indipendenti, che includono tutto ciò di
cui ha bisogno l’applicazione per essere eseguita correttamente: codice, runtime,
tools di sistema, librerie di sistema e impostazioni. I Docker Image diventano
container una volta eseguiti dalla Docker Engine, una piattaforma che consente alle applicazioni containerizzate di essere eseguite ovunque in modo coerente su
qualsiasi infrastruttura, eliminando il problema dalle dipendenze software.
Al giorno d’oggi i Docker Container sono la soluzione migliore per gestire più
applicazioni su uno stesso server contemporaneamente, poiché non richiedono un
sistema operativo per applicazione. Inoltre, condividendo il kernel del sistema
operativo su cui sono in esecuzione risultano estremamente leggeri e molto più
veloci delle macchine virtuali.


Per implementare l'intero stack tecnologico e i layer di elaborazione dati in streaming, è stato configurato un ambiente Docker a microservizi che simula la divisione e la distribuzione dei layer e dei servizi. In particolare, sono stati creati container per:
\begin{itemize}
    \item \textbf{Data feed (Fonti di dati:)}:
    \begin{itemize}
        \item \textbf{Simulators}: Esegue i \textbf{simulatori di sensori} per la raccolta dei dati.
        \item Produce dati nel formato JSON definito nello Schema Registry;
        \item Non espone porte all’esterno.
    \end{itemize} 
    \item \textbf{Streaming layer}:
    \begin{itemize}
        \item Esegue \textbf{Apache Kafka} per la gestione del flusso di dati in tempo reale.
        \item Accessibile agli altri container tramite l'indirizzo \textit{\textbf{kafka:9092}}.
        \item Kafka facilita la comunicazione asincrona e disaccoppiata tra i microservizi.
        \item Permette ai microservizi di comunicare in modo indipendente dalla loro posizione, tecnologia o linguaggio di programmazione.
        \item Promuove la scalabilità e l'elasticità dei microservizi, consentendo loro di comunicare in modo efficiente anche in caso di aumento del carico.
        \item \textbf{Zookeper \& Schema Registry:} Vengono eseguiti anche due differenti container per l'esecuzione di zookeper e Schema Registry:
            \begin{itemize}
                \item \textbf{Zookeper}: Esegue il servizio di coordinamento per Kafka.
                \begin{itemize}
                    \item Esegue il servizio di coordinamento per Kafka, memorizzazione e controllo degli schemi.
                    \item Espone la porta 2181 all'esterno per permettere l'accesso al servizio di coordinamento.
                \end{itemize}
                \item \textbf{Schema Registry}:
                \begin{itemize}
                    \item Esegue il servizio di registrazione degli schemi per Kafka.
                    \item Espone la porta 8081 all'esterno per permettere l'accesso al servizio di registrazione degli schemi.
                \end{itemize}
            \end{itemize}
    \end{itemize} 
    \item \textbf{Processing layer}:
    \begin{itemize}
        \item Esegue l'app \textbf{Faust} per il processing e il calcolo del punteggio di salute.
        \item Non espone porte all'esterno.
    \end{itemize}
    \item \textbf{Storage layer}:
    \begin{itemize}
        \item Esegue \textbf{Clickhouse} per lo storage delle misurazioni.
        \item La banca dati è accessibile agli altri container tramite l'indirizzo \textit{\textbf{clickhouse:8123}}.
    \end{itemize}
    \item \textbf{Data Visualization Layer}:
    \begin{itemize}
        \item Esegue \textbf{Grafana} come interfaccia utente per la visualizzazione dei dati.
        \item Espone la porta 3000 all'esterno per permettere l'accesso al servizio di dashboarding.
    \end{itemize}
\end{itemize}
Questa struttura permette una distribuzione modulare e scalabile del sistema, semplificando la gestione e la manutenzione dei componenti e consentendo una rapida scalabilità in risposta alle esigenze emergenti. Grazie all'uso di Docker, si garantisce coerenza e riproducibilità dell'ambiente di esecuzione, semplificando il deployment e garantendo maggiore affidabilità nell'ambiente di produzione nonchè la possibilità di attribuire le risorse necessarie ad ogni servizio in modo mirato.

\subsection{Comunicazione tra i componenti}
Nel contesto di Docker, la comunicazione tra i container avviene tramite la rete Docker interna, che è una rete virtuale creata automaticamente da Docker per i servizi all'interno di uno stesso file Compose o di un ambiente Docker. Questa rete consente ai container di comunicare tra loro utilizzando i nomi dei servizi come hostnames.

Quando un container viene avviato, Docker assegna un hostname basato sul nome del servizio definito nel file Compose. Ad esempio, nel file Compose fornito, il servizio Kafka ha il nome "kafka" e il servizio ClickHouse ha il nome "clickhouse". Questi nomi sono utilizzati all'interno dei container stessi per identificare gli altri servizi. Quando un container desidera comunicare con un altro container sulla stessa rete Docker, può semplicemente utilizzare il nome del servizio come hostname.

Inoltre, Docker offre una funzionalità chiamata "Discovery", che consente ai container di scoprire automaticamente gli altri container sulla stessa rete Docker senza dover conoscere esplicitamente i loro indirizzi IP. Questo semplifica la configurazione e la gestione della comunicazione tra i container.

È anche possibile specificare dipendenze tra i servizi utilizzando l'attributo "depends\_on" nel file Compose. Questo assicura che un servizio venga avviato solo dopo che i servizi di cui dipende sono stati avviati e sono nella condizione desiderata.

Infine, per i servizi che espongono porte, come Kafka, ClickHouse e Grafana, è possibile mappare le porte del container su porte del sistema host utilizzando l'attributo "ports" nel file Compose. Questo consente ad altri componenti esterni al Docker network di comunicare con i container attraverso le porte esposte.

Quando un container invia dati a un altro container tramite la rete Docker, i dati vengono incapsulati in pacchetti di rete utilizzando il protocollo TCP/IP. Questi pacchetti vengono quindi instradati attraverso la rete Docker, che si occupa di consegnarli al destinatario corretto.

In sintesi, Docker fornisce un'infrastruttura di rete integrata che gestisce la comunicazione tra i container all'interno dello stesso ambiente di deployment, semplificando la configurazione e la gestione della comunicazione tra i diversi componenti del sistema.
\begin{itemize}
    \item \textbf{Comunicazione data feed Layer -> Streaming Layer}: Si utilizza libreria \textit{Confluent Kafka} per Python che offre un'API efficiente e flessibile per inviare dati dai simulatori dei sensori a specifici topic Kafka.
    \item \textbf{Comunicazione Processing Layer -> Streaming Layer}: Si utilizza Faust come interfaccia di alto livello per la comunicazione con Kafka. Faust offre un'interfaccia intuitiva e ben documentata per consumare dati da topic con flussi di dati in tempo reale.
    Ulteriori informazioni in: \ref{sec:faust}
    \item \textbf{Comunicazione Storage Layer -> Streaming Layer}: Per ottenere in tempo reale i dati dai topic kafka viene utilizzato l'engine kafka di Clickhouse. Ulteriori informazioni in: \ref{sec:kafka_engine}
    \item \textbf{Comunicazione Data Visualization Layer-> Storage Layer}: Grafana si connette a Clickhouse per ottenere i dati da visualizzare in tempo reale tramite lo specifico plugin ClickHouse che permette l'utilizzo di un database Clickhouse come \textit{data source}.
    Il plugin nasconde alcuni dettagli di implementazione sottostanti, come la gestione della connessione,protocolli di comunicazione e l'esecuzione delle query. Ulteriori informazioni in: \ref{sec:click_plugin}
\end{itemize}
    
\subsubsection{Dipendenze tra i servizi}
\begin{itemize}
    \item \textbf{Clickhouse -> kafka}: Il servizio ClickHouse dipende dal servizio Kafka. Questo assicura che il servizio ClickHouse venga avviato solo dopo che il servizio Kafka è stato avviato e è nella condizione desiderata.

    \item \textbf{Simulators -> kafka}: Il servizio Simulators dipende dal servizio Kafka. Questo assicura che il servizio Simulators venga avviato solo dopo che il servizio Kafka è stato avviato e è nella condizione desiderata.
    
    \item \textbf{ Faust app -> kafka}: Il servizio Processor dipende dal servizio Kafka. Questo assicura che il servizio Processor venga avviato solo dopo che il servizio Kafka è stato avviato e è nella condizione desiderata.
\end{itemize}


