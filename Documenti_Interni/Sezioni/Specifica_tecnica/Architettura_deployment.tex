\section{Architettura di deployment}

Al fine di implementare ed eseguire l'intero stack tecnologico ed i layer del modello architetturale, è stato configurato un ambiente Docker che simula la suddivisione e la distribuzione dei servizi.
Informazioni aggiuntive sulle immagini utilizzate e sulle configurazioni dell'ambiente sono disponibili nel file \textit{docker-compose.yml} presente nel repository \textit{MVP} del progetto oltre che nella sezione: \ref{sec:docker}.

In particolare,per l'ambiente di produzione, sono stati creati i seguenti container:
\begin{itemize}
    \item \textbf{Data feed}:
    \begin{itemize}
        \item Container \textbf{Simulators}: 
        \begin{itemize}
            \item Esegue i \textbf{simulatori dei sensori} per la raccolta dei dati;
            \item Produce dati nel formato JSON definito nello Schema Registry e li invia al broker Kafka;
        \end{itemize}
    \end{itemize} 
    \item \textbf{Streaming layer}:
    \begin{itemize}
        \item Container \textbf{Kafka}:  
        \begin{itemize}
            \item Esegue \textbf{Apache Kafka} per la gestione del flusso di dati in tempo reale;
            \item Accessibile agli altri container tramite l'indirizzo \textit{\textbf{kafka:9092}}.
        \end{itemize}
        \item Componenti di supporto:
       \begin{itemize}
        \item Container \textbf{Zookeper}:
        \begin{itemize}
            \item Esegue il servizio di coordinamento per Kafka, oltre alla memorizzazione e al controllo degli schemi;
            \item Accessibile dagli altri container attraverso l’indirizzo \textit{\textbf{zookeeper:2181}}.
        \end{itemize}
        \item Container \textbf{Schema Registry}:
        \begin{itemize}
            \item Esegue il servizio di registrazione degli schemi per Kafka;
            \item Accessibile dagli altri container attraverso l’indirizzo \textit{\textbf{schema\_registry:8081}};
            \item Per la registrazione degli schemi vengono eseguiti container che tramite API REST (fornita dallo schema registry) permettono la registrazione degli schemi JSON.
        \end{itemize}
       \end{itemize}
        
    \end{itemize} 
    \item \textbf{Processing layer}:
    \begin{itemize}
        \item Container \textbf{Faust}:
        \begin{itemize}
            \item Esegue l'app \textbf{Faust} per il processing e il calcolo del punteggio di salute;
        \end{itemize}
    \end{itemize}
    \item \textbf{Storage layer}:
    \begin{itemize}
        \item Container \textbf{Clickhouse}:
        \begin{itemize}
            \item Esegue \textbf{Clickhouse} per lo storage delle misurazioni;
            \item La banca dati è accessibile agli altri container tramite l'indirizzo \textit{\textbf{clickhouse:8123}}.
        \end{itemize}
    \end{itemize}
    \item \textbf{Data Visualization Layer}:
    \begin{itemize}
        \item Container \textbf{Grafana}:
        \begin{itemize}
            \item Esegue \textbf{Grafana} come interfaccia utente per la visualizzazione dei dati;
            \item Espone la porta \textbf{\textit{3000}} all'esterno per permettere l'accesso al servizio di dashboarding.
        \end{itemize}
    \end{itemize}
    
\end{itemize}
Questa struttura permette una distribuzione modulare e scalabile del sistema, semplificando la gestione e la manutenzione dei componenti e consentendo una rapida scalabilità in risposta alle esigenze emergenti. Grazie all'uso di Docker, si garantisce coerenza e riproducibilità dell'ambiente di esecuzione, semplificando il deployment e garantendo maggiore affidabilità nell'ambiente di produzione nonchè la possibilità di attribuire le risorse necessarie ad ogni servizio in modo mirato.
