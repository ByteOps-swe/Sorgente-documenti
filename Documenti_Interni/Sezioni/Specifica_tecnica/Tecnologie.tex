\section{Tecnologie}
In questa sezione vengono definiti gli strumenti e le tecnologie impiegati per lo sviluppo e l'implementazione del software relativo al progetto InnovaCity. Si procederà quindi con la descrizione delle tecnologie e dei linguaggi di programmazione utilizzati, delle librerie e dei framework necessari, nonché delle infrastrutture richieste. L'obiettivo principale è garantire che il software sia sviluppato utilizzando le tecnologie più appropriate e selezionando le opzioni ottimali in termini di efficienza, sicurezza e affidabilità.

\paragraph*{Ambiente di sviluppo}
Per lo sviluppo del progetto sono stati utilizzati container Docker per garantire un ambiente di sviluppo consistente e riproducibile. Di seguito sono elencate le immagini Docker utilizzate:

\begin{itemize}
  \item \textbf{Python:} \href{https://hub.docker.com/_/python}{Python:3.9}
  \item \textbf{Apache Kafka:} \href{https://hub.docker.com/layers/bitnami/kafka/latest/images/sha256-4894d89d28f8e06a7d8a064efdc2dc9cb61dd205721c61296b6d033ad4824a91?context=explore}{bitnami/kafka:latest}
  \item \textbf{ClickHouse:} 
  \item \textbf{Grafana:} 
\end{itemize}
\subsection{Linguaggi e formato dati}
\subsubsection{Python}
Linguaggio di programmazione ad alto livello, interpretato e multi-paradigma.

\paragraph{Versione:}
Versione utilizzata: 3.9
\paragraph{Documentazione:}
\url{https://docs.python.org/release/3.9.0/}

\paragraph{Utilizzo nel progetto} 
\begin{itemize}
    \item Creazione delle simulazioni dei sensori, incluse le logiche di scrittura e invio dei dati registrati;
    \item Modello per il calcolo del punteggio di salute della città;
    \item Testing.
\end{itemize}

\paragraph{Librerie o framework}

\begin{itemize}
    \item \textbf{Confluent Kafka}
    \begin{itemize}
        \item \textbf{Documentazione:} \url{https://developer.confluent.io/get-started/python/}~(consultato: 19/03/2024);
        \item \textbf{Versione:} 2.3.0;
        \item Libreria Python che fornisce un insieme completo di strumenti per agevolare la produzione e il consumo di messaggi da Apache Kafka.
    \end{itemize}
    
    \item \textbf{Faust}
    \begin{itemize}
        \item \textbf{Documentazione:} \url{https://faust.readthedocs.io/en/latest/}~(consultato: 19/03/2024);
        \item \textbf{Versione:} 1.10.4;
        \item Framework Python per la creazione di applicazioni di data streaming in tempo reale. Fornisce un'API dichiarativa e funzionale per definire i flussi di dati e le trasformazioni, consentendo agli sviluppatori di scrivere facilmente applicazioni scalabili e affidabili per il trattamento di grandi volumi di dati in tempo reale.
        
        Faust si integra nativamente con Apache Kafka e offre funzionalità avanzate come il bilanciamento del carico, la gestione dello stato, la gestione delle query, e la tolleranza ai guasti, rendendolo una scelta ottimale per lo sviluppo di sistemi di data streaming complessi e robusti.
    \end{itemize}
    
    \item \textbf{Pytest}
    \begin{itemize}
        \item \textbf{Documentazione:} \url{https://docs.pytest.org/en/7.1.x/contents.html}~(consultato: 19/03/2024);
        \item \textbf{Versione:} 8.0.2;
        \item Framework di testing per Python, noto per la sua semplicità. Consente agli sviluppatori di scrivere test chiari e concisi utilizzando una sintassi intuitiva e flessibile.
        
        Pytest supporta una vasta gamma di funzionalità, tra cui test di unità, integrazione e accettazione, parametrizzazione dei test e gestione delle fixture.

        Merita menzione anche l'utilizzo di \textit{Pytest-asyncio} per testare codice asincrono e \textit{Pytest-cov} per la copertura del codice.
    \end{itemize}
    
    \item \textbf{Pylint}
    \begin{itemize}
        \item \textbf{Documentazione:} \url{https://pylint.readthedocs.io/en/stable/}~(consultato: 19/03/2024);
        \item \textbf{Versione:} 3.1.0;
        \item Strumento di analisi statica per il linguaggio di programmazione Python. Esamina il codice sorgente per individuare potenziali errori, conformità alle linee guida stilistiche e altre possibili fonti di bug nel codice Python. Inoltre, valuta anche la qualità del codice in termini di \textit{good practice} di programmazione.
        
        Pylint fornisce un punteggio di qualità del codice e suggerimenti per migliorare la leggibilità, la manutenibilità, sicurezza e la correttezza del codice Python.
    \end{itemize}
    
    \item \textbf{Clickhouse-connect}
    \begin{itemize}
        \item \textbf{Documentazione:} \url{https://clickhouse.com/docs/en/integrations/python}~(consultato: 19/03/2024);
        \item \textbf{Versione:} 0.7.2;
        \item ClickHouse Connect è una libreria open source sviluppata per semplificare l'interazione con il database ClickHouse tramite il linguaggio di programmazione Python, viene utilizzata nei test.
        
        Essa fornisce un'interfaccia per comunicare con ClickHouse, consentendo agli sviluppatori di eseguire query, inserire dati e gestire altri aspetti dell'interazione con il database in modo efficiente e conveniente.
    \end{itemize}
\end{itemize}

\subsubsection{SQL (Structured Query Language)}
Linguaggio standard per la gestione e la manipolazione dei
database che lo supportano. \todo{arricchire un po' questa parte}

\paragraph{Utilizzo nel progetto}
Gestione e interrogazione database Clickhouse.


\subsection{JSON (JavaScript Object Notation)}
JSON è un formato di scrittura leggibile dalle persone e facilmente interpretabile dai computer. È utilizzato principalmente per lo scambio di dati strutturati attraverso le reti, come Internet.

Il formato JSON si basa su due strutture di dati principali:

\begin{itemize}
  \item \textbf{Oggetti}: Rappresentati da coppie chiave-valore racchiuse tra parentesi graffe \{ \}, dove la chiave è una stringa e il valore può essere un altro oggetto, un array, una stringa, un numero, un booleano o \texttt{null}.
  \item \textbf{Array}: Una raccolta ordinata di valori, racchiusi tra parentesi quadre [ ], in cui ogni elemento può essere un oggetto, un array, una stringa, un numero, un booleano o \texttt{null}.
\end{itemize}

JSON offre una sintassi semplice e chiara per la rappresentazione dei dati, che lo rende ampiamente utilizzato in molti contesti, inclusi lo sviluppo web, le API di servizi web e lo scambio di dati tra applicazioni. La sua leggibilità e la sua natura basata su testo lo rendono particolarmente adatto per l'interazione tra sistemi eterogenei.

Nel nostro contesto viene utilizzato per scambiare i dati dai simulatori \textit{Python} a \textit{kafka}, e dal server \textit{kafka} a \textit{Clickhouse}.
\subsubsection{YAML (YAML Ain't Markup Language)}
Formato di serializzazione leggibile dall'uomo utilizzato per rappresentare dati strutturati in modo chiaro e semplice.

\paragraph{Utilizzo nel progetto}
\begin{itemize}
    \item Configurazione
    docker compose;
    \item Configurazione pipeline Git-Hub workflow per Countinuous Integration;
    \item Configurazione provisioning Grafana e politiche di notifica allerte.
\end{itemize}

\subsection{Database e servizi}
\subsubsection{Apache Kafka}
Apache Kafka è una piattaforma open-source di streaming distribuito sviluppata dall'Apache Software Foundation. Progettata per gestire flussi di dati in tempo reale in modo scalabile e affidabile, è ampiamente utilizzata nel data streaming e nell'integrazione dei dati nelle moderne applicazioni.

\paragraph{Versione}
La versione utilizzata è: 3.7.0
\paragraph{Documentazione}
\href{https://kafka.apache.org/20/documentation.html}{https://kafka.apache.org/20/documentation.html}

\paragraph{Funzionalità e vantaggi di Apache Kafka}
Le principali funzionalità e vantaggi di Apache Kafka includono:

\begin{itemize}
  \item \textbf{Pub-Sub Messaging:} Kafka utilizza un modello di messaggistica publish-subscribe, dove i produttori di dati inviano messaggi ad uno o più topic e i consumatori possono sottoscriversi a tali topic per ricevere i messaggi;
  
  \item \textbf{Disaccoppiamento Produttore - Consumatore:} questo principio si realizza grazie al fatto che i Produttori e i Consumatori non necessitano di essere consapevoli l'uno dell'altro o di interagire direttamente. Invece, essi comunicano attraverso il broker Kafka, che svolge il ruolo di intermediario per la trasmissione dei messaggi. Ciò consente una maggiore scalabilità e flessibilità nell'architettura del sistema, facilitando la gestione e il mantenimento delle applicazioni;
  
  \item \textbf{Architettura Distribuita:} Kafka è progettato per essere distribuito su un cluster di nodi, consentendo una scalabilità orizzontale per gestire grandi volumi di dati e carichi di lavoro. Questo approccio distribuito offre resilienza e alta disponibilità, garantendo che il sistema possa crescere in modo flessibile con l'aumentare delle richieste;
  
  \item \textbf{Persistenza e Affidabilità:} Kafka offre la possibilità di definire politiche specifiche per la conservazione dei dati, garantendo la durabilità dei messaggi. Questo non solo assicura la disponibilità dei dati anche in caso di eventuali interruzioni del servizio, ma consente anche ai consumatori di recuperare i messaggi dopo tali anomalie, garantendo un alto livello di affidabilità nel sistema.
  
  \item \textbf{Alta Disponibilità:} Kafka assicura un'elevata disponibilità e tolleranza ai guasti grazie alla sua architettura distribuita e al meccanismo di replica dei dati. Anche in caso di malfunzionamenti dei nodi o delle componenti, i cluster di Kafka mantengono la loro operatività, garantendo la continuità del servizio.
  
  \item \textbf{Elaborazione degli Stream:} Kafka supporta anche l'elaborazione degli stream di dati in tempo reale tramite API come Kafka Streams e Kafka Connect, consentendo agli sviluppatori di scrivere applicazioni per l'analisi e l'elaborazione dei dati in tempo reale.
\end{itemize}

\paragraph{Casi d'uso di Apache Kafka}

Apache Kafka è utilizzato in una vasta gamma di casi d'uso, tra cui:

\begin{itemize}
  \item \textbf{Data Integration:} Kafka viene utilizzato per integrare dati provenienti da diverse fonti e sistemi, consentendo lo scambio di dati in tempo reale tra applicazioni e sistemi eterogenei.
  
  \item \textbf{Streaming di Eventi:} Molte applicazioni moderne, come le applicazioni IoT (Internet of Things) e le applicazioni di monitoraggio in tempo reale, utilizzano Kafka per lo streaming di eventi in tempo reale e l'analisi dei dati.
  
  \item \textbf{Analisi dei Log:} Kafka è spesso utilizzato per l'analisi dei log di sistema e applicativi in tempo reale, consentendo il monitoraggio delle prestazioni, la rilevazione degli errori e l'analisi dei pattern di utilizzo.
  
  \item \textbf{Elaborazione di Big Data:} Kafka è integrato con tecnologie di big data come Apache Hadoop e Apache Spark, consentendo l'elaborazione di grandi volumi di dati in tempo reale.
  
  \item \textbf{Messaggistica Real-time:} Kafka è ampiamente utilizzato per la messaggistica real-time in applicazioni di social media, e-commerce e finanziarie, dove la velocità e l'affidabilità della messaggistica sono cruciali.
\end{itemize}

\paragraph{Utilizzo nel progetto}
\textit{Kafka} funge da intermediario dei messaggi, ricevendo i dati dai produttori e rendendoli disponibili ai consumatori. Nel contesto del progetto, i dati provenienti dalle simulazioni di sensori vengono inviati a \textit{Kafka} come messaggi in formato \textit{JSON}.

\paragraph*{Consumatori di dati:}
\begin{itemize}
  \item \textbf{\textit{ClickHouse:}} \textit{Kafka} invia \todo{è Kafka che li invia o i consumatori che se li prendono da Kafka?} i dati ai consumatori, inclusi i database come \textit{ClickHouse}, dove i dati vengono salvati per l'analisi e l'archiviazione a lungo termine.
  \item \textbf{\textit{Faust:}} per soddisfare il requisito opzionale del calcolo del punteggio di salute, \textit{Kafka} rende disponibili i dati in tempo reale a un'applicazione di Faust\todo{è corretto applicazione di Faust?}. Quest'ultima elabora i dati utilizzando una funzione di aggregazione per calcolare il punteggio e quindi mette a disposizione il risultato in una coda dedicata di Kafka per i servizi interessati.
\end{itemize}

In breve, \textit{Kafka} funge da ponte tra i produttori di dati (simulazioni di sensori) e i consumatori di dati (\textit{ClickHouse} o altri servizi futuri). Gestisce il flusso dei dati in tempo reale e garantisce che i dati siano disponibili per l'elaborazione e la visualizzazione in modo efficiente e scalabile.
\subsubsection{Clickhouse} \label{sec:clickHouse}
\textit{Clickhouse}\textsubscript{\textit{G}} è un \textit{sistema}\textsubscript{\textit{G}} di gestione di \textit{database}\textsubscript{\textit{G}} (DBMS) di tipo column-oriented, progettato principalmente per l'analisi di grandi volumi di dati in tempo reale. È un progetto \textit{open-source}\textsubscript{\textit{G}} creato per rispondere alle esigenze di elaborazione analitica ad alte prestazioni.

\paragraph{Versione}
Versione utilizzata: 24.2.1.2248
\paragraph{Documentazione:}
\url{https://clickhouse.com/docs/en/intro}

\paragraph{Funzionalità e Vantaggi di Clickhouse}
\begin{itemize}
    \item \textbf{Modello di dati column-oriented:} a differenza dei tradizionali DBMS che memorizzano i dati in modo row-oriented, dove le righe complete sono memorizzate in sequenza, \textit{clickhouse}\textsubscript{\textit{G}} memorizza i dati in modo column-oriented. Questo significa che i dati di una stessa colonna vengono memorizzati contiguamente, permettendo una maggiore compressione e velocità di \textit{query}\textsubscript{\textit{G}} per le analisi che coinvolgono molte colonne;
    \item \textbf{Architettura distribuita e scalabilità:} \textit{Clickhouse}\textsubscript{\textit{G}} è progettato per funzionare in un ambiente distribuito, consentendo la scalabilità orizzontale per gestire grandi carichi di lavoro;
    \item \textbf{Compressione dei dati:} utilizza algoritmi efficienti per ridurre lo spazio di archiviazione richiesto per i dati, riducendo i costi di archiviazione;
    \item \textbf{Alte prestazioni:} \textit{Clickhouse}\textsubscript{\textit{G}} è ottimizzato per eseguire \textit{query}\textsubscript{\textit{G}} analitiche su grandi volumi di dati in tempo reale, garantendo tempi di risposta bassi anche con carichi di lavoro elevati.
    \item \textbf{Supporto per SQL:} \textit{Clickhouse}\textsubscript{\textit{G}} supporta un sottoinsieme del linguaggio SQL;
    \item \textbf{Integrazione con Strumenti di Business Intelligence (BI):} può essere integrato con strumenti di BI popolari come \textit{Grafana}\textsubscript{\textit{G}} per la visualizzazione e l'analisi dei dati.
\end{itemize}

\paragraph{Casi d'uso di Clickhouse}
\textit{Clickhouse}\textsubscript{\textit{G}} è adatto per una vasta gamma di casi d'uso, tra cui:
\begin{itemize}
    \item \textbf{Analisi dei log:} \textit{Clickhouse}\textsubscript{\textit{G}} può essere utilizzato per analizzare i \textit{log}\textsubscript{\textit{G}} di grandi dimensioni generati da server, applicazioni web e dispositivi IoT;
    \item \textbf{Analisi dei dati in tempo reale:} \textit{Clickhouse}\textsubscript{\textit{G}} è ideale per l'analisi dei dati in tempo reale, consentendo agli utenti di eseguire \textit{query}\textsubscript{\textit{G}} complesse su flussi di dati in continua evoluzione;
\end{itemize}

\paragraph{Utilizzo nel progetto}
Nel contesto del progetto, \textit{Clickhouse}\textsubscript{\textit{G}} svolge una serie di ruoli cruciali per garantire l'efficacia e l'efficienza dell'analisi e delle persistenza dei dati provenienti dai sensori IoT:

\begin{itemize}
  \item \textbf{Integrazione con Kafka}: \textit{Clickhouse}\textsubscript{\textit{G}} viene utilizzato per recuperare in tempo reale i dati dal server \textit{Kafka}\textsubscript{\textit{G}}, consentendo una continua acquisizione dei dati dai sensori IoT. Questa \textit{integrazione}\textsubscript{\textit{G}} permette di assicurare che le informazioni più recenti siano immediatamente disponibili per l'analisi.
  
  \item \textbf{Organizzazione efficiente dei dati:} \textit{Clickhouse}\textsubscript{\textit{G}} è in grado di organizzare grandi volumi di dati in modo ottimale grazie alla sua \textit{architettura}\textsubscript{\textit{G}} columnar che consente una compressione dei dati efficace e un accesso rapido alle informazioni, migliorando le prestazioni complessive del \textit{sistema}\textsubscript{\textit{G}}.
  
  \item \textbf{Aggregazione rapida dei dati:} \textit{Clickhouse}\textsubscript{\textit{G}} offre potenti funzionalità per eseguire operazioni di aggregazione sui dati in modo rapido e incrementale. Ciò significa che è possibile ottenere risposte rapide alle \textit{query}\textsubscript{\textit{G}} di aggregazione anche su enormi quantità di dati, consentendo analisi in \textit{quasi-real-time} delle misurazioni dei sensori IoT.
  
  \item \textbf{Integrazione con Grafana:} I dati elaborati e aggregati da \textit{Clickhouse}\textsubscript{\textit{G}} sono resi disponibili per il reperimento tramite \textit{Grafana}\textsubscript{\textit{G}}. \textit{Grafana}\textsubscript{\textit{G}} consente di creare \textit{dashboard}\textsubscript{\textit{G}} interattive e \textit{report}\textsubscript{\textit{G}} visivi basati sui dati ricevuti offrendo agli utenti un'interfaccia intuitiva per l'analisi e la visualizzazione dei dati.
\end{itemize}

\subsection{Grafana}
Grafana è una piattaforma open-source per la visualizzazione e l'analisi dei dati, utilizzata per creare dashboard interattive e grafici da fonti di dati eterogenee. 
\subsubsection{Versione}
La versione utilizzata è: x.x.x
\subsubsection{Link download}
\href{https://clickhouse.com/}{https://clickhouse.com/}

\subsubsection{Funzionalità e Vantaggi di Grafana}
\begin{itemize}
    \item \textbf{Dashboard interattive}: Creazione di dashboard personalizzate e interattive per visualizzare dati provenienti da diverse fonti in un'unica interfaccia.
    
    \item \textbf{Connessione a sorgenti di dati eterogenee}: Supporto per una vasta gamma di sorgenti di dati, inclusi database, servizi cloud, sistemi di monitoraggio, API e altro ancora.
    
    \item \textbf{Ampia varietà di visualizzazioni}: Selezione di pannelli e visualizzazioni, tra cui grafici a linea, a barre, a torta, termometri, mappe geografiche e altro ancora, per adattarsi alle esigenze specifiche di visualizzazione dei dati.
    
    \item \textbf{Query e aggregazioni flessibili}: Esecuzione di query flessibili e aggregazione dei dati in modi personalizzati per ottenere insight approfonditi dai dati.
    
    \item \textbf{Notifiche e allarmi}: Impostazione di avvisi in base a criteri predefiniti, come soglie di performance, e ricezione di notifiche tramite diversi canali, tra cui email, Slack e molti altri.
    
    \item \textbf{Gestione degli accessi e dei permessi}: Controllo degli accessi e dei permessi degli utenti in modo granulare, gestendo chi può visualizzare, modificare o creare dashboard e pannelli.
    
    \item \textbf{Integrazione con altre applicazioni e strumenti}: Integrazione con una vasta gamma di applicazioni e strumenti, tra cui sistemi di log management, strumenti di monitoraggio delle prestazioni, sistemi di allerta e altro ancora.
    
   \end{itemize}
\subsubsection{Casi d'Uso di Grafana}
\begin{itemize}
    \item \textbf{Monitoraggio delle prestazioni}: Monitoraggio in tempo reale delle metriche di sistema come CPU, memoria e rete per identificare e risolvere rapidamente problemi di prestazioni.
    
    \item \textbf{Analisi dei log}: Analisi e visualizzazione dei log delle applicazioni e dell'infrastruttura per individuare pattern e risolvere problemi operativi.
    
    \item \textbf{Monitoraggio dell'infrastruttura}: Monitoraggio dello stato e delle prestazioni di server, servizi cloud, database e altri componenti IT per garantire un funzionamento ottimale dell'infrastruttura.
    
    \item \textbf{DevOps e CI/CD}: Monitoraggio dei processi di sviluppo, test e distribuzione del software per migliorare la collaborazione e l'efficienza del team.
    
    \item \textbf{Monitoraggio di dispositivi IoT}: Monitoraggio dei dispositivi IoT per raccogliere e visualizzare dati di sensori e dispositivi connessi, consentendo una gestione efficiente degli ambienti IoT.
\end{itemize}
\subsubsection{Utilizzo nel progetto}
Nel contesto di un progetto che coinvolge la visualizzazione e l'analisi di miliardi di misurazioni di sensori IoT, Grafana viene utilizzato principalmente per:

\begin{itemize}
  \item \textbf{Visualizzazione dei dati}: Grafana consente agli utenti di creare dashboard personalizzate e grafici interattivi che mostrano i dati provenienti dai sensori IoT in modo chiaro e comprensibile. Questi grafici possono essere configurati per visualizzare metriche specifiche nel formato desiderato, consentendo agli utenti di monitorare facilmente le prestazioni dei sensori e rilevare eventuali pattern o anomalie nei dati.
  
  \item \textbf{Analisi dei dati}: Grafana offre una vasta gamma di opzioni per analizzare i dati, inclusi filtri, aggregazioni, calcoli e altro ancora. Gli utenti possono eseguire query sui dati direttamente da Grafana e visualizzare i risultati in grafici, permettendo loro di ottenere una comprensione più approfondita delle tendenze e dei modelli presenti nei dati dei sensori IoT.
  
  \item \textbf{Monitoraggio in tempo reale}: Grafana supporta il monitoraggio in tempo reale dei dati, consentendo agli utenti di visualizzare aggiornamenti istantanei sui valori dei sensori e le metriche correlate. Ciò è particolarmente utile per l'analisi delle prestazioni in tempo reale e per la rilevazione immediata di problemi o anomalie nei dati dei sensori.
  
  \item \textbf{Allerta e notifica}: Grafana offre funzionalità avanzate di allerta e notifica che consentono agli utenti di impostare avvisi basati su condizioni specifiche dei dati. Ad esempio, è possibile configurare Grafana per inviare notifiche via email o tramite servizi di messaggistica istantanea quando un determinato sensore supera una soglia prestabilita o quando si verifica un'anomalia nei dati.
\end{itemize} 


