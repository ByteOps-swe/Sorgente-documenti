\section{Tecnologie}
In questa sezione sono definiti gli strumenti e le tecnologie impiegati per lo sviluppo e l'implementazione del software relativo al progetto InnovaCity.

Si procederà quindi con la descrizione delle tecnologie e dei linguaggi di programmazione utilizzati, delle librerie e dei framework necessari, nonché delle infrastrutture richieste.

L'obiettivo principale è garantire che il software sia sviluppato utilizzando le tecnologie più appropriate e selezionando le opzioni ottimali in termini di efficienza, sicurezza e affidabilità.

\subsubsection*{Premessa}
Per lo sviluppo, il testing e il rilascio del prodotto sono stati utilizzati container Docker per garantire ambienti consistenti e riproducibili.
\begin{itemize}
  \item \textbf{Ambiente di sviluppo:}
    \begin{itemize}
      \item È l'ambiente dove i software developer scrivono, testano e modificano il codice sorgente.
      \item Può includere strumenti di debug e monitoraggio per facilitare lo sviluppo e la correzione di errori.
      \item Non è accessibile agli utenti finali.
    \end{itemize}
    \item \textbf{Ambiente di test:}
    \begin{itemize}
      \item Simula l'ambiente di produzione, replicando l'hardware, il software e le configurazioni reali.
      \item Viene utilizzato per testare il software in modo completo e realistico prima del rilascio in produzione.
      \item I test possono essere automatizzati o manuali e includono test di funzionalità, integrazione, sicurezza e prestazioni.
    \end{itemize}
    \item \textbf{Ambiente di produzione:}
    \begin{itemize}
      \item È l'ambiente finale dove il software viene rilasciato e utilizzato dagli utenti finali.
      \item essere stabile, sicuro e performante per garantire un'esperienza utente ottimale.
      \item modifiche al software in produzione sono controllate rigorosamente per minimizzare i rischi di errori o downtime.
    \end{itemize}
\end{itemize}



\paragraph*{Ambiente docker}

Di seguito sono elencate le immagini Docker utilizzate:
\begin{itemize}
  \item \textbf{Simulators - Python:} 
  \begin{itemize}
    \item \textbf{Image:}\href{https://hub.docker.com/_/python}{Python:3.9}
    \item \textbf{Ambiente:} [develop,production]
  \end{itemize}
  \item \textbf{Apache Kafka} 
 \begin{itemize}
    \item \textbf{Image:}\href{https://hub.docker.com/layers/bitnami/kafka/latest/images/sha256-4894d89d28f8e06a7d8a064efdc2dc9cb61dd205721c61296b6d033ad4824a91?context=explore}{bitnami/kafka:latest}
    \item \textbf{Ambiente:} [develop,production,testing]
  \end{itemize}
  \item \textbf{Apache Kafka UI} 
 \begin{itemize}
    \item \textbf{Image:}
    \item \textbf{Ambiente:} [develop]
  \end{itemize}
  \item \textbf{Schema registry} 
 \begin{itemize}
    \item \textbf{Image:}
    \item \textbf{Ambiente:} [develop,production,testing]
  \end{itemize}
  \item \textbf{Schema registry UI} 
 \begin{itemize}
    \item \textbf{Image:}
    \item \textbf{Ambiente:} [develop]
  \end{itemize}
  \item \textbf{Faust processing - Python} 
 \begin{itemize}
    \item \textbf{Image:}
    \item \textbf{Ambiente:} [develop,production,testing]
  \end{itemize}
  \item \textbf{ClickHouse} 
  \begin{itemize}
    \item \textbf{Image:}
    \item \textbf{Ambiente:} [develop,production,testing]
  \end{itemize}
  \item \textbf{Grafana:} 
  \begin{itemize}
    \item \textbf{Image:}
    \item \textbf{Ambiente:} [develop,production]
  \end{itemize}
\end{itemize}

\subsection{Linguaggi e formato dati}
\subsubsection{Python}
Linguaggio di programmazione ad alto livello, interpretato e multi-paradigma.

\paragraph{Versione:}
Versione utilizzata: 3.9
\paragraph{Documentazione:}
\url{https://docs.python.org/release/3.9.0/}

\paragraph{Utilizzo nel progetto} 
\begin{itemize}
    \item Creazione delle simulazioni dei sensori, incluse le logiche di scrittura e invio dei dati registrati;
    \item Modello per il calcolo del punteggio di salute della città;
    \item Testing.
\end{itemize}

\paragraph{Librerie o framework}

\begin{itemize}
    \item \textbf{Confluent Kafka}
    \begin{itemize}
        \item \textbf{Documentazione:} \url{https://developer.confluent.io/get-started/python/}~(consultato: 19/03/2024);
        \item \textbf{Versione:} 2.3.0;
        \item Libreria Python che fornisce un insieme completo di strumenti per agevolare la produzione e il consumo di messaggi da Apache Kafka.
    \end{itemize}
    
    \item \textbf{Faust}
    \begin{itemize}
        \item \textbf{Documentazione:} \url{https://faust.readthedocs.io/en/latest/}~(consultato: 19/03/2024);
        \item \textbf{Versione:} 1.10.4;
        \item Framework Python per la creazione di applicazioni di data streaming in tempo reale. Fornisce un'API dichiarativa e funzionale per definire i flussi di dati e le trasformazioni, consentendo agli sviluppatori di scrivere facilmente applicazioni scalabili e affidabili per il trattamento di grandi volumi di dati in tempo reale.
        
        Faust si integra nativamente con Apache Kafka e offre funzionalità avanzate come il bilanciamento del carico, la gestione dello stato, la gestione delle query, e la tolleranza ai guasti, rendendolo una scelta ottimale per lo sviluppo di sistemi di data streaming complessi e robusti.
    \end{itemize}
    
    \item \textbf{Pytest}
    \begin{itemize}
        \item \textbf{Documentazione:} \url{https://docs.pytest.org/en/7.1.x/contents.html}~(consultato: 19/03/2024);
        \item \textbf{Versione:} 8.0.2;
        \item Framework di testing per Python, noto per la sua semplicità. Consente agli sviluppatori di scrivere test chiari e concisi utilizzando una sintassi intuitiva e flessibile.
        
        Pytest supporta una vasta gamma di funzionalità, tra cui test di unità, integrazione e accettazione, parametrizzazione dei test e gestione delle fixture.

        Merita menzione anche l'utilizzo di \textit{Pytest-asyncio} per testare codice asincrono e \textit{Pytest-cov} per la copertura del codice.
    \end{itemize}
    
    \item \textbf{Pylint}
    \begin{itemize}
        \item \textbf{Documentazione:} \url{https://pylint.readthedocs.io/en/stable/}~(consultato: 19/03/2024);
        \item \textbf{Versione:} 3.1.0;
        \item Strumento di analisi statica per il linguaggio di programmazione Python. Esamina il codice sorgente per individuare potenziali errori, conformità alle linee guida stilistiche e altre possibili fonti di bug nel codice Python. Inoltre, valuta anche la qualità del codice in termini di \textit{good practice} di programmazione.
        
        Pylint fornisce un punteggio di qualità del codice e suggerimenti per migliorare la leggibilità, la manutenibilità, sicurezza e la correttezza del codice Python.
    \end{itemize}
    
    \item \textbf{Clickhouse-connect}
    \begin{itemize}
        \item \textbf{Documentazione:} \url{https://clickhouse.com/docs/en/integrations/python}~(consultato: 19/03/2024);
        \item \textbf{Versione:} 0.7.2;
        \item ClickHouse Connect è una libreria open source sviluppata per semplificare l'interazione con il database ClickHouse tramite il linguaggio di programmazione Python, viene utilizzata nei test.
        
        Essa fornisce un'interfaccia per comunicare con ClickHouse, consentendo agli sviluppatori di eseguire query, inserire dati e gestire altri aspetti dell'interazione con il database in modo efficiente e conveniente.
    \end{itemize}
\end{itemize}

\subsubsection{SQL (Structured Query Language)}
Linguaggio standard per la gestione e la manipolazione dei
database che lo supportano. \todo{arricchire un po' questa parte}

\paragraph{Utilizzo nel progetto}
Gestione e interrogazione database Clickhouse.


\subsection{JSON (JavaScript Object Notation)}
JSON è un formato di scrittura leggibile dalle persone e facilmente interpretabile dai computer. È utilizzato principalmente per lo scambio di dati strutturati attraverso le reti, come Internet.

Il formato JSON si basa su due strutture di dati principali:

\begin{itemize}
  \item \textbf{Oggetti}: Rappresentati da coppie chiave-valore racchiuse tra parentesi graffe \{ \}, dove la chiave è una stringa e il valore può essere un altro oggetto, un array, una stringa, un numero, un booleano o \texttt{null}.
  \item \textbf{Array}: Una raccolta ordinata di valori, racchiusi tra parentesi quadre [ ], in cui ogni elemento può essere un oggetto, un array, una stringa, un numero, un booleano o \texttt{null}.
\end{itemize}

JSON offre una sintassi semplice e chiara per la rappresentazione dei dati, che lo rende ampiamente utilizzato in molti contesti, inclusi lo sviluppo web, le API di servizi web e lo scambio di dati tra applicazioni. La sua leggibilità e la sua natura basata su testo lo rendono particolarmente adatto per l'interazione tra sistemi eterogenei.

Nel nostro contesto viene utilizzato per scambiare i dati dai simulatori \textit{Python} a \textit{kafka}, e dal server \textit{kafka} a \textit{Clickhouse}.
\subsubsection{YAML (YAML Ain't Markup Language)}
Formato di serializzazione leggibile dall'uomo utilizzato per rappresentare dati strutturati in modo chiaro e semplice.

\paragraph{Utilizzo nel progetto}
\begin{itemize}
    \item Configurazione
    docker compose;
    \item Configurazione pipeline Git-Hub workflow per Countinuous Integration;
    \item Configurazione provisioning Grafana e politiche di notifica allerte.
\end{itemize}

\subsection{Database e servizi}
\subsubsection{Apache Kafka}
Apache Kafka è una piattaforma open-source di streaming distribuito sviluppata dall'Apache Software Foundation. Progettata per gestire flussi di dati in tempo reale in modo scalabile e affidabile, è ampiamente utilizzata nel data streaming e nell'integrazione dei dati nelle moderne applicazioni.

\paragraph{Versione}
La versione utilizzata è: 3.7.0
\paragraph{Documentazione}
\href{https://kafka.apache.org/20/documentation.html}{https://kafka.apache.org/20/documentation.html}

\paragraph{Funzionalità e vantaggi di Apache Kafka}
Le principali funzionalità e vantaggi di Apache Kafka includono:

\begin{itemize}
  \item \textbf{Pub-Sub Messaging:} Kafka utilizza un modello di messaggistica publish-subscribe, dove i produttori di dati inviano messaggi ad uno o più topic e i consumatori possono sottoscriversi a tali topic per ricevere i messaggi;
  
  \item \textbf{Disaccoppiamento Produttore - Consumatore:} questo principio si realizza grazie al fatto che i Produttori e i Consumatori non necessitano di essere consapevoli l'uno dell'altro o di interagire direttamente. Invece, essi comunicano attraverso il broker Kafka, che svolge il ruolo di intermediario per la trasmissione dei messaggi. Ciò consente una maggiore scalabilità e flessibilità nell'architettura del sistema, facilitando la gestione e il mantenimento delle applicazioni;
  
  \item \textbf{Architettura Distribuita:} Kafka è progettato per essere distribuito su un cluster di nodi, consentendo una scalabilità orizzontale per gestire grandi volumi di dati e carichi di lavoro. Questo approccio distribuito offre resilienza e alta disponibilità, garantendo che il sistema possa crescere in modo flessibile con l'aumentare delle richieste;
  
  \item \textbf{Persistenza e Affidabilità:} Kafka offre la possibilità di definire politiche specifiche per la conservazione dei dati, garantendo la durabilità dei messaggi. Questo non solo assicura la disponibilità dei dati anche in caso di eventuali interruzioni del servizio, ma consente anche ai consumatori di recuperare i messaggi dopo tali anomalie, garantendo un alto livello di affidabilità nel sistema.
  
  \item \textbf{Alta Disponibilità:} Kafka assicura un'elevata disponibilità e tolleranza ai guasti grazie alla sua architettura distribuita e al meccanismo di replica dei dati. Anche in caso di malfunzionamenti dei nodi o delle componenti, i cluster di Kafka mantengono la loro operatività, garantendo la continuità del servizio.
  
  \item \textbf{Elaborazione degli Stream:} Kafka supporta anche l'elaborazione degli stream di dati in tempo reale tramite API come Kafka Streams e Kafka Connect, consentendo agli sviluppatori di scrivere applicazioni per l'analisi e l'elaborazione dei dati in tempo reale.
\end{itemize}

\paragraph{Casi d'uso di Apache Kafka}

Apache Kafka è utilizzato in una vasta gamma di casi d'uso, tra cui:

\begin{itemize}
  \item \textbf{Data Integration:} Kafka viene utilizzato per integrare dati provenienti da diverse fonti e sistemi, consentendo lo scambio di dati in tempo reale tra applicazioni e sistemi eterogenei.
  
  \item \textbf{Streaming di Eventi:} Molte applicazioni moderne, come le applicazioni IoT (Internet of Things) e le applicazioni di monitoraggio in tempo reale, utilizzano Kafka per lo streaming di eventi in tempo reale e l'analisi dei dati.
  
  \item \textbf{Analisi dei Log:} Kafka è spesso utilizzato per l'analisi dei log di sistema e applicativi in tempo reale, consentendo il monitoraggio delle prestazioni, la rilevazione degli errori e l'analisi dei pattern di utilizzo.
  
  \item \textbf{Elaborazione di Big Data:} Kafka è integrato con tecnologie di big data come Apache Hadoop e Apache Spark, consentendo l'elaborazione di grandi volumi di dati in tempo reale.
  
  \item \textbf{Messaggistica Real-time:} Kafka è ampiamente utilizzato per la messaggistica real-time in applicazioni di social media, e-commerce e finanziarie, dove la velocità e l'affidabilità della messaggistica sono cruciali.
\end{itemize}

\paragraph{Utilizzo nel progetto}
\textit{Kafka} funge da intermediario dei messaggi, ricevendo i dati dai produttori e rendendoli disponibili ai consumatori. Nel contesto del progetto, i dati provenienti dalle simulazioni di sensori vengono inviati a \textit{Kafka} come messaggi in formato \textit{JSON}.

\paragraph*{Consumatori di dati:}
\begin{itemize}
  \item \textbf{\textit{ClickHouse:}} \textit{Kafka} invia \todo{è Kafka che li invia o i consumatori che se li prendono da Kafka?} i dati ai consumatori, inclusi i database come \textit{ClickHouse}, dove i dati vengono salvati per l'analisi e l'archiviazione a lungo termine.
  \item \textbf{\textit{Faust:}} per soddisfare il requisito opzionale del calcolo del punteggio di salute, \textit{Kafka} rende disponibili i dati in tempo reale a un'applicazione di Faust\todo{è corretto applicazione di Faust?}. Quest'ultima elabora i dati utilizzando una funzione di aggregazione per calcolare il punteggio e quindi mette a disposizione il risultato in una coda dedicata di Kafka per i servizi interessati.
\end{itemize}

In breve, \textit{Kafka} funge da ponte tra i produttori di dati (simulazioni di sensori) e i consumatori di dati (\textit{ClickHouse} o altri servizi futuri). Gestisce il flusso dei dati in tempo reale e garantisce che i dati siano disponibili per l'elaborazione e la visualizzazione in modo efficiente e scalabile.
\subsubsection{Schema Registry}
Schema Registry è un componente importante nell'ecosistema di Apache Kafka, progettato per la gestione e la convalida degli schemi dei dati utilizzati all'interno di un sistema di messaggistica distribuita.

\paragraph{Funzionalità e Vantaggi di Schema Registry}
Le funzionalità principali di Schema Registry includono:
\begin{itemize}
    \item \textbf{Gestione centralizzata degli schemi}: Fornisce un repository centralizzato per la gestione degli schemi dei dati.
    \item \textbf{Convalida degli schemi}: Assicura la validità e la compatibilità degli schemi dei dati.
    \item \textbf{Serializzazione e deserializzazione}: Supporta la serializzazione e la deserializzazione dei dati basati sugli schemi su reti distribuite.
    \item \textbf{Governance dei dati}: Contribuisce alla governance dei dati garantendo la qualità, la conformità agli standard e la tracciabilità dei dati.
\end{itemize}

\paragraph{Casi d'uso di Schema Registry}

Schema Registry è utilizzato in una vasta gamma di casi d'uso, tra cui:

\begin{itemize}
\item \textbf{Garanzia della compatibilità dei dati:} Schema Registry garantisce la compatibilità dei dati tra produttori e consumatori, consentendo l'evoluzione degli schemi dei dati senza interruzioni nei flussi di lavoro.

\item \textbf{Gestione della versione degli schemi:} Fornisce un sistema per gestire diverse versioni degli schemi dei dati, permettendo agli sviluppatori di aggiornare gli schemi in modo controllato e gestire la migrazione dei dati tra le versioni.

\item \textbf{Conformità agli standard e governance dei dati:} Aiuta a garantire la conformità agli standard aziendali e normativi, fornendo strumenti per la convalida degli schemi e la tracciabilità delle modifiche nel tempo.

\item \textbf{Collaborazione tra team e integrazione dei sistemi:} Funge da punto centrale per la collaborazione tra team di sviluppo, consentendo loro di condividere, discutere e approvare gli schemi dei dati per un'integrazione più efficace dei sistemi.

\item \textbf{Controllo della qualità dei dati:} Schema Registry contribuisce a garantire la qualità dei dati, riducendo il rischio di errori dovuti a incompatibilità o a dati non validi all'interno del sistema.
\end{itemize}


\paragraph{Utilizzo nel progetto}
Nell'ambito del progetto didattico Schema registry permette di validare i messaggi nell'ambito del topic kakfa di appartenenza definendo un contratto che i produttori, ovvero i sensori, dovranno rispettare nell'invio delle misurazioni.
\subsubsection{Zookeper}
Apache ZooKeeper è un servizio di coordinamento open-source sviluppato dalla Apache Software Foundation. È progettato per fornire funzionalità di coordinazione affidabili e scalabili per applicazioni distribuite.

\paragraph*{Versione:}


\paragraph*{Documentazione:}
\href{https://zookeeper.apache.org/documentation.html}{https://zookeeper.apache.org/documentation.html}
\paragraph*{Funzionalità e vantaggi di Apache ZooKeeper:}
Le principali funzionalità e vantaggi di Apache ZooKeeper includono:
\begin{itemize}
    \item \textbf{Servizio di coordinazione centralizzato:}ZooKeeper fornisce un servizio centralizzato per la gestione delle configurazioni, l'elezione del leader, la sincronizzazione dei dati e la notifica di eventi.
    \item \textbf{Affidabilità e scalabilità:} ZooKeeper è progettato per essere affidabile e scalabile, in grado di gestire grandi cluster di applicazioni distribuite.
    \item \textbf{Integrazione con altri software:} ZooKeeper è integrato con molti altri software open-source, tra cui Apache Kafka, Apache Hadoop e Apache HBase.
\end{itemize}
 
\paragraph*{Casi d'uso di Apache ZooKeeper:}

Apache ZooKeeper è utilizzato in una vasta gamma di casi d'uso, tra cui:
\begin{itemize}
    \item Naming service;
    \item Configuration management;
    \item Data Synchronization;
    \item Leader election;
    \item Message queue;
    \item Notification system.
\end{itemize}

\paragraph*{Utilizzo nel progetto:}
ZooKeeper è utilizzato principalmente:
\begin{itemize}
    \item \textbf{Sincronizzazione dei nodi Kafka}: Memorizza la configurazione del cluster Kafka, inclusa la lista dei broker attivi.
    Quando un nuovo broker viene aggiunto, ZooKeeper aggiorna la configurazione e notifica gli altri broker.
    Questo garantisce che tutti i broker abbiano una visione coerente del cluster e possano comunicare correttamente.
    \item \textbf{Coordinamento dello Schema Registry}:Memorizza lo schema per tutti i topic Kafka utilizzati nel progetto.
    Quando un client tenta di produrre un messaggio su un topic, lo Schema Registry verifica lo schema con ZooKeeper.
    Se lo schema è compatibile, il messaggio viene accettato. In caso contrario, il messaggio viene rifiutato.
    Questo garantisce che solo messaggi con schemi validi vengano pubblicati sui topic.
    
\end{itemize}

\subsection{ClickHouse} \label{sec:clickHouse}
ClickHouse è un sistema di gestione di database (DBMS) di tipo column-oriented, progettato principalmente per l'analisi di grandi volumi di dati in tempo reale. È un progetto open-source sviluppato da Yandex, un motore di ricerca russo, ed è stato creato per rispondere alle esigenze di elaborazione analitica ad alte prestazioni.
\subsubsection{Versione}
La versione utilizzata è: 24.1.5.6
\subsubsection{Link download}
\href{https://clickhouse.com/}{https://clickhouse.com/}

\subsubsection*{Funzionalità e Vantaggi di ClickHouse}
\begin{itemize}
    \item \textbf{ Modello di dati column-oriented:} a differenza dei tradizionali DBMS che memorizzano i dati in modo row-oriented, dove le righe complete sono memorizzate in sequenza, ClickHouse memorizza i dati in modo column-oriented. Questo significa che i dati di ogni colonna sono memorizzati insieme, permettendo una maggiore compressione e velocità di query per le analisi che coinvolgono molte colonne;
    \item \textbf{Architettura Distribuita e scalabilità:} ClickHouse è progettato per funzionare in un ambiente distribuito, consentendo la scalabilità orizzontale per gestire grandi carichi di lavoro;
    \item \textbf{Compressione dei Dati:} utilizza algoritmi efficienti per ridurre lo spazio di archiviazione richiesto per i dati, riducendo i costi di archiviazione;
    \item \textbf{Alte Prestazioni:} ottimizzato per eseguire query analitiche su grandi volumi di dati in tempo reale, garantendo tempi di risposta bassi anche con carichi di lavoro elevati.
    \item \textbf{Supporto per SQL:} supporta un sottoinsieme del linguaggio SQL, consentendo agli sviluppatori di scrivere query complesse per l'analisi dei dati;
    \item \textbf{Integrazione con Strumenti di Business Intelligence (BI):} può essere integrato con strumenti di BI popolari come Tableau, Power BI, Qlik, Grafana per la visualizzazione e l'analisi dei dati.
\end{itemize}


\subsubsection*{Casi d'Uso di ClickHouse}
ClickHouse è adatto per una vasta gamma di casi d'uso, tra cui:
\begin{itemize}
    \item \textbf{Analisi dei Log:} clickHouse può essere utilizzato per analizzare i log di grandi dimensioni generati da server, applicazioni web e dispositivi IoT;
    \item \textbf{Analisi dei Dati in Tempo Reale:} ClickHouse è ideale per l'analisi dei dati in tempo reale, consentendo agli utenti di eseguire query complesse su flussi di dati in continua evoluzione;
    \item \textbf{Reporting e Dashboard:} ClickHouse può essere utilizzato per generare report e dashboard interattivi per monitorare le prestazioni del business e identificare tendenze.
\end{itemize}



\subsection{Grafana}
Grafana è una piattaforma open-source per la visualizzazione e l'analisi dei dati, utilizzata per creare dashboard interattive e grafici da fonti di dati eterogenee. 
\subsubsection{Versione}
La versione utilizzata è: x.x.x
\subsubsection{Link download}
\href{https://clickhouse.com/}{https://clickhouse.com/}

\subsubsection{Funzionalità e Vantaggi di Grafana}
\begin{itemize}
    \item \textbf{Dashboard interattive}: Creazione di dashboard personalizzate e interattive per visualizzare dati provenienti da diverse fonti in un'unica interfaccia.
    
    \item \textbf{Connessione a sorgenti di dati eterogenee}: Supporto per una vasta gamma di sorgenti di dati, inclusi database, servizi cloud, sistemi di monitoraggio, API e altro ancora.
    
    \item \textbf{Ampia varietà di visualizzazioni}: Selezione di pannelli e visualizzazioni, tra cui grafici a linea, a barre, a torta, termometri, mappe geografiche e altro ancora, per adattarsi alle esigenze specifiche di visualizzazione dei dati.
    
    \item \textbf{Query e aggregazioni flessibili}: Esecuzione di query flessibili e aggregazione dei dati in modi personalizzati per ottenere insight approfonditi dai dati.
    
    \item \textbf{Notifiche e allarmi}: Impostazione di avvisi in base a criteri predefiniti, come soglie di performance, e ricezione di notifiche tramite diversi canali, tra cui email, Slack e molti altri.
    
    \item \textbf{Gestione degli accessi e dei permessi}: Controllo degli accessi e dei permessi degli utenti in modo granulare, gestendo chi può visualizzare, modificare o creare dashboard e pannelli.
    
    \item \textbf{Integrazione con altre applicazioni e strumenti}: Integrazione con una vasta gamma di applicazioni e strumenti, tra cui sistemi di log management, strumenti di monitoraggio delle prestazioni, sistemi di allerta e altro ancora.
    
   \end{itemize}
\subsubsection{Casi d'Uso di Grafana}
\begin{itemize}
    \item \textbf{Monitoraggio delle prestazioni}: Monitoraggio in tempo reale delle metriche di sistema come CPU, memoria e rete per identificare e risolvere rapidamente problemi di prestazioni.
    
    \item \textbf{Analisi dei log}: Analisi e visualizzazione dei log delle applicazioni e dell'infrastruttura per individuare pattern e risolvere problemi operativi.
    
    \item \textbf{Monitoraggio dell'infrastruttura}: Monitoraggio dello stato e delle prestazioni di server, servizi cloud, database e altri componenti IT per garantire un funzionamento ottimale dell'infrastruttura.
    
    \item \textbf{DevOps e CI/CD}: Monitoraggio dei processi di sviluppo, test e distribuzione del software per migliorare la collaborazione e l'efficienza del team.
    
    \item \textbf{Monitoraggio di dispositivi IoT}: Monitoraggio dei dispositivi IoT per raccogliere e visualizzare dati di sensori e dispositivi connessi, consentendo una gestione efficiente degli ambienti IoT.
\end{itemize}
\subsubsection{Utilizzo nel progetto}
Nel contesto di un progetto che coinvolge la visualizzazione e l'analisi di miliardi di misurazioni di sensori IoT, Grafana viene utilizzato principalmente per:

\begin{itemize}
  \item \textbf{Visualizzazione dei dati}: Grafana consente agli utenti di creare dashboard personalizzate e grafici interattivi che mostrano i dati provenienti dai sensori IoT in modo chiaro e comprensibile. Questi grafici possono essere configurati per visualizzare metriche specifiche nel formato desiderato, consentendo agli utenti di monitorare facilmente le prestazioni dei sensori e rilevare eventuali pattern o anomalie nei dati.
  
  \item \textbf{Analisi dei dati}: Grafana offre una vasta gamma di opzioni per analizzare i dati, inclusi filtri, aggregazioni, calcoli e altro ancora. Gli utenti possono eseguire query sui dati direttamente da Grafana e visualizzare i risultati in grafici, permettendo loro di ottenere una comprensione più approfondita delle tendenze e dei modelli presenti nei dati dei sensori IoT.
  
  \item \textbf{Monitoraggio in tempo reale}: Grafana supporta il monitoraggio in tempo reale dei dati, consentendo agli utenti di visualizzare aggiornamenti istantanei sui valori dei sensori e le metriche correlate. Ciò è particolarmente utile per l'analisi delle prestazioni in tempo reale e per la rilevazione immediata di problemi o anomalie nei dati dei sensori.
  
  \item \textbf{Allerta e notifica}: Grafana offre funzionalità avanzate di allerta e notifica che consentono agli utenti di impostare avvisi basati su condizioni specifiche dei dati. Ad esempio, è possibile configurare Grafana per inviare notifiche via email o tramite servizi di messaggistica istantanea quando un determinato sensore supera una soglia prestabilita o quando si verifica un'anomalia nei dati.
\end{itemize} 


