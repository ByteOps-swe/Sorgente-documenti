\section{Architettura di sistema}
\subsection{Architettura di implementazione}
Il sistema richiede la capacità di elaborare dati provenienti da diverse fonti in tempo reale e di fornire una visualizzazione immediata e continua di tali dati, permettendo di monitorarne gli andamenti e di rilevare eventuali anomalie. 
Per tale scopo, l'architettura di sistema adottata è la \textit{$\kappa$-architecture}.

\subsubsection{$\kappa$-architecture}
L'architettura Kappa è un modello di elaborazione dati in streaming che offre un'alternativa all'architettura Lambda. Il suo obiettivo principale è unificare l'elaborazione in tempo reale e batch (per i dati storici) all'interno di un unico stack tecnologico.
\paragraph{Vantaggi}
\begin{itemize}
    \item Semplice da implementare e gestire, costi di manutenzione ridotti;
    \item Assicura coerenza tra l'analisi in tempo reale e batch.
\end{itemize}
\paragraph*{Svantaggi}
\begin{itemize}
    \item Potenziale rallentamento dell'analisi in tempo reale, meno flessibile rispetto a Lambda.
\end{itemize}

\subsubsection{Componenti di sistema}
\begin{figure}[H]
    \centering
    \includegraphics[width=1\textwidth]{../Images/SpecificaTecnica/Architettura_PB_microservices2.png}
    \caption{Componenti dell'architettura - Innovacity}
    \label{fig: fdf}
\end{figure}

\begin{itemize}
    \item \textbf{Data feed}:Le sorgenti dati sono costituite da sensori IoT dislocati sul territorio cittadino. Questi sensori sono in grado di inviare, ad intervalli regolari, messaggi contenenti misurazioni allo streaming layer;
    \item \textbf{Streaming layer}:Lo streaming layer gestisce i dati in arrivo in tempo reale, per poi archiviarli sistematicamente nello storage layer. Lo streaming Layer è composto da:
    \begin{itemize}
        \item \textbf{Apache Kafka}: Kafka è un sistema di messaggistica distribuito che consente di pubblicare, sottoscrivere e archiviare messaggi in tempo reale. Kafka è utilizzato per ricevere i dati dai sensori IoT e renderli disponibili per l'elaborazione in tempo reale e batch.
        \item \textbf{Clickhouse Kafka table engine}:consumatore che legge i
        dati dal server Kafka per persisterli nello storage layer.
    \end{itemize}
    \item \textbf{Processing Layer:} Il processing Layer è costituito da Faust che consuma i dati dallo streaming layer e li processa in tempo reale. Faust è un framework Python che consente di scrivere applicazioni di streaming in tempo reale. Faust è utilizzato per elaborare i dati in arrivo tramite un modello per il calcolo del punteggio di salute che poi viene reso nuovamente disponibili allo streaming layer.
    \item \textbf{Storage layer}:Lo storage layer è costituito da un database column-oriented, ClickHouse, che archivia i dati in arrivo dallo streaming layer. Questi dati sono disponibili per l'analisi e la visualizzazione in tempo reale e batch.
    \item \textbf{Data Visualization Layer}: composto da Grafana, si occupa della visualizzazione dei dati elaborati ottenuti dallo storage layer e della gestione delle notifiche in caso di anomalie rilevate.
\end{itemize}
\subsection{Architettura dei simulatori} \label{sec:architettura_simulatori}
Nonostante i simulatori non siano ufficialmente considerati come parte fondamentale del prodotto dalla proponente, ma necessari solamente per dimostrare il corretto funzionamento del sistema, il nostro team ha comunque scelto di dedicare alcune risorse alla progettazione di questa componente nell'ambito del progetto didattico.

Inoltre, abbiamo deciso di implementare e tenere conto delle possibili logiche dei microcontrollori associati ai sensori IoT, che possono effettuare operazioni per rendere più efficiente l'intero sistema.

Nei paragrafi successivi, verrà presentata l'architettura individuata mediante l'utilizzo di diagrammi delle classi e relative descrizioni. Inoltre, saranno motivate le scelte dei design pattern individuati e le decisioni progettuali rilevanti. Successivamente, per ogni classe, saranno illustrati metodi e attributi.

\subsubsection{Modulo simulatori sensori}
\begin{figure}[H]
    \centering
    \includegraphics[width=1.1\textwidth]{../Images/SpecificaTecnica/simulatoriSensori.PNG}
    \caption{Modulo simulatori sensori - InnovaCity}
    \label{fig: Modulo_simulatori_sensori}
\end{figure}
Questo modulo si occupa della generazione di dati per diverse tipologie di sensori.
In particolare sono stati implementati simulatori per i seguenti tipi di sensori:
\begin{itemize}
    \item Sensori di temperatura;
    \item Sensori di umidità;
    \item Sensori di polveri sottili PM10;
    \item Sensori di stato occupazione colonnine di ricarica;
    \item Sensori di stato riempimento isole ecologiche;
    \item Sensori di presenza d'acqua;
    \item Sensori di guasto elettrico.
\end{itemize}
\paragraph{Design pattern Template Method} \label{sec:templateSIM}

La classe astratta \textit{simulator} implementa il design pattern \textit{Template Method}. Il metodo \textit{simulate()} fornisce lo scheletro dell'algoritmo per la generazione delle misurazioni e non può essere ridefinito nelle implementatazioni. Le classi che estendono \textit{simulator} implementano i metodi:
\begin{itemize}
    \item \textbf{\textit{generate\_measure()}:} Per la generazione semi-randomica della misurazione associata al tipo di sensore, di fatto non è richiesta una generazione di misurazioni realistiche dal prodotto, ma per poter avere una visualizzazione finale delle misurazioni non altalenanti ogni misurazione viene generata sulla base di quella precedente con una variazione limitata;
    
    \item \textbf{\textit{adapt()} - "hook method":} L'implementazione di default non modifica in alcun modo la misurazione generata. Può essere ridefinito per implementare la logica di adattamento della misurazione.
    
    Ad esempio:
    \begin{itemize}
        \item Per i sensori di polveri sottili PM10, la misurazione può essere adattata cambiando il valore dell'unità di misura µg/m³ in mg/m³ senza modifcare la logica di generazione in \textit{generate\_measure()};
        
        \item Per i sensori di temperatura, la misurazione può essere adattata cambiando il valore dell'unità di misura da °C a Fahrenheit  senza modifcare la logica di generazione in \textit{generate\_measure()};
        
        ~•~ Fahrenheit = (Celsius × 9/5) + 32
 
        \item Questo apre le porte alla possibilità di creare per una stessa tipoligia di sensore diverse implementazioni, che generano misurazioni con unità di misura diverse, senza dover modificare la logica di generazione che dichiara l'unità di misura generata di default, ma semplicemente estendendo la classe e ridefinendo il metodo \textit{adapt()} inserendo la formula di trasformazione.
        
    \end{itemize}
    \item Per garantire il rispetto del \textbf{Liskov Substitution Principle (LSP)} \textit{"Objects of a superclass should be replaceable with objects of its subtypes without altering the program's correctness."}:
    \begin{itemize}
        \item Il metodo \textit{adapt()} si limita a convertire le misurazioni senza alterare il comportamento generale del metodo \textit{simulate()} e senza modificare il tipo della misurazione generata da \textit{generate\_measure()};
        
        \item Il metodo \textit{generate\_measure()} viene reso \textbf{final} nelle concretizzazioni di \textit{simulator} impedendo la ridefinizione del comportamento.
    \end{itemize}

    In generale le postcondizioni sono piu forti nelle classi derivate, mentre non variano le precondizioni garantendo il rispetto di \textit{LSP}.
    
    Ad esempio:

    \begin{itemize}
        \item In \textit{simulator}:

        \begin{itemize}
            \item La postcondizione del metodo \textit{simulate()} in simulator è la generazione di un oggetto di tipo \textit{misurazione};
            
            \item La postcondizione del metodo \textit{generate\_measure} è la generazione del valore della misurazione.
            
            \item La postcondizione del metodo \textit{adapt()} è la possibilità di convertire il valore della misurazione.
        \end{itemize}

        \item In \textit{temperature\_simulator}, che eredita da \textit{simulator}:
        \begin{itemize}
            \item La postcondizione del metodo \textit{simulate()} è la generazione di un oggetto di tipo \textit{misurazione} di temperatura;
            
            \item La postcondizione del metodo \textit{generate\_measure} è la generazione del valore di una misurazione di temperatura;
            
            \item La postcondizione del metodo \textit{adapt()} è la possibilità di convertire il valore della misurazione ad un altra unità di misura (Kelvin, Fahrenheit).
        \end{itemize}

        \item In una possibile estensione futura \textit{temperature\_simulator\_fahrenheit}, che eredita da \textit{temperature\_simulator}:
        \begin{itemize}
            \item La postcondizione del metodo \textit{simulate()} è la generazione di un oggetto di tipo \textit{misurazione} di temperatura espresso in gradi Fahrenheit;
            
            \item La postcondizione del metodo \textit{generate\_measure} rimane invariata;
            
            \item La postcondizione del metodo \textit{adapt()} è di convertire il valore della misurazione dall'unità di misura di default a Fahrenheit.
        \end{itemize}
    \end{itemize}
\end{itemize}

Al termine delle operazioni di generazione e adattamento, il metodo \textit{simulate()} crea e restituisce un oggetto di tipo \textit{misurazione}.

Il design pattern \textit{Template Method} è stato scelto per:
\begin{itemize}
    \item Permettere una facile estensione del sistema con nuovi tipi di sensori che dovranno unicamente implementare la loro logica di generazione delle misurazioni e di adapting se necessario;
    
    \item Standardizzare i passi per la generazione delle misurazioni, garantendo coerenza e manutenibilità del codice;
    
    \item Ridurre la duplicazione del codice.
\end{itemize}

Come già esposto, una volta ottenuto lo stato del sensore, esso viene inserito in un oggetto di tipo \textit{misurazione}. Questo oggetto contiene informazioni di contesto come:
\begin{itemize}
    \item Identificativo del sensore;
    \item Cella della città in cui è presente il sensore;
    \item Timestamp della misurazione;
    \item Valore della misurazione;
    \item Coordinate;
    \item Tipologia di misurazione.
\end{itemize}

L'oggetto \textit{misurazione} viene poi ritornato al chiamante che si occuperà di inviarlo al server Kafka.
Un oggetto di tipo \textit{simulator} infatti, verrà assegnato ad ogni \textit{simulator\_thread}. Esso chiamerà ad intervalli regolari il metodo \textit{simulate()} ottenendo appunto la misurazione che invierà poi al server Kafka tramite un modulo apposito e indipendendente.

\paragraph{Design pattern Factory}
sensor\_factory implementa il design pattern \textit{Factory} per la creazione di simulatori dei sensori.
Il pattern \textit{Factory} è un pattern di tipo “Creazionale” secondo la classificazione della GoF.

I pattern di tipo creazionali si occupano della costruzione delle simulazioni dei sensori e delle problematiche che si possono originare, astraggono il processo di creazione degli oggetti, nascondono i dettagli della creazione e rendono i sistemi indipendenti da come gli oggetti sono creati e composti.

Il pattern \textit{Factory} incapsula la creazione concreta dei sensori, consentendo al client
(l’utilizzatore) di non conoscere i dettagli.

\paragraph{Classi, interfacce metodi e attributi:}
\begin{itemize}
    \item {\textbf{Classe astratta: \textit{simulator}}}
    \begin{itemize}
        \item \textbf{Attributi}: 
        \begin{itemize}
            \item \textbf{ID\_sensor: string [private]} - Identificatore univoco del sensore;
            \item \textbf{cella\_sensore: string [private]} - Identificatore della cella del sensore;
            \item \textbf{coordinate: coordinate [private]} - Coordinate geografiche del sensore;
            \item \textbf{misurazione: T [protected]} - Misurazione corrente del sensore;
            \item \textbf{type: string [private]} - Tipo di sensore.
        \end{itemize}
        \item \textbf{Metodi}:
        \begin{itemize}
            \item \textbf{simulate(): misurazione [public]} - Metodo principale per simulare la generazione di una misurazione.
            \item 
            Si basa sul design pattern Template Method:
            \begin{enumerate}
                \item Chiama generate\_measure() per generare un valore di misurazione;
                \item Chiama adapt() per effettuare adattamenti se necessari;
                \item Restituisce un oggetto \textit{misurazione} con data e ora corrente, valore misurato, tipo di sensore, coordinate e identificativo del sensore.
            \end{enumerate}
            \item \textbf{generate\_measure(): None [protected]} - Metodo astratto da implementare nelle classi concrete per generare un valore di misurazione semi-casuale coerente con la tipolgia di sensore da salvare nell'attributo \textit{misurazione};
            \item \textbf{adapt(): void [protected]} - Fornisce un'implementazione di default che non modifica in alcun modo la misurazione generata. Può essere ridefinito per implementare la logica di adattamento della misurazione come conversioni di unità di misura.
        \end{itemize}
        \item \textbf{Note}:
        \begin{itemize}
            \item La classe \textit{simulator} è astratta e definisce il comportamento generale della simulazione della misurazione, pattern \textit{Template method};
            \item Le classi concrete che ereditano da \textit{simulator} devono implementare il metodo astratto generate\_measure();
            \item Il metodo adapt() può essere ridefinito nelle classi concrete per implementare conversioni o adattamenti necessari;
            \item Il metodo \textit{simulate()} è final e non può essere ridefinito;
            \item Spiegazioni esaustive sono state presentate in: \ref{sec:templateSIM}.
        \end{itemize}
    \end{itemize}
        
    \item{\textbf{Enumerazione: \textit{sensor\_types}}}
    \begin{itemize}
        \item \textbf{Costanti}: 
        \begin{itemize}
            \item \textbf{TEMPERATURE: string [public]} - Rappresenta la nomenclatura dei sensore di temperatura;
            \item \textbf{HUMIDITY: string [public]} - Rappresenta la nomenclatura dei sensore di umidità;
            \item \textbf{DUST\_PM10: string [public]} - Rappresenta la nomenclatura dei sensore di "polvere PM10";
            \item \textbf{CHARGING\_STATION: string [public]} - Rappresenta la nomenclatura dei sensore di stato delle colonnine di ricarica;
            \item \textbf{ECOLOGICAL\_ISLAND: string [public]} - Rappresenta la nomenclatura dei sensore di stato riempimento isole ecologica;
            \item \textbf{WATER\_PRESENCE: string [public]} - Rappresenta la nomenclatura dei sensore di presenza d'acqua;
            \item \textbf{ELECTRICAL\_FAULT: string [public]} - Rappresenta la nomenclatura dei sensore di guasti elettrici.
        \end{itemize}

        \item \textbf{Note}:
        \begin{itemize}
            \item L'enumerazione viene utilizzata per centralizzare la gestione della nomenclatura dei tipi di sensori che verrà salvata nelle misurazioni.
        \end{itemize}
    \end{itemize}
        
    \item{\textbf{Classe: \textit{temperature\_simulator}}}
    \begin{itemize}
        \item \textbf{Attributi:}
        \begin{itemize}
            \item \textbf{count: int [private, static]} - Contatore statico per generare un ID univoco per ogni istanza.
        \end{itemize}
        \item\textbf{Metodi}: 
        \begin{itemize}
            \item \textbf{generate\_measure(): None [protected,final]} - Genera una misurazione di temperatura in gradi Celsius semi-casuale e aggiorna lo stato interno con il valore della misurazione corrente.
        \end{itemize}
        \item\textbf{Note}:
        \begin{itemize}
            \item La classe \textit{temperature\_simulator} è una classe concreta che eredita dalla classe astratta \textit{simulator};
            \item Il costruttore genera automaticamente un ID sensore univoco per ogni istanza;
            \item Dichiara di generare misurazioni di temperatura con unità di default (Gradi Celsius), possibili classi derivate possono effettuare conversioni ad altre unità di misura (Kelvin,Fahrenheit) tramite il metodo \textit{adapt()}, senza dover modificare la logica di generazione.
        \end{itemize}
    \end{itemize}
    
    \item{\textbf{Classe: \textit{humidity\_simulator}}}
    \begin{itemize}
        \item\textbf{Attributi:}
        \begin{itemize}
            \item \textbf{count: int [private, static]} - Contatore statico per generare un ID univoco per ogni istanza.
        \end{itemize}
        \item \textbf{Metodi}: 
        \begin{itemize}
            \item \textbf{generate\_measure(): None [protected,final]} - Genera una misurazione di umidità in percentuale semi-casuale e aggiorna lo stato interno con il valore della misurazione corrente.
        \end{itemize}
        \item \textbf{Note}:
        \begin{itemize}
        \item La classe \textit{humidity\_simulator} è una classe concreta che eredita dalla classe astratta \textit{simulator};
        \item Il costruttore genera automaticamente un ID sensore univoco per ogni istanza;
        \item Dichiara di generare misurazioni di umidità con unità di default (Percentuale), possibili classi derivate possono effettuare conversioni ad altre unità di misura (g/m³) tramite il metodo \textit{adapt()}, senza dover modificare la logica di generazione.
        \end{itemize}
    \end{itemize}

    \item{\textbf{Classe: \textit{charging\_station\_simulator}}}
    \begin{itemize}
        \item \textbf{Attributi}: 
        \begin{itemize}
            \item \textbf{count: int [private, static]} - Contatore statico per generare un ID univoco per ogni istanza.
        \end{itemize}
        \item \textbf{Metodi}:
        \begin{itemize}
            \item \textbf{generate\_measure(): None [protected,final]} - Genera lo stato della colonnina di ricarica (Occupato: \textsc{True}, Libero: \textsc{False}) basata su una probabilità di transizione e aggiorna lo stato interno con il valore della misurazione corrente.
        \end{itemize}
        \item \textbf{Note}:
        \begin{itemize}
            \item La classe \textit{charging\_station\_simulator} è una classe concreta che eredita dalla classe astratta \textit{simulator};
            \item Il costruttore genera automaticamente un ID sensore univoco per ogni istanza;
            \item Implementa il metodo astratto \textit{generate\_measure()} per generare una misurazione basata sulla probabilità di transizione.
        \end{itemize}
    \end{itemize}

    \item{\textbf{Classe: \textit{dust\_PM10\_simulator}}}
    \begin{itemize}
        \item \textbf{Attributi}: 
        \begin{itemize}
            \item \textbf{count: int [private, static]} - Contatore statico per generare un ID univoco per ogni istanza.
        \end{itemize}
        \item \textbf{Metodi}: 
        \begin{itemize}
            \item \textbf{generate\_measure(): None [protected,final]} - Genera una variazione di quantità di polvere PM10 semi-casuale e aggiorna lo stato interno con il valore della misurazione corrente.
        \end{itemize}
        \item \textbf{Note}:
        \begin{itemize}
            \item La classe \textit{dust\_PM10\_simulator} è una classe concreta che eredita dalla classe astratta \textit{simulator};
            \item Il costruttore genera automaticamente un ID sensore univoco per ogni istanza;
            \item Dichiara di generare misurazioni di polvere PM10 in con unità di default (µg/m³), possibili classi derivate possono effettuare conversioni ad altre unità di misura (mg/m³) tramite il metodo \textit{adapt()}, senza dover modificare la logica di generazione.
        \end{itemize}
    \end{itemize}

    \item{\textbf{Classe: \textit{electrical\_fault\_simulator}}}
    \begin{itemize}
        \item \textbf{Attributi}: 
        \begin{itemize}
            \item \textbf{count: int [private, static]} - Contatore statico per generare un ID univoco per ogni istanza.
        \end{itemize}
        \item \textbf{Metodi}: 
        \begin{itemize}
            \item \textbf{generate\_measure(): None [protected,final]} - Genera lo stato di una centralina elettrica (Guasto verificato: \textsc{True}, Operativa: \textsc{False}) basandosi su una probabilità di guasto e aggiorna lo stato interno con il valore della misurazione corrente.
        \end{itemize}
        \item \textbf{Note}:
        \begin{itemize}
            \item La classe \textit{electrical\_fault\_simulator} è una classe concreta che eredita dalla classe astratta \textit{simulator};
            \item Il costruttore genera automaticamente un ID sensore univoco per ogni istanza.
        \end{itemize}
    \end{itemize}

    \item{\textbf{Classe: \textit{ecological\_island\_simulator}}}
    \begin{itemize}
        \item \textbf{Attributi}: 
        \begin{itemize}
            \item \textbf{count: int [private, static]} - Contatore statico per generare un ID univoco per ogni istanza.
        \end{itemize}
        \item \textbf{Metodi}: 
        \begin{itemize}
            \item \textbf{generate\_measure(): None [protected,final]} - Genera una misurazione della percentuale di riempimento di un isola ecologica e aggiorna lo stato interno con il valore della misurazione corrente.
        \end{itemize}
        \item \textbf{Note}:
        \begin{itemize}
            \item La classe \textit{ecological\_island\_simulator} è una classe concreta che eredita dalla classe astratta \textit{simulator};
            \item Il costruttore genera automaticamente un ID sensore univoco per ogni istanza.
        \end{itemize}
    \end{itemize}

    \item{\textbf{Classe: \textit{water\_presence\_sensor}}}
    \begin{itemize}
        \item \textbf{Attributi}: 
        \begin{itemize}
            \item \textbf{count: int [private, static]} - Contatore statico per generare un ID univoco per ogni istanza.
        \end{itemize}
        \item \textbf{Metodi}: 
        \begin{itemize}
            \item \textbf{generate\_measure(): None [protected,final]} - Genera una misurazione basata sulla soglia di presenza dell'acqua (Acqua rilevata: \textsc{True}, Acqua non rilevata: \textsc{False}) e aggiorna lo stato interno con il valore della misurazione corrente.
        \end{itemize}
        \item \textbf{Note}:
        \begin{itemize}
            \item La classe \textit{ecological\_island\_simulator} è una classe concreta che eredita dalla classe astratta \textit{simulator};
            \item Il costruttore genera automaticamente un ID sensore univoco per ogni istanza.
        \end{itemize}
    \end{itemize}

    \item{\textbf{Classe: \textit{misurazione}}}
    \begin{itemize}
        \item \textbf{Attributi}: 
        \begin{itemize}
            \item \textbf{timestamp: datetime [private]} - Timestamp della misurazione;
            \item \textbf{value: T [private]} - Valore della misurazione;
            \item \textbf{type: string [private]} - Tipo della misurazione;
            \item \textbf{coord: coordinate [private]} - Coordinate della misurazione;
            \item \textbf{ID\_sensore: string [private]} - ID del sensore che ha effettuato la misurazione;
            \item \textbf{cella: string [private]} - Cella in cui è stata effettuata la misurazione.
        \end{itemize}
        \item \textbf{Metodi}: 
        \begin{itemize}
            \item \textbf{\_\_eq\_\_(other: misurazione): bool [public]} - Ridefinizione dell'operatore di uguaglianza per confrontare due oggetti \textit{misurazione}.
        \end{itemize}
    \end{itemize}

    \item{\textbf{Classe: \textit{coordinate}}}
    \begin{itemize}
        \item \textbf{Attributi}: 
        \begin{itemize}
            \item \textbf{latitude: float [private]} - Latitudine della coordinata;
            \item \textbf{longitude: float [private]} - Longitudine della coordinata.
        \end{itemize}
        \item \textbf{Metodi}: 
        \begin{itemize}
            \item \textbf{\_\_eq\_\_(other: coordinate):bool [public]} - Ridefinizione dell'operatore di uguaglianza per confrontare due oggetti \textit{coordinate}.
        \end{itemize}
    \end{itemize}

    \item{\textbf{Classe: \textit{sensor\_factory}}}
    \begin{itemize}
        \item \textbf{Metodi}: 
        \begin{itemize}
            \item \textbf{create\_temperature\_sensor(latitude: float, longitude: float, cella: string, initial\_value: float): temperature\_simulator [public, static]} - Crea un simulatore di temperatura;

            \item \textbf{create\_humidity\_sensor(latitude: float, longitude: float, cella: string, initial\_value: float): humidity\_simulator [public, static]} - Crea un simulatore di umidità;
            
            \item \textbf{create\_charging\_station\_sensor(latitude: float, longitude: float, cella: string, probabilita\_occupazione: float): charging\_station\_simulator [public, static]} - Crea un simulatore di stazione di ricarica;
            
            \item \textbf{create\_ecological\_island\_sensor(latitude: float, longitude: float, cella: string, initial\_value: float): ecological\_island\_simulator [public, static]} - Crea un simulatore di isola ecologica;
            
            \item \textbf{create\_water\_presence\_sensor(latitude: float, longitude: float, cella: string, soglia\_rilevamento: float): water\_presence\_sensor [public, static]} - Crea un sensore di presenza d'acqua;
            
            \item \textbf{create\_dust\_PM10\_sensor(latitude: float, longitude: float, cella: string, initial\_value: float): dust\_PM10\_simulator [public, static]} - Crea un simulatore di polvere PM10;
            
            \item \textbf{create\_eletrical\_fault\_sensor(latitude: float, longitude: float, cella: string, fault\_probability: float): electrical\_fault\_simulator [public, static]} - Crea un simulatore di guasto elettrico.
        \end{itemize}
    \textbf{Note}:
        \begin{itemize}
            \item Implementazione del pattern Factory;
            \item Fornisce metodi per la creazione di simulatori di sensori;
            \item Astrae il processo di creazione dei sensori, nascondendo i dettagli della creazione.
        \end{itemize}
    \end{itemize}
\end{itemize}

\subsubsection{Modulo Writers} \label{sec:writersModule}

\begin{figure}[H]
    \centering
    \includegraphics[width=1.1\textwidth]{../Images/SpecificaTecnica/writerModule.PNG}
    \caption{Modulo writers - InnovaCity}
    \label{fig: writersModule}
\end{figure}

Questo modulo si occupa della scrittura e/o invio di informazioni a diverse tipologie di servizi e vuole essere completamentemente indipendendente e non influenzato dal modulo della simulazione dei sensori, così da poter consentire un suo riutilizzo.
Per quanto riguarda la scrittura in Kafka, l'impiego della connessione allo Schema Registry (vedi sezione \S\ref{sec:schema_registry}) consente la convalida del formato del messaggio prima della sua scrittura nei topic Kafka. Questa pratica permette di ridurre il carico di rete nel caso di messaggi malformati, consentendo un filtraggio "alla fonte".

Il modulo è stato progettato per rispettare il Dependecy Inversion Principle \textit{(DIP)}, di conseguenza sia i moduli di alto livello che quelli di basso livello dipendono da astrazioni (interfacce o classi astratte).

\paragraph{Design pattern Strategy + Composite:}
Il modulo presenta un'interfaccia \textit{writer} che offre il metodo di scrittura \textit{write()} di oggetti che implementano \textit{writable}.
Questo metodo è implementato da diverse classi concrete che rappresentano i vari servizi a cui è possibile inviare le informazioni.
L'approccio adottato implementa il design pattern \textit{Strategy} per la scrittura/invio dei dati su diverse piattaforme/servizi e il design pattern \textit{Composite} per la scrittura ad uno o più servizi in modo uniforme.
Nello specifico sono state implentate tre strategie di scrittura: la prima, (\textit{kafka\_writer}), atta a permettere al simulatore di inviare messaggi a topic Kafka,  la seconda (\textit{std\_out\_writer}) atta a permettere di stampare gli \textit{writable} su terminale e la terza (\textit{list\_writer}) per il salvataggio su una lista degli \textit{writable}.
L'utilizzo del design pattern Composite e Strategy in questo caso ha diverse motivazioni:

\begin{enumerate}
    \item \textit{kafka\_writer}, atta a permettere al simulatore di inviare messaggi a Kafka;
    \item \textit{std\_out\_writer}, atta a permettere di stampare i \textit{writable} su terminale;
    \item \textit{list\_writer}, per il salvataggio su una lista degli oggetti di tipo \textit{writable}.
\end{enumerate}
L'utilizzo dei design pattern Composite e Strategy in questo caso ha diverse motivazioni:
\begin{itemize}
    \item \textbf{Gestione uniforme dei servizi}: Il pattern Strategy consente di definire una famiglia di algoritmi, incapsularli e renderli intercambiabili. In questo caso, i servizi di scrittura sono trattati come algoritmi intercambiabili, consentendo di scrivere informazioni su diversi servizi senza dover conoscere i dettagli di implementazione di ciascuno;
    \item \textbf{Gestione gerarchica dei servizi}: Il pattern Composite consente di trattare gli oggetti singoli e le loro composizioni (gruppi di oggetti) allo stesso modo. \\
    Nel contesto del modulo, potrebbe esserci la necessità di gestire non solo singoli servizi, ma anche gruppi di servizi. Ad esempio, potrebbe essere utile inviare informazioni al servizio Kafka e contemporaneamente stamparle nel terminale e memorizzarle in un'apposita lista per il testing. Il Composite consente di comporre questi servizi in modo gerarchico e trattarli uniformemente.
\end{itemize}

\paragraph{Design pattern Object Adapter:}
Nello specifico, la classe \textit{kafka\_writer} realizza la sua funzionalità attraverso l'utilizzo del design pattern \textit{Adapter}, nella sua variante \textit{Object Adapter}. Tale scelta è stata motivata dall'impiego della classe \textit{Producer} della libreria \textit{confluent\_kafka}, la quale potrebbe subire variazioni non controllabili da noi. Per garantire la capacità di rispondere prontamente a tali cambiamenti senza dover modificare la classe \textit{kafka\_writer} o altri parti di sistema, si è optato per l'utilizzo di questo pattern, trasferendo così la complessità derivante da tali modifiche proprio nell'adapter.
Inoltre grazie all'interfaccia \textit{kafka\_target}, si è garantita la possibilità di estendere il sistema con nuovi metodi di scrittura su Kafka o l'utilizzo di nuove librerie senza dover modificare la classe \textit{kafka\_writer} ma solamente aggiungendo una nuova classe adapter che implementi \textit{kafka\_target}.

\paragraph{Classi: metodi e attributi}

\begin{itemize}
    \item{\textbf{Interfaccia: \textit{writable}}}
    \begin{itemize}
        \item\textbf{Metodi}: 
        \begin{itemize}
            \item \textbf{to\_json(): string [public, abstract]} - Metodo astratto che deve essere implementato nelle sottoclassi per convertire l'oggetto in una stringa JSON.
        \end{itemize}
        \item\textbf{Note}:
        \begin{itemize}
            \item L'interfaccia \textit{writable} definisce un insieme di metodi che una classe deve implementare perchè possa essere utilizzata dalle strategie di scrittura.
        \end{itemize}
    \end{itemize}
    \item{\textbf{Interfaccia: \textit{writer}}}
     \begin{itemize}
        \item \textbf{Metodi}:
         \begin{itemize}
            \item \textbf{write(to\_write: writable): None [public, abstract]} - Metodo astratto che deve essere implementato nelle sottoclassi per scrivere un oggetto writable.
        \end{itemize}
        \item\textbf{Note}:
        \begin{itemize}
            \item L'interfaccia \textit{writer} definisce il metodo che una classe deve implementare perchè possa essere utilizzata come strategia di scrittura. Rappresenta il componente "\textit{strategy}" del pattern "\textit{Strategy}";
            \item Rappresenta l'interfaccia "\textit{Component}" del pattern \textit{Composite} che descrive le operazioni comuni sia agli elementi semplici che a quelli complessi dell'albero.
        \end{itemize}
    \end{itemize}
    \item{\textbf{Classe: \textit{composite\_writer}}}
    \begin{itemize}
    \item\textbf{Attributi}:
        \begin{itemize}
        \item \textbf{writers: writer [private]} - Lista di oggetti \textit{writer}.
    \end{itemize}
    \item \textbf{Metodi}:
    \begin{itemize}
        \item \textbf{add\_writer(writer: writer ): composite\_writer [public]} - Aggiunge un oggetto che implementa \textit{writer} alla lista writers;
        \item \textbf{add\_kafka\_confluent\_writer(topic: string, host: string, port: int, schema\_registry\_url: string): composite\_writer [public]} - Crea un \textit{kafka\_writer} con un \textit{kafka\_confluent\_adapter} e lo aggiunge alla lista writers. Ritorna se stesso per permettere operazione concatenate;
        \item \textbf{add\_std\_out\_writer(): composite\_writer [public]} - Crea un \textit{std\_out\_writer} e lo aggiunge alla lista writers. Ritorna se stesso per permettere operazione concatenate;
        \item \textbf{add\_list\_writer(writer\_list: list\_writer): composite\_writer [public]} - Aggiunge un \textit{list\_writer} alla lista writers. Ritorna se stesso per permettere operazione concatenate;
        \item \textbf{remove\_writer(writer: writer): composite\_writer [public]} - Rimuove un \textit{writer} dalla lista writers.  Ritorna se stesso per permettere operazione concatenate;
        \item \textbf{write(to\_write: writable): \textit{composite\_writer} [public]} - Chiama il metodo write su ogni writer nella lista writers passando come attributo il \textit{writable} ricevuto.
    \end{itemize}
    \item\textbf{Note}:
        \begin{itemize}
            \item La classe è la componente "Composite" del pattern \textit{Composite}, ovvero l'elemento che può avere sottoelementi;
            \item Dopo aver ricevuto una richiesta, il contenitore (detto composite) delega il lavoro ai suoi sottoelementi: foglie o altri contenitori.
        \end{itemize}
    \end{itemize}
    \item{\textbf{Classe: \textit{std\_out\_writer}}}
    \begin{itemize}
    \item\textbf{Attributi}:
        \begin{itemize}
        \item \textbf{lock:threading.Lock [private]} - Lock per garantire l'accesso esclusivo alla stampa ed un esecuzione Thread safe.
    \end{itemize}
    \item \textbf{Metodi};
    \begin{itemize}
        \item \textbf{write(to\_write: writable): None [public]} - Stampa l'oggetto writable come stringa JSON nella console;
    \end{itemize}
    \item\textbf{Note}:
        \begin{itemize}
            \item La classe è una strategia di scrittura del pattern \textit{Strategy} ma anche la componente "Leaf" del pattern \textit{Composite}, ovvero l'elemento base che non ha sottoelementi.
        \end{itemize}
    \end{itemize}
    \item{\textbf{Classe: \textit{list\_writer}}}
    \begin{itemize}
    \item\textbf{Attributi}:
        \begin{itemize}
        \item \textbf{data\_list:list [private]} - Lista per memorizzare gli oggetti writable;
        \item \textbf{lock:threading.Lock [private]} - Lock per garantire l'accesso esclusivo alla lista ed un esecuzione Thread safe.
    \end{itemize}
    \item \textbf{Metodi}:
    \begin{itemize}
        \item \textbf{write(to\_write: writable): None [public]} - Aggiunge l'oggetto writable alla lista;
        \item \textbf{get\_data\_list(): list [public]} - Restituisce la lista di oggetti writable.
    \end{itemize}
    \item\textbf{Note}:
        \begin{itemize}
            \item La classe è una strategia di scrittura del pattern \textit{Strategy} ma anche la componente "Leaf" del pattern \textit{Composite}, ovvero l'elemento base che non ha sottoelementi.
        \end{itemize}
    \end{itemize}
    \item{\textbf{Classe: \textit{kafka\_writer}}}
    \begin{itemize}
    \item\textbf{Attributi}:
        \begin{itemize}
        \item \textbf{lock:threading.Lock [private]} - Lock per garantire l'accesso esclusivo alla scrittura su Kafka ed un esecuzione Thread safe;
        \item \textbf{kafka\_target:kafka\_target [private]} - Riferimento ad un implementazione di kafka\_target per effettuare l'effettiva scrittura in Kafka tramite librerie.
    \end{itemize}
    \item \textbf{Metodi}:
    \begin{itemize}
        \item \textbf{write(to\_write: writable): None [public]} - Scrive l'oggetto \textit{writable} come stringa JSON su Kafka.
    \end{itemize}
    \item\textbf{Note}:
        \begin{itemize}
            \item La classe è una strategia di scrittura del pattern \textit{Strategy} ma anche la componente "Leaf" del pattern \textit{Composite}, ovvero l'elemento base che non ha sottoelementi;
            \item La costruzione dell'oggetto \textit{kafka\_writer} richiede un riferimento ad un oggetto che implementi l'interfaccia kafka\_target.
        \end{itemize}
    \end{itemize}
    \item{\textbf{Interfaccia: kafka\_target}}
    \begin{itemize}
        \item \textbf{Metodi}:
        \begin{itemize}
            \item \textbf{write\_to\_kafka(data: string): None [public, abstract]} - Metodo astratto che deve essere implementato nelle sottoclassi per scrivere dati su Kafka.
        \end{itemize}
        \item\textbf{Note}:
        \begin{itemize}
            \item La classe è una interfaccia che fornisce un contratto per le operazioni di scrittura/invio a topic Kafka;
            \item Rappresenta il componente Target del pattern \textit{Object Adapter}.
        \end{itemize}
    \end{itemize}
    \item{\textbf{Classe: \textit{KafkaConfluentAdapter}}}
    \begin{itemize}
        \item\textbf{Attributi}:
        \begin{itemize}
            \item \textbf{topic:string [private]} - Il topic su cui scrivere in Kafka;
            \item \textbf{producer:Producer [private]} - Il producer Kafka per inviare messaggi;
            \item \textbf{schema\_registry:CachedSchemaRegistryClient [private]} - Consente di interagire con Confluent Schema Registry in modo efficiente, memorizza in cache gli schemi recuperati da Schema Registry, riducendo le chiamate di rete e migliorando la velocità di accesso agli schemi.
        \end{itemize}
    \end{itemize}
    \item \textbf{Metodi}:
    \begin{itemize}
        \item \textbf{write\_to\_kafka(data: string): None [public]} - Scrive i dati su Kafka dopo averli validati rispetto allo schema registrato per il topic di destinazione.
    \item\textbf{Note}:
        \begin{itemize}
            \item La classe è un'implementazione concreta dell'interfaccia \textit{kafka\_target}, utilizzando la libreria \textit{confluent-kafka} per interagire con Kafka;
            \item Rappresenta il componente "Adapter" del pattern \textit{Object Adapter};
            \item Il Producer kafka rappresenta la componente "service" del pattern \textit{Object Adapter};
            \item Prima di inviare il messaggio (ovvero la misurazione) controlla che sia nel formato/schema corretto per il topic di destinazione evintando cosi sovraccarico inutile di rete;
            \item Nel caso in cui il formato del messaggio sia scorretto questo viene scartato.
        \end{itemize}
    \end{itemize}
\end{itemize}

\subsubsection{Modulo Threading/Scheduling}
\begin{figure}[H]
    \centering
    \includegraphics[width=1.1\textwidth]{../Images/SpecificaTecnica/simulatorThread.PNG}
    \caption{Modulo Threading/Scheduling simulatori sensori - InnovaCity}
    \label{fig: Modulo_simulatori_sensori_thread}
\end{figure}
Questo modulo si propone di gestire la logica di pianificazione per il recupero dei dati dai simulatori dei sensori e di inviare/scrivere tali dati utilizzando il modulo Writer.

Funge da orchestratore per i due moduli appena descritti, offrendo la possibilità di configurare la frequenza di campionamento e il numero di misurazioni da eseguire.

Inoltre, incorpora una logica di ottimizzazione, simile a quella impiegata dai microcontrollori dei sensori nella realtà, al fine di evitare la trasmissione di dati ridondanti, inviando solo i cambiamenti di stato dei sensori.

Il modulo è stato progettato per rispettare il Dependecy Inversion Principle \textit{(DIP)}, di conseguenza sia i moduli di alto livello che quelli di basso livello dipendono da astrazioni (interfacce o classi astratte).

\paragraph{Design pattern Composite:}
Come il modulo di scrittura anche questo è sviluppato secondo il pattern \textit{Composite} che permette di gestire un singolo thread di esecuzione o un gruppo di thread in modo uniforme.

\paragraph{Design pattern Object Adapter:}
Inoltre, considerando l'impiego di più thread per un esecuzione parallela, per delegare l'orchestrazione delle operazioni, si è deciso di utilizzare delle ThreadPool.

Al fine di evitare modifiche dirette al codice di \textit{simulator\_thread\_pool}, è stato adottato il pattern \textit{Object Adapter} per adattare la ThreadPool di Python a un'interfaccia comune con cui \textit{simulator\_thread\_pool} possa interagire.

Questo approccio consente di modificare la logica o la libreria utilizzata per la gestione dei thread senza richiedere modifiche al codice di \textit{simulator\_thread\_pool}, ma semplicemente aggiungendo una nuova classe adapter che implementi \textit{thread\_pool\_target}.

Un'altra implementazione del pattern \textit{Object Adapter} viene impiegata per adattare gli oggetti \textit{misurazione} del modulo dei Simulatori all'interfaccia \textit{writable} del modulo Writers. La classe \textit{adapter\_misurazione}, implementando l'interfaccia \textit{writable}, fornisce un'implementazione del metodo \textit{to\_json()} che consente di convertire un oggetto \textit{misurazione} nel formato JSON, compatibile con il formato definito nello Schema Registry e riconosciuto da Kafka.
\ref*{sec:formatoMessaggi}

\paragraph{Classi, interfacce, metodi e attributi}

\begin{itemize}
    \item{\textbf{Interfaccia: \textit{component\_simulator\_thread}}}
    \begin{itemize}
        \item \textbf{Metodi:}
        \begin{itemize}
            \item \textbf{run(): None [public, abstract]} - Metodo astratto che deve essere implementato nelle sottoclassi per definire il comportamento del thread quando viene avviato;
            \item \textbf{task(): None [public, abstract]} - Metodo astratto che deve essere implementato nelle sottoclassi per definire il compito specifico che il thread deve eseguire;
            \item \textbf{stop(): None [public, abstract]} - Metodo astratto che deve essere implementato nelle sottoclassi per definire come fermare il thread.
        \end{itemize}
        \item\textbf{Note}:
        \begin{itemize}
            \item Eredita le proprietà e i metodi della classe \textit{Thread} della \textit{Standard Library};
            \item \textit{component\_simulator\_thread} è un interfaccia di threading per la simulazione dei sensori, fornisce un contratto per le operazioni di avvio, esecuzione del compito e arresto;
            \item Rappresenta il componente "Component" del pattern \textit{Composite}, descrive le operazioni comuni sia ai singoli Thread sia a composizioni di questi;
            \item L'utilizzatore dei simulatori può lavorare allo stesso modo con elementi semplici (singoli Thread) o complessi (insiemi di Thread in forma di albero).
        \end{itemize}
    \end{itemize}
    \item{\textbf{Classe: \textit{simulator\_thread}}}
    \begin{itemize}
        \item\textbf{Attributi}:
        \begin{itemize}
            \item \textbf{simulator: \textit{simulator} [private]} - Il simulatore da utilizzare per generare i dati;
            \item \textbf{frequency: float [private]} - La frequenza con cui generare i dati;
            \item \textbf{is\_running: bool [private]} - Flag per controllare se il thread è in esecuzione;
            \item \textbf{data\_to\_generate: int [private]} - Il numero di dati da generare;
            \item \textbf{writers: writer [private]} - L'oggetto implementazione di \textit{writer} per scrivere i dati generati. (Singolo o albero - Composite pattern)
        \end{itemize}

        \item \textbf{Metodi:}
        \begin{itemize}
            \item \textbf{run(): None [public]} - Avvia il thread del simulatore;
            \item \textbf{task(): None [public]} - Definisce il compito specifico che il thread deve eseguire, contiene la logica per generare il numero di misurazioni richieste con l'intervallo specificato alla costruzione.
            Inoltre evita l'invio di misurazioni consecutive uguali cosi da ridurre il carico scartando dati ridondanti e deducibili inviando ai \textit{Writers} solo i cambi di stato del sensore da cui acquisisce la misurazione.
            All'interno del metodo, la misurazione restituita dal simulatore, viene adattata ad un oggetto \textit{writable} tramite \textit{adapter\_misurazione} ed inviata ai \textit{Writers};
            \item \textbf{stop(): None [public]} - Ferma il thread del simulatore.
        \end{itemize}
        \item\textbf{Note}:
        \begin{itemize}
            \item La classe è un'implementazione concreta dell'interfaccia \textit{component\_simulator\_thread};
            \item Utilizza un oggetto \textit{\textit{simulator}} per generare dati a una certa frequenza e un oggetto che implementa\textit{component\_writer} per scrivere i dati generati;
            \item Rappresenta il componente Leaf del pattern \textit{Composite};
            \item Se \textit{data\_to\_generate} < 0, allora genera misurazioni finchè il thread non viene interroto dall'esterno;
            \item Sebbene i simulatori non siano considerati dalla proponente parte del prodotto, la logica di ottimizzazione per inviare solo i cambi di stato dei sensori viene implementata nella realtà IoT.
            
            Di conseguenza, è stata presa la decisione di replicarla. È importante notare che questa logica non è incorporata nel simulatore del sensore, il quale ha unicamente il compito semantico di generare dati come un vero sensore. Invece, essa è implementata in \textit{simulator\_thread}, il quale agisce in modo simile a un microcontrollore, responsabile sia della gestione dell'intervallo di campionamento che della logica per l'invio delle misurazioni;
            \item Nel corso dello sviluppo futuro, potrebbe risultare vantaggioso considerare l'implementazione di un pattern \textit{Strategy} per gestire la strategia/criterio di invio dei dati, che possa distinguere tra un invio continuo e la trasmissione solo in caso di cambiamenti di stato. Tuttavia, al momento della decisione, si è optato per non includerlo al fine di evitare un'eccessiva complessità nell'architettura, nota come sovraingegnerizzazione. Tale scelta è stata dettata dalla volontà di mantenere un equilibrio tra la completezza del sistema e la sua semplicità, favorendo un'implementazione più diretta e immediata delle funzionalità richieste.
        \end{itemize}
    \end{itemize}

    \item{\textbf{Classe: \textit{adapter\_misurazione}}}
    \begin{itemize}
        \item\textbf{Attributi}:
        \begin{itemize}
            \item \textbf{misurazione: \textit{misurazione} [private]} - L'oggetto \textit{misurazione} da adattare.
        \end{itemize}
        \item \textbf{Metodi: }
        \begin{itemize}
            \item \textbf{to\_json(): string [public]} - Converte l'oggetto \textit{misurazione} in una stringa JSON conforme a quanto definito in \ref{sec:formatoMessaggi};
            \item \textbf{from\_json(json\_data: dict): misurazione [staticmethod, public]} - Crea un oggetto \textit{misurazione} da un dizionario JSON.
        \end{itemize}
        \item\textbf{Note}:
        \begin{itemize}
            \item La classe è un'implementazione concreta dell'interfaccia \textit{writable}. Fornisce metodi per convertire un oggetto \textit{misurazione} in un formato JSON e viceversa;
            \item Rappresenta la componente "Adapter" del pattern \textit{Object Adapter}.
        \end{itemize}
    \end{itemize}

    \item{\textbf{Classe: \textit{composite\_simulator\_thread}}}
    \begin{itemize}
        \item\textbf{Attributi}:
        \begin{itemize}
            \item \textbf{simulator\_executor: simulator\_thread\_pool [private]} - L'executor per gestire l'esecuzione di più thread dei simulatori.
        \end{itemize}
        \item \textbf{Metodi: }
        \begin{itemize}
            \item \textbf{add\_simulator(simulator: simulator, writers: component\_writer, frequency: float, data\_to\_generate: int): composite\_simulator\_thread [public]} - Aggiunge un simulatore all'executor;
            \item \textbf{add\_simulator\_thread(thread\_simulator: component\_simulator\_thread): composite\_simulator\_thread [public]} - Aggiunge un \textit{simulator\_thread} all'executor;
            \item \textbf{run(): None [public]} - Avvia tutti i \textit{simulator\_thread} nell'executor;
            \item \textbf{stop(): None [public]} - Ferma tutti i \textit{simulator\_thread} nell'executor;
            \item \textbf{task(): None [public]} - Avvia tutti i \textit{simulator\_thread} nell'executor.
        \end{itemize}
        \item\textbf{Note}:
        \begin{itemize}
            \item La classe è un'implementazione concreta dell'interfaccia \textit{component\_simulator\_thread}, utilizza un oggetto \textit{simulator\_thread\_pool} per gestire l'esecuzione di vari simulatori;
            \item Rappresenta il componente "Composite" del pattern \textit{Composite}.
            \item Utilizzando il metodo \textit{run()} viene creato un nuovo thread che permette di eseguire la funzione \textit{task()} in modo sincrono.
            \item Utilizzando il metodo \textit{task()} direttamente non è possibile avere un'esecuzione asincrona. La funzione \textit{task()} viene eseguita sul thread chiamante, bloccando l'esecuzione del codice fino al termine della sua esecuzione. Questa funzione viene utilizzata dalle ThreadPool che creano dei thread per l'esecuzione asincrona delle task.
        \end{itemize}
    \end{itemize}

    \item{\textbf{Classe: \textit{simulator\_thread\_pool}}}
    \begin{itemize}
        \item\textbf{Attributi:}
        \begin{itemize}
            \item \textbf{simulators: List[component\_simulator\_thread] [private]} - La lista dei \textit{component\_simulator\_thread} da eseguire (Singoli thread o alberi di thread);
            \item \textbf{thread\_pool\_adapter: thread\_pool\_target [private]} - Thread pool per gestire l'esecuzione parallela dei simulatori.
        \end{itemize}
        \item \textbf{Metodi:}
        \begin{itemize}
            \item \textbf{run\_all(): None [public]} - Avvia tutti i simulatori nella thread pool, utilizzando l'interfaccia fornita da \textit{thread\_pool\_target} per l'esecuzione controllata di attività in parallelo.
            Per farlo viene utilizzato il metodo \textit{map()} di \textit{thread\_pool\_target} per mappare la funzione statica \textit{start\_simulator()} su ogni \textit{component\_simulator\_thread} in \textit{simulators};
            \item \textbf{stop\_all(): None [public]} - Ferma tutti i simulatori nel thread pool, utilizzando l'interfaccia fornita da \textit{thread\_pool\_target} per l'esecuzione controllata di attività in parallelo.
            Per farlo utilizza il metodo \textit{map()} di \textit{thread\_pool\_target} per mappare la funzione statica \textit{stop\_simulator()} su ogni \textit{component\_simulator\_thread} in \textit{simulators};
            \item \textbf{append\_simulator(simulator: component\_simulator\_thread): None [public]} - Aggiunge un \textit{component\_simulator\_thread} alla threadpool;
            \item \textbf{start\_simulator(simulator: component\_simulator\_thread): None [private, static]} - Avvia un \textit{component\_simulator\_thread};
            \item \textbf{stop\_simulator(simulator: component\_simulator\_thread): None [private, static]} - Ferma un \textit{component\_simulator\_thread}.
        \end{itemize}
        \item\textbf{Note}:
        \begin{itemize}
            \item La classe gestisce una pool di thread per l'esecuzione di vari simulatori, utilizzando un oggetto che implementa \textit{thread\_pool\_target} per gestire l'esecuzione dei simulatori;
            \item I metodi \textit{run\_all()} e \textit{stop\_all()} utilizzano l'interfaccia fornita da \textit{thread\_pool\_target} per mappare rispettivamente la funzione statica \textit{start\_simulator()} e \textit{stop\_simulator()} per ogni \textit{component\_simulator\_thread} in \textit{simulators};
            \item Grazie all'utilizzo di \textit{thread\_pool\_target} è possibile estendere il sistema con nuovi metodi o utilizzare nuove librerie senza dover modificare la classe \textit{simulator\_thread\_pool}, ma solamente aggiungendo una nuova classe adapter che implementi \textit{thread\_pool\_target}.
        \end{itemize}
    \end{itemize}

    \item{\textbf{Interfaccia: \textit{thread\_pool\_target}}}
    \begin{itemize}
        \item\textbf{Metodi:}
        \begin{itemize}
            \item \textbf{map(func, iterable): [abstractmethod]} - Un metodo astratto che deve essere implementato nelle classi derivate. Questo metodo applica la funzione \textit{func} a ogni elemento nell'\textit{iterable}.
        \end{itemize}
        \item\textbf{Note}:
        \begin{itemize}
            \item L'interfaccia rappresenta la componente "Target" del pattern \textit{Object Adapter} fornendo un contratto per le operazioni di esecuzione controllata di attività in parallelo.
        \end{itemize}
    \end{itemize}

    \item{\textbf{Classe: \textit{thred\_pool\_executor\_adapter}}}
    \begin{itemize}
        \item\textbf{Attributi:}
        \begin{itemize}
            \item \textbf{executor: concurrent.futures.ThreadPoolExecutor [private]} - L'executor della thread pool per gestire l'esecuzione dei thread dalla libreria \textit{concurrent.futures}. 
        \end{itemize}
        \item \textbf{Metodi:}
        \begin{itemize}
            \item \textbf{map(func, iterable): [public]} - Applica la funzione \textit{func} a ogni elemento nell'\textit{iterable} utilizzando l'executor del thread pool (\textit{executor}).
        \end{itemize}
        \item\textbf{Note}:
        \begin{itemize}
            \item La classe è un'implementazione concreta dell'interfaccia \textit{thread\_pool\_target}, utilizzando un oggetto \textit{concurrent.futures.ThreadPoolExecutor} per gestire l'esecuzione dei thread;
            \item Rappresenta il componente "Adapter" del pattern \textit{Object Adapter};
            \item Adatta l'oggetto ThreadPoolExecutor dalla libreria \textit{concurrent.futures};
            \item Al momento della costruzione deve essere fornito il parametro intero "workers" ovvero
            Il numero massimo di thread che è possibile utilizzare per eseguire le task indicate.
        \end{itemize}
    \end{itemize}
\end{itemize}

\subsubsection{Progettazione - Panoramica UML }
\begin{figure}[H]
    \centering
    \includegraphics[width=1.1\textwidth]{../Images/SpecificaTecnica/progettazioneCompSimulatori.PNG}
    \caption{Panoramica progettazione simulatori sensori UML - InnovaCity}
    \label{fig: panor_sim}
\end{figure}
L'immagine vuole mostrare sinteticamente la struttura dei moduli Simulatori, Writers e Threading/Scheduling e le relazioni tra le classi principali di questi moduli.

In particolare, si evidenzia la presenza del pattern \textit{Composite} per la gestione di più servizi di scrittura e di più thread di esecuzione, il pattern \textit{Strategy} per la scrittura su diversi servizi e il pattern \textit{Object Adapter} per adattare le ThreadPool di Python e gli oggetti \textit{misurazione} ai \textit{writer}.

Viene riportato solo il sensore di temperatura come implementazione concreta di \textit{simulator} in quanto le altre implementazioni sono analoghe.
\subsection{Apache Kafka}
\subsubsection{Kafka topic}
I topic in Kafka possono essere considerati come le tabelle di un database, utili per separare logicamente diversi tipi di messaggi o eventi che vengono inseriti nel sistema. Noi li utilizziamo per separare le diverse misurazioni dei sensori, quindi per ogni tipo di sensore è presente un topic dedicato. Ciò ci consente di creare all'interno di ClickHouse delle "tabelle consumatrici" che acquisiscono automaticamente i dati. Questo è possibile grazie alla separazione logica dei topic, che garantisce che tutti i messaggi all'interno di ciascun topic abbiano lo stesso formato.
\subsubsection{Formato messaggi} \label{sec:formatoMessaggi}
La struttura di un messaggio contenente le informazioni della misurazione è la seguente in formato Json e rispetta il contratto definito nello Schema Registry, vedi sez.\ref{sec:schema_registry}, in particolare \ref{sec:schema_registry_sez_schema}:
\begin{lstlisting}[style=code]
    {
      "timestamp": "AAAA-MM-DD HH:MM:SS.sss", 
      "value": "Valore della misurazione",  
      "type": "Tipologia Simulatore",
      "latitude": "Latitudine",
      "longitude": "Longitudine",
      "ID_sensore": "ID sensore",
      "cella": "Partizione della citt\`{a} dove \`{e} presente il sensore" 
     }
\end{lstlisting}
Mentre la struttura di un messaggio contenente le informazioni di una misurazione del punteggio di salute è la seguente in formato Json:
\begin{lstlisting}[style=code]
    {
      "timestamp": "AAAA-MM-DD HH:MM:SS.sss", 
      "value": "Valore della misurazione",  
      "type": "Tipologia Simulatore",
      "cella": "Cella relativa al punteggio di salute"
    }
\end{lstlisting}


Sebbene le misurazioni vengano divise in topic diversi a seconda della tipoligia di sensore che ha effettuato la misurazione si è comunque deciso di inviare e salvare il campo della tipoligia di misurazione per i seguenti motivi:
\begin{itemize}
    \item \textbf{Backup e ripristino dei dati:} Se per qualche motivo si dovesse perdere la struttura dei topic o occorre ripristinare i dati in un altro sistema, il campo type può aiutare a identificare il tipo di sensore che ha effettuato la misurazione, anche se i dati sono stati conservati insieme in un unico topic.
    \item \textbf{Flessibilità futura:} 
    \begin{itemize}
        \item Potrebbero sorgere esigenze future che richiedono l'analisi dei dati provenienti da diversi tipi di sensori all'interno dello stesso topic. In questo caso, il campo type sarebbe utile per distinguere le misurazioni provenienti da sensori diversi;
        \item includere il campo type potrebbe essere particolarmente utile se si prevede di supportare diverse unità di misura per una stessa tipologia di sensore in futuro. Ad esempio, potrebbe essere necessario gestire misurazioni di temperatura in gradi Celsius, Fahrenheit o Kelvin. In tal caso, includendo il campo type, si può associare ad ogni misurazione l'unità di misura corretta.
    \end{itemize}
\end{itemize}
    

\subsubsection{Kafka patterns}
\paragraph{Pattern di Pub/Sub}
\begin{itemize}
    \item \textbf{Descrizione:} Il pattern Pub/Sub (Publish/Subscribe) permette ai producer di inviare messaggi a topic e ai consumer di ricevere messaggi da tali topic.
    \item \textbf{Funzione in Kafka:} Decoupling tra producer e consumer, favorendo la scalabilità e l'asincronia.
    \item \textbf{Esempio:} Un sensore invia dati a Kafka come producer. I dati vengono pubblicati su un topic specifico, e più consumer, come un'applicazione di analisi in tempo reale e un sistema di archiviazione, si iscrivono al topic.
\end{itemize}

\paragraph{Partizionamento}
\begin{itemize}
    \item \textbf{Descrizione:} Distribuisce i messaggi su più partizioni all'interno di un topic per migliorare la scalabilità e le prestazioni.
    \item \textbf{Funzione in Kafka:} Permette di distribuire il carico di lavoro su più broker e di aumentare la resilienza ai guasti.
    \item \textbf{Esempio:} I dati di un sensore possono essere partizionati in base al tipo di sensore o alla posizione geografica.
\end{itemize}

\paragraph{Replicazione}
\begin{itemize}
    \item \textbf{Descrizione:} Duplica i dati su più broker per garantire la disponibilità e la tolleranza ai guasti.
    \item \textbf{Funzione in Kafka:} I messaggi vengono replicati su un numero configurabile di broker per massimizzare la ridondanza.
    \item \textbf{Esempio:} Se un broker fallisce, i dati sono ancora disponibili su altri broker.
\end{itemize}



\paragraph{Leader Election}
\begin{itemize}
    \item \textbf{Descrizione:} Algoritmo per eleggere un leader per ogni partizione, responsabile dell'ordinamento e della replica dei messaggi.
    \item \textbf{Funzione in Kafka:} Garantisce la coerenza dei dati e la gestione efficiente delle partizioni.
    \item \textbf{Esempio:} Un leader viene eletto per ogni partizione del topic, garantendo che solo un broker riceva e replichi i messaggi per quella partizione.
\end{itemize}


\paragraph{Log Compaction}
\begin{itemize}
    \item \textbf{Descrizione:} Rimuove i messaggi obsoleti da un topic per ottimizzare l'utilizzo dello storage.
    \item \textbf{Funzione in Kafka:} Le vecchie versioni dei messaggi vengono eliminate dopo un periodo di tempo configurabile.
    \item \textbf{Esempio:} I messaggi di sensore con valori vecchi possono essere compattati per risparmiare spazio di archiviazione.
\end{itemize}

\paragraph{Altri Pattern}
Oltre a quelli sopra elencati, Kafka implementa altri pattern come:
\begin{itemize}
    \item \textbf{Consumer Group}: Raggruppamento di consumer che collaborano per ricevere messaggi da un topic.
    \item \textbf{Coordinated Commit}: Meccanismo per garantire che tutti i consumer in un gruppo ricevano correttamente tutti i messaggi di una partizione.
    \item \textbf{Rate Limiting}: Controllo del numero di messaggi che possono essere inviati o ricevuti da un topic in un determinato intervallo di tempo.
    \item \textbf{Dead Letter Queue (DLQ)}: Coda speciale dove vengono inviati i messaggi che non possono essere elaborati correttamente.
    \item \textbf{Monitoring \& Metrics}: Fornisce un'ampia gamma di metriche per monitorare le prestazioni e l'utilizzo del sistema.
\end{itemize}

\paragraph{Conclusione}
L'utilizzo di questi design pattern rende Kafka una piattaforma di messaggistica robusta, scalabile e affidabile per una varietà di casi d'uso. L'implementazione di questi pattern permette di ottenere un'architettura efficiente e performante per l'elaborazione dati in streaming.
\subsection{Faust - Processing Layer} \label{sec:faust}

\subsubsection{Introduzione}
Per soddisfare il requisito opzionale del calcolo del punteggio di salute, si è scelto di utilizzare Faust, una libreria \textit{Python}\textsubscript{\textit{G}} ispirata al modello di Kafka\textsubscript{\textit{G}} Streams.

Faust facilita l'elaborazione di flussi di dati distribuiti in tempo reale, rendendola una libreria ideale per questo caso d'uso.
Offre un'interfaccia di alto livello che astrae le complessità di Kafka\textsubscript{\textit{G}}, rendendo la raccolta dati semplice e intuitiva.
Inoltre Faust è progettato per essere scalabile e può essere utilizzato per gestire grandi volumi di dati.

\paragraph{Faust \& Schema Registry} 
Faust deserializza i messaggi dei topic in conformità allo schema definito nello Schema Registry. Qualora si riscontrino messaggi non conformi, essi vengono eliminati. I messaggi validi vengono successivamente processati e instradati verso un topic Kafka dedicato.

La gestione della connessione allo Schema Registry è integrata in Faust. Dopo aver fornito l'indirizzo dello Schema Registry nella configurazione dell'applicazione Faust, essa si occupa di validare i messaggi in base allo schema registrato per il relativo topic.

\paragraph{Calcolo del punteggio di salute}
Il punteggio di salute rappresenta un indicatore sintetico del benessere generale di una città, viene calcolato sulla base di tre tipologie di misurazioni:
\begin{itemize}
    \item Temperatura;
    \item Umidità;
    \item Livello di polveri sottili nell'aria (PM10).
\end{itemize}

Il calcolo del punteggio di salute avviene in due fasi:
\begin{enumerate}
    \item \textbf{Incrementi:} 
    \begin{itemize}
        \item A intervalli regolari, per ciascuna tipoligia di misurazione, viene calcolato un incremento basato sulle sole misurazioni acquisite all'interno dell'intervallo;
        \item Ciascuna tipologia di misurazione ha un suo algoritmo di calcolo dell'incremento, basato su soglie predefinite di benessere.
    \end{itemize}
    \item \textbf{Punteggio Finale:}
    \begin{itemize}
        \item Il punteggio di salute finale si ottiene sommando gli incrementi calcolati per le tre tipologie di misurazioni;
        \item Punteggi più alti indicano un minore stato di benessere, con la necessità di interventi per migliorare la qualità della vita.
    \end{itemize}
\end{enumerate}

\subsubsection{Componenti Faust \& Processing Layer}
\begin{itemize}
    \item \textbf{Applicazione Faust:}
    \begin{lstlisting}[style=code]
    faust.App(<nome_app>, broker=<broker_kafka>)
    \end{lstlisting} 
    \begin{itemize}
        \item Un'applicazione Faust è un \textit{programma}\textsubscript{\textit{G}} \textit{Python}\textsubscript{\textit{G}} che elabora flussi di dati in tempo reale da Kafka\textsubscript{\textit{G}};
        \item \textbf{nome\_app:} Identifica l'applicazione, coincide con il \textit{ConsumerGroup} di Kafka;
        \item \textbf{broker\_kafka:} Indirizzo del \textit{broker}\textsubscript{\textit{G}} Kafka\textsubscript{\textit{G}} (hostname: porta);
        \item Importante configurare l'applicazione con l'indirizzo dello Schema Registry per garantire la validazione e deserializzazione dei messaggi.
    \end{itemize}

    \item \textbf{Topic:}
    \begin{lstlisting}[style=code]
    app.topic(<nome_topic>, value_type=<tipo_dato>)
    \end{lstlisting}  
    \begin{itemize}
        \item \textbf{nome\_topic:} Nome del topic Kafka\textsubscript{\textit{G}} a cui iscrivere l'app Faust;
        \item \textbf{tipo\_dato:} Classe che rappresenta il tipo di dato del topic (es. faust\_measurement);
        \item Nel caso si vogliano aggiungere altri topic da cui consumare dati basterà aggiungerli prima del parametro value\_type separati da virgole.
    \end{itemize}

    \item \textbf{Tipo di dato atteso:}
     \begin{lstlisting}[style=code]
    class faust\_measurement(faust.Record, serializer='json')
    \end{lstlisting}  
    \begin{itemize}
        \item È una classe che eredita da \textbf{faust.Record};
        \item \textbf{faust.Record} è una classe fornita dalla libreria Faust che semplifica la definizione di record per la rappresentazione dei dati in streaming;
        \item Rappresenta una singola misurazione proveniente da un \textit{sensore}\textsubscript{\textit{G}}. Viene usata nella applicazione Faust per definire il tipo di dati atteso nei topic Kafka\textsubscript{\textit{G}}.
    \end{itemize}

    \item \textbf{Modello per il calcolo del punteggio di salute:}
    \begin{itemize}
        \item \textbf{Processore di misurazioni:}
        Tramite il \textit{pattern}\textsubscript{\textit{G}} \textit{Object Adapter} e l'interfaccia \textit{processor} l'app Faust invia le misurazioni ottenute dai topic al modello per il calcolo del punteggio di salute che verrà adattato come \textit{processor}.
    \end{itemize}

    \item \textbf{Agente di elaborazione:} 
    \begin{lstlisting}[style=code]
    @app.agent(<topic>)
    \end{lstlisting}  
    \begin{itemize}
        \item Permette di definire una funzione asincrona che elabora i dati dai topic;
        \item La funzione riceve un iteratore di oggetti del tipo specificato per il topic;
        \item La funzione esegue l'elaborazione desiderata su ogni misurazione.
    \end{itemize}

    \item \textbf{Interfaccia \textit{processor}:}
    \begin{itemize}
        \item Viene utilizzata per definire un contratto di logiche di elaborazione;
        \item Viene utilizzata dagli agenti di elaborazione per inviare le misurazioni al modello per il calcolo del punteggio di salute.
    \end{itemize}

    \item \textbf{Task aggiuntivo} (opzionale): 
    \begin{lstlisting}[style=code]
    @app.task()
        \end{lstlisting}  
    \begin{itemize}
        \item Definisce una funzione eseguita una sola volta all'avvio dell'applicazione;
        \item Nel nostro progetto, viene chiamato il metodo \textit{start()} del thread adibito all'ottenimento e scrittura periodico dei punteggi di salute.
    \end{itemize}
\end{itemize}

\subsubsection{Processing layer data-flow}
Di seguito esposto il flusso dei dati per il calcolo del punteggio di salute.

\begin{figure}[H]
    \centering
    \includegraphics[width=0.8\textwidth]{../Images/SpecificaTecnica/faustFlow.png}
    \caption{Faust data-flow}
    \label{fig: FaustDataflow}
\end{figure}
In breve:
\begin{enumerate}
    \item Faust ottiene le misurazioni dai topic;
    \item Tramite una porta di accesso le fornisce al modello per il calcolo del punteggio di salute;
    \item Quando l'applicazione Faust viene avviata questa avvia a sua volta un Thread che periodicamente ottiene i punteggi di salute calcolati sulla base delle misurazioni ottenute da Faust e tramite il modulo Writer li invia al topic Kafka dedicato.
\end{enumerate}

\subsubsection{Modello per il calcolo del punteggio di salute}
\begin{figure}[H]
    \centering
    \includegraphics[width=1\textwidth]{../Images/SpecificaTecnica/healthModel.PNG}
    \caption{Modello per il calcolo del punteggio di salute - InnovaCity}
    \label{fig: healthModello}
\end{figure}

Il modello per il calcolo del punteggio di salute riceve le letture dei sensori attraverso gli agenti di elaborazione dell'applicazione Faust, i quali sono in ascolto sui topic Kafka\textsubscript{\textit{G}} relativi alle misurazioni di temperatura, umidità e polveri sottili PM10.

Ad intervalli regolari, il \textit{sistema}\textsubscript{\textit{G}} calcola il punteggio di salute della città basandosi su tali misurazioni. Una volta effettuato il calcolo, il risultato è reso disponibile in un topic Kafka\textsubscript{\textit{G}} dedicato.
Il modello che racchiude la logica per il calcolo, richiamato a intervalli regolari, è quello attualmente preso in esame.

In sintesi, il modello:
\begin{itemize}
    \item Riceve le misurazioni di temperatura, umidità e polveri sottili dall'agente dell'app Faust;
    \item Riceve da un thread la richiesta ad intervalli regolari di calcolare i punteggi di salute per le celle della città con le misurazioni ottenute in tempo reale.
\end{itemize}

In modo simile a quanto accade in un'\textit{architettura}\textsubscript{\textit{G}} esagonale, la logica del modello è completamente disaccoppiata dai suoi utilizzatori, i quali interagiscono con esso tramite specifiche classi adapter. Questo approccio promuove la separazione delle preoccupazioni e favorisce la modularità del \textit{sistema}\textsubscript{\textit{G}}. Gli adapter fungono da ponte tra il modello e gli utilizzatori, consentendo una comunicazione fluida e senza dipendenze dirette. Così, eventuali cambiamenti nella logica del modello possono essere implementati senza influenzare gli utilizzatori, garantendo una maggiore flessibilità e manutenibilità del \textit{sistema}\textsubscript{\textit{G}} nel suo complesso.

Il presente modulo è concepito per fornire la logica relativa al puro calcolo del punteggio di salute della città. Tale calcolo si basa su un modello che tiene conto delle misurazioni sopra citate ed è stato progettato al fine di determinare un punteggio di salute per ciascuna cella della città in cui sono presenti misurazioni delle suddette tipologie.

\paragraph{Design pattern Strategy}

Il modello per il calcolo del punteggio di salute è stato ideato mediante l'utilizzo del design \textit{pattern}\textsubscript{\textit{G}} Strategy. Tale \textit{pattern}\textsubscript{\textit{G}} consente di definire una famiglia di algoritmi, di incapsularli e renderli intercambiabili. Ciò permette di variare l'algoritmo impiegato per il calcolo del punteggio di salute senza incidere sui suoi utilizzatori. In particolare, l'interfaccia \textit{health\_algorithm} stabilisce il contratto che deve essere rispettato da tutti gli algoritmi per il calcolo del punteggio di salute.

Inoltre, un'implementazione del \textit{pattern}\textsubscript{\textit{G}} \textit{Strategy} è presente anche negli "Incrementers". Questi, a partire dalle misurazioni fornite, restituiscono un valore da sommare al punteggio di salute della città. Tale incremento è determinato in base a delle soglie predefinite di temperatura, umidità e polveri sottili PM10, le quali sono definite di default in \textit{health\_constants} ma possono essere impostate al momento della costruzione.

In particolare, l'interfaccia \textit{incrementer} specifica il contratto che deve essere rispettato da tutti gli Incrementers. Vengono implementati tre Incrementers, uno per la temperatura, uno per l'umidità e uno per le polveri sottili PM10, come strategie del \textit{pattern}\textsubscript{\textit{G}} Strategy.

\paragraph{Classi, interfacce, metodi e attributi}
\begin{itemize}
    \item{\textbf{Interfaccia: \textit{health\_algorithm}}}
    \begin{itemize}
    \item \textbf{Metodi:}
    \begin{itemize}
        \item \textbf{generate\_new\_health\_score(): List[misurazione\_salute] [abstractmethod]} - Un metodo astratto che, nelle sue implementezioni concrete genera un punteggio di salute.
    \end{itemize}
    \item\textbf{Note:}
        \begin{itemize}
            \item L'interfaccia definisce il contratto per un algoritmo di calcolo per un punteggio di salute. Le sottoclassi devono implementare il metodo \textit{generate\_new\_health\_score()};
            \item Rappresenta la componente "Strategy" del \textit{pattern}\textsubscript{\textit{G}} omonimo;
            \item Per rispettare il Single Responsibility Principle \textit{(SRP)}, \textit{health\_algorithm} è stata divisa dalla logica di buffering delle misurazioni presente nella classe astratta \textit{health\_processor\_buffer} poiché l'utilizzatore \textit{health\_calculator\_thread} non utilizza i metodi per il buffering.
        \end{itemize}
    \end{itemize}

    \item\textbf{Classe astratta: \textit{health\_processor\_buffer}}
    \begin{itemize}
    \item\textbf{Attributi:}
        \begin{itemize}
        \item \textbf{lista\_misurazioni: lista\_misurazioni [private]} - Una lista di oggetti \textit{misurazione};
        \item \textbf{lock: threading.Lock [private]} - Un oggetto lock per gestire l'accesso concorrente alla lista di misurazioni.
    \end{itemize}
    \item \textbf{Metodi:}
    \begin{itemize}
        \item \textbf{add\_misurazione(timestamp, value, type\_, latitude, longitude, ID\_sensore, cella): None [public]} - Aggiunge una nuova misurazione alla lista di misurazioni;
        \item \textbf{clear\_list(): None [public]} - Svuota la lista di misurazioni.
    \end{itemize}
    \item\textbf{Note:}
        \begin{itemize}
            \item La classe astratta definisce un buffer di misurazioni per effettuare processing su di esse;
            \item La logica di buffering e quella dell'algoritmo per il calcolo del punteggio di salute vengono separate in due astrazioni per rispettare il Single Responsibility Principle \textit{(SRP)}. Gli utilizzatori di questa classe, ovvero i Processor, sono interessati esclusivamente al metodo per l'invio del dato al buffer;
            \item La classe astratta definisce un'interfaccia per la comunicazione con gli utilizzatori esterni al modello.
        \end{itemize}
    \end{itemize}
    \item{\textbf{Classe: \textit{health\_calculator}}}
    \begin{itemize}
    \item\textbf{Attributi:}
        \begin{itemize}
        \item \textbf{tmpInc: temperature\_incrementer [private]} - Utilizzato per il calcolo dell'incremento di temperatura;
        \item \textbf{umdInc: humidity\_incrementer [private]}; - Utilizzato per il calcolo dell'incremento di umidità;
        \item \textbf{dstPm10Inc: dust\_PM10\_incrementer [private]} - Utilizzato per il calcolo dell'incremento di PM10;
        \item \textbf{temperature\_measure\_type\_naming: string [private]} - Nomenclatura dei tipi di misurazione di temperatura;
        \item \textbf{humidity\_measure\_type\_naming: string [private]} - Nomenclatura dei tipi di misurazione di umidità;
        \item \textbf{ dtsPm10\_measure\_type\_naming: string [private]} - Nomenclatura dei tipi di misurazione di PM10;
        \item \textbf{ health\_score\_measure\_type\_naming: string [private]} - Nomenclatura dei tipi di misurazione di punteggio di salute;
        \item \textbf{lock [private]} - Un oggetto lock per gestire l'accesso concorrente.
    \end{itemize}
    \item \textbf{Metodi:}
    \begin{itemize}
        \item \textbf{generate\_new\_health\_score(): List[misurazione\_salute] [public]} - Genera e restituisce una nuova lista di punteggi di salute, uno per ogni cella della città di cui sono state fornite misurazioni;
        \item \textbf{calcola\_incremento\_tmp(cella: str, lista\_misurazioni): int [private]} - Calcola e restituisce l'incremento della temperatura;
        \item \textbf{calcola\_incremento\_umd(cella: str, lista\_misurazioni): int [private]} - Calcola e restituisce l'incremento dell'umidità;
        \item \textbf{calcola\_incremento\_dstPm10(cella: str, lista\_misurazioni): int [private]} - Calcola e restituisce l'incremento della polvere PM10.
    \end{itemize}
    \item\textbf{Note:}
        \begin{itemize}
            \item La classe implementa l'interfaccia \textit{health\_algorithm} e la classe astratta \textit{health\_processor\_buffer} per calcolare il punteggio di salute tramite la strategia concreta definita in \textit{generate\_new\_health\_score()} che genera una nuova lista di punteggi di salute;
            \item Questa classe rappresenta il cervello del calcolo del punteggio di salute in quanto utilizzatore di tutti gli incrementatori e delle misurazioni bufferizzate per creare una strategia di calcolo.
        \end{itemize}
    \end{itemize}

    \item\textbf{Classe: \textit{misurazione}}
    \begin{itemize}
        \item \textbf{Attributi:} 
    \begin{itemize}
        \item \textbf{timestamp: datetime [private]} - Timestamp della misurazione;
        \item \textbf{value [private]} - Valore della misurazione;
        \item \textbf{type: str [private]} - Tipo della misurazione;
        \item \textbf{coordinates: coordinate [private]} - Coordinate della misurazione;
        \item \textbf{ID\_sensore: str [private]} - \textit{ID}\textsubscript{\textit{G}} del \textit{sensore}\textsubscript{\textit{G}} che ha effettuato la misurazione;
        \item \textbf{cella: str [private]} - Cella in cui è stata effettuata la misurazione.
    \end{itemize}
    \item \textbf{Metodi:} 
    \begin{itemize}
        \item \textbf{\_\_eq\_\_(other: misurazione): bool [public]} - Ridefinizione dell'operatore di uguaglianza per confrontare due oggetti \textit{misurazione}.
    \end{itemize}
\end{itemize}

    \item\textbf{Classe: \textit{coordinate}}
    \begin{itemize}
        \item    \textbf{Attributi:} 
    \begin{itemize}
        \item \textbf{latitude:float [private]} - Latitudine della coordinata.
        \item \textbf{longitude:float [private]} - Longitudine della coordinata.
    \end{itemize}
    \item     \textbf{Metodi:} 
    \begin{itemize}
        \item \textbf{\_\_eq\_\_(other:coordinate):bool [public]} - Ridefinizione dell'operatore di uguaglianza per confrontare due oggetti Coordinate.
    \end{itemize}
\end{itemize}
\item\textbf{Classe: \textit{misurazione\_salute}}
    \begin{itemize}
    \item\textbf{Attributi:}
        \begin{itemize}
        \item \textbf{timestamp:datetime [private]} - Il timestamp della misurazione di salute.
        \item \textbf{value:float [private]} - Il valore della misurazione di salute.
        \item \textbf{type:string [private]} - Il tipo della misurazione.
        \item \textbf{cella:string [private]} - La cella della misurazione di salute.
    \end{itemize}
    \item\textbf{Note:}
        \begin{itemize}
            \item La classe rappresenta una misurazione di salute. Contiene informazioni sul timestamp, il valore (ovvero il punteggio di salute calcolato), il tipo della misurazione e la cella relativa alla misurazione.
        \end{itemize}
    \end{itemize}
    \item\textbf{Classe: \textit{lista\_misurazioni}}
    \begin{itemize}
    \item\textbf{Attributi:}
        \begin{itemize}
        \item \textbf{list:List[Misurazione] [private]} - Una lista di oggetti \textit{misurazione}.
    \end{itemize}
    \item \textbf{Metodi: }
    \begin{itemize}
        \item \textbf{add\_misurazione(timestamp, value, type\_, latitude, longitude, ID\_sensore, cella): None [public]} - Aggiunge una nuova misurazione alla lista.
        \item \textbf{clear\_list(): None [public]} - Svuota la lista di misurazioni.
        \item \textbf{get\_list\_by\_cella\_and\_type(cella: str, tipo\_dato: str): List[Misurazione] [public]} - Restituisce una lista di misurazioni che corrispondono alla cella e al tipo di misurazione specificati (temperatura,umidità,ecc.).
        \item \textbf{get\_unique\_celle(): List[str] [public]} - Restituisce la lista di celle presenti nelle misurazioni senza ripetzioni.
    \end{itemize}
    \item\textbf{Note:}
        \begin{itemize}
            \item La classe rappresenta una lista di misurazioni. Fornisce metodi per aggiungere misurazioni, svuotare la lista, ottenere misurazioni per cella e tipo di misurazioni, e ottenere le celle di cui si hanno misurazioni.
        \end{itemize}
    \end{itemize}
    \item\textbf{Enumerazione: \textit{SensorTypes}}
        \begin{itemize}
            \item \textbf{Costanti:} 
            \begin{itemize}
                \item \textbf{TEMPERATURE:str [public]} - Rappresenta la nomenclatura dei \textit{sensore}\textsubscript{\textit{G}} di temperatura.
                \item \textbf{HUMIDITY:str [public]} - Rappresenta la nomenclatura dei \textit{sensore}\textsubscript{\textit{G}} di umidità.
                \item \textbf{DUST\_PM10:str [public]} - Rappresenta la nomenclatura dei \textit{sensore}\textsubscript{\textit{G}} di "polvere PM10".
                \item \textbf{CHARGING\_STATION:str [public]} - Rappresenta la nomenclatura dei \textit{sensore}\textsubscript{\textit{G}} di stato delle colonnine di ricarica.
                \item \textbf{ECOLOGICAL\_ISLAND:str [public]} - Rappresenta la nomenclatura dei \textit{sensore}\textsubscript{\textit{G}} di stato riempimento isole ecologica.
                \item \textbf{WATER\_PRESENCE:str [public]} - Rappresenta la nomenclatura dei \textit{sensore}\textsubscript{\textit{G}} di presenza d'acqua.
                \item \textbf{ELECTRICAL\_FAULT:str [public]} - Rappresenta la nomenclatura dei \textit{sensore}\textsubscript{\textit{G}} di guasti elettrici.
            \end{itemize}

            \item \textbf{Note:}
            \begin{itemize}
                \item L'enumerazione viene utilizzata per centralizzare la gestione della nomenclatura dei tipi di sensori che verrà salvata nelle misurazioni.
            \end{itemize}
        \end{itemize}
\item \textbf{Interfaccia: \textit{incrementer}}
    \begin{itemize}
    \item \textbf{Metodi: }
    \begin{itemize}
        \item \textbf{get\_incrementation(misurazioni: List[Misurazione]): int [abstractmethod]} - Un metodo astratto che deve essere implementato nelle sottoclassi. Questo metodo calcola e restituire un incremento basato sulla lista di misurazioni fornita.
    \end{itemize}
    \item\textbf{Note:}
        \begin{itemize}
            \item L'interfaccia definisce il contratto per un incrementatore. Le sottoclassi devono implementare il metodo \textit{get\_incrementation()}.
            \item Rappresenta la componente "Strategy" del \textit{pattern}\textsubscript{\textit{G}} omonimo.
        \end{itemize}
    \end{itemize}
    \item \textbf{Classe: \textit{temperature\_incrementer}}
    \begin{itemize}
    \item \textbf{Attributi:}
        \begin{itemize}
        \item \textbf{upper\_health\_soglia:int [private]} - La soglia superiore di benessere per la temperatura;
        \item \textbf{under\_health\_soglia:int [private]} - La soglia inferiore di benessere per la temperatura.
    \end{itemize}
    \item \textbf{Metodi: }
    \begin{itemize}
        \item \textbf{get\_incrementation(misurazioni: List[Misurazione]): float [public]} - Calcola e restituisce un incremento basato sulle sole misurazioni di temperatura della lista fornita.
    \end{itemize}
    \item\textbf{Note:}
        \begin{itemize}
            \item La classe implementa l'interfaccia \textit{incrementer};
            \item I valori di default per le soglie vengono presi dall'enumerazione \textit{HealthConstant} altrimenti sono impostabili alla costruzione.
            \item Rappresenta una strategia concreta del \textit{pattern}\textsubscript{\textit{G}} \textit{Strategy} per il calcolo dell'incremento di temperatura.
        \end{itemize}
    \end{itemize}
    \item{\textbf{Classe: \textit{humidity\_incrementer}}}
    \begin{itemize}
    \item\textbf{Attributi:}
        \begin{itemize}
        \item \textbf{upper\_health\_soglia:int [private]} - La soglia superiore di benessere per l'umidità;
        \item \textbf{under\_health\_soglia:int [private]} - La soglia inferiore di benessere per l'umidità.
    \end{itemize}
    \item \textbf{Metodi: }
    \begin{itemize}
        \item \textbf{get\_incrementation(misurazioni: List[Misurazione]): float [public]} - Calcola e restituisce un incremento basato sulle sole misurazioni di umidità della lista fornita.
    \end{itemize}
    \item\textbf{Note:}
        \begin{itemize}
            \item La classe implementa l'interfaccia \textit{incrementer};
            \item I valori di default per le soglie vengono presi dall'enumerazione \textit{HealthConstant} altrimenti sono impostabili alla costruzione.
            \item Rappresenta una strategia concreta del \textit{pattern}\textsubscript{\textit{G}} \textit{Strategy} per il calcolo dell'incremento di umidità.
        \end{itemize}
    \end{itemize}\item{\textbf{Classe: \textit{dust\_PM10\_incrementer}}}
    \begin{itemize}
    \item \textbf{Metodi: } 
    \begin{itemize}
        \item \textbf{get\_incrementation(misurazioni: List[Misurazione]): float [public]} - Calcola e restituisce un incremento basato sulle sole misurazioni di polveri sottili della lista fornita.
    \end{itemize}
    \item\textbf{Note:}
        \begin{itemize}
            \item La classe implementa l'interfaccia \textit{incrementer};
            \item Rappresenta una strategia concreta del \textit{pattern}\textsubscript{\textit{G}} \textit{Strategy} per il calcolo dell'incremento di polveri sottili PM10.
            \item A differenza degli altri \textit{Incrementer}, \textit{dust\_PM10\_incrementer} non definisce soglie di benessere in quanto è scontato che il valore ottimale di inquinamento è zero.
        \end{itemize}
    \end{itemize}

\end{itemize}

\subsubsection{Modulo Writer}
\begin{figure}[H]
    \centering
    \includegraphics[width=1\textwidth]{../Images/SpecificaTecnica/writerModule.PNG}
    \caption{Modulo Writer - InnovaCity}
    \label{fig: healthModelloWriter}
\end{figure}
Il modulo Writer è il medesimo descritto in \ref*{sec:writersModule} e viene nella sua totalità riutilizzato per la scrittura dei punteggi di salute calcolati.
Non viene riportata la strategia di scrittura su di una lista poichè non ne è stato ritenuto necessario l'utilizzo.

\paragraph*{Classi: metodi e attributi}
Tutte le informazioni sono già state esposte in: \ref*{sec:writersModule}.

\subsubsection{Modulo Threading/Scheduling}
\begin{figure}[H]
    \centering
    \includegraphics[width=1\textwidth]{../Images/SpecificaTecnica/healthThreading.PNG}
    \caption{Modulo Threading/Scheduling health Model - InnovaCity}
    \label{fig: threadHealth}
\end{figure}

Questo modulo si occupa di integrare la logica di Scheduling e Threading per il calcolo periodico del punteggio di salute della città e quella di scrittura/invio di \textit{writable}. In particolare, fornisce un'implementazione di un thread che, a intervalli regolari, richiama il calcolo del punteggio di salute della città. Successivamente, utilizzando il modulo \textit{Writer}, adatta le misurazioni di salute ottenute all'interfaccia \textit{writable} per inviare i dati al topic Kafka e/o stamparli su terminale. Pertanto, questo modulo è utilizzatore del modello per il calcolo del punteggio di salute e del modulo di \textit{Writer}.
Il modulo è stato progettato per rispettare il Dependecy Inversion Principle \textit{(DIP)}, di conseguenza sia i moduli di alto livello che quelli di basso livello dipendono da astrazioni (interfacce o classi astratte).

In sintesi, il modulo:
    \begin{enumerate}
        \item Richiama l'algoritmo per calcolo del punteggio di salute della città a intervalli regolari;
        \item Scrive il risultato ottenuto sul topic Kafka\textsubscript{\textit{G}} dedicato.
    \end{enumerate}

\paragraph*{Classi, interfacce, metodi e attributi}
\begin{itemize}
    \item{\textbf{Classe: \textit{health\_calculator\_thread}}}
    \begin{itemize}
        \item\textbf{Attributi:}
        \begin{itemize}
            \item \textbf{health\_calculator: health\_algorithm [private]} - Un implementatazione dell'interfaccia \textit{health\_algorithm}, ovvero una strategia per il calcolo del punteggio di salute;
            \item \textbf{frequency: float [private]} - La frequenza con cui il thread genera nuovi punteggi di salute;
            \item \textbf{is\_running: bool [private]} - Un flag che indica se il thread è in esecuzione;
            \item \textbf{data\_to\_generate: int [private]} - Il numero di misurazioni di salute da generare;
            \item \textbf{writers: writer [private]} - Un oggetto che implementa la classe \textit{writer}. (Singolo scrittore o albero, Composite \textit{pattern}\textsubscript{\textit{G}})
        \end{itemize}
        \item \textbf{Metodi: }
        \begin{itemize}
            \item \textbf{run(): None [public]} - Esegue il thread, generando nuovi punteggi di salute ad una certa frequenza;
            \item \textbf{stop(): None [public]} - Ferma l'esecuzione del thread.
        \end{itemize}
        \item\textbf{Note:}
        \begin{itemize}
            \item La classe estende la classe \textit{threading.Thread};
            \item Se \textit{data\_to\_generate} è minore di zero, genera misurazioni di salute finchè il thread non viene interroto dall'esterno;
            \item   Grazie al \textit{pattern}\textsubscript{\textit{G}} \textit{Strategy} è possibile cambiare agevolmente l'algoritmo volto al calcolo del punteggio di salute della città.
        \end{itemize}
    \end{itemize}
    \item{\textbf{Classe: \textit{adapter\_misurazione}}}
    \begin{itemize}
        \item \textbf{Metodi: }
        \begin{itemize}
            \item \textbf{to\_json(): String [public]} - Ritorna una stringa JSON compatibile con lo schema richiesto nello Schema Registry per le misurazione di salute.
        \end{itemize}
        \item\textbf{Note:}
        \begin{itemize}
            \item Rappresenta il componente adapter del pattern \textit{Object adapter};
            \item L'adapter adatta le misurazioni di salute ottenute dall'algoritmo di calcolo del punteggio di salute all'interfaccia \textit{writable} per l'invio al topic Kafka\textsubscript{\textit{G}}.
        \end{itemize}
    \end{itemize}
\end{itemize}

\subsubsection{Modulo Processing}
\begin{figure}[H]
    \centering
    \includegraphics[width=1\textwidth]{../Images/SpecificaTecnica/processorFaust.PNG}
    \caption{Modulo Processing - InnovaCity}
    \label{fig: processHealth}
\end{figure}
Al fine garantire un'interfaccia uniforme delle operazioni di elaborazione dei dati provenienti da Kafka\textsubscript{\textit{G}} tramite Faust e per stabilire un canale di comunicazione con il modello per il calcolo del punteggio di salute, viene sviluppato il modulo di Processing. Il modulo offre l'interfaccia target denominata \textit{processor}, e un suo adapter denominato \textit{health\_model\_processor\_adapter} per l'invio delle misurazioni al modello per il calcolo del punteggio di salute.

\paragraph*{Design Pattern Object Adapter}
Nel contesto dell'applicazione Faust, all'interno del ruolo svolto dagli agenti, ogni volta che una misurazione viene ricevuta, viene invocato il metodo \textit{process()} dell'implementazione dell'interfaccia \textit{processor} denominata \textit{health\_model\_processor\_adapter}.

In particolare, \textit{health\_model\_processor\_adapter} adatta la classe astratta \textit{health\_processor\_buffer}, che rappresenta un buffer di misurazioni utilizzato per eseguire il calcolo periodico del punteggio di salute della città, all'interfaccia \textit{processor}.

Questo \textit{pattern}\textsubscript{\textit{G}} consente di definire dei contratti per le logiche di elaborazione e di rendere il modello indipendente dall'implementazione specifica dell'applicazione Faust. Allo stesso tempo, facilita la sostituzione dell'operazione di elaborazione eseguita su ogni misurazione dagli agenti grazie al contratto definito nell'interfaccia \textit{processor}.
\paragraph*{Classi, interfacce, metodi e attributi}
\begin{itemize}
    \item{\textbf{Interfaccia: \textit{processor}}}
    \begin{itemize}
    \item\textbf{Metodi: }
    \begin{itemize}
        \item \textbf{process(misurazione: faust\_measurement): None [public, abstract]} - Metodo astratto che deve essere implementato nelle sottoclassi per effettuare elaborazioni di una misurazione.
    \end{itemize}
    \item\textbf{Note:}
        \begin{itemize}
            \item  Le sottoclassi devono implementare il metodo astratto \textit{process()} definendo la propria operazione da effettuare su ogni misurazione ricevuta dai topic di iscrizione;
            \item Rappresenta la componente "Target" del \textit{pattern}\textsubscript{\textit{G}} \textit{Object Adapter};
            \item L'interfaccia è stata progettata per garantire e rappresentare un contratto uniforme per i metodi di elaborazione dei dati provenienti da Kafka\textsubscript{\textit{G}} tramite Faust;
            \item Gli agenti in ascolto sul topic utilizzeranno un implementatazione di \textit{processor} per effettuare l'elaborazione delle misurazioni ottenute.
        \end{itemize}
    \end{itemize}
    \item{\textbf{Classe: \textit{faust\_measurement}}}
    \begin{itemize}
    \item\textbf{Attributi:}
        \begin{itemize}
        \item \textbf{timestamp: str} - Il timestamp della misurazione;
        \item \textbf{value: float} - Il valore della misurazione;
        \item \textbf{type: str} - Il tipo della misurazione;
        \item \textbf{latitude: float} - La latitudine della misurazione;
        \item \textbf{longitude: float} - La longitudine della misurazione;
        \item \textbf{ID\_sensore: str} - L'\textit{ID}\textsubscript{\textit{G}} del \textit{sensore}\textsubscript{\textit{G}} che ha effettuato la misurazione;
        \item \textbf{cella: str} - La cella in cui è stata effettuata la misurazione.
    \end{itemize}
    \item\textbf{Note:}
        \begin{itemize}
            \item La classe \textit{faust\_measurement} definita eridatando da \textit{faust.Record} rappresenta un singolo record di misurazione proveniente da un \textit{sensore}\textsubscript{\textit{G}} consumata da un'applicazione Faust;
            \item Faust si occupa automaticamente della conversione dei dati in formato JSON sulla base degli attributi definiti, facilitando la ricezione e la deserializzazione dei dati nei topic Kafka\textsubscript{\textit{G}};
            \item \textbf{In sintesi:}
            Questa classe viene utilizzata in un'applicazione Faust per definire il tipo dei dati attesi nei topic Kafka\textsubscript{\textit{G}} di interesse. I dati provenienti dai sensori, contenenti timestamp, valore, tipo, coordinate geografiche, identificativo del \textit{sensore}\textsubscript{\textit{G}} e eventuale cella di appartenenza, verranno convertiti in oggetti di tipo \textit{faust\_measurement} prima di essere elaborati dall'applicazione.
        \end{itemize}
    \end{itemize}
    \item{\textbf{Classe: \textit{health\_model\_processor\_adapter}}}
    \begin{itemize}
    \item\textbf{Attributi:}
        \begin{itemize}
        \item \textbf{health\_calculator: health\_processor\_buffer} - Un implementazione della classe astratta \textit{health\_processor\_buffer}.
    \end{itemize}
    \item \textbf{Metodi: }
    \begin{itemize}
        \item \textbf{process(misurazione: faust\_measurement): None [public, async]} - Aggiunge la misurazione ad un implementazione della classe astratta \textit{health\_processor\_buffer} adattando un oggetto di tipo \textit{faust\_measurement} alla porta di accesso fornita da \textit{health\_processor\_buffer} per l'elaborazione volta al calcolo del punteggio di salute.
    \end{itemize}
    \item\textbf{Note:}
        \begin{itemize}
            \item La classe implementa l'interfaccia \textit{processor} ed implementa il metodo astratto \textit{process()} per aggiungere/adattare la misurazione del tipo \textit{faust\_measurement} ad un implementatazione di \textit{health\_processor\_buffer};
            \item Rappresenta la componente "Adapter" del \textit{pattern}\textsubscript{\textit{G}} \textit{Object Adapter};
            \item Il fine è adattare la classe astratta \textit{health\_processor\_buffer} o più in generale il modello per il calcolo del punteggio di salute, all'interfaccia \textit{processor} per l'elaborazione delle misurazioni provenienti dai topic Kafka\textsubscript{\textit{G}} e consumate dagli agenti Faust.
        \end{itemize}
    \end{itemize}
\end{itemize}


\subsection{Configurazione Database}
Si è optato per l'utilizzo di ClickHouse per il salvataggio dei dati, le motivazioni sono descritte nella sezione \ref{sec:clickHouse}. In particolare, per ogni sensore dei quali si desidera memorizzare i dati, viene creata una tabella che acquisisce i dati dal relativo topic Kafka.
Le tipologie di sensori cui misurazioni si vogliono trattare nel progetto sono:
\begin{itemize}
    \item Sensori di temperatura;
    \item Sensori di umidità;
    \item Sensori di rilevamento polveri sottili; 
    \item Sensori stato riempimento isole ecologiche;
    \item Sensori di stato occupazione colonnine di ricarica;
    \item Sensori di guasti elettrici;
    \item Sensori del livello dell'acqua.
\end{itemize}

La configurazione del database ClickHouse è stata cruciale nella progettazione, poiché un'adeguata ottimizzazione consente di garantire prestazioni ottimali per un sistema orientato al tempo reale e in grado di gestire analisi su enormi volumi di dati.


\subsubsection{Funzionalità Clickhouse utilizzate}
\paragraph{Materialized Views}
Link alla documentazione: \href{https://clickhouse.com/docs/en/guides/developer/cascading-materialized-views}{https://clickhouse.com/docs/en/guides/developer/cascading-materialized-views} (Consultato 25/03/2024).\newline
Le Materialized Views in ClickHouse sono un meccanismo potente per migliorare le prestazioni delle query e semplificare l'accesso ai dati. Funzionano mantenendo una copia fisica dei risultati di una query di selezione, che viene quindi memorizzata su disco. Questa copia è aggiornata periodicamente in base ai dati sottostanti.

\paragraph{Utilizzi Principali delle Materialized Views}
\begin{itemize}
    \item \textbf{Calcolo aggregazioni e popolamento tabelle}:Spesso le delle materialized Views sono state utilizzate per calcolare aggregazioni su dati e quindi popolare altre tabelle con i risultati aggregati. Ad esempio, nel caso specifico in cui una Materialized View calcola la media delle temperature per ogni sensore ogni secondo, i risultati di questa vista possono essere utilizzati per popolare una tabella principale contenente i dati di temperatura aggregati, aggiornando i valori di temperatura medi per ogni sensore ogni secondo;
    \item \textbf{Ottimizzazione delle Prestazioni}: memorizzando i risultati di una query complessa, le Materialized Views consentono di eseguire rapidamente le Query successive senza dover ricalcolare i dati ogni volta. Ciò è particolarmente utile in applicazioni che richiedono interrogazioni frequenti su grandi volumi di dati;
    \item \textbf{Decomposizione delle \textit{Query} Complesse}: le Materialized Views consentono di decomporre query complesse in passaggi più semplici e riutilizzabili, migliorando la leggibilità del codice e semplificando lo sviluppo e la manutenzione delle query.
\end{itemize}

Nel progetto le materialized view sono fondamentali per spostare automaticamente i dati da Kafka alla tabella di destinazione.
\begin{figure}[H]
  \centering
  \includegraphics[width=1\textwidth]{../Images/SpecificaTecnica/enginePipeline.jpg}
  \caption{Data pipeline - ClickHouse}
  \label{fig:datapip}
\end{figure}

\paragraph{MergeTree}\label{sec:MergeTree}
Link alla documentazione: \href{https://clickhouse.com/docs/en/engines/table-engines/mergetree-family/mergetree#mergetree}{ClickHouse - MergeTree} (Consultato 25/03/2024).\newline
MergeTree è uno dei motori di tabella più potenti e utilizzati in ClickHouse, noto per la sua capacità di gestire e memorizzare grandi volumi di dati in modo efficiente. È una scelta ideale per applicazioni che richiedono l'archiviazione e l'analisi di dati cronologicamente ordinati, come i dati di log o di monitoraggio. L'architettura di MergeTree organizza i dati in parti, ciascuna contenente una serie di punti dati ordinati cronologicamente. Questa organizzazione ottimizzata consente di eseguire rapidamente le query che richiedono l'accesso a dati specifici all'interno di un intervallo di tempo definito, garantendo prestazioni elevate anche su grandi dataset. Oltre alla gestione efficiente dei dati, MergeTree supporta funzionalità avanzate come la compressione dei dati e la gestione automatica delle partizioni. Queste caratteristiche consentono di ottimizzare ulteriormente le prestazioni e la gestione complessiva dei dati, rendendo MergeTree una scelta affidabile per una vasta gamma di scenari di utilizzo in ClickHouse.



\paragraph{Time To Live in ClickHouse} \label{sec:RollupTTL}
Link alla documentazione: \href{https://clickhouse.com/docs/en/guides/developer/ttl#implementing-a-rollup}{https://clickhouse.com/docs/en/guides/developer/ttl\#implementing-a-rollup} \newline
In ClickHouse, la funzionalità TTL (Time To Live) è un elemento chiave per gestire grandi volumi di dati in modo efficiente e garantire la pulizia automatica di informazioni obsolete o non più rilevanti. \\
Quando si specifica il motore Rollup per definire una tabella in ClickHouse, si abilita la creazione di tabelle che supportano il TTL. Questo consente di impostare un periodo temporale dopo il quale i dati saranno eliminati automaticamente dalla tabella. La struttura a Rollup organizza i dati in parti, ciascuna contenente una serie di punti dati ordinati cronologicamente. Il TTL può essere configurato per ciascuna parte dei dati, offrendo un controllo preciso sulla conservazione delle informazioni nel tempo. Questa flessibilità è particolarmente utile per applicazioni che richiedono la conservazione di dati storici per un periodo limitato, come ad esempio i dati di log o di monitoraggio. \newline
Un esempio di come potrebbe può venire utilizzato il motore Rollup per il TTL in ClickHouse è il seguente:
\begin{verbatim}
    TTL toDateTime(timestamp) + INTERVAL 1 MONTH
\end{verbatim}
L'uso del TTL di tipo Rollup in questo contesto è cruciale per garantire che la tabella rimanga efficiente e gestibile nel tempo, eliminando automaticamente i dati più vecchi e non più necessari dopo un periodo di tempo specificato. Questo aiuta a ottimizzare le prestazioni complessive del sistema e a gestire in modo efficiente i grandi volumi di dati accumulati nel tempo.


\paragraph{Partition}\label{sec:Partition}
Link alla documentazione: \href{https://clickhouse.com/docs/en/engines/table-engines/mergetree-family/mergetree#partition-by}{ClickHouse - Partitioning} (Consultato 25/03/2024).\\
Le partizioni sono una funzionalità fondamentale di ClickHouse che consente di organizzare in modo efficiente e gestire grandi volumi di dati. Questa caratteristica permette di suddividere i dati in gruppi logici in base a criteri specifici, come il valore di una colonna o un intervallo di tempo. Grazie a questa organizzazione ottimizzata, le query che richiedono l'accesso a dati specifici all'interno di una partizione possono essere eseguite rapidamente, garantendo prestazioni elevate anche su dataset di grandi dimensioni.\\
L'utilizzo delle partizioni nel nostro contesto viene giustificato dall'utilizzo di un TTL (Time To Live), infatti l'utilizzo combinato di queste due funzionalità consente:
\begin{itemize}
    \item Una gestione efficace dei dati nel tempo;
    \item Migliori prestazioni del sistema;
    \item Una semplificazione nella manutenzione del database.
\end{itemize}
Il partizionamento basato sul timestamp è una pratica comune in ClickHouse, poiché consente di organizzare i dati in partizioni in base al periodo temporale, ad esempio mensilmente. Questo approccio ottimizza l'archiviazione e facilita l'analisi dei dati di serie temporali, come le temperature o i log di eventi. Grazie a questa struttura, le query che coinvolgono dati all'interno di specifici intervalli temporali diventano più efficienti, consentendo un accesso rapido e una migliore analisi dei dati.




    
\paragraph{Projection}\label{sec:projections}
Link alla documentazione: \href{https://clickhouse.com/docs/en/sql-reference/statements/alter/projection}{https://clickhouse.com/docs/en/sql-reference/statements/alter/projection}\newline
Le proiezioni memorizzano i dati in un formato che ottimizza l'esecuzione delle \textit{Query}, questa caratteristica è utile per:

\begin{itemize}
    \item Eseguire \textit{Query} su una colonna che non fa parte della chiave primaria;
    \item Pre-aggregare colonne, riducendo sia i calcoli che l'I/O.
\end{itemize}

Puoi definire una o più proiezioni per una tabella e durante l'analisi della \textit{Query} la proiezione con meno dati da esaminare sarà selezionata da ClickHouse senza modificare la \textit{Query} fornita dall'utente.
\\
In generale l'introduzione delle PROJECTIONS produce risultati di notevole importanza, come illustrato di seguito. Consideriamo una tipica query eseguita per l'analisi tramite Grafana:
    
    \begin{lstlisting}[caption={Query tipica - Grafana}, captionpos=b]
      SELECT ID_sensore, avgMerge(value) AS value, timestamp
      FROM innovacity.temperatures
      WHERE (cella IN ('Arcella')) AND ((timestamp >= toDateTime64(1708338633507 / 1000, 3)) AND (timestamp <= toDateTime64(1708338933507 / 1000, 3) + INTERVAL 1 DAY))
      GROUP BY timestamp, ID_sensore
      HAVING (value >= -100) AND (value <= 100)

      --Query id: 48635435-9b35-4727-b580-9e33a9db92d4
    \end{lstlisting}

    \begin{figure}[H]
        \centering
        \includegraphics[width=1\textwidth]{../Images/SpecificaTecnica/ProjectionQuery.jpg}
        \caption{Query tipica - Grafana}
        \label{fig:ProjectionsQuery}
      \end{figure}
      Senza l'utilizzo delle PROJECTIONS, il risultato ottenuto è il seguente:
    \begin{figure}[H]
        \centering
        \includegraphics[width=0.9\textwidth]{../Images/SpecificaTecnica/SenzaProectionResult.jpg}
        \caption{Query tipica risultato senza projections}
        \label{fig:ProjectionsQueryWthout}
      \end{figure}
      ovvero sono state processate per ottenere il risultato della query \textbf{16,38} migliaia di righe. Invece in seguito all’aggiunta delle PROJECTIONS:

      \begin{figure}[H]
        \centering
        \includegraphics[width=0.9\textwidth]{../Images/SpecificaTecnica/ConProjectionRisultato.jpg}
        \caption{Query tipica risultato con projections}
        \label{fig:ProjectionsQueryWith}
      \end{figure}   
  Sono state elaborate approssimativamente \textbf{8,19} migliaia di righe per ottenere il risultato della query, circa la metà rispetto al conteggio precedente, evidenziando un miglioramento significativo. Inoltre, mediante un'interrogazione specifica è possibile confermare che le PROJECTIONS sono state effettivamente impiegate per generare il risultato della query in questione.
\begin{figure}[H]
    \centering
    \includegraphics[width=1\textwidth]{../Images/SpecificaTecnica/ProjectionUsedByClickHouse.jpg}
    \caption{Uso della Projection}
    \label{fig:ProjectionsUsed}
\end{figure}

Considerando un'altra query eseguita dall'applicativo, che calcola la media globale di \textbf{170.000} misurazioni di temperatura, è possibile riconoscere i benefici derivanti dall'utilizzo delle PROJECTIONS. Alla conclusione dell'analisi, è evidente anche il loro effettivo impiego nel calcolo del risultato. Grazie all'adozione delle PROJECTIONS, si ottiene:
\begin{figure}[H]
    \centering
    \includegraphics[width=1\textwidth]{../Images/SpecificaTecnica/query2ProjectionsWith.jpg}
    \caption{Query esempio Projection 2 - ClickHouse}
    \label{fig:with2proj}
  \end{figure}
Ovvero il totale di righe processate per ottenere il risultato è di \textbf{49,95 migliaia} con \textbf{0,07 secondi} di tempo utilizzati.
Si puo notare invece la differenza delle righe processate una volta rimossa la \textit{PROJECTIONS}:
\begin{figure}[H]
    \centering
    \includegraphics[width=1\textwidth]{../Images/SpecificaTecnica/query2ProjectionsWithout.jpg}
    \caption{Query esempio senza Projection 2 - ClickHouse}
    \label{fig:without2proj}
  \end{figure}

 Il totale di righe processate per ottenere il risultato è ora di \textbf{170,09 migliaia}, ovvero la totalità delle righe presenti nella tabella, con \textbf{0,09 secondi} di tempo utilizzati.

\paragraph*{Utilizzo dello spazio su disco}
\textbf{Attenzione:} le proiezioni creeranno internamente una nuova tabella nascosta, ciò significa che saranno necessari più I/O e spazio su disco. Ad esempio, se la proiezione ha definito una chiave primaria diversa, tutti i dati dalla tabella originale verranno duplicati.

\subsubsection{Integrazione tramite Kafka Engine in ClickHouse}\label{sec:kafka_engine}
ClickHouse supporta l'integrazione con Kafka tramite Kafka Engine, permettendo la lettura dei dati da un topic Kafka e il loro salvataggio in una tabella ClickHouse. Tale funzionalità riveste un'importanza notevole per applicazioni che richiedono l'elaborazione in tempo reale di dati provenienti da fonti esterne, una necessità frequente nel contesto del monitoraggio urbano. L'integrazione con Kafka consente l'acquisizione e la memorizzazione efficiente dei dati, garantendo prestazioni elevate anche su grandi volumi di dati.\\
Kafka Engine è progettato per il recupero di dati una sola volta. Ciò significa che una volta che i dati vengono interrogati da una tabella Kafka, vengono considerati consumati dalla coda. Pertanto, non si dovrebbero mai selezionare dati direttamente da una tabella di Kafka Engine, ma utilizzare invece una vista materializzata. Una vista materializzata viene attivata una volta che i dati sono disponibili in una tabella di Kafka Engine. Automaticamente sposta i dati da una tabella Kafka a una tabella di tipo MergeTree o Distributed. Quindi, sono necessarie almeno 3 tabelle:
\begin{itemize}
  \item La tabella di origine del motore Kafka;
  \item La tabella di destinazione (famiglia MergeTree o distribuita);
  \item Vista materializzata per spostare i dati;
\end{itemize}
\begin{figure}[H]
  \centering
  \includegraphics[width=.7\textwidth]{../Images/SpecificaTecnica/kafka_engine_architecture.png}
  \caption{Architettura di Kafka Engine in ClickHouse}
  \label{fig:Architettura_kafka_engine}
\end{figure}

\subsubsection{Trasferimento dati tramite Materialized View} \label{sec:materializedView}
Una materialized view funge da ponte tra la fonte dei dati (Kafka Engine) e la destinazione dei dati (MergeTree). Quando nuovi dati vengono scritti nella tabella Kafka Engine, la materialized view viene attivata automaticamente.\\
La materialized view esegue una query sulla tabella Kafka Engine per selezionare i dati più recenti. Una volta selezionati, questi dati vengono inseriti nella tabella di destinazione (ad esempio, una tabella MergeTree). Questo processo avviene in modo automatico e immediato, senza bisogno di intervento manuale.\\
In pratica, la materialized view si assicura che la tabella di destinazione sia sempre aggiornata con i dati più recenti presenti nella tabella Kafka Engine. Questo offre numerosi vantaggi:
\begin{itemize}
  \item \textbf{Automatizzazione del processo}: Non è necessario eseguire manualmente operazioni di trasferimento dati da una tabella all'altra. La materialized view si occupa di tutto in modo automatico;
  \item \textbf{Efficienza}: Il trasferimento dei dati avviene in tempo reale, garantendo che la tabella di destinazione sia sempre allineata con la fonte dei dati senza ritardi;
  \item \textbf{Ottimizzazione delle risorse}: Il processo di trasferimento dei dati è gestito in modo efficiente, utilizzando al meglio le risorse disponibili e garantendo prestazioni elevate.
\end{itemize}
Nel contesto specifico, le materialized view sono responsabili di eseguire controlli sui dati, come ad esempio la verifica della loro correttezza ed affidabilità nel contesto di utilizzo, prima di inserirli nella tabella di destinazione. Questo processo assicura che i dati siano sempre affidabili e pronti per l'analisi, senza la necessità di ulteriori operazioni di pulizia o preparazione.\\
Per esempio, nel caso dei dati di umidità raccolti da sensori in un'area urbana, la materialized view potrebbe eseguire controlli per assicurarsi che i valori rientrino all'interno di un intervallo plausibile e che non ci siano discrepanze improbabili. Ciò garantirebbe che i dati di umidità inseriti nella tabella di destinazione siano accurati e affidabili per l'analisi meteorologica o ambientale.


\subsubsection{Tabella di origine di Kafka Engine per un sensore generico}
Le tabelle del database impiegate per registrare le misurazioni di ciascuna tipologia di sensore presentano una configurazione sostanzialmente simile, differenziandosi principalmente per il tipo di dato della colonna relativa alla misurazione e per il \textit{topic} di riferimento utilizzato per ottenere le misurazioni.
Nello specifico per ogni sensore si avrà la seguente tabella Clickhouse:
\begin{figure}[H]
    \centering
    \includegraphics[width=.6\textwidth]{../Images/SpecificaTecnica/sensorType_kafka.PNG}
    \caption{Tabella sensore generico per il reperimento da kafka - ClickHouse}
    \label{fig:Reperimento_kafka_clickhouse}
  \end{figure}

    La tabella è configurata con il motore di storage \textit{Kafka}, il che significa che i dati verranno letti da un \textit{topic Kafka}. 

    I campi sono:
    \begin{itemize}
        \item \textbf{ID\_sensore}: un campo di tipo \textit{String} che identifica univocamente il sensore che ha effettuato la misurazione;
        \item \textbf{cella}: un campo di tipo \textit{String} che rappresenta la cella della città in cui è stata effettuata la misurazione;
        \item \textbf{value}: un campo di tipo variabile a seconda del tipo di misurazione che contiene il valore della temperatura;
        \item \textbf{timestamp}: campo di tipo \textit{DATETIME64} che rappresenta il timestamp della misurazione della temperatura;
        \item \textbf{latitude}: un campo di tipo \textit{Float64} che rappresenta la latitudine del luogo dove è stata effettuata la misurazione;
        \item \textbf{longitude}: un campo di tipo \textit{Float64} che rappresenta la longitudine del luogo dove è stata effettuata la misurazione.
    \end{itemize}

    Mentre i parametri esposti racchiusi da parentesi graffe variano per ogni tipolgia di sensore correlato alla misurazione e sono:
    \begin{itemize}
        \item \textbf{tipologiaSensore}: viene sostituito con la tipologia del sensore che effettua le misurazioni salvate nella tabella; (ex. temperatures)
        \item \textbf{TipoDatoMisurazione}: viene sostituito con il tipo del dato che rappresenta la misurazione (ex. Float32, UInt8);
        \item \textbf{IndirizzoServerKafka}: specifica l'indirizzo del server Kafka.
        Nel nostro caso il server Kafka è in esecuzione su un container \textit{Docker} raggiungibile tramite l'indirizzo:
         \textit{'kafka:9092'};
        \item \textbf{topicTipologiaSensore}: specifica il nome del topic Kafka da cui leggere i dati (ex.temperature). Accetta anche liste di topic Kafka separati da virgole.
        \item \textbf{ConsumerGroupKafka}: specifica il nome del consumer group Kafka che verrà utilizzato per leggere i messaggi dal topic \textit{Kafka} denominato 'temperature'.
        Un consumer group in \textit{Kafka} è un gruppo di consumatori che lavorano insieme per consumare i messaggi da uno o più topic. Ogni messaggio inviato a un \textit{topic Kafka} può essere consumato da uno dei consumatori nel gruppo. I consumer all'interno di uno stesso gruppo condividono l'elaborazione dei messaggi all'interno dei topic: ogni messaggio viene elaborato da uno e un solo consumatore all'interno del gruppo. Nel nostro caso sarà sempre '\textit{CG\_Clickhouse\_1}' per indicare il servizio di salvataggio \textit{Clickhouse}.
        \item \textbf{FormatoDatiTopicKafka}: specifica il formato dei dati nel \textit{topic Kafka}. Nel nostro caso, i dati sono nel formato JSONEachRow, che è un formato di serializzazione JSON di \textit{ClickHouse} che consente di scrivere o leggere record JSON separati da una riga. Quindi avremo che <<FormatoDatiTopicKafka>> = \textit{JSONEachRow}.
        \item \textbf{KafkaSkipBrokenMessages}:
        \begin{itemize}
          \item è un'opzione di configurazione utilizzata nel motore Kafka di ClickHouse. Determina il comportamento del motore quando incontra messaggi Kafka considerati "corrotti" o non processabili.
          Un messaggio Kafka può essere considerato corrotto per diversi motivi, tra cui:
          \begin{itemize}
            \item \textbf{Formato non valido}: Il messaggio potrebbe avere un formato JSON o Avro non valido, impedendo a ClickHouse di decodificarlo correttamente.
            \item \textbf{Dati mancanti}: Il messaggio potrebbe contenere dati mancanti o incompleti, violando lo schema previsto.
            \item \textbf{Errori di codifica}: Il messaggio potrebbe avere errori di codifica che impediscono la lettura dei dati.
          \end{itemize}
         \item Per impostazione predefinita, \textit{kafka\_skip\_broken\_messages} è impostato su 0. Ciò significa che ClickHouse interrompe l'elaborazione del flusso di dati da Kafka e registra un errore quando incontra un messaggio corrotto.
        \item Puoi configurare \textit{kafka\_skip\_broken\_messages} su un valore diverso da zero per modificare il comportamento. Il valore rappresenta il numero massimo di messaggi corrotti consecutivi per blocco, considerato nel contesto di  \textit{kafka\_max\_block\_size}, che ClickHouse ignorerà prima di interrompere l'elaborazione.
        \item Bisogna anche ricordare che per come è stato progettato il sistema i messaggi corrotti vengono scartati "alla fonte" dallo Schema Registry di Kafka.
        \item Nel nostro caso vogliamo che ogni messaggio malformato nel blocco venga ignorato.
        \end{itemize}
        \item \textbf{input\_format\_skip\_unknown\_fields}:  è un'impostazione utilizzata con alcuni formati di input di ClickHouse, compreso quello da noi utilizzato \textit{JSONEachRow} per specificare come gestire i dati in entrata che contengono colonne sconosciute alla tabella di destinazione. Impostando \textit{input\_format\_skip\_unknown\_fields} su 1, ClickHouse ignorerà le colonne sconosciute nei dati in entrata e importerà solo le colonne che corrispondono alle colonne della tabella di destinazione. Questo è utile quando si desidera importare solo una parte dei dati in entrata, ignorando le colonne non necessarie o non rilevanti.
        Nel nostro caso l'impostazione di default è quella richiesta.
    \end{itemize}

    
    \subsubsection{Misurazioni temperatura} \label{sec:tab_temperatures}
    Di seguito viene presentata una configurazione dettagliata per l'archiviazione delle misurazioni di temperatura. Tale configurazione si applica alla tabella 'temperatures\_kafka', progettata per acquisire dati da un topic Kafka. La tabella è strutturata per includere l'ID del sensore (String), la posizione della cella (String), il valore della temperatura misurato (Float32), il timestamp della misurazione (DATETIME64), la latitudine (Float64) e la longitudine (Float64) del sensore. Ogni campo è definito con un tipo di dato specifico al fine di garantire la precisione e l'integrità dei dati.
    
    \begin{figure}[H]
        \centering
        \includegraphics[width=1\textwidth]{../Images/SpecificaTecnica/temperatures.png}
        \caption{Tabella temperatures\_kafka e temperatures}
        \label{fig:temperatures}
      \end{figure}
    
    La tabella 'temperatures\_kafka' è essenziale nel contesto dell'architettura dei dati, poiché funge da tramite tra un topic Kafka e il sistema di gestione dei dati ClickHouse. Questa tabella agisce come un'interfaccia di origine, trasformando i flussi di dati provenienti dal topic Kafka in un formato comprensibile per ClickHouse. Successivamente, una Materialized View, in questo caso 'mv\_temperatures', opera su questa tabella per trasferire i dati ottenuti verso la tabella di destinazione 'temperatures' come spiegato in \ref{sec:materializedView}.
    
    \paragraph{Projections per misurazioni di temperatura} \label{sec:temp_projections}
    Durante la fase di progettazione, è stata dedicata particolare attenzione all'utilizzo delle tabelle precedentemente descritte e alle richieste che verranno formulate su di esse. È emerso che, considerando il requisito di suddividere la città in una serie di celle e specificare la cella di origine della misurazione, la filtrazione delle misurazioni per celle diventerà una richiesta frequente al database. Di conseguenza, si è optato per l'utilizzo delle PROJECTIONS, le quali sono dettagliatamente descritte nella sezione \ref{sec:projections}.
    \vspace{0,3cm}
    \begin{lstlisting}[caption={Esempio di proiezione e materializzazione in una tabella}, captionpos=b]
      --Projection per tabella temperatures
      ALTER TABLE innovacity.temperatures ADD PROJECTION tmp_sensor_cell_projection (SELECT * ORDER BY cella);
      ALTER TABLE innovacity.temperatures MATERIALIZE PROJECTION tmp_sensor_cell_projection;
  \end{lstlisting}
    \vspace{0,3cm}
    La proiezione ci consentirà di effettuare rapidamente filtraggi basati sulle celle, anche se tale attributo non è definito come \textit{PRIMARY\_KEY} nella tabella originale.

\subsubsection{Misurazioni umidità}
Le considerazioni relative al salvataggio delle misurazioni di umidità coincidono con quelle espresse nella sezione \ref{sec:temp_projections} riguardo alle misurazioni di temperatura.
In questa situazione, dove le misure riguardano l’umidità, la tabella di destinazione ClickHouse è nominata ‘humidity’:

\begin{figure}[H]
    \centering
    \includegraphics[width=1\textwidth]{../Images/SpecificaTecnica/humidity.png}
    \caption{Tabella humidity\_kafka e humidity}
    \label{fig:humidity_tables}
  \end{figure}

\paragraph{Projections per misurazioni di umidità} 
Dopo aver considerato le stesse argomentazioni presentate nella sezione \ref{sec:tab_temperatures} riguardanti le misurazioni di temperatura, abbiamo deciso di estendere l'utilizzo delle PROJECTION anche alle misurazioni di umidità. I vantaggi ottenuti risultano essere simili a quelli evidenziati per le misurazioni di temperatura, come descritto nella stessa sezione. A seguire, vengono illustrate le configurazioni delle PROJECTION relative alle tabelle delle misurazioni di umidità:

\begin{lstlisting}
    --Projection per tabella humidity
    ALTER TABLE innovacity.humidity ADD PROJECTION umd_sensor_cell_projection (SELECT * ORDER BY cella);
    ALTER TABLE innovacity.humidity MATERIALIZE PROJECTION umd_sensor_cell_projection;
\end{lstlisting}


\subsubsection{Misurazioni di polveri sottili}Le considerazioni concernenti l'archiviazione delle misurazioni di polveri sottili corrispondono a quelle espresse nella sezione \ref{sec:temp_projections} in merito alle misurazioni di temperatura.

\begin{figure}[H]
    \centering
    \includegraphics[width=1\textwidth]{../Images/SpecificaTecnica/dust_PM10.png}
    \caption{Tabella dustPM10\_kafka e dustPM10}
    \label{fig:dust_table}
  \end{figure}

\paragraph{Projections per misurazioni di polveri sottili} 
Dopo aver considerato le stesse argomentazioni presentate nella sezione \ref{sec:tab_temperatures} riguardanti le misurazioni di temperatura, abbiamo deciso di estendere l'utilizzo delle PROJECTION anche alle misurazioni di polveri sottili. I vantaggi ottenuti risultano essere simili a quelli evidenziati per le misurazioni di temperatura, come descritto nella stessa sezione. A seguire, vengono illustrate le configurazioni delle PROJECTION relative alle tabelle delle misurazioni di polveri sottili:

\begin{lstlisting}
  --Projection per tabella dust_PM10
  ALTER TABLE innovacity.dust_PM10 ADD PROJECTION dust_sensor_cell_projection (SELECT * ORDER BY cella);
  ALTER TABLE innovacity.dust_PM10 MATERIALIZE PROJECTION dust_sensor_cell_projection;
\end{lstlisting}

\subsubsection{Misurazioni guasti elettrici} \label{sec:tab_guasti}
Segue una dettagliata configurazione per l'archiviazione delle misurazioni relative ai guasti elettrici, applicabile alla tabella 'electricalFault\_kafka' progettata per acquisire dati dal topic Kafka. La struttura della tabella include l'ID del sensore (String), la posizione della cella (String), il valore misurato (UInt8), il timestamp della misurazione (DATETIME64), la latitudine (Float64) e la longitudine (Float64) del sensore, ciascuno definito con un tipo di dato specifico per garantire la precisione e l'integrità dei dati.

\begin{figure}[H]
    \centering
    \includegraphics[width=1\textwidth]{../Images/SpecificaTecnica/electricalFault.png}
    \caption{Tabella electricalFault\_kafka e electricalFault}
    \label{fig:electricalFault_tables}
  \end{figure}

Le considerazioni concernenti l'archiviazione delle misurazioni di guasti elettrici corrispondono a quelle espresse nella sezione \ref{sec:temp_projections} in merito alle misurazioni di temperatura.

\paragraph{Projections per misurazioni di guasti elettrici} 
Dopo aver considerato le stesse argomentazioni presentate nella sezione \ref{sec:tab_temperatures} riguardanti le misurazioni di temperatura, abbiamo deciso di estendere l'utilizzo delle PROJECTION anche alle misurazioni di guasti elettrici. I vantaggi ottenuti risultano essere simili a quelli evidenziati per le misurazioni di temperatura, come descritto nella stessa sezione. A seguire, vengono illustrate le configurazioni delle PROJECTION relative alle tabelle delle misurazioni di guasti elettrici:

\begin{lstlisting}
  --Projection per tabella electricalFault
  ALTER TABLE innovacity.electricalFault ADD PROJECTION elctF_sensor_cell_projection (SELECT * ORDER BY cella);
  ALTER TABLE innovacity.electricalFault MATERIALIZE PROJECTION elctF_sensor_cell_projection;
\end{lstlisting}

\subsubsection{Misurazioni stazioni di ricarica} Le considerazioni concernenti l'archiviazione delle misurazioni delle stazioni di ricarica corrispondono a quelle espresse nella sezione \ref{sec:tab_guasti} in merito alle misurazioni guasti elettrici.

\begin{figure}[H]
    \centering
    \includegraphics[width=1\textwidth]{../Images/SpecificaTecnica/chargingStations.png}
    \caption{Tabella chargingStation\_kafka e chargingStation}
    \label{fig:chargingStation_tables}
  \end{figure}

\paragraph{Projections per misurazioni delle stazioni di ricarica}
Dopo aver considerato le stesse argomentazioni presentate nella sezione \ref{sec:tab_temperatures} riguardanti le misurazioni di temperatura, abbiamo deciso di estendere l'utilizzo delle PROJECTION anche alle misurazioni delle stazioni di ricarica. I vantaggi ottenuti risultano essere simili a quelli evidenziati per le misurazioni di temperatura, come descritto nella stessa sezione. A seguire, vengono illustrate le configurazioni delle PROJECTION relative alle tabelle delle misurazioni delle stazioni di ricarica:

\begin{lstlisting}
  --Projection per tabella chargingStations
  ALTER TABLE innovacity.chargingStations ADD PROJECTION chS_sensor_cell_projection (SELECT * ORDER BY cella);
  ALTER TABLE innovacity.chargingStations MATERIALIZE PROJECTION chS_sensor_cell_projection;
\end{lstlisting}

\subsubsection{Misurazioni isole ecologiche}
Le considerazioni concernenti l'archiviazione delle misurazioni delle isole ecologiche corrispondono a quelle espresse nella sezione \ref{sec:tab_temperatures} in merito alle misurazioni di temperatura.

\begin{figure}[H]
  \centering
  \includegraphics[width=1\textwidth]{../Images/SpecificaTecnica/ecoIslands.png}
  \caption{Tabella ecoIslands\_kafka e ecoIslands}
  \label{fig:ecoIslands_tables}
\end{figure}

\paragraph{Projections per misurazioni delle isole ecologiche} 
Dopo aver considerato le stesse argomentazioni presentate nella sezione \ref{sec:tab_temperatures} riguardanti le misurazioni di temperatura, abbiamo deciso di estendere l'utilizzo delle PROJECTION anche alle misurazioni delle isole ecologiche. I vantaggi ottenuti risultano essere simili a quelli evidenziati per le misurazioni di temperatura, come descritto nella stessa sezione. A seguire, vengono illustrate le configurazioni delle PROJECTION relative alle tabelle delle misurazioni delle isole ecologiche:

\begin{lstlisting}
  --Projection per tabella ecoIslands
  ALTER TABLE innovacity.ecoIslands ADD PROJECTION umd_sensor_cell_projection (SELECT * ORDER BY cella);
  ALTER TABLE innovacity.ecoIslands MATERIALIZE PROJECTION umd_sensor_cell_projection;
\end{lstlisting}

\subsubsection{Misurazioni sensori livello dell’acqua}
Le considerazioni concernenti l'archiviazione delle misurazioni dei sensori di livello dell'acqua corrispondono a quelle espresse nella sezione \ref{sec:tab_guasti} in merito alle misurazioni guasti elettrici.

\begin{figure}[H]
  \centering
  \includegraphics[width=1\textwidth]{../Images/SpecificaTecnica/waterPresence.png}
  \caption{Tabella waterPresence\_kafka e waterPresence}
  \label{fig:waterPresence_tables}
\end{figure}

\paragraph{Projections per misurazioni del livello dell'acqua} 
Dopo aver considerato le stesse argomentazioni presentate nella sezione \ref{sec:tab_temperatures} riguardanti le misurazioni di temperatura, abbiamo deciso di estendere l'utilizzo delle PROJECTION anche alle misurazioni del livello d'acqua. I vantaggi ottenuti risultano essere simili a quelli evidenziati per le misurazioni di temperatura, come descritto nella stessa sezione. A seguire, vengono illustrate le configurazioni delle PROJECTION relative alle tabelle delle misurazioni del livello d'acqua:
\begin{lstlisting}
  --Projection per tabella waterPresence
  ALTER TABLE innovacity.waterPresence ADD PROJECTION waPr_sensor_cell_projection (SELECT * ORDER BY cella);
  ALTER TABLE innovacity.waterPresence MATERIALIZE PROJECTION waPr_sensor_cell_projection;
\end{lstlisting}
\subsection{Grafana}
Grafana è un software open source per la visualizzazione e l'analisi dei dati progettato per interagire con vari data-source, tra cui Clickhouse. Grafana offre un'interfaccia utente intuitiva e flessibile che consente di creare e condividere dashboard personalizzate per monitorare i dati di diversa natura in tempo reale.

\subsubsection{Utenti}
L'accesso a Grafana è vincolato a due utenze e non permette uteriori registrazioni per l'accesso alla piattaforma di monitoraggio.
\begin{itemize}
    \item \textbf{Amministratore:} 
    \begin{itemize}
        \item Accesso riservato all'amministratore di sistema per permettere manutenzione e modifiche alle impostazioni sensibili della piattaforma;
        \item Non accessibile in produzione;
        \item Credenziali:
        \begin{itemize}
            \item \textbf{Username:} admin
            \item \textbf{Password:} admin
        \end{itemize} 
    \end{itemize}
    
    \item \textbf{User:} 
    \begin{itemize}
        \item Accesso riservato alle autorità locali per la visualizzazione e il monitoraggio dei dati;
        \item Credenziali:
        \begin{itemize}
            \item \textbf{Username:} user
            \item \textbf{Password:} user
        \end{itemize} 
    \end{itemize}
\end{itemize}

\subsubsection{Dashboards}
Per soddisfare tutti i requisiti definiti in \textit{Analisi dei requisiti v2.0.0 - Sez. Req. Funzionali} sono state create due dashboard:
\begin{itemize}
    \item \textbf{Dashboard Principale:} Questa dashboard fornisce una visualizzazione chiara e intuitiva delle misurazioni provenienti da tutti i sensori, di tutte le tipologie, distribuiti nell'area urbana. La dashboard include una mappa interattiva della città che mostra la posizione geografica di ciascun sensore e la relativa ultima misurazione. Inoltre, viene presentato il punteggio di salute della città o di celle specifiche;
    \item \textbf{Dashboard dedicata:} Mostra le misurazioni di una specifica tipologia di sensore selezionata dall'utente in modo più dettagliato e permette di effettuare le attività di filtraggio e aggregazione definite in \textit{Analisi dei requisiti v2.0.0 - Sez. Req. Funzionali}.
\end{itemize}



\paragraph*{Dashboard Principale - Progettazione in dettaglio}
La dashboard principale è suddivisa in righe comprimibili. Di seguito, vengono elencate le informazioni da visualizzare per ciascuna riga, nell’ordine dato dall'enumerazione, \textbf{ l'intervallo di tempo scelto dall'utente tramite interfaccia di default Grafana verrà chiamato \textit{UserInterval}.}
\begin{enumerate}
    \item Titolo riga: \textbf{City manager}
    \begin{enumerate}
        \item Pannello per la scelta delle celle della città di cui si intende visualizzare le misurazioni;
        \item Mappa interattiva della città, presenta i sensori interni alle celle selezionate;
        \item Pannello per la visualizzazione del punteggio di salute relativo alla celle selezionate;
        \item Pannello per la visualizzazione dello stato degli alert relativi alle celle selezionate.
    \end{enumerate}
    \item Titolo riga: \textbf{Temperatura}
    \begin{enumerate}
        \item Pannello per la scelta dei sensori di temperatura da analizzare (vengono proposti per la scelta solo i sensori interni alle celle selezionate per l'analisi).\\
        \textbf{Verrà chiamato \textit{tmps} l'insieme dei sensori di temperatura scelti nel pannello appena esposto e presenti nelle celle selezionate nel pannello apposito};
        \item Pannello con vista time-series delle misurazioni di temperatura. Vengono presentate le misurazioni dei sensori \textit{tmps} in \textit{UserInterval};
        \item Pannello con vista della media delle misurazioni dei sensori \textit{tmps} in \textit{UserInterval}.
    \end{enumerate}
    \item Titolo riga: \textbf{Umidità}
    \begin{enumerate}
        \item Pannello per la scelta dei sensori di umidità da analizzare (vengono proposti per la scelta solo i sensori interni alle celle selezionate per l'analisi).\\
        \textbf{Verrà chiamato \textit{umds} l'insieme dei sensori di umidità scelti nel pannello appena esposto e presenti nelle celle selezionate nel pannello apposito};
        \item Pannello con vista time-series delle misurazioni di umidità. Vengono presentate le misurazioni dei sensori \textit{umds} in \textit{UserInterval};
        \item Pannello con vista della media delle misurazioni dei sensori \textit{umds} in \textit{UserInterval}.
    \end{enumerate}
    \item Titolo riga: \textbf{Polveri sottili}
    \begin{enumerate}
        \item Pannello per la scelta dei sensori di polveri sottili da analizzare (vengono proposti per la scelta solo i sensori interni alle celle selezionate per l'analisi).\\
        \textbf{Verrà chiamato \textit{pm\_sensors} l'insieme dei sensori di polveri sottili scelti nel pannello appena esposto e presenti nelle celle selezionate nel pannello apposito};
        \item Pannello con vista time-series delle misurazioni di polveri sottili. Vengono presentate le misurazioni dei sensori \textit{pm\_sensors} in \textit{UserInterval};
        \item Pannello con vista della media delle misurazioni dei sensori \textit{pm\_sensors} in \textit{UserInterval}.
    \end{enumerate}
    \item Titolo riga: \textbf{Isole ecologiche}
    \begin{enumerate}
        \item Pannello per la scelta delle isole ecologiche da analizzare (vengono proposte per la scelta solo le isole ecologiche interne alle celle selezionate per l'analisi).\\
        \textbf{Verrà chiamato \textit{islands} l'insieme delle isole ecologiche scelte nel pannello appena esposto};
        \item Pannello con vista time-series delle misurazioni delle isole ecologiche. Vengono presentate le misurazioni delle isole ecologiche \textit{islands} in \textit{UserInterval};
        \item Pannello con vista della media delle misurazioni delle isole ecologiche \textit{islands} in \textit{UserInterval}.
    \end{enumerate}
\item Titolo riga: \textbf{Colonnine di ricarica}
\begin{enumerate}
    \item Pannello per la scelta delle colonnine di ricarica da analizzare (vengono proposte per la scelta solo le  colonnine di ricarica interne alle celle selezionate per l'analisi).\\
    \textbf{Verrà chiamato \textit{chsSt} l'insieme delle  colonnine di ricarica scelte nel pannello appena esposto};
    \item Pannello con vista sul numero di colonnine di ricarica libere in \textit{chsSt} considerando l'ultima misurazione in
    \textit{UserInterval};
    \item Pannello con vista sul numero di colonnine di ricarica occupate in \textit{chsSt} considerando l'ultima misurazione in \textit{UserInterval};
    \item Pannello con vista tabellare delle ultime misurazioni in \textit{UserInterval} delle colonnine di ricarica in \textit{chsSt}.
\end{enumerate}
\item Titolo riga: \textbf{Guasti elettrici}
\begin{enumerate}
    \item Pannello per la scelta dei sensori di guasti elettrici da analizzare (vengono proposti per la scelta solo i sensori di guasti elettrici interni alle celle selezionate per l'analisi).\\
    \textbf{Verrà chiamato \textit{GstEl} l'insieme dei sensori di guasti elettrici scelti nel pannello appena esposto};
    \item Pannello con vista sul numero di sensori che hanno rilevato anomalie in \textit{GstEl} considerando l'ultima misurazione in \textit{UserInterval};
    \item Pannello con vista sul numero di sensori che non hanno rilevato anomalie in \textit{GstEl} considerando l'ultima misurazione in \textit{UserInterval};
    \item Pannello con vista tabellare delle ultime misurazioni in \textit{UserInterval} dei sensori \textit{GstEl}.
\end{enumerate}
\item Titolo riga: \textbf{Sensori di presenza dell'acqua}
\begin{enumerate}
    \item Pannello per la selezione dei sensori di presenza dell'acqua da analizzare (vengono proposti solo i sensori di presenza dell'acqua interni alle celle selezionate per l'analisi).\\
    \textbf{Verrà chiamato \textit{PresAcq} l'insieme dei sensori di presenza dell'acqua scelti nel pannello appena esposto};
    \item Pannello con vista sul numero di sensori che hanno rilevato acqua in \textit{PresAcq} considerando l'ultima misurazione nell'intervallo temporale definito dall'utente (\textit{UserInterval});
    \item Pannello con vista sul numero di sensori che non hanno rilevato acqua in \textit{PresAcq} considerando l'ultima misurazione nell'intervallo temporale definito dall'utente (\textit{UserInterval});
    \item Pannello con vista tabellare delle ultime misurazioni nell'intervallo (\textit{UserInterval}) dei sensori \textit{PresAcq}.
\end{enumerate}
\end{enumerate}


\paragraph*{Dashboard dedicata - Progettazione in dettaglio}
La dashboard dedicata è suddivisa in due sezioni: la prima, posta nella parte superiore della dashboard, è dedicata ai pannelli per le variabili di input, mentre la seconda, posta nella parte inferiore della dashboard, è dedicata alla visualizzazione delle misurazioni dei sensori secondo le impostazioni selezionate.\\
Di seguito vengono esposti i dettagli relativi alla progettazione della selezione delle variabili di input:
\begin{enumerate}
    \item Pannello "\textbf{Selezione cella}": Permette di selezionare le celle della città da analizzare;
    \item Pannello "\textbf{Tipologia misurazioni}": Permette di selezionare la tipologia di sensori da analizzare;
    \item Pannello "\textbf{Selezione sensori}": Permette di selezionare i sensori da analizzare relativi alla tipologia e alla cella selezionata;
    \item Pannello "\textbf{Aggregazione temporale}": Permette di selezionare l'intervallo temporale di aggregazione delle misurazioni:{Automatico, Secondo, Minuto, Ora, Giorno, Mese, Nessuno}. Maggiori dettagli in \ref{sec:var_dedicate};
    \item Pannello "\textbf{Misurazine minima}": Permette di selezionare il valore minimo delle misurazioni da visualizzare;
    \item Pannello "\textbf{Misurazione massima}": Permette di selezionare il valore massimo delle misurazioni da visualizzare.
\end{enumerate}
Di seguito vengono esposti i dettagli relativi alla progettazione della visualizzazione delle misurazioni secondo le variabili selezionate:
\begin{enumerate}
    \item Pannello "\textbf{Grafico a linee}": Visualizzazione delle misurazioni attraverso un grafico a linee time-series. Le misurazioni sono mostrate in base ai parametri specificati dall’utente. In particolare le misurazioni esposte rispettano i parametri scelti dall'utente nella sezione di selezione delle variabili di input sopra elencate;
    \item Pannello "\textbf{Tabella misurazioni}": Visualizzazione delle misurazioni in forma tabellare. Come nel pannello precedente, le misurazioni sono mostrate in base ai parametri specificati dall'utente.
\end{enumerate}

\subsubsection{ClickHouse data source plugin} \label{sec:click_plugin}
\paragraph{Documentazione:}
\href{https://grafana.com/grafana/plugins/grafana-clickhouse-datasource/}{https://grafana.com/grafana/plugins/grafana-clickhouse-datasource/}

Questo plugin di grafana consente di connettersi a un'istanza di ClickHouse e di visualizzarne i dati in tempo reale. È possibile eseguire query SQL personalizzate e visualizzare i risultati in forma di grafici, tabelle e pannelli personalizzati. Il plugin offre anche funzionalità di aggregazione e di calcolo dei dati, consentendo di analizzare e visualizzare i dati in modo flessibile e personalizzato.

\paragraph{Data sources configuration}
La configurazione del data source avviene tramite file \textit{yaml} che deve essere presente in \textit{"/provisioning/datasources"}.
Il protocollo di trasporto utilizzato è \textit{TLS} e, se necessario, può essere modificato nel file appena citato grazie al parametro di configurazione \textit{protocol}.

\paragraph{Macro utilizzate}\label{sec:macros}
Per semplificare la sintassi e consentire operazioni dinamiche, come filtri per intervalli temporali, le query al database Clickhouse possono contenere macro.
Le macro utilizzate sono:
\begin{itemize}
    \item \textbf{\$\_\_timeFilter(columnName)}: Permette di effettuare il filtro temporale alla query per ottenere le sole misurazioni all'interno dell'intervallo di tempo selezionato dall'utente;
    \item  \textbf{\$\_\_timeInterval(columnName)}: Permette di modificare il raggruppamento temporale delle misurazioni in automatico sulla base dell'ampiezza dell'intervallo temporale selezionato dall'utente.
    In questo modo è possibile avere una visione ottimizzate delle misurazioni.
\end{itemize}

\subsubsection{Variabili Grafana}
\textbf{Documentazione:}
\url{https://grafana.com/docs/grafana/latest/dashboards/variables/} (Consultato:~25/03/2024)


Le variabili in Grafana sono un potente strumento per rendere le dashboard dinamiche e interattive. Permettono di filtrare i dati visualizzati in base a valori scelti dall'utente, rendendo la dashboard più versatile e adattabile a differenti esigenze.
\paragraph*{Variabili nella dashboard principale:}
Nella dashboard principale, le variabili sono:
\begin{itemize}
    \item \textbf{variabile (\$cella)}: Per mostrare solo le misurazioni provenienti da determinate celle della città;
    \item \textbf{variabili (\$<TipoSensore>\_sensors\_id)}: Per mostrare le misurazioni di determinati sensori di un certo tipo.
\end{itemize}
Queste variabili, all'interno delle query al database, permettono il filtraggio delle misurazioni sulla base di quanto selezionato dall'utente.
Un esempio di query per la visualizzazione delle misurazioni time-series di temperatura è:
\begin{lstlisting}[style=code]
    SELECT    ID_sensore, avg(value) as value,
              $__timeInterval(timestamp) as timestamp
    FROM    innovacity.temperatures 
    WHERE    $__timeFilter(timestamp) AND cella IN ($Cella) AND ID_sensore in (${tmp_sensors_id})
    GROUP BY ID_sensore, timestamp;
\end{lstlisting}

La query mostra anche l'utilizzo delle macro esposte in: \ref{sec:macros}

\paragraph*{Variabili nella dashboard dedicata:} \label{sec:var_dedicate}
Nella dashboard dedicata alla visualizzazione specifica delle misurazioni di una sola tipologia sono presenti le seguenti variabili:
\begin{itemize}
    \item \textbf{variabile (\$cella)}: Per mostrare solo le misurazioni provenienti da determinate celle della città;
    \item \textbf{variabili (\$<TipoSensore>\_sensors\_id)}: Per mostrare le misurazioni di determinati sensori di un certo tipo;
    \item \textbf{variabili (\$tabella)}: Per selezionare la tipoligia di sensore di cui si vuole visualizzare la dashboard dedicata e quindi la tabella del database da cui ricavare i dati;
    \item \textbf{(\$aggregazione)}: Per selezionare l'intervallo temporale di aggregazione delle misurazioni
    (Automatico, Secondo, Minuto, Ora, Giorno, Mese, Nessuno).
    Nel caso della selezione della modalità "Automatico" si utilizza l'intervallo temporale di aggregazione più opportuno sulla base dell'ampiezza dell'intervallo temporale selezionato dall'utente;
    \item \textbf{(\$Max\_value)}: Variabile ad input numerico per filtrare le misurazioni con valore al di sotto di quello indicato;
    \item \textbf{(\$Min\_value)}: Variabile ad input numerico per filtrare le misurazioni con valore al di sopra di quello indicato.
\end{itemize}



\subsubsection{Grafana alerts}
\textbf{Documentazione:}
\url{https://grafana.com/docs/grafana/latest/alerting/} (Consultato:~25/03/2024)

Grafana offre un sistema di alerting completo per monitorare i dati e inviare notifiche quando si verificano determinate condizioni. Le notifiche possono essere inviate tramite diversi canali, tra cui email, Slack, Telegram e Discord.

\paragraph{Alert Rule}
Per poter configurare un alert è necessario creare una regola di alert. Tale regola viene impostata tramite query al data source e fa scattare l'alert quando la query restituisce un risultato che soddisfa le condizioni impostate.
Gli alert sono configurati per i seguenti eventi:
\begin{itemize}
    \item Quando un sensore di temperatura registra una temperatura superiore ai 40°C o inferiore ai -10°C;
    \item Quando un sensore di polveri sottili supera i 50 microgrammi al metro cubo;
    \item Quando un sensore di guasti elettrici rileva un guasto.
\end{itemize}

Gli alert attraversano 3 stati:
\begin{itemize}
    \item \textbf{Pending:} La condizione per l'attivazione dell'avviso è stata soddisfatta, ma il periodo di valutazione dell'avviso non è ancora trascorso;
    \item \textbf{Firing:} Indica che un alert è stato attivato e la sua valutazione ha confermato che la condizione di alert è soddisfatta per il periodo impostato nella regola e quindi viene inviata la notifica ai canali impostati;
    \item \textbf{OK:} Indica che un alert è stato disattivato e la sua valutazione ha confermato che la condizione di alert non è più soddisfatta.

Le regole di allerta sono configurabili tramite l'interfaccia grafica di Grafana e vengono esportate in formato \textit{yaml} ed inserite in \textit{"/provisioning/alerting"}.
                            
\end{itemize}

\paragraph{Configurazione canale di notifica}
Per configurare i canali di notifica è necessario andare in \textit{Alerting} e selezionare \textit{Notification channels} dall'interfaccia grafica di Grafana.

Per il progetto è stato scelto Discord come unico canale di notifica.

Per configurare il canale di notifica è necessario seguire i seguenti passaggi:
\begin{itemize}
    \item Seleziona Discord come canale di notifica;
    \item Configurazione Server Discord "InnovaCity":
    \begin{itemize}
        \item Il server Discord è stato creato ed è raggiungibile tramite l'indirizzo: \url{https://discord.gg/cCp9qxK7};
        \item Per ottenere il webhook URL del canale Discord andare in: \textit{Impostazioni server/Integrazioni} e seleziona \textit{visualizza webhook}.
    \end{itemize}
    \item Inserisci il webhook URL del tuo canale Discord;
    \item Personalizza il messaggio di notifica.
    
\end{itemize}

Anche le impostazioni di configurazione del canale di notifica sono esportabili in formato \textit{yaml} e vengono inserite in \textit{/provisioning/alerting}.



\paragraph{Notification policies}
Le norme di notifica negli alert di Grafana sono un modo potente per gestire l'invio degli alert a diversi canali di notifica.

Per una spiegazione dettagliata della configurazione si rimanda alla documentazione ufficiale di Grafana: \url{https://grafana.com/docs/grafana/latest/alerting/alerting-rules/create-notification-policy/} (Consultato:~25/03/2024).

Anche le impostazioni delle notification policies sono esportabili in formato \textit{yaml} e vengono inserite in \textit{/provisioning/alerting}.

\subsubsection{Altri plugin utilizzati}
\paragraph{Orchestra Cities Map plugin}
\textbf{Documentazione:}

\url{https://grafana.com/grafana/plugins/orchestracities-map-panel/} (Consultato:~25/03/2024)

Il plugin Orchestra Cities Map per Grafana estende il pannello Geomap con diverse funzionalità avanzate per la visualizzazione di dati geolocalizzati su mappe.
Viene utilizzato al fine di consentire una rappresentazione differenziata delle icone corrispondenti ai vari tipi di sensori distribuiti in città, nonché la visualizzazione dell'ultima misurazione effettuata, ovvero lo stato attuale del sensore stesso.

Funzionalità principali:
\begin{itemize}
    \item \textbf{Supporto per GeoJSON}: Permette di visualizzare dati GeoJSON su mappe, come shapefile di città, regioni o stati;
    \item \textbf{Icone personalizzate}: Permette di utilizzare icone personalizzate per rappresentare diversi tipi di dati sui punti mappa;
    \item \textbf{Popup informativi}: Permette di visualizzare popup con informazioni dettagliate quando si clicca su un punto mappa;
    \item \textbf{Strati multipli}: Permette di creare più strati sovrapposti per visualizzare diversi set di dati sulla stessa mappa;
    \item \textbf{Filtraggio e ricerca}: Permette di filtrare i punti mappa in base a diversi criteri, come proprietà dei dati o valori delle metriche;
    \item \textbf{Colorazione dei punti}: Permette di colorare i punti mappa in base a valori di metriche o ad altri criteri;
    \item \textbf{Legende personalizzate}: Permette di creare legende personalizzate per spiegare il significato dei colori e delle icone utilizzati nella mappa.
\end{itemize}

\paragraph{Variable Panel plugin}
\textbf{Documentazione:}
\url{https://volkovlabs.io/plugins/volkovlabs-variable-panel/} (Consultato:~25/03/2024)


Il plugin permette di creare dei pannelli Grafana che possono essere posizionati ovunque nella dashboard e che consentono di selezionare i valori delle variabili.
In aggiunta, il sistema consente la rappresentazione a forma di albero delle variabili, la quale risulta vantaggiosa nel nostro contesto in cui i sensori sono localizzati all'interno delle celle della città.



