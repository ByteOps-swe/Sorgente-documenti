\documentclass{article}
\usepackage[utf8]{inputenc}
\usepackage[absolute]{textpos}
\usepackage[default]{raleway}
\usepackage{titlesec, comment, tabularx, makecell, listings, array, setspace, geometry, graphicx, xcolor, xparse, fancyvrb, relsize, fancyhdr, booktabs, hyperref}
\usepackage{makecell}
%\geometry{a4paper, left=2cm, right=2cm, top=2cm, bottom=2.5cm}
\renewcommand{\headrulewidth}{0pt}

% Definisci uno stile per i comandi git
\definecolor{light-gray}{gray}{0.92}

\lstdefinestyle{code}{
    frame=single,
    framesep=1mm,    
    rulecolor=\color{light-gray},
    backgroundcolor=\color{light-gray},
    basicstyle=\ttfamily,
}

% ----------------------------- Definizione tabella ----------------------------

\newcolumntype{C}[1]{>{\centering\arraybackslash}m{#1}}

%\setcellgapes{2ex} % Imposta l'altezza dell'header (2ex)


% ------------------------------Metadati indice -------------------------------
\title{\textbf{\fontsize{28}{6}\selectfont Indice}}
\author{\fontsize{14}{6}\selectfont ByteOps} 
\date{Ottobre 18, 2023}
% -----------------------------Creazione footer --------------------------------

\pagestyle{fancy}
\fancyhf{}
\renewcommand{\footrulewidth}{0.4pt}
\lfoot{
    \parbox[c]{2cm}{\includegraphics[width=2cm]{../../Images/logo.png}}
    \textcolor[RGB]{120, 120, 120}{$\cdot$ Verbale Interno}
}
\rfoot{\thepage}


% --------------------------Modifica formato hyperlinks ------------------------

\hypersetup{
    colorlinks=true,
    linkcolor=black,
    filecolor=black,      
    pdftitle={Verbale 18-10-2023},
    pdfpagemode=FullScreen,
}

% ------------------------------- Valore sotto-paragrafi indice --------------------------------------

\setcounter{secnumdepth}{4}
\setcounter{tocdepth}{4}

\titleformat{\section}
{\normalfont\huge\bfseries}{\thesection}{0.2cm}{}
\titlespacing*{\paragraph}{0pt}{0.5cm}{0.1cm}

\titleformat{\subsection}
{\normalfont\Large\bfseries}{\thesubsection}{0.2cm}{}
\titlespacing*{\paragraph}{0pt}{0.5cm}{0.1cm}

\titleformat{\subsubsection}
{\normalfont\large\bfseries}{\thesubsubsection}{0.2cm}{}
\titlespacing*{\paragraph}{0pt}{0.5cm}{0.1cm}

\titleformat{\paragraph}
{\normalfont\normalsize\bfseries}{\theparagraph}{0.2cm}{}
\titlespacing*{\paragraph}{0pt}{0.5cm}{0.1cm}

% ------------------------------- Front Page ---------------------------------------

\begin{document}

% --------------------------Aggiunta firma finale ------------------------
\begin{textblock*}{\textwidth}(0.85\textwidth, 1.16\textheight)
    Il responsabile: Davide Diotto
\end{textblock*}
% ------------------------------------------------------------------------

\pagestyle{fancy}
\begin{center}
\includegraphics[width = 0.7\textwidth]{../../Images/logo.png} \\
\vspace{0.2cm}
\textcolor[RGB]{60, 60, 60}{\textit{ByteOps.swe@gmail.com}} \\
\vspace{1cm}
\fontsize{16}{6}\selectfont Verbale Interno $\cdot$ Data: 18/10/2023 \\
\vspace{0.5cm}
\end{center}

\section*{Informazioni documento}
\def\arraystretch{1.2} \begin{tabular}{>{\raggedleft\arraybackslash}p{0.2\textwidth}|>{\raggedright\arraybackslash}p {0.6\textwidth}c}
\hline
\addlinespace
\textbf{Luogo} & Discord\vspace{10pt} \\
\textbf{Orario} & 15:00 - 17:30 \vspace{10pt} \\
\textbf{Redattori} & A. Barutta \\ & R. Smanio \\ & N. Preto \vspace{10pt} \\
\textbf{Verificatori} & E. Hysa \\ & L. Skenderi \\ & D. Diotto \vspace{10pt} \\ \textbf{Amministratore} & F. Pozza \vspace{10pt} \\
\textbf{Destinatari} & T. Vardanega \\ & R. Cardin \vspace{10pt} \\
\textbf{Partecipanti} & A. Barutta \\ & E. Hysa \\ & R. Smanio \\ & D. Diotto \\ & F. Pozza \\ & L. Skenderi \\ & N. Preto \vspace{10pt} \\
\end{tabular}
\pagebreak

% ------------------------- Changelog ----------------------------
\section*{Registro delle modifiche}
\begin{tabular}{|C{2.5cm}|C{2.5cm}|C{2.5cm}|C{2.5cm}|C{2.5cm}|} \hline
\textbf{Versione} & \textbf{Data} & \textbf{Autore} & \textbf{Verificatore} & \textbf{Dettaglio} \\
\hline \hline
0.0.2 & 07/11/2023 & A. Barutta & \makecell{E. Hysa\\L. Skenderi} & Nuovo template verbali \\ \hline
0.0.1 & 18/10/2023 & \makecell{A. Barutta\\R. Smanio} & D. Diotto & Redazione documento\\ \hline
\end{tabular} % \pagebreak

% ------------------------- Generazione automatica indice ----------------------
\setstretch{1.5}
\maketitle
\thispagestyle{fancy}
\tableofcontents
\setstretch{1.2}
\pagebreak

% ------------------------ INIZIO DOCUMENTO ----------------------
\flushleft

\section{Revisione del periodo precedente}
Si evidenzia che alcune delle \textit{attività}\textsubscript{\textit{G}} delineate nel verbale precedente non sono ancora state portate a termine, tuttavia le decisioni e le direttive stabilite durante l’incontro del 17/10/2023 sono rimaste valide e accurate anche dopo un’attenta riflessione di giornata. Ora si intende continuare ad aderire alle decisioni precedentemente concordate e ci si impegna a completare le \textit{attività}\textsubscript{\textit{G}} rimanenti nel minor tempo possibile. 

\section{Ordine del giorno}

    \subsection{Creazione repository Git}
        È stato deciso di creare un \textit{repository}\textsubscript{\textit{G}} Git per il versionamento dei file relativi al progetto e
        si è deciso di utilizzare \textit{Github}\textsubscript{\textit{G}} come \textit{piattaforma}\textsubscript{\textit{G}} di hosting del \textit{repository}\textsubscript{\textit{G}}. \subsection{Gestione dei documenti relativi al progetto}
        Si è scelto di creare una cartella condivisa su \textit{Google Drive} per la gestione dei documenti in fase di elaborazione. Si è scelto inoltre di utilizzare \textit{Google Docs} per poter apportare modifiche ai documenti sia in modalità sincrona che asincrona. Ogni documento dopo essere stato completato e revisionato viene redatto con \LaTeX\ e caricato nell’apposita cartella all’interno della \textit{repository}\textsubscript{\textit{G}} dai verificatori. 
    
    \subsection{Gestione delle attività}
        Abbiamo scelto di utilizzare \textit{Jira} come strumento per la suddivisione ed il tracciamento delle \textit{attività}\textsubscript{\textit{G}} relative al progetto, così da associare ad ogni \textit{attività}\textsubscript{\textit{G}} dettagli come, ad esempio, assegnatario/i, scadenza, priorità e stima ore richieste.\\
        In questo modo è possibile monitorare l'avanzamento del progetto ed organizzare al meglio il lavoro di gruppo permettendo ad ogni componente di avere una chiara visione delle \textit{attività}\textsubscript{\textit{G}} svolte, delle \textit{attività}\textsubscript{\textit{G}} in corso e delle \textit{attività}\textsubscript{\textit{G}} da svolgere. 
        Ogni qualvolta ci siano \textit{attività}\textsubscript{\textit{G}} da aggiungere, l’amministratore del gruppo provvederà a farlo specificando i dettagli sopracitati. 
    
    \subsection{Programmazione riunioni settimanali}
        Si è deciso di svolgere dei meeting di gruppo a cadenza settimanale, (salvo necessità straordinarie) in cui analizzeremo il lavoro svolto, discuteremo di eventuali dubbi e/o problemi sorti durante la settimana e definiremo le successive \textit{attività}\textsubscript{\textit{G}} da svolgere.
        I meeting si terranno ogni lunedì su \textit{Discord}.\\
        Essendo che la maggior parte dei diari di bordo viene svolta il lunedì ci sarà modo di discutere anche di eventuali problemi riscontrati e di pianificare delle azioni di miglioramento.
    
        \subsection{Stesura di una traccia per i primi documenti}
            \begin{itemize}
                \item Valutazione dei capitolati
                \item Lettera di presentazione
                \item Preventivi costi
            \end{itemize}

    \subsection{Fissato meeting con proponente capitolato C6}
        Data: 20/10/2023 \\
        Ora: 11:30 - 12:00 \\
        In merito a ciò, abbiamo concordato le interrogazioni da porre e le abbiamo documentate in un apposito file, gli esiti saranno visionabili nel relativo verbale esterno che verrà redatto successivamente all'incontro.

\section{Attività da svolgere}
    Non sono state definite nuove \textit{attività}\textsubscript{\textit{G}}.

\end{document}
