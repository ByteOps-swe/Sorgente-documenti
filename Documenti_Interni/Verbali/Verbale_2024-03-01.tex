\documentclass{article}
\usepackage[utf8]{inputenc}
\usepackage[absolute]{textpos}
\usepackage[default]{raleway}
\usepackage{titlesec, comment, tabularx, makecell, listings, array, setspace, geometry, graphicx, xcolor, xparse, fancyvrb, relsize, fancyhdr, booktabs, multirow, hyperref}
\usepackage{colortbl}
%\geometry{a4paper, left=2cm, right=2cm, top=2cm, bottom=2.5cm}
\renewcommand{\headrulewidth}{0pt}

% Definisci uno stile per i comandi git
\definecolor{light-gray}{gray}{0.92}

\lstdefinestyle{code}{
    frame=single,
    framesep=1mm,
    rulecolor=\color{light-gray},
    backgroundcolor=\color{light-gray},
    basicstyle=\ttfamily,
}

% ----------------------------- Definizione tabella ---------------------------

\newcolumntype{C}[1]{>{\centering\arraybackslash}m{#1}}

%\setcellgapes{2ex} % Imposta l'altezza dell'header (2ex)


% ------------------------------Metadati indice --------------------------------
\title{\textbf{\fontsize{28}{6}\selectfont Indice}}
\author{\fontsize{14}{6}\selectfont ByteOps}
\date{Marzo 01, 2024}


% -----------------------------Creazione footer --------------------------------

\pagestyle{fancy}
\fancyhf{}
\renewcommand{\footrulewidth}{0.4pt}
\lfoot{
    \parbox[c]{2cm}{\includegraphics[width=2cm]{../../Images/logo.png}}
    \textcolor[RGB]{120, 120, 120}{$\cdot$ Verbale Interno}
}
\rfoot{\thepage}

% --------------------------Modifica formato hyperlinks ------------------------

\hypersetup{
    colorlinks=true,
    linkcolor=black,
    filecolor=black,      
    pdftitle={Verbale Interno 01/03/2024},  %inserisci data verbale
    pdfpagemode=FullScreen,
}

% ------------------------------- Valore sotto-paragrafi indice --------------------------------------

\setcounter{secnumdepth}{4}
\setcounter{tocdepth}{4}

\titleformat{\section}
{\normalfont\huge\bfseries}{\thesection}{0.2cm}{}
\titlespacing*{\paragraph}{0pt}{0.5cm}{0.1cm}

\titleformat{\subsection}
{\normalfont\Large\bfseries}{\thesubsection}{0.2cm}{}
\titlespacing*{\paragraph}{0pt}{0.5cm}{0.1cm}

\titleformat{\subsubsection}
{\normalfont\large\bfseries}{\thesubsubsection}{0.2cm}{}
\titlespacing*{\paragraph}{0pt}{0.5cm}{0.1cm}

\titleformat{\paragraph}
{\normalfont\normalsize\bfseries}{\theparagraph}{0.2cm}{}
\titlespacing*{\paragraph}{0pt}{0.5cm}{0.1cm}

% ------------------------------- Front Page ---------------------------------------

\begin{document}

% --------------------------Aggiunta firma finale ------------------------
\begin{textblock*}{\textwidth}(0.85\textwidth, 1.16\textheight)
    Il responsabile: R. Smanio
\end{textblock*}
% ------------------------------------------------------------------------

\pagestyle{fancy}
\begin{center}
\includegraphics[width = 0.7\textwidth]{../../Images/logo.png} \\
\vspace{0.2cm}
\textcolor[RGB]{60, 60, 60}{\textit{ByteOps.swe@gmail.com}} \\
\vspace{1cm}
\fontsize{16}{6}\selectfont Verbale Interno $\cdot$ Data: 01/03/2024 \\
\vspace{0.5cm}
\end{center}

\section*{Informazioni documento}
\def\arraystretch{1.2}
\begin{tabular}{>{\raggedleft\arraybackslash}p{0.3\textwidth}|>{\raggedright\arraybackslash}p{0.6\textwidth}c}
\hline
\addlinespace
\textbf{Luogo} & Discord \vspace{10pt} \\
\textbf{Orario} & 17:30 - 18:15 \vspace{10pt} \\
\textbf{Redattore} & N. Preto \vspace{10pt} \\
\textbf{Verificatore} & A. Barutta \vspace{10pt} \\
\textbf{Amministratore} & E. Hysa \vspace{10pt} \\
\textbf{Destinatari} & T. Vardanega \\ & R. Cardin \vspace{10pt} \\
\multirow[t]{7}{*}{\textbf{Partecipanti}} & A. Barutta \\ & E. Hysa \\ & R. Smanio \\ & D. Diotto \\ & F. Pozza \\ & L. Skenderi \\ & N. Preto \vspace{10pt} \\
\end{tabular}
\pagebreak 

% ------------------------- Changelog ----------------------------

\section*{Registro delle modifiche}

\begin{tabular}{|C{2.5cm}|C{2.5cm}|C{2.5cm}|C{2.5cm}|C{2.5cm}|}
    \hline
    \textbf{Versione} & \textbf{Data} & \textbf{Autore} & \textbf{Verificatore} & \textbf{Dettaglio} \\
    \hline \hline
    0.0.2 & 02/03/2024 & N. Preto & A. Barutta & Aggiunta issue mancanti \\
    \hline
    0.0.1 & 01/03/2024 & N. Preto & A. Barutta & Redazione documento \\
    \hline
\end{tabular}
\pagebreak

% ------------------------- Generazione automatica indice ----------------------
\setstretch{1.5}
\maketitle
\thispagestyle{fancy}
\tableofcontents
\setstretch{1.2}
\pagebreak

% ------------------------ INIZIO DOCUMENTO ----------------------
\flushleft

\section{Revisione del periodo precedente}
Nella settimana successiva all'ultimo meeting interno, è stato integrato con successo Faust nel prodotto \textit{software}\textsubscript{\textit{G}}, rispondendo efficacemente alle criticità evidenziate dal Professor Cardin durante la revisione \textit{RTB}\textsubscript{\textit{G}} e garantendo l'aderenza ai principi di ingegneria del \textit{software}\textsubscript{\textit{G}}. \\
Successivamente all'\textit{integrazione}\textsubscript{\textit{G}} di Faust, le principali \textit{attività}\textsubscript{\textit{G}} a cui si sono dedicati i membri del team riguardano lo sviluppo dei \textit{test}\textsubscript{\textit{G}} e la redazione del \textit{Manuale Utente}.
Inoltre, l'Amministratore si è dedicato al miglioramento della struttura e dell'organizzazione del \textit{repository}\textsubscript{\textit{G}} GitHub, seguendo i suggerimenti ricevuti dal Professor Vardanega nella valutazione della revisione \textit{RTB}\textsubscript{\textit{G}}. \\
In aggiunta, sono state apportate modifiche all'aspetto di alcuni pannelli della \textit{dashboard}\textsubscript{\textit{G}} \textit{Grafana}\textsubscript{\textit{G}}, aumentando così la qualità dell'interfaccia e dell'user experience.

\section{Ordine del giorno}
    \subsection{Discussione sulle difficoltà emerse nell'attività di testing}
    All'inizio dell'incontro, è stata posta l'attenzione sui \textit{test}\textsubscript{\textit{G}} che sono attualmente in fase di sviluppo. \\
    Nei giorni precedenti infatti, sono state riscontrate difficoltà nella realizzazione dei \textit{test}\textsubscript{\textit{G}} di \textit{integrazione}\textsubscript{\textit{G}} tra il producer e il server \textit{Kafka}\textsubscript{\textit{G}}, principalmente a causa della poca esperienza del team in tale ambito. \\
    A seguito delle discussioni tenute durante il \textit{SAL}\textsubscript{\textit{G}} odierno, l'azienda \textit{proponente}\textsubscript{\textit{G}} ha proposto di fornirci supporto per la risoluzione dei problemi riscontrati. Inoltre, si è deciso di fare pratica con dei minimal working examples in modo tale da acquisire le competenze e l'esperienza necessaria per poter realizzare i \textit{test}\textsubscript{\textit{G}} di \textit{integrazione}\textsubscript{\textit{G}} nel modo corretto in un contesto più complesso come quello del progetto.

    \subsection{Discussione sul livello di dettaglio del Manuale Utente}
    Dopo un'attenta riflessione sul livello di dettaglio da mantenere riguardo alle istruzioni e alle procedure nel \textit{Manuale Utente}, si era inizialmente optato per una descrizione più semplice delle funzionalità offerte e delle relative modalità di utilizzo. Questa scelta mirava a rendere la lettura e la comprensione del manuale accessibili, evitando complicazioni. Tuttavia, una successiva riflessione più approfondita del team ha portato alla conclusione che fosse necessario ampliare ulteriormente le descrizioni delle funzionalità e fornire indicazioni più dettagliate sul loro scopo e sul modo migliore per utilizzarle. In quest'ottica, diventa fondamentale trovare un equilibrio tra precisione, dettaglio e approfondimento, mantenendo allo stesso tempo la semplicità di lettura e comprensione. \\
    L'adozione di una maggior granularità delle informazioni mira a garantire una comprensione completa degli aspetti e dei dettagli meno evidenti del prodotto, riducendo al contempo il rischio di fraintendimenti e ambiguità. L'obiettivo finale è rendere il Manuale Utente più accessibile, preciso e utile per il lettore, fornendo informazioni dettagliate sulle modalità di utilizzo di tutte le funzionalità offerte, soprattutto su quelle che potrebbero non risultare immediate o intuitive.

    \subsection{Discussione riguardante il sistema di autenticazione}
    Si è evidenziato un ulteriore punto di interesse riguardante il mantenimento del \textit{sistema}\textsubscript{\textit{G}} di autenticazione fornito da \textit{Grafana}\textsubscript{\textit{G}} e, dopo aver analizzato attentamente le diverse opinioni espresse in merito, si è concluso che mantenere l'autenticazione è fondamentale poiché garantisce un livello di sicurezza superiore. \\
    È da sottolineare che, come concordato con la \textit{proponente}\textsubscript{\textit{G}}, non saranno previsti \textit{account}\textsubscript{\textit{G}} con differenti livelli di privilegio all'interno del \textit{software}\textsubscript{\textit{G}}. Ciò è dovuto al fatto che gli utenti finali, ovvero i membri delle autorità locali, devono avere accesso completo a tutte le funzionalità del \textit{sistema}\textsubscript{\textit{G}}. \\
    In conclusione, per accedere alla \textit{dashboard}\textsubscript{\textit{G}}, gli utenti facenti parte delle autorità locali non dovranno registrarsi in autonomia, ma verranno fornite loro delle credenziali di accesso (nome utente e password) tramite un canale affidabile.

    \subsection{Controllo conformità ai requisiti}
    È stato riconosciuto l'importante bisogno di condurre un'attenta revisione dei requisiti delineati nel documento "\textit{Analisi dei Requisiti}". Ciò si configura come un passaggio cruciale per garantire che le modifiche apportate alla \textit{dashboard}\textsubscript{\textit{G}} siano conformi ai requisiti stabiliti all'inizio del progetto. L'obiettivo principale è garantire che il gruppo non abbia introdotto elementi superflui o deviazioni dalle specifiche inizialmente definite.

    \section{Attività da svolgere}
    \begin{center}
        \begin{tabular}{|C{7cm}|C{1,5cm}|C{3cm}|}
            \hline
            \textbf{Titolo} & \textbf{\# Issue} & \textbf{Verificatore} \\
            \hline
            \hline
            Approfondimento paragrafi Manuale Utente & 96 & A. Barutta \\
            \hline
            Ricontrollo AdR dopo prima stesura MU & 97 & A. Barutta \\
            \hline
            Redazione verbale interno 01/03/2024 & 99 & N. Preto \\
            \hline
            Stesura periodo 10 - PdP & 100 & A. Barutta \\
            \hline
            Sviluppo \textit{test}\textsubscript{\textit{G}} di unità sensori & 101 & R. Smanio \\
            \hline
            Sviluppo \textit{test}\textsubscript{\textit{G}} di \textit{integrazione}\textsubscript{\textit{G}} \textit{kafka}\textsubscript{\textit{G}} - \textit{python}\textsubscript{\textit{G}} & 102 & R. Smanio \\
            \hline
            Sviluppo \textit{test}\textsubscript{\textit{G}} di \textit{integrazione}\textsubscript{\textit{G}} \textit{Kafka}\textsubscript{\textit{G}} - \textit{Python}\textsubscript{\textit{G}} - \textit{Clickhouse}\textsubscript{\textit{G}} & 103 & A. Barutta \\
            \hline
        \end{tabular}
    \end{center}
\end{document}
