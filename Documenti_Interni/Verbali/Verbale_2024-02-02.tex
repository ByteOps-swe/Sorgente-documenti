\documentclass{article}
\usepackage[utf8]{inputenc}
\usepackage[absolute]{textpos}
\usepackage[default]{raleway}
\usepackage{titlesec, comment, tabularx, makecell, listings, array, setspace, geometry, graphicx, xcolor, xparse, fancyvrb, relsize, fancyhdr, booktabs, hyperref}
\usepackage{colortbl}
\usepackage{float}
%\geometry{a4paper, left=2cm, right=2cm, top=2cm, bottom=2.5cm}
\renewcommand{\headrulewidth}{0pt}

% Definisci uno stile per i comandi git
\definecolor{light-gray}{gray}{0.92}

\lstdefinestyle{code}{
    frame=single,
    framesep=1mm,
    rulecolor=\color{light-gray},
    backgroundcolor=\color{light-gray},
    basicstyle=\ttfamily,
}

% ----------------------------- Definizione tabella ---------------------------

\newcolumntype{C}[1]{>{\centering\arraybackslash}m{#1}}

%\setcellgapes{2ex} % Imposta l'altezza dell'header (2ex)


% ------------------------------Metadati indice --------------------------------
\title{\textbf{\fontsize{28}{6}\selectfont Indice}}
\author{\fontsize{14}{6}\selectfont ByteOps}
\date{Febbraio 02, 2024}


% -----------------------------Creazione footer --------------------------------

\pagestyle{fancy}
\fancyhf{}
\renewcommand{\footrulewidth}{0.4pt}
\lfoot{
    \parbox[c]{2cm}{\includegraphics[width=2cm]{../../Images/logo.png}}
    \textcolor[RGB]{120, 120, 120}{$\cdot$ Verbale Interno}
}
\rfoot{\thepage}

% --------------------------Modifica formato hyperlinks ------------------------

\hypersetup{
    colorlinks=true,
    linkcolor=black,
    filecolor=black,      
    pdftitle={Verbale Interno 02/02/2024},  %inserisci data verbale
    pdfpagemode=FullScreen,
}

% ------------------------------- Valore sotto-paragrafi indice --------------------------------------

\setcounter{secnumdepth}{4}
\setcounter{tocdepth}{4}

\titleformat{\section}
{\normalfont\huge\bfseries}{\thesection}{0.2cm}{}
\titlespacing*{\paragraph}{0pt}{0.5cm}{0.1cm}

\titleformat{\subsection}
{\normalfont\Large\bfseries}{\thesubsection}{0.2cm}{}
\titlespacing*{\paragraph}{0pt}{0.5cm}{0.1cm}

\titleformat{\subsubsection}
{\normalfont\large\bfseries}{\thesubsubsection}{0.2cm}{}
\titlespacing*{\paragraph}{0pt}{0.5cm}{0.1cm}

\titleformat{\paragraph}
{\normalfont\normalsize\bfseries}{\theparagraph}{0.2cm}{}
\titlespacing*{\paragraph}{0pt}{0.5cm}{0.1cm}

% ------------------------------- Front Page ---------------------------------------

\begin{document}

% --------------------------Aggiunta firma finale ------------------------
\begin{textblock*}{\textwidth}(0.85\textwidth, 1.16\textheight)
    Il responsabile: A. Barutta
\end{textblock*}
% ------------------------------------------------------------------------

\pagestyle{fancy}
\begin{center}
\includegraphics[width = 0.7\textwidth]{../../Images/logo.png} \\
\vspace{0.2cm}
\textcolor[RGB]{60, 60, 60}{\textit{ByteOps.swe@gmail.com}} \\
\vspace{1cm}
\fontsize{16}{6}\selectfont Verbale Interno $\cdot$ Data: 02/02/2024 \\
\vspace{0.5cm}
\end{center}

\section*{Informazioni documento}
\def\arraystretch{1.2}
\begin{tabular}{>{\raggedleft\arraybackslash}p{0.2\textwidth}|>{\raggedright\arraybackslash}p{0.6\textwidth}c}
    \hline
    \addlinespace
    \textbf{Luogo} & Discord \vspace{10pt} \\
    \textbf{Orario} & 9:30 - 10:00 \vspace{10pt} \\
    \textbf{Redattore} & N. Preto \vspace{10pt} \\
    \textbf{Verificatore} & A. Barutta \vspace{10pt} \\
    \textbf{Amministratore} & E. Hysa \vspace{10pt} \\
    \textbf{Destinatari} & T. Vardanega \\ & R. Cardin \vspace{10pt} \\
    \textbf{Partecipanti} & A. Barutta \\ & E. Hysa \\ & R. Smanio \\ & D. Diotto \\ & F. Pozza \\ & L. Skenderi \\ & N. Preto \vspace{10pt} \\
\end{tabular}
\pagebreak 

% ------------------------- Changelog ----------------------------

\section*{Registro delle modifiche}

\begin{tabular}{|C{2.5cm}|C{2.5cm}|C{2.5cm}|C{2.5cm}|C{2.5cm}|}
    \hline
    \textbf{Versione} & \textbf{Data} & \textbf{Autore} & \textbf{Verificatore} & \textbf{Dettaglio} \\
    \hline \hline
    0.0.1 & 02/02/2024 & N. Preto & A. Barutta & Redazione verbale \\
    \hline
\end{tabular}
\pagebreak

% ------------------------- Generazione automatica indice ----------------------
\setstretch{1.5}
\maketitle
\thispagestyle{fancy}
\tableofcontents
\setstretch{1.2}
\pagebreak

% ------------------------ INIZIO DOCUMENTO ----------------------
\flushleft

\section{Revisione del periodo precedente}
Nel periodo antecedente, conformemente a quanto anticipato nel Verbale Interno del 11/01/2024, sono state ridotte le \textit{attività}\textsubscript{\textit{G}} relative al Progetto didattico al fine di agevolare un'approfondita preparazione agli esami della sessione di Gennaio. Durante tale periodo, sono state completate le modifiche pianificate al documento Analisi dei Requisiti, in vista della revisione del professore Cardin.

\section{Ordine del giorno}

\subsection{Scrittura mail prof. Cardin}
All'apertura dell'incontro, avendo già stabilito in precedenza la nostra intenzione di presentare il \textit{PoC}\textsubscript{\textit{G}} nella settimana compresa tra il 05/02/2024 e l'11/02/2024, è stata elaborata una candidatura formale da inviare via email al professore. Una volta redatta, la suddetta è stata trasmessa al mittente al fine di ottenere una risposta.

\subsection{Definizione dei ruoli per la presentazione RTB}
Dopo aver inviato al professore l'email di candidatura, si è convenuto sul metodo di suddivisione del contenuto della presentazione tra la maggior parte dei membri del gruppo. Per coloro i quali non avrebbero ricoperto un ruolo specifico nell'esposizione, è stato stabilito che avrebbero contribuito alla stesura delle motivazioni riguardanti le decisioni tecnologiche adottate per la realizzazione del \textit{PoC}\textsubscript{\textit{G}}.

\subsection{Creazione di un video illustrativo }
È stato deliberato che, in occasione della presentazione, risulterebbe vantaggioso disporre di un video preregistrato, realizzato in precedenza, che illustri le funzionalità del \textit{PoC}\textsubscript{\textit{G}}. Tale decisione è motivata dalla consapevolezza che, in linea con le consuetudini operative, gli imprevisti possono verificarsi e, pertanto, nel caso di eventuali inconvenienti nell'utilizzo delle tecnologie impiegate nel \textit{PoC}\textsubscript{\textit{G}} durante la presentazione, si avrà comunque a disposizione un supporto multimediale affidabile.

\subsection{Rotazione ruoli}
    \begin{table}[H]
        \centering
        \begin{tabular}{|c|c|} 
            \hline
            \textbf{Ruolo} & \textbf{Nome Cognome} \\
            \hline \hline
            Responsabile (Re) & A. Barutta \\ 
            \hline
            Amministratore (Am) & E. Hysa \\ 
            \hline
            Analista (An) & \makecell{E. Hysa\\N. Preto} \\
            \hline
            Verificatore (Ve) & \makecell{A. Barutta\\F.Pozza} \\
            \hline
            Programmatore (Pr) & \makecell{D. Diotto\\L. Skenderi} \\
            \hline
        \end{tabular}
        \caption{Tabella che rappresenta i membri per ogni ruolo}
    \end{table}

\section{Attività da svolgere}
    \begin{center}
        \begin{tabular}{|C{7cm}|C{1,5cm}|C{3cm}|}
            \hline
            \textbf{Titolo} & \textbf{\# Issue} & \textbf{Verificatore} \\
            \hline\hline
            Correzione Riferimenti a File & 80 & F. Pozza \\
            \hline
        \end{tabular}
    \end{center}

\end{document}
