\documentclass{article}
\usepackage[utf8]{inputenc}
\usepackage[absolute]{textpos}
\usepackage[default]{raleway}
\usepackage{titlesec, comment, tabularx, makecell, listings, array, setspace, geometry, graphicx, xcolor, xparse, fancyvrb, relsize, fancyhdr, booktabs, hyperref}
\usepackage{colortbl}
\usepackage{float}
%\geometry{a4paper, left=2cm, right=2cm, top=2cm, bottom=2.5cm}
\renewcommand{\headrulewidth}{0pt}

% Definisci uno stile per i comandi git
\definecolor{light-gray}{gray}{0.92}

\lstdefinestyle{code}{
    frame=single,
    framesep=1mm,
    rulecolor=\color{light-gray},
    backgroundcolor=\color{light-gray},
    basicstyle=\ttfamily,
}

% ----------------------------- Definizione tabella ---------------------------

\newcolumntype{C}[1]{>{\centering\arraybackslash}m{#1}}

%\setcellgapes{2ex} % Imposta l'altezza dell'header (2ex)


% ------------------------------Metadati indice --------------------------------
\title{\textbf{\fontsize{28}{6}\selectfont Indice}}
\author{\fontsize{14}{6}\selectfont ByteOps}
\date{Febbraio 12, 2024}


% -----------------------------Creazione footer --------------------------------

\pagestyle{fancy}
\fancyhf{}
\renewcommand{\footrulewidth}{0.4pt}
\lfoot{
    \parbox[c]{2cm}{\includegraphics[width=2cm]{../../Images/logo.png}}
    \textcolor[RGB]{120, 120, 120}{$\cdot$ Verbale Interno}
}
\rfoot{\thepage}

% --------------------------Modifica formato hyperlinks ------------------------

\hypersetup{
    colorlinks=true,
    linkcolor=black,
    filecolor=black,      
    pdftitle={Verbale Interno 12/02/2024},  %inserisci data verbale
    pdfpagemode=FullScreen,
}

% ------------------------------- Valore sotto-paragrafi indice --------------------------------------

\setcounter{secnumdepth}{4}
\setcounter{tocdepth}{4}

\titleformat{\section}
{\normalfont\huge\bfseries}{\thesection}{0.2cm}{}
\titlespacing*{\paragraph}{0pt}{0.5cm}{0.1cm}

\titleformat{\subsection}
{\normalfont\Large\bfseries}{\thesubsection}{0.2cm}{}
\titlespacing*{\paragraph}{0pt}{0.5cm}{0.1cm}

\titleformat{\subsubsection}
{\normalfont\large\bfseries}{\thesubsubsection}{0.2cm}{}
\titlespacing*{\paragraph}{0pt}{0.5cm}{0.1cm}

\titleformat{\paragraph}
{\normalfont\normalsize\bfseries}{\theparagraph}{0.2cm}{}
\titlespacing*{\paragraph}{0pt}{0.5cm}{0.1cm}

% ------------------------------- Front Page ---------------------------------------

\begin{document}

% --------------------------Aggiunta firma finale ------------------------
\begin{textblock*}{\textwidth}(0.85\textwidth, 1.16\textheight)
    Il responsabile: A. Barutta
\end{textblock*}
% ------------------------------------------------------------------------

\pagestyle{fancy}
\begin{center}
\includegraphics[width = 0.7\textwidth]{../../Images/logo.png} \\
\vspace{0.2cm}
\textcolor[RGB]{60, 60, 60}{\textit{ByteOps.swe@gmail.com}} \\
\vspace{1cm}
\fontsize{16}{6}\selectfont Verbale Interno $\cdot$ Data: 12/02/2024 \\
\vspace{0.5cm}
\end{center}

\section*{Informazioni documento}
\def\arraystretch{1.2}
\begin{tabular}{>{\raggedleft\arraybackslash}p{0.2\textwidth}|>{\raggedright\arraybackslash}p{0.6\textwidth}c}
    \hline
    \addlinespace
    \textbf{Luogo} & Discord \vspace{10pt} \\
    \textbf{Orario} & 9:30 - 11:00 \vspace{10pt} \\
    \textbf{Redattore} & A. Barutta \vspace{10pt} \\
    \textbf{Verificatore} & F. Pozza \vspace{10pt} \\
    \textbf{Amministratore} & E. Hysa \vspace{10pt} \\
    \textbf{Destinatari} & T. Vardanega \\ & R. Cardin \vspace{10pt} \\
    \textbf{Partecipanti} & A. Barutta \\ & E. Hysa \\ & R. Smanio \\ & D. Diotto \\ & F. Pozza \\ & L. Skenderi \\ & N. Preto \vspace{10pt} \\
\end{tabular}
\pagebreak 

% ------------------------- Changelog ----------------------------

\section*{Registro delle modifiche}

\begin{tabular}{|C{2.5cm}|C{2.5cm}|C{2.5cm}|C{2.5cm}|C{2.5cm}|}
    \hline
    \textbf{Versione} & \textbf{Data} & \textbf{Autore} & \textbf{Verificatore} & \textbf{Dettaglio} \\
    \hline \hline
    0.0.1 & 12/02/2024 & A.Barutta & F. Pozza & Redazione verbale \\
    \hline
\end{tabular}
\pagebreak

% ------------------------- Generazione automatica indice ----------------------
\setstretch{1.5}
\maketitle
\thispagestyle{fancy}
\tableofcontents
\setstretch{1.2}
\pagebreak

% ------------------------ INIZIO DOCUMENTO ----------------------
\flushleft

\section{Revisione del periodo precedente}
Nel periodo intercorso tra l'ultimo incontro e quello attuale, è stata condotta la prima parte della revisione \textit{RTB}\textsubscript{\textit{G}} con il professor Cardin. L'esito è stato positivo e abbiamo ottenuto l'approvazione per procedere con le \textit{attività}\textsubscript{\textit{G}} e per affrontare la seconda parte della revisione \textit{RTB}\textsubscript{\textit{G}} con il professor Vardanega.

Durante l'incontro con il \textit{proponente}\textsubscript{\textit{G}} del 09/02/2024, abbiamo discusso e analizzato i commenti del Professor Cardin relativi al \textit{PoC}\textsubscript{\textit{G}}, prendendo anche decisioni basate su tali osservazioni presenti nel verbale esterno correlato.

\section{Ordine del giorno}

\subsection{Riscontro RTB Prof. Cardin}
Si è effettuata una discussione collaborativa esaminando le osservazioni critiche del Professore Cardin sul documento di Analisi dei Requisiti \textit{v.} 1.0.0. In seguito a questa analisi, abbiamo concordato che gli elementi individuati come errati devono essere risolti con la alta priorità.

\subsection{Connessione diretta Kafka-ClickHouse}
Durante il recente incontro del \textit{SAL}\textsubscript{\textit{G}} del 09/02/2024, abbiamo preso la decisione di stabilire una connessione diretta tra \textit{Kafka}\textsubscript{\textit{G}} e \textit{ClickHouse}\textsubscript{\textit{G}}, eliminando così eventuali rischi associati a "edge-case" non affrontati. In seguito a questa scelta, abbiamo discusso le modalità di elaborazione dei dati per soddisfare il requisito relativo al calcolo del punteggio di salute. Tale elaborazione verrà gestita mediante l'implementazione di una libreria o \textit{framework}\textsubscript{\textit{G}} appropriato, come \textit{Kafka}\textsubscript{\textit{G}} Streams o Faust. Una volta eseguito il calcolo sui dati, il risultato verrà reinserito in \textit{Kafka}\textsubscript{\textit{G}}, specificamente in una coda dedicata, per renderlo disponibile a tutti i servizi che ne necessitano.

\subsection{Progettazione simulazione sensori}
Tra le \textit{attività}\textsubscript{\textit{G}} pianificate, abbiamo deliberato di dedicare risorse alla fase iniziale della progettazione del diagramma delle classi per la simulazione dei sensori, nonostante ci sia stato chiarito, nel più recente diario di bordo, che il focus del progetto non è su questa specifica area.

\subsection{Prova presentazione prof. Vardanega}
Come ultima fase dell'incontro, abbiamo condotto una prova della presentazione relativa alla seconda parte della revisione \textit{RTB}\textsubscript{\textit{G}}.
L'esito è stato positivo e soddisfacente, rispettando il limite di tempo richiesto di 15 minuti.

\section{Attività da svolgere}
    \begin{center}
        \begin{tabular}{|C{7cm}|C{1,5cm}|C{3cm}|}
            \hline
            \textbf{Titolo} & \textbf{\# Issue} & \textbf{Verificatore} \\
            \hline\hline
            Correzione AdR - Cardin & 83 & F. Pozza \\
            \hline
            Connessione diretta \textit{Kafka}\textsubscript{\textit{G}}-\textit{ClickHouse}\textsubscript{\textit{G}} & 84 & A. Barutta \\
            \hline
            Progettazione \textit{sistema}\textsubscript{\textit{G}} simulazione sensori & 85 & A. Barutta \\
            \hline
        \end{tabular}
    \end{center}

\end{document}
