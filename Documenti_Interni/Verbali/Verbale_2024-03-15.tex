\documentclass{article}
\usepackage[utf8]{inputenc}
\usepackage[absolute]{textpos}
\usepackage[default]{raleway}
\usepackage{titlesec, comment, tabularx, makecell, listings, array, setspace, geometry, graphicx, xcolor, xparse, fancyvrb, relsize, fancyhdr, booktabs, multirow, todonotes, hyperref}
\usepackage{colortbl}
%\geometry{a4paper, left=2cm, right=2cm, top=2cm, bottom=2.5cm}
\renewcommand{\headrulewidth}{0pt}

% Definisci uno stile per i comandi git
\definecolor{light-gray}{gray}{0.92}

\lstdefinestyle{code}{
    frame=single,
    framesep=1mm,
    rulecolor=\color{light-gray},
    backgroundcolor=\color{light-gray},
    basicstyle=\ttfamily,
}

% ----------------------------- Definizione tabella ---------------------------

\newcolumntype{C}[1]{>{\centering\arraybackslash}m{#1}}

%\setcellgapes{2ex} % Imposta l'altezza dell'header (2ex)


% ------------------------------Metadati indice --------------------------------
\title{\textbf{\fontsize{28}{6}\selectfont Indice}}
\author{\fontsize{14}{6}\selectfont ByteOps}
\date{Marzo 15, 2024}


% -----------------------------Creazione footer --------------------------------

\pagestyle{fancy}
\fancyhf{}
\renewcommand{\footrulewidth}{0.4pt}
\lfoot{
    \parbox[c]{2cm}{\includegraphics[width=2cm]{../../Images/logo.png}}
    \textcolor[RGB]{120, 120, 120}{$\cdot$ Verbale Interno}
}
\rfoot{\thepage}

% --------------------------Modifica formato hyperlinks ------------------------

\hypersetup{
    colorlinks=true,
    linkcolor=black,
    filecolor=black,      
    pdftitle={Verbale Interno 15/03/2024},  %inserisci data verbale
    pdfpagemode=FullScreen,
}

% ------------------------------- Valore sotto-paragrafi indice --------------------------------------

\setcounter{secnumdepth}{4}
\setcounter{tocdepth}{4}

\titleformat{\section}
{\normalfont\huge\bfseries}{\thesection}{0.2cm}{}
\titlespacing*{\paragraph}{0pt}{0.5cm}{0.1cm}

\titleformat{\subsection}
{\normalfont\Large\bfseries}{\thesubsection}{0.2cm}{}
\titlespacing*{\paragraph}{0pt}{0.5cm}{0.1cm}

\titleformat{\subsubsection}
{\normalfont\large\bfseries}{\thesubsubsection}{0.2cm}{}
\titlespacing*{\paragraph}{0pt}{0.5cm}{0.1cm}

\titleformat{\paragraph}
{\normalfont\normalsize\bfseries}{\theparagraph}{0.2cm}{}
\titlespacing*{\paragraph}{0pt}{0.5cm}{0.1cm}

% ------------------------------- Front Page ---------------------------------------

\begin{document}

% --------------------------Aggiunta firma finale ------------------------
\begin{textblock*}{\textwidth}(0.85\textwidth, 1.16\textheight)
    Il responsabile: D. Diotto
\end{textblock*}
% ------------------------------------------------------------------------

\pagestyle{fancy}
\begin{center}
\includegraphics[width = 0.7\textwidth]{../../Images/logo.png} \\
\vspace{0.2cm}
\textcolor[RGB]{60, 60, 60}{\textit{ByteOps.swe@gmail.com}} \\
\vspace{1cm}
\fontsize{16}{6}\selectfont Verbale Interno $\cdot$ Data: 15/03/2024 \\
\vspace{0.5cm}
\end{center}

\section*{Informazioni documento}
\def\arraystretch{1.2}
\begin{tabular}{>{\raggedleft\arraybackslash}p{0.3\textwidth}|>{\raggedright\arraybackslash}p{0.6\textwidth}c}
\hline
\addlinespace
\textbf{Luogo} & Discord \vspace{10pt} \\
\textbf{Orario} & 17:30 - 18:30 \vspace{10pt} \\
\textbf{Redattore} & E. Hysa \vspace{10pt} \\
\textbf{Verificatore} & A. Barutta \vspace{10pt} \\
\textbf{Amministratore} & N. Preto \vspace{10pt} \\
\textbf{Destinatari} & T. Vardanega \\ & R. Cardin \vspace{10pt} \\
\multirow[t]{7}{*}{\textbf{Partecipanti}} & A. Barutta \\ & E. Hysa \\ & R. Smanio \\ & D. Diotto \\ & F. Pozza \\ & L. Skenderi \\ & N. Preto \vspace{10pt} \\
\end{tabular}
\pagebreak 

% ------------------------- Changelog ----------------------------

\section*{Registro delle modifiche}

\begin{tabular}{|C{2.5cm}|C{2.5cm}|C{2.5cm}|C{2.5cm}|C{2.5cm}|}
    \hline
    \textbf{Versione} & \textbf{Data} & \textbf{Autore} & \textbf{Verificatore} & \textbf{Dettaglio} \\
    \hline \hline
    0.0.1 & 15/03/2024 & E. Hysa & A. Barutta & Redazione documento \\
    \hline
\end{tabular}
\pagebreak

% ------------------------- Generazione automatica indice ----------------------
\setstretch{1.5}
\maketitle
\thispagestyle{fancy}
\tableofcontents
\setstretch{1.2}
\pagebreak

% ------------------------ INIZIO DOCUMENTO ----------------------
\flushleft

\section{Revisione del periodo precedente}
Dal momento dell’ultimo incontro, le \textit{attività}\textsubscript{\textit{G}} pianificate sono state completate con successo e nei tempi previsti. In particolare, è stato implementato un \textit{sistema}\textsubscript{\textit{G}} di notifiche \textit{Grafana}\textsubscript{\textit{G}}-\textit{Discord}\textsubscript{\textit{G}}, progettato per informare l’utente in modo tempestivo su eventuali problemi o anomalie rilevate nel \textit{sistema}\textsubscript{\textit{G}}. \\
In aggiunta, è stato introdotta la funzionalità Time-To-Live (TTL) per i dati archiviati in \textit{Clickhouse}\textsubscript{\textit{G}}. Questa funzionalità permette una gestione efficace di grandi volumi di dati e consente una pulizia automatica di dati obsoleti, non più pertinenti o troppo remoti. \\
Infine, è stata adottata la pratica della Continuous Integration attraverso \textit{Github}\textsubscript{\textit{G}} Actions. Questo approccio assicura un processo di sviluppo efficiente, affidabile e di alta qualità, garantendo che ogni nuova modifica apportata non provochi anomalie nel codice preesistente.

\section{Ordine del giorno}
\subsection{Schema registry}
All’inizio dell’incontro, il gruppo ha discusso relativamente all’implementazione dello Schema Registry. Questo componente fornisce un nodo centralizzato per la gestione e la validazione degli schemi utilizzati nei messaggi dei topic \textit{Kafka}\textsubscript{\textit{G}}. La sua importanza risiede nella capacità di garantire coerenza e compatibilità dei dati man mano che gli schemi evolvono. \\
L’utilizzo dello Schema Registry offre numerosi vantaggi. Innanzitutto, valida i dati provenienti dai sensori, assicurando che siano conformi agli schemi attesi ed evitando di introdurre nel topic \textit{Kafka}\textsubscript{\textit{G}} messaggi in un formato errato. Inoltre, contribuisce a mantenere la coerenza tra i dati nuovi e quelli già presenti nel \textit{sistema}\textsubscript{\textit{G}}. Grazie al versionamento degli schemi, è possibile gestire le modifiche nel tempo senza compromettere l’integrità dei dati. Infine, semplifica le pipeline dati, riducendo il rischio di problemi di compatibilità o perdita di informazioni.
In sintesi, lo Schema Registry è uno strumento fondamentale per garantire la qualità e l’affidabilità dei dati all’interno del \textit{sistema}\textsubscript{\textit{G}} e per queste motivazioni, si è deciso di integrare questo componente nel \textit{sistema}\textsubscript{\textit{G}}.

\subsection{Proposta incontro con la proponente}
Durante il \textit{SAL}\textsubscript{\textit{G}} con la \textit{proponente}\textsubscript{\textit{G}}, è stata avanzata la proposta di organizzare un incontro presso la loro sede per l’approvazione dell’\textit{MVP}\textsubscript{\textit{G}}. Successivamente, durante una discussione tenuta dopo il \textit{SAL}\textsubscript{\textit{G}}, il gruppo ha deliberato di accettare tale proposta e ha comunicato alla \textit{proponente}\textsubscript{\textit{G}} la disponibilità a partecipare all’incontro. Tuttavia, la data e l’orario specifici per questo incontro devono ancora essere concordati con la \textit{proponente}\textsubscript{\textit{G}}. 

\section{Attività da svolgere}
\begin{center}
    \begin{tabular}{|C{7cm}|C{1,5cm}|C{3cm}|}
        \hline
        \textbf{Titolo} & \textbf{\# Issue} & \textbf{Verificatore} \\
        \hline
        \hline
        Implementazione Schema Registry & 114 & A. Barutta \\
        \hline
        Redazione periodo \#12 - PdP & 115 & E. Hysa \\
        \hline
        Redazione verbale esterno 15/03/2024 & 116 & A. Barutta \\
        \hline
    \end{tabular}
\end{center}

\end{document}
