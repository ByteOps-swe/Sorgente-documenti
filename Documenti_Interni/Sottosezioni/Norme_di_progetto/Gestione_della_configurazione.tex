\subsection{Gestione della configurazione}
\subsubsection{Introduzione}
La gestione della configurazione del progetto è un processo che norma il tracciamento e il controllo delle modifiche a documenti e prodotti del \textit{software}\textsubscript{\textit{G}} detti Configuration Item (CI).

La gestione della configurazione viene applicata a qualunque categoria di “artefatti” coinvolti nel ciclo di vita del \textit{software}\textsubscript{\textit{G}}. \\
Secondo lo \textit{standard}\textsubscript{\textit{G}} ISO/IEC 12207:1995, la gestione della configurazione del \textit{software}\textsubscript{\textit{G}} è un processo di identificazione, organizzazione e controllo delle modifiche apportate ai prodotti \textit{software}\textsubscript{\textit{G}} durante il loro ciclo di vita. \\
Lo \textit{standard}\textsubscript{\textit{G}} IEEE 828-2012 definisce la gestione della configurazione del \textit{software}\textsubscript{\textit{G}} come “un processo disciplinato per gestire l’evoluzione del \textit{software}\textsubscript{\textit{G}}”.

\subsubsection{Numeri di versionamento}\label{subsubsec:versionamento}
La convenzione per identificare la versione di un documento è nel formato X.Y.Z con:
\begin{itemize}
    \item \textbf{X:} viene incrementato al raggiungimento di \textit{RTB}\textsubscript{\textit{G}}, \textit{PB}\textsubscript{\textit{G}} ed eventualmente CA;
    \item \textbf{Y:} viene incrementato quando vengono apportate modifiche significative al documento, come cambiamenti strutturali, nuove sezioni importanti o modifiche sostanziali nel contenuto;
    \item \textbf{Z:} viene incrementato per modifiche minori o aggiornamenti al documento. Questi potrebbero includere correzioni di errori, miglioramenti marginali o l'aggiunta di nuovi contenuti meno rilevanti.
\end{itemize}

L'incremento dei valori più significativi porta i meno significativi a zero. Ad esempio, se in un documento nella versione 0.6.4 vengono effettuate modifiche significative e viene conseguentemente incrementato il valore Y, si passa alla versione 0.7.0 e non alla versione 0.7.4. \\
Ogni variazione di versione deve essere presente nel registro delle modifiche.

\subsubsection{Repository}
Di seguito sono elencate le \textit{repository}\textsubscript{\textit{G}} del team ByteOps con i relativi riferimenti:
\begin{itemize}
    \item \textbf{Sorgente-documenti:} \textit{repository}\textsubscript{\textit{G}} per il versionamento, la gestione e lo sviluppo dei file sorgente relativi alla documentazione. \\
    Riferimento: \url{https://github.com/ByteOps-swe/Sorgente-documenti} (Consultato:~16/03/2024);
    \item \textbf{Documents:} \textit{repository}\textsubscript{\textit{G}} destinata ai committenti/proponenti, dove vengono esclusivamente condivisi i file in formato PDF relativi alla documentazione, ottenuti dalla compilazione dei file sorgente presenti nella cartella "Sorgente-documenti". \\
    Riferimento: \url{https://github.com/ByteOps-swe/Documents} (Consultato:~16/03/2024);
    \item \textbf{proof-of-concept:} \textit{repository}\textsubscript{\textit{G}} destinata al \textit{POC}\textsubscript{\textit{G}}. \\
    Riferimento: \url{https://github.com/ByteOps-swe/proof-of-concept} (Consultato:~16/03/2024);
    \item \textbf{MVP:} contiene il codice sorgente relativo al Minimum Viable Product. \\
    Riferimento: \url{https://github.com/ByteOps-swe/MVP} (Consultato:~16/03/2024);
    \item \textbf{ByteOps-swe.github.io:} contiene il codice sorgente relativo alla pagina web dedicata all'esposizione dei repository sopracitati. \\
    Riferimento: \url{https://github.com/ByteOps-swe/ByteOps-swe.github.io} (Consultato:~16/03/2024).
\end{itemize}
\paragraph{Struttura repository}
Di seguito è riportata la struttura dei \textit{repository}\textsubscript{\textit{G}} destinati alla documentazione. Si sottolinea che i nomi in grassetto nell'elenco seguente identificano directory.

\begin{itemize}
    \item \textbf{Documentazione esterna:}
        \begin{itemize}
            \item \textbf{Verbali}
            \item Analisi dei requisiti;
            \item Manuale utente;
            \item Piano di progetto;
            \item Piano di qualifica;
            \item Preventivo costi e assunzione impegni;
            \item Specifica tecnica.
        \end{itemize}
    \item \textbf{Documentazione interna:}
        \begin{itemize}
            \item \textbf{Verbali};
            \item Glossario;
            \item Norme di progetto;
            \item Valutazione Capitolati;
        \end{itemize}
    \item Lettera di presentazione Candidatura;
    \item Lettera di presentazione RTB;
    \item Lettera di presentazione PB.
\end{itemize}

\hypertarget{subsubsec:sincronizzazione&branching}{\subsubsection{Sincronizzazione e Branching}}
\paragraph{Documentazione}
Il processo operativo utilizzato per la redazione della documentazione, noto come "Feature Branch workflow" (\url{https://www.atlassian.com/git/tutorials/comparing-workflows/feature-branch-workflow} Consultato:~19/03/2024), implica la creazione di un ramo dedicato per ciascun documento o sezione da elaborare.
Tale metodologia permette una parallelizzazione agile dei lavori evitando sovrascritture indesiderate di altri lavori e permette l'adozione dei principi "Documentation as code" definiti nella \textit{sezione \S~\ref{sec:DocumentationAscode}}. \\ 

\paragraph*{\hypertarget{par:convezioninomenclaturabranchdocumenti}{\textbf{Convenzioni per la nomenclatura dei branch relativi alle attività di redazione o modifica di documenti}}}

\begin{itemize}
    \item Il nome del \textit{branch}\textsubscript{\textit{G}} deve presentare l'identificativo del documento che si vuole redarre o modificare. \\
    Ad ogni documento è associato un identificativo, come descritto nella seguente tabella:
\end{itemize}

\begin{table}[H]
    \centering
    \begin{tabular}{|C{4cm}|C{2cm}|}
        \hline
        \textbf{Documento} & \textbf{ID} \\
        \hline \hline
        Verbale interno         & VI  \\
        Verbale esterno         & VE  \\
        Norme di Progetto       & NdP \\
        Piano di Qualifica      & PdQ \\
        Piano di Progetto       & PdP \\ 
        Analisi dei Requisiti   & AdR \\
        \hline
    \end{tabular}
    \caption{ID per la nomenclatura dei branch relativi alla documentazione}
\end{table}

\begin{itemize}
    \item Nel caso dei verbali, dopo l'identificativo del documento si aggiunge un underscore seguito dalla data: \\
    \texttt{IdDocumento\_dd-mm-yyyy} (ex. VI\_27-12-2023);
    \item Nel caso della redazione di una specifica sezione di un documento, il nome del \textit{branch}\textsubscript{\textit{G}} deve avere il formato: \\
    \texttt{IdDocumento\_NomeSezione} (ex. NdP\_Documentazione);
    \item Nel caso di modifica di una specifica sezione di un documento, il nome del \textit{branch}\textsubscript{\textit{G}} deve avere il formato: \\
    \texttt{IdDocumento\_ModNomeSezione} (ex. NdP\_ModDocumentazione).
\end{itemize}

\paragraph{Sviluppo}
Gitflow è il workflow Git che utilizza il team ByteOps per lo sviluppo.\\
Riferimento: \url{https://www.atlassian.com/it/git/tutorials/comparing-workflows/gitflow-workflow} (Consultato:~19/03/2024).

\paragraph*{Flusso generale di Gitflow}

\begin{enumerate}
    \item \textbf{Branch \texttt{develop}:} viene creato a partire dal \textit{branch}\textsubscript{\textit{G}} principale (\texttt{main}). È il punto di partenza per lo sviluppo di nuove funzionalità;

    \item \textbf{Branch \texttt{release}:} creato da \texttt{develop}, questo \textit{branch}\textsubscript{\textit{G}} gestisce la preparazione del \textit{software}\textsubscript{\textit{G}} per un rilascio. Durante questa fase, sono consentite solo correzioni di bug e miglioramenti minori;

    \item \textbf{Branch \texttt{feature}:} creati da \texttt{develop}, sono utilizzati per lo sviluppo di nuove funzionalità o miglioramenti; 

    \item \textbf{Merge di \texttt{feature} in \texttt{develop}:} quando una funzionalità è completa, il \textit{branch}\textsubscript{\textit{G}} \texttt{feature} viene "fuso" nel \textit{branch}\textsubscript{\textit{G}} \texttt{develop};

    \item \textbf{Merge di \texttt{release} in \texttt{develop} e \texttt{main}:} dopo il completamento del \textit{branch}\textsubscript{\textit{G}} \texttt{release}, quest'ultimo viene unito sia in \texttt{develop} che in \texttt{main}, segnalando un nuovo rilascio stabile;

    \item \textbf{Branch \texttt{hotfix}:} creato da \texttt{main} in caso di problemi critici rilevati nell'ambiente di produzione;

    \item \textbf{Merge di \texttt{hotfix} in \texttt{develop} e \texttt{main}:} una volta risolto il problema, il \textit{branch}\textsubscript{\textit{G}} \texttt{hotfix} viene unito sia in \texttt{develop} che in \texttt{main} per garantire coerenza tra le versioni.
\end{enumerate}

\paragraph*{Comandi di comodo}
\vspace{0.2cm}
\begin{itemize}
    \item Inizializzare Gitflow
    \vspace{0.1cm}
    \begin{lstlisting}[style=code]
        git flow init
    \end{lstlisting}
    \item Sviluppo di una Nuova Funzionalità
    \vspace{0.1cm}
    \begin{lstlisting}[style=code]
        git flow feature start nome_feature
    \end{lstlisting}
    \begin{lstlisting}[style=code]
        git flow feature finish nome_feature
    \end{lstlisting}
    \pagebreak
    \item Risoluzione di un bug
    \vspace{0.1cm}
    \begin{lstlisting}[style=code]
        git flow hotfix start nome_hotfix
    \end{lstlisting}
    \begin{lstlisting}[style=code]
        git flow hotfix finish nome_hotfix
    \end{lstlisting}
    \item Rilascio di una nuova versione
    \vspace{0.1cm}
    \begin{lstlisting}[style=code]
        git flow release start X.X.X
    \end{lstlisting}
    \begin{lstlisting}[style=code]
        git flow release finish X.X.X
    \end{lstlisting}
    \item Pubblicazione delle modifiche
    \vspace{0.1cm}
    \begin{lstlisting}[style=code]
        git push origin develop
        git push origin master
        git push origin --tags
    \end{lstlisting}
    \begin{lstlisting}[style=code]
        git push origin feature/nome_feature
        git push origin hotfix/nome_hotfix
        git push origin release/X.X.X
    \end{lstlisting}
\end{itemize}

\paragraph{Pull Request}
Dopo aver portato a termine le \textit{attività}\textsubscript{\textit{G}} nel proprio \textit{branch}\textsubscript{\textit{G}}, il responsabile del suo sviluppo è tenuto ad avviare una Pull Request per incorporare le modifiche effettuate nel ramo principale (main). La Pull Request viene approvata solo dopo che le modifiche sono state verificate sulla base di quanto descritto nella \textit{sezione \S~\ref{subsec:verifica}}.

\hypertarget{par:creazionePR}{\subparagraph{Procedura per la creazione di Pull Request}}
Per creare una Pull Request eseguire i seguenti passaggi:
\begin{enumerate}
    \item Accedere al \textit{repository}\textsubscript{\textit{G}} GitHub e cliccare sulla scheda "Pull requests";
    \item Cliccare il pulsante "New Pull Request";
    \item Selezionare il \textit{branch}\textsubscript{\textit{G}} di partenza ed il \textit{branch}\textsubscript{\textit{G}} target;
    \item Cliccare il pulsante "Create Pull Request";
    \item Aggiungere un titolo alla Pull Request;
    \item Aggiungere una descrizione;
    \item Selezionare i verificatori;
    \item Come convenzione, l'assegnatario è colui che richiede la Pull Request;
    \item Selezionare le \textit{label}\textsubscript{\textit{G}};
    \item Selezionare il progetto;
    \item Selezionare le \textit{milestone}\textsubscript{\textit{G}};
    \item Clicca il pulsante "Create Pull Request";
    \item Nel caso siano presenti conflitti seguire le istruzioni in \textit{Github}\textsubscript{\textit{G}} per rimuovere tali conflitti.
\end{enumerate}

\subsubsection{Controllo di configurazione}
\paragraph{Change request (Richiesta di modifica)}
Seguendo lo \textit{standard}\textsubscript{\textit{G}} ISO/IEC 12207:1995 per affrontare questo processo in modo strutturato le \textit{attività}\textsubscript{\textit{G}} sono:
\begin{enumerate}
    \item \textbf{Identificazione e registrazione} \\
    Le change request vengono identificate, registrate e documentate accuratamente. Questo include informazioni come la natura della modifica richiesta, l'urgenza e l'impatto sul \textit{sistema}\textsubscript{\textit{G}} esistente. \\
    L'identificazione avviene tramite la creazione di una \textit{issue}\textsubscript{\textit{G}} nell'ITS con \textit{label}\textsubscript{\textit{G}}: "Change request";
    \item \textbf{Valutazione e analisi} \\
    Le change request vengono valutate dal team per determinare la loro fattibilità, importanza e impatto sul progetto. Si analizzano i costi e i benefici associati all'implementazione della modifica;
    \item \textbf{Approvazione o rifiuto} \\
    Il responsabile valuta le informazioni raccolte e decide se approvare o respingere la change request. Questa decisione può essere basata su criteri come il budget, il tempo, le priorità degli \textit{stakeholder}\textsubscript{\textit{G}};
    \item \textbf{Pianificazione delle modifiche} \\
    Se una change request viene approvata, viene pianificata e integrata nel ciclo di sviluppo del \textit{software}\textsubscript{\textit{G}}. Questo può richiedere, ad esempio, la rinegoziazione dei tempi di consegna e la revisione del piano di progetto;
    \item \textbf{Implementazione delle modifiche} \\
    Le modifiche vengono effettivamente implementate. Durante questo processo, è fondamentale mantenere una traccia accurata di ciò che viene fatto per consentire una corretta documentazione e, se necessario, la possibilità di un \textit{rollback}\textsubscript{\textit{G}};
    \item \textbf{Verifica e validazione} \\
    Le modifiche apportate vengono verificate per assicurarsi che abbiano raggiunto gli obiettivi previsti e non abbiano introdotto nuovi problemi o errori;
    \item \textbf{Documentazione} \\
    Tutti i passaggi del processo di gestione delle change request vengono documentati accuratamente per garantire la trasparenza e la tracciabilità. Questa documentazione è utile per futuri riferimenti e per l'apprendimento delle modifiche apportate;
    \item \textbf{Comunicazione agli stakeholder} \\Durante tutto il processo è importante comunicare in modo chiaro e tempestivo con gli \textit{stakeholder}\textsubscript{\textit{G}}, per mantenere trasparenza e fiducia.
\end{enumerate}

\subsubsection{Configuration Status Accounting (Contabilità dello Stato di Configurazione)}
La Contabilità dello Stato di Configurazione è un'\textit{attività}\textsubscript{\textit{G}} dedicata a registrare e monitorare lo stato di tutte le configurazioni di un \textit{sistema}\textsubscript{\textit{G}} \textit{software}\textsubscript{\textit{G}} durante il suo ciclo di vita.

Questo processo è cruciale per mantenere la trasparenza e la tracciabilità nel ciclo di vita del \textit{software}\textsubscript{\textit{G}}, aiutando a gestire le configurazioni in modo uniforme e a mantenere un registro accurato di tutte le \textit{attività}\textsubscript{\textit{G}} e le modifiche relative agli elementi di configurazione.\\

\begin{itemize}
    \item \hypertarget{item:registrazioneconfigurazioni}{\textit{\textbf{Registrazione delle configurazioni}}}: registrazione delle informazioni dettagliate su ogni Configuration Item;
          \begin{itemize}
              \item  \textbf{Documentazione:} le informazioni relative alla configurazione sono presenti nella prima pagina di ciascuno;
              \item  \textbf{Sviluppo:} le informazioni relative alla configurazione sono inserite come prime righe di ciascun file sotto forma di commento.
          \end{itemize}
    \item \textbf{Stato e cambiamenti}: tenere traccia dello stato attuale di ciascun elemento di configurazione e di tutti i cambiamenti che avvengono nel corso del tempo. \\
    Ciò include le versioni attuali, le revisioni, le modifiche e le baseline;
        \begin{itemize}
            \item \textbf{Registro delle modifiche:} per monitorare lo stato di ciascun Configuration Item si utilizza il registro delle modifiche incorporato in ognuno di essi;
            \item \textbf{Branching \& DashBoard:} per verificare se vi sono \textit{attività}\textsubscript{\textit{G}} in corso su un Configuration Item, è sufficiente controllare se ci sono \textit{branch}\textsubscript{\textit{G}} attivi correlati e, dato che ogni \textit{issue}\textsubscript{\textit{G}} è associata a un CI tramite \textit{label}\textsubscript{\textit{G}}, verificare le eventuali \textit{issue}\textsubscript{\textit{G}} contrassegnate come "In Progress" nella colonna corrispondente della Dashboard del progetto.
        \end{itemize}
    \item \textbf{Supporto per la gestione delle change request}: registra e documenta le modifiche apportate agli elementi di configurazione in risposta alle richieste di modifica. \\
    Per gestire le richieste di modifica, si utilizza l'Issue Tracking System (ITS) di GitHub, creando una \textit{issue}\textsubscript{\textit{G}} contrassegnata con l'etichetta "Change request".
\end{itemize}

Per approfondimenti su come creare issues e associare labels a una \textit{issue}\textsubscript{\textit{G}}, consultare la sezione \S~\hyperlink{par:ticketing}{4.1.2.10}.

\subsubsection{Release management and delivery}
Nello \textit{standard}\textsubscript{\textit{G}} ISO/IEC 12207:1995, il processo "Release management and delivery" è definito come un processo che si occupa della pianificazione, del coordinamento e del controllo delle \textit{attività}\textsubscript{\textit{G}} necessarie per preparare e distribuire una versione di un prodotto \textit{software}\textsubscript{\textit{G}} per l'uso operativo.

Questo processo comprende la gestione delle release, la distribuzione del \textit{software}\textsubscript{\textit{G}}, la preparazione della documentazione correlata e altre \textit{attività}\textsubscript{\textit{G}} correlate al rilascio e alla consegna del prodotto \textit{software}\textsubscript{\textit{G}}.

\vspace{0.2cm}

Il team ByteOps ha stabilito che la creazione di una release deve avvenire in relazione alle baseline stabilite per il progetto didattico, ovvero \textit{RTB}\textsubscript{\textit{G}} (Requirements and Technology Baseline), \textit{PB}\textsubscript{\textit{G}} (Product Baseline) ed eventualmente CA (Customer Acceptance). Questo processo inoltre deve essere attuato solo dopo una completa verifica e validazione del prodotto seguendo le procedure dettagliate nelle \textit{sezioni Verifica (~\S~\ref{subsec:verifica})} e \textit{Validazione (~\S~\ref{subsec:validazione})}.

\vspace{0.2cm}

In conformità con quanto definito nella \textit{sezione~\S~\ref{subsubsec:versionamento}}, relativa al versionamento dei numeri, ogni nuova release comporta un incremento di 1 nell'elemento numerico X del formato X.Y.Z.

\paragraph{Procedura per la creazione di una release}
Per la creazione di una release eseguire la seguente procedura:
\begin{itemize}
    \item Accedi alla \textit{repository}\textsubscript{\textit{G}} GitHub;
    \item Accedi alla sezione "Releases";
    \item Clicca sul pulsante "Create a new release";
    \item Clicca sul pulsante "Choose a tag" e selezionare il tag che identifica la versione della release. Nel caso il tag desiderato non fosse presente nella lista, è possibile crearne uno nuovo scrivendo la versione desiderata e cliccando sul pulsante "Create new tag". Per la corretta definizione della versione vedi la \textit{sezione~\S~\ref{subsubsec:versionamento}} relativa ai numeri di versionamento;
    \item Seleziona il \textit{branch}\textsubscript{\textit{G}} main come \textit{branch}\textsubscript{\textit{G}} target di cui si vuole creare una release;
    \item Scegliere un titolo per la release. La convenzione adottata è quella di inserire come titolo il nome della baseline per la quale si vuole creare la release.
    \item Scrivi una breve descrizione per la release;
    \item Clicca "Publish release".
\end{itemize}

Dopo aver seguito la procedura, nella sezione "Releases" di GitHub sarà visibile la release appena creata e sarà possibile scaricare il file.zip relativo.

\subsubsection{Strumenti}
Le tecnologie adottate per la gestione dei Configuration Item sono:
\begin{itemize}
    \item \textbf{Git}: Version Control System distribuito utilizzato per il versionamento dei Configuration Item;
    \item \textbf{GitHub}: \textit{piattaforma}\textsubscript{\textit{G}} web per il controllo di versione (tramite Git) dei Configuration Item e per il Ticketing (\hyperlink{par:ticketing}{\S 4.1.2.10}). È impiegata per gestire le richieste di modifica tramite \textit{issue}\textsubscript{\textit{G}} e \textit{label}\textsubscript{\textit{G}}, oltre che per la contabilità dello stato di configurazione.
\end{itemize}

