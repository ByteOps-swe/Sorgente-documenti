\subsection{Documentazione}

\subsubsection{Introduzione}
La documentazione costituisce l'insieme di informazioni scritte che accompagnano un prodotto \textit{software}\textsubscript{\textit{G}}, fornendo dettagli utili a sviluppatori, distributori e utenti.\\
Tra gli scopi della documentazione troviamo:
\begin{itemize}
    \item Comprensione del prodotto senza supporto umano, usandola come mezzo di comunicazione;
    \item Segnare il confine tra creatività e disciplina;
    \item Assicurare che i \textit{processi}\textsubscript{\textit{G}} produttivi si svolgano con la qualità attesa.
\end{itemize}

L'obiettivo della sezione è:
\begin{itemize}
    \item  Definire delle procedure ripetibili che permettano di uniformare la documentazione prodotta dal gruppo ed il metodo di lavoro;
    \item  Raccogliere ed organizzare le norme che i membri del team devono seguire così da semplificare l'operazione di scrittura dei documenti.
\end{itemize}
Tali norme dovranno essere applicate da tutti i membri del team, ed i sorgenti \LaTeX\ che contengono tale documentazione verranno inseriti nel \textit{repository}\textsubscript{\textit{G}} "Sorgente-documenti".

\begin{comment} \paragraph*{Primi approcci alla redazione di documenti e problematiche riscontrate}
Per la composizione iniziale dei documenti richiesti per la candidatura, è stato sperimentato un'approccio che impiegava gli strumenti di Google Drive. Tale metodologia consentiva ai redattori di redigere agevolmente i documenti senza la necessità di padroneggiare la sintassi LaTeX, con l'intenzione di trasporre successivamente il contenuto in LaTeX una volta che fosse stato validato dai verificatori. Tuttavia, questo approccio ha suscitato problematiche, tra cui:
\begin{itemize}
    \item Rischio di incoerenza tra il contenuto presente negli strumenti di Google Drive.
    \item Prolungato impiego di tempo per la riscrittura in LaTeX, dovuto alla necessità di un passaggio aggiuntivo.
\end{itemize}
Per tali ragioni si è presa la decisione di adottare un nuovo approccio.
\end{comment}
\subsubsection{Documentation as Code}\label{sec:DocumentationAscode}
L'approccio che si intende adottare è quello di "Documentation as Code" (Documentazione come Codice) che consiste nel trattare la documentazione di un progetto \textit{software}\textsubscript{\textit{G}} allo stesso modo in cui si tratta il codice sorgente.

Questo approccio è incentrato sull'utilizzo di pratiche e strumenti tipici dello sviluppo \textit{software}\textsubscript{\textit{G}} per creare, gestire e distribuire la documentazione.
Alcuni aspetti chiave della "Documentation as Code" includono:
\begin{itemize}
    \item \textbf{Versionamento};
    \item \textbf{Scrittura in formato testuale};
    \item \textbf{Automazione};
    \item \textbf{Collaborazione};
    \item \textbf{Integrazione Continua};
    \item \textbf{Distribuzione}.
\end{itemize}
Questo approccio porta diversi vantaggi, tra cui una maggiore coerenza, una migliore tracciabilità delle modifiche e facilità di manutenzione. Inoltre, il concetto di "Documentation as Code" si allinea con la filosofia DevOps, dove la collaborazione e l'automazione sono valori chiave.\\

\subsubsection{Sorgente documenti}
Poiché i documenti vengono redatti in \LaTeX, per favorire una migliore collaborazione tra diversi autori, si è scelto di creare un file sorgente per ciascuna sezione e sottosezione di ogni documento. Questo approccio consente a ciascun autore di lavorare sulle singole sezioni o sottosezioni, evitando di apportare modifiche a parti del documento non pertinenti al proprio ambito. \\
Infine, viene creato un file principale con il nome del documento, nel quale verranno collegati tutti i file delle varie sezioni e sottosezioni mediante l'uso del comando \verb|\input{Sezione.tex}| o, nel caso delle sottosezioni, \verb|\input{Sottosezione.tex}|.

Nel caso dei verbali, essendo documenti composti da sezioni molto brevi, non si utilizza questo approccio.

\subsubsection{Ciclo di vita dei documenti}
Il ciclo di vita dei documenti è una sequenza di stati e \textit{attività}\textsubscript{\textit{G}}:
\begin{enumerate}
    \item \textbf{Stato: Necessità}
          \begin{enumerate}
              \item Nasce la necessità di una documentazione, per obbligo o per opportunità;
              \item Pianicazione della sua stesura;
              \item Suddivisione in sezioni;
              \item Durante le riunioni, si procede alla discussione e alla definizione collettiva di una traccia per il contenuto;
              \item Assegnazione delle sezioni ai redattori tramite task su ITS. 
          \end{enumerate}
    \item \textbf{Stato: Redazione}
          \begin{enumerate}
              \item  Ogni tipo di documento viene creato secondo la struttura specificata nella \textit{sezione~\ref{subsubsec:strutturadocumento}};
              \item Il team realizza il documento redigendone il contenuto rispettando le norme definite.
          \end{enumerate}
    \item \textbf{Stato: Verifica}
          \begin{enumerate}
              \item Quando la redazione del documento è conclusa, questo viene revisionato dai verificatori;
              \item Il documento viene compilato e il PDF generato viene inserito nella repository.
          \end{enumerate}
    \item \textbf{Stato: Manutenzione}
          \begin{enumerate}
              \item Manutenzione: Ogni documento soggetto a \textit{configuration management}\textsubscript{\textit{G}} deve essere adeguatamente modificato in conformità alle regole di versionamento e \textit{change management}\textsubscript{\textit{G}};
              \item La richiesta di modifica derivata da nuove esigenze relative al contenuto del documento determina il ritorno del documento allo stato di modifica.
          \end{enumerate}
\end{enumerate}

\subsubsection{Procedure correlate alla redazione di documenti} \label{subsubsec:ProcedureRedazioneDocumenti}

\hypertarget{par:Redattori}{\paragraph{I redattori}}
Il redattore responsabile della redazione di un documento o di una sua sezione deve seguire lo stesso approccio richiesto per la codifica del \textit{software}\textsubscript{\textit{G}}, adottando il \textit{workflow}\textsubscript{\textit{G}} noto come "Feature Branch" (\url{https://www.atlassian.com/git/tutorials/comparing-workflows/feature-branch-workflow} Consultato:~27/02/2024).

\paragraph*{\textbf{Caso redazione nuovo documento/sezione o modifica documento/sezione già verificato/a}}
Dalla \textit{repository}\textsubscript{\textit{G}} "Sorgente-documenti" contenente i sorgenti \LaTeX, il redattore dovrà creare un nuovo \textit{branch}\textsubscript{\textit{G}} Git in locale e spostarsi su di esso mediante l'utilizzo del comando:

\vspace{0.1cm}

\begin{lstlisting}[style=code]
    git checkout main
    git checkout -b identificativoBranch
\end{lstlisting}

Il nome del \textit{branch}\textsubscript{\textit{G}} destinato alla redazione o modifica del documento o di una sua sezione deve essere descrittivo al fine di consentire un'identificazione immediata e precisa del documento o della sezione su cui si sta lavorando. Pertanto, relativamente alla redazione dei documenti, devonono essere adottatate le specifiche \textit{\textbf{convenzioni per la nomenclatura dei branch}} esposte nella sezione \S~\hyperlink{par:convezioninomenclaturabranchdocumenti}{3.4.4.1} .

A questo punto, dopo aver creato/modificato il documento o la sezione assegnata e avviato la stesura, affinché gli altri redattori possano continuare il lavoro, è necessario rendere il \textit{branch}\textsubscript{\textit{G}} accessibile anche nella \textit{repository}\textsubscript{\textit{G}} remota, seguendo i seguenti passaggi:

\vspace{0.1cm}
\begin{lstlisting}[style=code]
    git add .
    git commit -m "Descrizione del commit"
    git push origin identificativoBranch
\end{lstlisting}

\paragraph*{\textbf{Caso modifica documento in stato di redazione}}
Qualora il redattore intenda continuare la stesura di un documento (o sezione) iniziato da un altro redattore, sono necessari i comandi:

\vspace{0.1cm}
\begin{lstlisting}[style=code]
    git pull
    git checkout identificativoBranch
\end{lstlisting}

\hypertarget{par:salvataggioecondivisioneprogressitasknoncompletate}{\paragraph*{{\textbf{Salvataggio e condivisione progressi di task non completate}}}}
Alla fine di una sessione di modifiche di un file, nel caso si desideri rendere accessibile ai membri il lavoro non ancora completato e, pertanto, non pronto alla verifica, è necessario assicurarsi di essere nel \textit{branch}\textsubscript{\textit{G}} corretto e successivamente:
\begin{enumerate}
    \item Eseguire il push\textsubscript{\textit{G}}  delle modifiche fatte nel \textit{branch}\textsubscript{\textit{G}}
    \vspace{0.1cm}
    \begin{lstlisting}[style=code]
    git add .
    git commit -m "Descrizione del commit"
    git push origin identificativoBranch
    \end{lstlisting}

    \item Nel caso di problemi con il punto 1:
    \vspace{0.1cm}
    \begin{lstlisting}[style=code]
    git pull origin identificativoBranch
    \end{lstlisting}
    \item Risolvi i conflitti e ripeti punto 1.
\end{enumerate}

\paragraph*{\textbf{Completamento attività di redazione}}
Una volta terminata la redazione di un documento o di una sezione, ogni redattore deve:
\begin{enumerate}
    \item Segnalare il completamento dell'\textit{attività}\textsubscript{\textit{G}} a loro assegnata (essa sia la stesura completa di un documento o di una sua parte) posizionando l'\textit{issue}\textsubscript{\textit{G}} relativa nella colonna "Da revisionare" della DashBoard documentazione (\url{https://github.com/orgs/ByteOps-swe/projects/1/views/1} Consultato:~27/02/2024).
    \item Attuare una \textit{\textbf{Pull Request}} (la procedura si può trovare nella sezione \S~\hyperlink{par:creazionePR}{3.4.4.3.1}):
          \begin{enumerate}
              \item Aggiornare il registro delle modifiche inserendo i dati richiesti in una nuova riga e incrementando la versione (il verificatore è definito all'assegnazione della task, presente nella descrizione della Issue del'ITS).
              È importante sottolineare che l'incremento della versione deve avvenire seguendo la convenzione specificata nella \textit{sezione \S~\ref{subsubsec:versionamento}} e tramite il comando "\verb|{\label{Git_Action_Version}X.Y.Z}|", come specificato nel paragrafo "\textit{Informazioni utili}" all'interno della sezione \S~\hyperlink{par:infoUtiliRegistroMod}{3.1.6.1};
              \item Eseguire i passaggi dettagliati nel caso "\textit{\textbf{Salvataggio e condivisione progressi di task non completate}}" all'interno della sezione \S~\hyperlink{par:Redattori}{3.1.5.1};
              \item Accedere al repository GitHub >> Apri la sezione "Pull requests" >> Clicca "New Pull Request";
              \item Selezionare come \textit{branch}\textsubscript{\textit{G}} di destinazione "main" e come \textit{branch}\textsubscript{\textit{G}} sorgente il ramo utilizzato per la redazione del documento (o sezione) da validare;
              \item Cliccare "Create Pull Request";
              \item Inserire un titolo significativo e una breve descrizione alla Pull Request >> Seleziona i verificatori su "Reviewers" >> Clicca "Create Pull Request".
          \end{enumerate}
\end{enumerate}

Se i verificatori non convalidano il documento (o la sezione), i redattori riceveranno dei feedback allegati alla Pull Request relativi ai problemi identificati.

\paragraph{I Verificatori}
I compiti e le procedure dei verificatori sono dettagliate nella \textit{sezione \S~\ref{sec:verificatori}}.

\paragraph{Il responsabile}
Nel processo di redazione dei documenti, il compito del responsabile consiste nel:
\begin{itemize}
    \item \textbf{Identificare i documenti o le sezioni da redigere};
    \item \textbf{Stabilire le scadenze entro cui devono essere completati};
    \item \textbf{Assegnare i redattori e i verificatori ai task};
    \item \textbf{Approvazione:} il responsabile si riserva il diritto di approvare il lavoro o richiedere eventuali ulteriori modifiche su tutti i documenti redatti e verificati nel periodo in cui ha esercitato la propria carica.
\end{itemize}

\paragraph{L'amministratore}
Nel processo di redazione dei documenti, il compito dell'amministratore consiste nell'inserire nell'ITS le \textit{attività}\textsubscript{\textit{G}} specificate dal responsabile:
    \begin{itemize}
        \item Redattori: assegnatari della \textit{issue}\textsubscript{\textit{G}};
        \item Verificatori: specificati nella descrizione della \textit{issue}\textsubscript{\textit{G}};
        \item Scadenza.
    \end{itemize}

\subsubsection{Struttura del documento}
\label{subsubsec:strutturadocumento}
I documenti seguono un rigoroso e uniforme schema strutturale, il quale richiede un'osservanza scrupolosa.

\paragraph{Registro delle modifiche}
\label{sec:RegistroModifiche}
Tutti i documenti devono essere provvisti di un registro delle modifiche in formato tabellare che contiene un riassunto
delle modifiche apportate al documento nel corso del tempo. \\
La tabella deve essere inserita nella sezione registro delle modifiche subito prima dell’indice del documento e deve registrare le seguenti informazioni:
\begin{itemize}
    \item \textbf{Versione del file};
    \item \textbf{Data di rilascio};
    \item \textbf{Autore};
    \item \textbf{Verificatore};
    \item \textbf{Descrizione}: una breve sintesi delle modifiche apportate.
\end{itemize}

\hypertarget{par:infoUtiliRegistroMod}{\paragraph*{Informazioni utili:}}
\begin{itemize}
    \item La convezione per il versionamento è presente alla \textit{sezione \S~\ref{subsubsec:versionamento}};
    \item La prima riga del registro delle modifiche identifica la versione attuale del documento;
    \item Durante la compilazione del file sorgente \LaTeX, viene eseguito un processo automatizzato che integra la versione corretta nel nome del documento, basandosi sulla prima riga del registro delle modifiche. Affinché ciò avvenga correttamente, è necessario specificare l'ultima versione all'interno del registro delle modifiche del documento utilizzando il comando \verb|{\label{Git_Action_Version}X.Y.Z}|.
\end{itemize}

\paragraph{Prima pagina}
Nella prima pagina di ogni documento deve essere presente:
\begin{itemize}
    \item Nome e mail del gruppo;
    \item Nome del documento;
    \item Redattori;
    \item Verificatori;
    \item Destinatari.
\end{itemize}

\paragraph{Indice}
Tutti i documenti devono essere provvisti di un indice dove saranno elencate le varie sezioni e sottosezioni, con la possibilità di raggiungerle direttamente tramite click.
In caso di presenza di figure nel documento, sarà presente anche un indice relativo.

\paragraph{Pié di pagina}
In ogni pié di pagina deve essere presente:
\begin{itemize}
    \item Nome del gruppo;
    \item Nome del documento;
    \item Numero di pagina.
\end{itemize}

\paragraph{Verbali: struttura generale}
\label{sec:Verbali}
I verbali assumono l'importante ruolo di costituire un registro ufficiale dei meeting, atto a riportare in maniera accurata gli argomenti trattati, le decisioni adottate, le azioni da intraprendere e le figure coinvolte.\\
In particolare, è possibile distinguere tra verbali interni, destinati all'uso interno dell'organizzazione, e verbali esterni, che trovano applicazione quando vi sono terze parti coinvolte nelle discussioni o nelle decisioni documentate.\\

La struttura dei verbali è la seguente:
\begin{itemize}
    \item \textbf{Prima pagina} \\
        Oltre alle informazioni comuni a ogni documento vengono specificate:
        \begin{itemize}
            \item Data della riunione e tipologia (Interna,Esterna) nel nome del documento;
            \item Luogo (nel caso non sia un luogo fisico si specifica il canale di comunicazione adottato);
            \item Ora di inizio e fine dell'incontro;
            \item Amministratore;
            \item Partecipanti della riunione (interni ed eventuali esterni);
            \item Il Responsabile (in basso a destra);
        \end{itemize}
    \item \textbf{Corpo del documento} \\
        Nel corpo del documento sono presenti le seguenti sottosezioni:
        \begin{itemize}
            \item \textbf{Revisione del periodo precedente} \\
            Si analizza lo stato delle \textit{attività}\textsubscript{\textit{G}} e, relativamente all'approccio lavorativo, si valutano gli aspetti positivi e le difficoltà incontrate in modo tale da identificare azioni di miglioramento per ottimizzare i \textit{processi}\textsubscript{\textit{G}};
            \item \textbf{Ordine del giorno} \\
            Elenco delle tematiche discusse durante la riunione, accompagnate dai relativi esiti. \\
            Nel caso del verbale esterno, se sono presenti richieste di chiarimenti effettuate alle terze parti coinvolte, saranno incluse nella sottosezione "\textbf{Richieste e chiarimenti}".
        
            \item \textbf{Attività da svolgere}:
            \begin{itemize}
                \item Nel caso dei verbali esterni è presente un elenco in cui ogni voce identifica un attività scaturita da una decisione presa durante il meeting. Queste attività sono descritte in modo discorsivo, eventualmente ad alto livello, e saranno oggetto di discussione nel successivo meeting interno. Durante tale meeting, verranno definite in modo formale le issue, specificando inoltre gli incaricati dell'esecuzione e della verifica;
                \item Nel caso dei verbali interni è presente una tabella dove ogni riga identifica un'\textit{attività}\textsubscript{\textit{G}} e in cui viene specificato:
                    \begin{itemize}
                        \item Nome della task da svolgere;
                        \item Id \textit{issue}\textsubscript{\textit{G}} dell'ITS;
                        \item Verificatore dell'\textit{attività}\textsubscript{\textit{G}}.
                    \end{itemize}
            \end{itemize}
            
        \end{itemize}
    \item \textbf{Ultima pagina} \\
        Nel caso di verbale esterno, in ultima pagina, deve essere presente la firma delle terze parti coinvolte, il luogo e la data.
\end{itemize}

Il template per i verbali è raggiungibile tramite il seguente link: \url{https://github.com/ByteOps-swe/Sorgente-documenti/tree/main/Templates} (Consultato:~26/02/2024)

\subsubsection{Regole tipografiche}
\paragraph*{Documenti del progetto e nome dei file}\label{sec:NomeFile}
Il nome dei documenti generati deve essere omogeneo, con la prima lettera maiuscola ed il resto minuscolo e, ad eccezione dei verbali, deve contenere un riferimento alla versione del documento (vedi notazione per il versionamento nella \textit{sezione \S~\ref{subsubsec:versionamento}}). \\
Nello specifico devono seguire la seguente convenzione:
\begin{itemize}
    \item \textbf{Verbali:} Verbale\_AAAA-MM-DD ;
    \item \textbf{Norme di Progetto:} Norme\_di\_progetto\_vX.Y.Z ;
    \item \textbf{Analisi dei requisiti:} Analisi\_dei\_requisiti\_vX.Y.Z ;
    \item \textbf{Piano di progetto:} Piano\_di\_progetto\_vX.Y.Z ;
    \item \textbf{Glossario:} Glossario\_vX.Y.Z .
\end{itemize}

\paragraph*{Regole sintattiche:}
\begin{itemize}
    \item I titoli delle sezioni iniziano con la lettera maiuscola;
    \item Ciascuna voce di un elenco deve essere seguita da un punto e virgola, ad eccezione dell'ultima che deve terminare con un punto;
    \item Le date vengono scritte nel formato GG/MM/AAAA (giorno/mese/anno);
    \item I numeri razionali si scrivono utilizzando la virgola come separatore tra parte intera e parte decimale;
    \item Quando si fa riferimento a informazioni contenute in un documento diverso da quello attuale, è fondamentale indicare la versione del documento e, se necessario, specificare anche la sezione pertinente (Ex: Norme di Progetto v.1.0.0 sez. 1.2.3).
\end{itemize}

\paragraph*{Stile del testo:}
\begin{itemize}
    \item \textbf{Grassetto}:
    \begin{itemize}
        \item Titoli delle sezioni;
        \item Parole o frasi ritenute di rilievo.
    \end{itemize}
    \item \textbf{Italico}:
    \begin{itemize}
        \item Riferimenti a paragrafi, sezioni e documenti;
        \item Nomi delle aziende;
        \item Termini presenti nel \textit{Glossario};
        \item Nomi delle classi.
    \end{itemize}
\end{itemize}

\paragraph*{Riferimenti}
    \begin{itemize}
        \item Per i riferimenti interni si usa il segunte colore: \textcolor{navyblue}{RGB(0.0,0.0,0.8)};
        \item Per i riferimenti esterni si usa il colore: \textcolor{darkgreen}{RGB(0.0, 0.5, 0.0)}.
\end{itemize}

\subsubsection{Abbreviazioni}
Nei documenti vi sono molte ripetizioni di termini per la quale si possono usare le seguenti abbreviazioni:\\
\vspace{0.2cm}
\begin{center}
    \begin{tabular}{|C{2.5cm}|C{5cm}|}
        \hline
        \textbf{Abbreviazione} & \textbf{Scrittura Estesa}            \\
        \hline
        \textit{RTB}\textsubscript{\textit{G}}                    & Requirements and Technology Baseline \\
        \textit{PB}\textsubscript{\textit{G}}                     & Product Baseline                     \\
        CA                     & Customer Acceptance                  \\
        ITS                    & Issue Tracking System                \\
        CI                     & Configuration Item                   \\
        \textit{SAL}\textsubscript{\textit{G}}                    & Stato Avanzamento Lavori             \\   
        \hline
    \end{tabular}
\end{center}
    
\subsubsection{Strumenti}
Gli strumenti utilizzati dalle \textit{attività}\textsubscript{\textit{G}} di redazione dei documenti vogliono soddisfare il principio adottato di "Documentation as Code".
\begin{itemize}
    \item \textbf{GitHub}: utilizzato per il versionamento, la collaborazione e la distribuzione dei documenti, nonché per automatizzare la loro compilazione;
    \item \textbf{\LaTeX}: utilizzato per la redazione dei documenti, sfruttando il suo potente \textit{sistema}\textsubscript{\textit{G}} di formattazione testuale per creare documenti di alta qualità;
    \item \textbf{Visual Studio Code}: IDE utilizzato dai redattori per la redazione dei documenti grazie al supporto del plugin \textbf{\LaTeX\ Workshop}.
\end{itemize}
