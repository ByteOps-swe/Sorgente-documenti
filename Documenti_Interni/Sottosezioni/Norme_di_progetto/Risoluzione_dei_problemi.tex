\subsection{Risoluzione dei problemi}

\subsubsection{Introduzione}
Il processo di risoluzione dei problemi è un processo che mira ad analizzare e risolvere i problemi (incluse le non conformità) di qualunque natura o fonte, che vengono rilevati durante l'esecuzione di sviluppo, manutenzione o altri \textit{processi}\textsubscript{\textit{G}}.

\vspace{0.2cm}

L'obiettivo è fornire un mezzo tempestivo, responsabile e documentato per garantire che tutti i problemi scoperti siano analizzati e risolti e che siano riconosciute le tendenze. \\
Il processo di risoluzione dei problemi mira ad analizzare e risolvere i problemi, inclusi i casi di non conformità, che emergono durante tutto il ciclo di vita del prodotto. L'obiettivo principale è fornire un approccio tempestivo, responsabile e documentato per affrontare in modo efficace ogni problematica individuata.

\vspace{0.2cm}

Questo processo non si limita a risolvere i problemi immediati, ma si propone anche di identificarne le cause profonde e di adottare misure preventive per evitare che si verifichino nuovamente in futuro. Inoltre, mira a promuovere un ambiente di miglioramento continuo, dove l'apprendimento dagli errori passati gioca un ruolo cruciale nella crescita e nell'ottimizzazione dei \textit{processi}\textsubscript{\textit{G}}.

\vspace{0.2cm}

Un aspetto importante della risoluzione dei problemi è l'adozione di approcci strutturati e metodologie efficaci, che includono la raccolta di dati, l'analisi delle cause alla radice, la valutazione degli impatti e la definizione di azioni correttive e preventive.

È essenziale anche mantenere una documentazione accurata di tutti i problemi identificati e delle relative soluzioni implementate, al fine di garantire la trasparenza, la tracciabilità e la possibilità di revisione nel tempo.

\subsubsection{Gestione dei rischi}
Nel documento \textit{Piano di Progetto}, nella sezione "Analisi dei rischi", vengono identificati dal responsabile tutti i potenziali rischi di progetto, inclusa la probabilità della loro occorrenza e le misure di mitigazione. Per ogni fase di avanzamento, è dedicata una sezione alla documentazione dei problemi riscontrati, con un'analisi del loro impatto e una valutazione dell'esito della mitigazione programmata. Un esito negativo evidenzia una mitigazione inadeguata che richiede modifiche.

\paragraph{Codifica dei rischi}
La convenzione utilizzata per la codifica dei rischi è la seguente: 
\begin{center}
    \texttt{R[Tipologia]-[Probabilità][Priorità]-[Indice]} : \textit{Nome associato al rischio}
\end{center} 

\begin{flushleft}
    \textbf{Tipologia}: \\
    Natura del rischio:
    \begin{itemize}
        \item \textbf{T}: Rischi legati all'utilizzo delle tecnologie;
        \item \textbf{O}: Rischi legati all'organizzazione del gruppo;
        \item \textbf{P}: Rischi legati agli impegni personali dei membri del gruppo.
    \end{itemize}
    \textbf{Probabilità}: \\
    Valore alfabetico che indica la probabilità di occorrenza del rischio:
    \begin{itemize}
        \item \textbf{1}: Alta;
        \item \textbf{2}: Media; 
        \item \textbf{3}: Bassa.
    \end{itemize}
    \textbf{Priorità}: \\
    Valore numerico che indica la pericolosità del rischio:
    \begin{itemize}
        \item \textbf{A}: Alta;
        \item \textbf{M}: Media;
        \item \textbf{B}: Bassa.
    \end{itemize}
    \textbf{Indice}: Valore numerico incrementale che determina univocamente il rischio relativamente ad una specifica tipologia.
\end{flushleft}

\paragraph{Metriche}
\begin{table}[H]
    \centering
    \begin{tabular}{|C{3cm}|C{4cm}|C{3cm}|}
    \hline
    \textbf{Metrica} & \textbf{Nome} & \textbf{Riferimento} \\
    \hline \hline
    M11RNP & \makecell{Rischi non previsti (RNP)} &  \hyperlink{item:M11RNP}{M11RNP}\\ 
    \hline
    \end{tabular}
    \caption{Metriche relative alla gestione dei processi}
\end{table}

\subsubsection{Identificazione dei problemi}
Nel caso in cui un membro del team identifichi un problema, è obbligatorio notificarlo immediatamente al gruppo, e contemporaneamente, deve essere aperta una segnalazione nel \textit{sistema}\textsubscript{\textit{G}} di tracciamento delle \textit{issue}\textsubscript{\textit{G}} (ITS) con l'etichetta "bug" e una descrizione completa del problema. La procedura per l'apertura di una \textit{issue}\textsubscript{\textit{G}} è descritta nel dettaglio nella sezione \hyperlink{par:ticketing}{\textit{Ticketing}}.

\subsubsection{Strumenti}
\begin{itemize}
    \item \textbf{GitHub:} per la segnalazione delle Issue.
\end{itemize}