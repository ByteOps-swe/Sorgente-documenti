\subsection{Riferimenti}
\subsubsection{Riferimenti normativi}
\begin{itemize}
    \item \textbf{Standard ISO/IEC 12207:1995}: \url{https://www.math.unipd.it/~tullio/IS-1/2009/Approfondimenti/ISO_12207-1995.pdf} (Consultato:~17/02/2024) \\
    Standard internazionale che definisce un modello di ciclo di vita del \textit{software}\textsubscript{\textit{G}} e in cui sono definite delle linee guida per la gestione dei \textit{processi}\textsubscript{\textit{G}} \textit{software}\textsubscript{\textit{G}} e le relative \textit{attività}\textsubscript{\textit{G}}.
    
    In sintesi, stabilisce una struttura per organizzare le diverse fasi e le \textit{attività}\textsubscript{\textit{G}} coinvolte nel ciclo di vita del \textit{software}\textsubscript{\textit{G}}, aiutando le organizzazioni a sviluppare prodotti \textit{software}\textsubscript{\textit{G}} in modo più efficiente ed efficace.

    \item \textbf{Capitolato C6}: \url{https://www.math.unipd.it/~tullio/IS-1/2023/Progetto/C6.pdf} (Consultato:~17/02/2024)

\end{itemize}

\subsubsection{Riferimenti informativi}
\begin{itemize}
    \item \textbf{\textit{Glossario v2.0.0}};
    \item \textbf{"Clean Code"} di Robert C. Martin;
    \item \textbf{Documentazione Git}: \url{https://git-scm.com/docs} (Consultato:~17/02/2024);
    \item \textbf{Documentazione \LaTeX}: \url{https://www.latex-project.org/help/documentation/} (Consultato:~17/02/2024);
    \item \textbf{Documentazione Python}: \url{https://docs.python.org/3/} (Consultato: 17/02/2024);
    \item \textbf{Documentazione Kafka}: \url{https://kafka.apache.org/documentation/} (Consultato:~17/02/2024);
    \item \textbf{Documentazione Clickhouse}: \url{https://clickhouse.com/docs} (Consultato:~17/02/2024);
    \item \textbf{Documentazione Grafana}: \url{https://grafana.com/docs/grafana/latest/} (Consultato:~17/02/2024).
\end{itemize}
