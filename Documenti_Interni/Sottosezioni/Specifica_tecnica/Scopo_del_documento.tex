\subsection{Scopo del documento}
Il presente documento si propone come una risorsa esaustiva per la comprensione degli aspetti tecnici chiave del progetto "InnovaCity". La sua finalità principale è fornire una descrizione dettagliata e approfondita di due aspetti centrali: l'architettura implementativa e l'architettura di deployment.

Nel contesto dell'architettura implementativa, è prevista un'analisi approfondita che si estenda anche al livello di design più dettagliato. Ciò include la definizione e la spiegazione dettagliata dei design pattern e degli idiomi utilizzati nel contesto del progetto.

Gli obbiettivi del presente documento sono: motivare le scelte di sviluppo adottate, fungere da guida fondamentale per l'attività di codifica ed infine garantire una completa copertura dei requisiti identificati nel documento \textit{Analisi dei Requisiti v2.0.0}.


\begin{comment}
    Questo documento sarà impiegato dal gruppo ByteOps al fine di fornire un'esposizione dell'architettura del prodotto in fase di sviluppo, offrire informazioni per l'estensione del progetto e descrivere le procedure per l'installazione e lo sviluppo in ambiente locale. Nello specifico, la Specifica Tecnica comprende:
\begin{itemize}
    \item  \textbf{L'architettura logica del sistema:} una visione generale della struttura del sistema, identificando i suoi componenti principali e le relazioni tra di essi;
    \item  \textbf{Le tecnologie impiegate:}le tecnologie e le librerie di terze parti utilizzate per la realizzazione del sistema;
    \item  I pattern architetturali adottati e quelli influenzati dalle tecnologie impiegate;
    \item \textbf{Gli idiomi:} pattern di basso livello architetturale, convenzioni o tecniche specifiche utilizzate nel codice sorgente per affrontare determinati problemi o situazioni comuni durante lo sviluppo del software.
\end{itemize}
\end{comment}

