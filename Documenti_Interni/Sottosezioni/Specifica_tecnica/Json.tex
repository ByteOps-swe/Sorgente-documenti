\subsubsection{JSON (JavaScript Object Notation)}
JSON è un formato di scrittura leggibile dalle persone e facilmente interpretabile dai computer. È utilizzato principalmente per lo scambio di dati strutturati attraverso le reti, come Internet. \todo{non mi piace questa parte}

Il formato JSON si basa su due strutture di dati principali:

\begin{itemize}
  \item \textbf{Oggetti}: Rappresentati da coppie chiave-valore racchiuse tra parentesi graffe \{ \}, dove la chiave è una stringa e il valore può essere un altro oggetto, un array, una stringa, un numero, un booleano o \texttt{null}.
  \item \textbf{Array}: Una raccolta ordinata di valori, racchiusi tra parentesi quadre [ ], in cui ogni elemento può essere un oggetto, un array, una stringa, un numero, un booleano o \texttt{null}.
\end{itemize}

JSON è ampiamente impiegato in diversi contesti, tra cui lo sviluppo web, le API di servizi web e lo scambio di dati tra applicazioni, grazie alla sua sintassi semplice e chiara per la rappresentazione dei dati. La sua struttura basata su testo e la facilità di lettura lo rendono ideale per facilitare l'interazione tra sistemi eterogenei.

\paragraph{Utilizzo nel progetto}
\begin{itemize}
  \item Formato dei messaggi trasmessi dai simulatori dei sensori al broker Kafka;
  \item Configurazione dashboard Grafana.
\end{itemize}
