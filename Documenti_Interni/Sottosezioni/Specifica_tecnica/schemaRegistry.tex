\subsubsection{Schema Registry}
Schema Registry è un componente importante nell'ecosistema di \textit{Apache Kafka}\textsubscript{\textit{G}}, progettato per la gestione e la convalida degli schemi dei dati utilizzati all'interno di un \textit{sistema}\textsubscript{\textit{G}} di messaggistica distribuita.
\paragraph{Versione}
La versione utilizzata è: 
\paragraph{Documentazione}
\href{https://docs.confluent.io/platform/current/schema-registry/index.html}{https://docs.confluent.io/platform/current/schema-registry/index.html}

\paragraph{Funzionalità e Vantaggi di Schema Registry}
Le funzionalità principali di Schema Registry includono:
\begin{itemize}
    \item \textbf{Gestione centralizzata degli schemi}: Fornisce un \textit{repository}\textsubscript{\textit{G}} centralizzato per la gestione degli schemi dei dati.
    \item \textbf{Convalida degli schemi}: Assicura la validità e la compatibilità degli schemi dei dati.
    \item \textbf{Serializzazione e deserializzazione}: Supporta la serializzazione e la deserializzazione dei dati basati sugli schemi su reti distribuite.
    \item \textbf{Governance dei dati}: Contribuisce alla governance dei dati garantendo la qualità, la conformità agli \textit{standard}\textsubscript{\textit{G}} e la tracciabilità dei dati.
\end{itemize}

\paragraph{Casi d'uso di Schema Registry}

Schema Registry è utilizzato in una vasta gamma di casi d'uso, tra cui:

\begin{itemize}
\item \textbf{Garanzia della compatibilità dei dati:} Schema Registry garantisce la compatibilità dei dati tra produttori e consumatori, consentendo l'evoluzione degli schemi dei dati senza interruzioni nei flussi di lavoro.

\item \textbf{Gestione della versione degli schemi:} Fornisce un \textit{sistema}\textsubscript{\textit{G}} per gestire diverse versioni degli schemi dei dati, permettendo agli sviluppatori di aggiornare gli schemi in modo controllato e gestire la migrazione dei dati tra le versioni.

\item \textbf{Conformità agli standard e governance dei dati:} Aiuta a garantire la conformità agli \textit{standard}\textsubscript{\textit{G}} aziendali e normativi, fornendo strumenti per la convalida degli schemi e la tracciabilità delle modifiche nel tempo.

\item \textbf{Collaborazione tra team e integrazione dei sistemi:} Funge da punto centrale per la collaborazione tra team di sviluppo, consentendo loro di condividere, discutere e approvare gli schemi dei dati per un'\textit{integrazione}\textsubscript{\textit{G}} più efficace dei sistemi.

\item \textbf{Controllo della qualità dei dati:} Schema Registry contribuisce a garantire la qualità dei dati, riducendo il rischio di errori dovuti a incompatibilità o a dati non validi all'interno del \textit{sistema}\textsubscript{\textit{G}}.
\end{itemize}


\paragraph{Utilizzo nel progetto}
Nell'ambito del progetto didattico Schema registry permette di validare i messaggi nell'ambito del topic kakfa di appartenenza definendo un contratto che i produttori, ovvero i sensori, dovranno rispettare nell'invio delle misurazioni.