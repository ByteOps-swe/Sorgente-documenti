\subsubsection{Zookeper}
Apache ZooKeeper è un servizio di coordinamento open-source sviluppato dalla Apache Software Foundation. È progettato per fornire funzionalità di coordinazione affidabili e scalabili per applicazioni distribuite.

\paragraph*{Versione:}

\paragraph*{Funzionalità e vantaggi di Apache ZooKeeper:}
Le principali funzionalità e vantaggi di Apache ZooKeeper includono:
\begin{itemize}
    \item \textbf{Servizio di coordinazione centralizzato:}ZooKeeper fornisce un servizio centralizzato per la gestione delle configurazioni, l'elezione del leader, la sincronizzazione dei dati e la notifica di eventi.
    \item \textbf{Affidabilità e scalabilità:} ZooKeeper è progettato per essere affidabile e scalabile, in grado di gestire grandi cluster di applicazioni distribuite.
    \item \textbf{Integrazione con altri software:} ZooKeeper è integrato con molti altri software open-source, tra cui Apache Kafka, Apache Hadoop e Apache HBase.
\end{itemize}
 
\paragraph*{Casi d'uso di Apache ZooKeeper:}

Apache ZooKeeper è utilizzato in una vasta gamma di casi d'uso, tra cui:
\begin{itemize}
    \item Naming service;
    \item Configuration management;
    \item Data Synchronization;
    \item Leader election;
    \item Message queue;
    \item Notification system.
\end{itemize}

\paragraph*{Utilizzo nel progetto:}
ZooKeeper è utilizzato principalmente:
\begin{itemize}
    \item \textbf{Sincronizzazione dei nodi Kafka}: Memorizza la configurazione del cluster Kafka, inclusa la lista dei broker attivi.
    Quando un nuovo broker viene aggiunto, ZooKeeper aggiorna la configurazione e notifica gli altri broker.
    Questo garantisce che tutti i broker abbiano una visione coerente del cluster e possano comunicare correttamente.
    \item \textbf{Coordinamento dello Schema Registry}:Memorizza lo schema per tutti i topic Kafka utilizzati nel progetto.
    Quando un client tenta di produrre un messaggio su un topic, lo Schema Registry verifica lo schema con ZooKeeper.
    Se lo schema è compatibile, il messaggio viene accettato. In caso contrario, il messaggio viene rifiutato.
    Questo garantisce che solo messaggi con schemi validi vengano pubblicati sui topic.
    
\end{itemize}
