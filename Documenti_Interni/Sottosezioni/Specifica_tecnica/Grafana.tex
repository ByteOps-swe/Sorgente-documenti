\subsubsection{Grafana}
Grafana è una piattaforma open-source per la visualizzazione e l'analisi dei dati, utilizzata per creare dashboard interattive e grafici da fonti di dati eterogenee. 
\paragraph{Versione}
La versione utilizzata è: x.x.x
\paragraph{Documentazione}
\href{https://grafana.com/docs/grafana/latest/}{https://grafana.com/docs/grafana/latest/}

\paragraph{Funzionalità e Vantaggi di Grafana}
\begin{itemize}
    \item \textbf{Dashboard interattive}: Creazione di dashboard personalizzate e interattive per visualizzare dati provenienti da diverse fonti in un'unica interfaccia.
    
    \item \textbf{Connessione a sorgenti di dati eterogenee}: Supporto per una vasta gamma di sorgenti di dati, inclusi database, servizi cloud, sistemi di monitoraggio, API e altro ancora.
    
    \item \textbf{Ampia varietà di visualizzazioni}: Selezione di pannelli e visualizzazioni, tra cui grafici a linea, a barre, a torta, termometri, mappe geografiche e altro ancora, per adattarsi alle esigenze specifiche di visualizzazione dei dati.
    
    \item \textbf{Query e aggregazioni flessibili}: Esecuzione di query flessibili e aggregazione dei dati in modi personalizzati per ottenere insight approfonditi dai dati.
    
    \item \textbf{Notifiche e allarmi}: Impostazione di avvisi in base a criteri predefiniti, come soglie di performance, e ricezione di notifiche tramite diversi canali, tra cui email, Slack e molti altri.
    
    \item \textbf{Gestione degli accessi e dei permessi}: Controllo degli accessi e dei permessi degli utenti in modo granulare, gestendo chi può visualizzare, modificare o creare dashboard e pannelli.
    
    \item \textbf{Integrazione con altre applicazioni e strumenti}: Integrazione con una vasta gamma di applicazioni e strumenti, tra cui sistemi di log management, strumenti di monitoraggio delle prestazioni, sistemi di allerta e altro ancora.
    
   \end{itemize}
\paragraph{Casi d'Uso di Grafana}
\begin{itemize}
    \item \textbf{Monitoraggio delle prestazioni}: Monitoraggio in tempo reale delle metriche di sistema come CPU, memoria e rete per identificare e risolvere rapidamente problemi di prestazioni.
    
    \item \textbf{Analisi dei log}: Analisi e visualizzazione dei log delle applicazioni e dell'infrastruttura per individuare pattern e risolvere problemi operativi.
    
    \item \textbf{Monitoraggio dell'infrastruttura}: Monitoraggio dello stato e delle prestazioni di server, servizi cloud, database e altri componenti IT per garantire un funzionamento ottimale dell'infrastruttura.
    
    \item \textbf{DevOps e CI/CD}: Monitoraggio dei processi di sviluppo, test e distribuzione del software per migliorare la collaborazione e l'efficienza del team.
    
    \item \textbf{Monitoraggio di dispositivi IoT}: Monitoraggio dei dispositivi IoT per raccogliere e visualizzare dati di sensori e dispositivi connessi, consentendo una gestione efficiente degli ambienti IoT.
\end{itemize}
\paragraph{Utilizzo nel progetto}
Nel contesto di un progetto che coinvolge la visualizzazione e l'analisi di miliardi di misurazioni di sensori IoT, Grafana viene utilizzato principalmente per:

\begin{itemize}
  \item \textbf{Visualizzazione dei dati}: Grafana consente agli utenti di creare dashboard personalizzate e grafici interattivi che mostrano i dati provenienti dai sensori IoT in modo chiaro e comprensibile. Questi grafici possono essere configurati per visualizzare metriche specifiche nel formato desiderato, consentendo agli utenti di monitorare facilmente le prestazioni dei sensori e rilevare eventuali pattern o anomalie nei dati.
  
  \item \textbf{Analisi dei dati}: Grafana offre una vasta gamma di opzioni per analizzare i dati, inclusi filtri, aggregazioni, calcoli e altro ancora. Gli utenti possono eseguire query sui dati direttamente da Grafana e visualizzare i risultati in grafici, permettendo loro di ottenere una comprensione più approfondita delle tendenze e dei modelli presenti nei dati dei sensori IoT.
  
  \item \textbf{Monitoraggio in tempo reale}: Grafana supporta il monitoraggio in tempo reale dei dati, consentendo agli utenti di visualizzare aggiornamenti istantanei sui valori dei sensori e le metriche correlate. Ciò è particolarmente utile per l'analisi delle prestazioni in tempo reale e per la rilevazione immediata di problemi o anomalie nei dati dei sensori.
  
  \item \textbf{Allerta e notifica}: Grafana offre funzionalità avanzate di allerta e notifica che consentono agli utenti di impostare avvisi basati su condizioni specifiche dei dati. Ad esempio, è possibile configurare Grafana per inviare notifiche via email o tramite servizi di messaggistica istantanea quando un determinato sensore supera una soglia prestabilita o quando si verifica un'anomalia nei dati.
\end{itemize} 