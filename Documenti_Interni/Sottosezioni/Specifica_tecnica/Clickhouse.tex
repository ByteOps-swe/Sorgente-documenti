\subsubsection{Clickhouse} \label{sec:clickHouse}
Clickhouse è un sistema di gestione di database (DBMS) di tipo column-oriented, progettato principalmente per l'analisi di grandi volumi di dati in tempo reale. È un progetto open-source sviluppato da Yandex, un motore di ricerca russo, ed è stato creato per rispondere alle esigenze di elaborazione analitica ad alte prestazioni.
\paragraph{Versione}
La versione utilizzata è: 24.1.6.52
\paragraph{Link download}
\href{https://clickhouse.com/}{https://clickhouse.com/}

\paragraph{Funzionalità e Vantaggi di Clickhouse}
\begin{itemize}
    \item \textbf{ Modello di dati column-oriented:} a differenza dei tradizionali DBMS che memorizzano i dati in modo row-oriented, dove le righe complete sono memorizzate in sequenza, Clickhouse memorizza i dati in modo column-oriented. Questo significa che i dati di ogni colonna sono memorizzati insieme, permettendo una maggiore compressione e velocità di query per le analisi che coinvolgono molte colonne;
    \item \textbf{Architettura Distribuita e scalabilità:} Clickhouse è progettato per funzionare in un ambiente distribuito, consentendo la scalabilità orizzontale per gestire grandi carichi di lavoro;
    \item \textbf{Compressione dei Dati:} utilizza algoritmi efficienti per ridurre lo spazio di archiviazione richiesto per i dati, riducendo i costi di archiviazione;
    \item \textbf{Alte Prestazioni:} ottimizzato per eseguire query analitiche su grandi volumi di dati in tempo reale, garantendo tempi di risposta bassi anche con carichi di lavoro elevati.
    \item \textbf{Supporto per SQL:} supporta un sottoinsieme del linguaggio SQL, consentendo agli sviluppatori di scrivere query complesse per l'analisi dei dati;
    \item \textbf{Integrazione con Strumenti di Business Intelligence (BI):} può essere integrato con strumenti di BI popolari come Tableau, Power BI, Qlik, Grafana per la visualizzazione e l'analisi dei dati.
\end{itemize}


\paragraph{Casi d'Uso di Clickhouse}
Clickhouse è adatto per una vasta gamma di casi d'uso, tra cui:
\begin{itemize}
    \item \textbf{Analisi dei Log:} clickhouse può essere utilizzato per analizzare i log di grandi dimensioni generati da server, applicazioni web e dispositivi IoT;
    \item \textbf{Analisi dei Dati in Tempo Reale:} Clickhouse è ideale per l'analisi dei dati in tempo reale, consentendo agli utenti di eseguire query complesse su flussi di dati in continua evoluzione;
    \item \textbf{Reporting e Dashboard:} Clickhouse può essere utilizzato per generare report e dashboard interattivi per monitorare le prestazioni del business e identificare tendenze.
\end{itemize}


\paragraph{Utilizzo nel progetto}
Nel contesto del progetto, \textbf{Clickhouse} svolge una serie di ruoli cruciali per garantire l'efficacia e l'efficienza dell'analisi dei dati provenienti dai sensori IoT:

\begin{itemize}
  \item \textbf{Integrazione con Kafka}: \textbf{Clickhouse} viene utilizzato per recuperare in tempo reale i dati dal server Kafka, consentendo una continua acquisizione dei dati dai sensori IoT. Questa integrazione permette di assicurare che le informazioni più recenti siano immediatamente disponibili per l'analisi.
  
  \item \textbf{Organizzazione efficiente dei dati}: Grazie alla sua architettura columnar, \textbf{Clickhouse} è in grado di organizzare i dati in modo ottimale per l'analisi di grandi volumi di dati. La struttura columnar consente una compressione dei dati efficace e un accesso rapido alle informazioni, migliorando le prestazioni complessive del sistema.
  
  \item \textbf{Aggregazione rapida dei dati}: \textbf{Clickhouse} offre potenti funzionalità per eseguire operazioni di aggregazione sui dati in modo rapido e incrementale. Ciò significa che è possibile ottenere risposte rapide alle query di aggregazione anche su enormi quantità di dati, consentendo analisi in tempo quasi reale delle misurazioni dei sensori IoT.
  
  \item \textbf{Integrazione con Grafana}: I dati elaborati e aggregati da \textbf{Clickhouse} sono resi disponibili per il reperimento tramite Grafana. Grafana consente di creare dashboard interattive e report visivi basati sui dati dei sensori IoT, offrendo agli utenti un'interfaccia intuitiva per l'analisi e la visualizzazione dei dati.
\end{itemize}
