\subsubsection{Clickhouse} \label{sec:clickHouse}
Clickhouse è un sistema di gestione di database (DBMS) di tipo column-oriented, progettato principalmente per l'analisi di grandi volumi di dati in tempo reale. È un progetto open-source creato per rispondere alle esigenze di elaborazione analitica ad alte prestazioni.

\paragraph{Versione}
Versione utilizzata: 24.2.1.2248
\paragraph{Documentazione:}
\url{https://clickhouse.com/docs/en/intro}

\paragraph{Funzionalità e Vantaggi di Clickhouse}
\begin{itemize}
    \item \textbf{Modello di dati column-oriented:} a differenza dei tradizionali DBMS che memorizzano i dati in modo row-oriented, dove le righe complete sono memorizzate in sequenza, clickhouse memorizza i dati in modo column-oriented. Questo significa che i dati di una stessa colonna vengono memorizzati contiguamente, permettendo una maggiore compressione e velocità di query per le analisi che coinvolgono molte colonne;
    \item \textbf{Architettura distribuita e scalabilità:} Clickhouse è progettato per funzionare in un ambiente distribuito, consentendo la scalabilità orizzontale per gestire grandi carichi di lavoro;
    \item \textbf{Compressione dei dati:} utilizza algoritmi efficienti per ridurre lo spazio di archiviazione richiesto per i dati, riducendo i costi di archiviazione;
    \item \textbf{Alte prestazioni:} Clickhouse è ottimizzato per eseguire query analitiche su grandi volumi di dati in tempo reale, garantendo tempi di risposta bassi anche con carichi di lavoro elevati.
    \item \textbf{Supporto per SQL:} Clickhouse supporta un sottoinsieme del linguaggio SQL;
    \item \textbf{Integrazione con Strumenti di Business Intelligence (BI):} può essere integrato con strumenti di BI popolari come Grafana per la visualizzazione e l'analisi dei dati.
\end{itemize}

\paragraph{Casi d'uso di Clickhouse}
Clickhouse è adatto per una vasta gamma di casi d'uso, tra cui:
\begin{itemize}
    \item \textbf{Analisi dei log:} Clickhouse può essere utilizzato per analizzare i log di grandi dimensioni generati da server, applicazioni web e dispositivi IoT;
    \item \textbf{Analisi dei dati in tempo reale:} Clickhouse è ideale per l'analisi dei dati in tempo reale, consentendo agli utenti di eseguire query complesse su flussi di dati in continua evoluzione;
\end{itemize}

\paragraph{Utilizzo nel progetto}
Nel contesto del progetto, Clickhouse svolge una serie di ruoli cruciali per garantire l'efficacia e l'efficienza dell'analisi e delle persistenza dei dati provenienti dai sensori IoT:

\begin{itemize}
  \item \textbf{Integrazione con Kafka}: Clickhouse viene utilizzato per recuperare in tempo reale i dati dal server Kafka, consentendo una continua acquisizione dei dati dai sensori IoT. Questa integrazione permette di assicurare che le informazioni più recenti siano immediatamente disponibili per l'analisi.
  
  \item \textbf{Organizzazione efficiente dei dati:} Clickhouse è in grado di organizzare grandi volumi di dati in modo ottimale grazie alla sua architettura columnar che consente una compressione dei dati efficace e un accesso rapido alle informazioni, migliorando le prestazioni complessive del sistema.
  
  \item \textbf{Aggregazione rapida dei dati:} Clickhouse offre potenti funzionalità per eseguire operazioni di aggregazione sui dati in modo rapido e incrementale. Ciò significa che è possibile ottenere risposte rapide alle query di aggregazione anche su enormi quantità di dati, consentendo analisi in \textit{quasi-real-time} delle misurazioni dei sensori IoT.
  
  \item \textbf{Integrazione con Grafana:} I dati elaborati e aggregati da Clickhouse sono resi disponibili per il reperimento tramite Grafana. Grafana consente di creare dashboard interattive e report visivi basati sui dati ricevuti offrendo agli utenti un'interfaccia intuitiva per l'analisi e la visualizzazione dei dati.
\end{itemize}
