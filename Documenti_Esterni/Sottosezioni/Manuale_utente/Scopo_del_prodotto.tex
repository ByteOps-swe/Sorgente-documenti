\subsection{Scopo del Prodotto}
L'obiettivo del progetto è quello di creare un'applicazione web per il monitoraggio di una "\textit{Smart City}\textsubscript{\textit{G}}", consentendo un controllo completo sul suo stato di salute. Ciò permetterà di prendere decisioni rapide ed efficaci, oltre ad analizzare gli effetti delle azioni intraprese.\\
La \textit{piattaforma}\textsubscript{\textit{G}} è in grado di fornire informazioni chiare e in tempo reale sullo stato della città tramite una \textit{dashboard}\textsubscript{\textit{G}} \textit{Grafana}\textsubscript{\textit{G}}, che mette a disposizione tutti gli strumenti necessari per l'analisi delle misurazioni provenienti dai sensori.\\
Come detto in precedenza, questa \textit{piattaforma}\textsubscript{\textit{G}} è destinata alle autorità cittadine desiderose di ottenere una visione globale della situazione urbana, fornendo informazioni chiare e in tempo reale sullo stato della città.

\subsection{Accesso alla piattaforma}
La \textit{piattaforma}\textsubscript{\textit{G}} è presentata come una web-application accessibile esclusivamente agli utenti autorizzati. L'accesso al \textit{servizio}\textsubscript{\textit{G}} avviene tramite un \textit{browser}\textsubscript{\textit{G}} web, senza richiedere l'installazione di alcun \textit{software}\textsubscript{\textit{G}} aggiuntivo sul dispositivo dell'utente. Al fine di garantire la massima sicurezza e riservatezza dei dati, l'accesso è limitato esclusivamente agli utenti in possesso del \textit{link}\textsubscript{\textit{G}} e delle credenziali di accesso, le quali vengono fornite dal team amministrativo o da personale autorizzato. Una volta ottenuto il \textit{link}\textsubscript{\textit{G}} e le credenziali, gli utenti possono accedere alla web-application da qualsiasi dispositivo connesso a Internet, garantendo un'esperienza di utilizzo flessibile e accessibile ovunque si trovi