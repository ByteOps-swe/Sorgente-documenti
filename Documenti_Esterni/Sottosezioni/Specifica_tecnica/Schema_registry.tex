\subsubsection{Schema Registry}
Schema Registry è un componente importante nell'ecosistema di \textit{Apache Kafka}\textsubscript{\textit{G}}, progettato per la gestione e la convalida degli schemi dei dati utilizzati all'interno di un \textit{sistema}\textsubscript{\textit{G}} di messaggistica distribuita.
\paragraph{Versione}
Versione utilizzata: 7.6.0
\paragraph{Documentazione}
\url{https://docs.confluent.io/platform/current/schema-registry/index.html} (Consultato: 19/03/2024).

\paragraph{Funzionalità e Vantaggi di Schema Registry}
Le funzionalità principali di Schema Registry includono:
\begin{itemize}
    \item \textbf{Gestione centralizzata degli schemi}: Fornisce un \textit{repository}\textsubscript{\textit{G}} centralizzato per la gestione degli schemi dei dati.
    Contribuisce alla governance dei dati garantendo la qualità, la conformità agli \textit{standard}\textsubscript{\textit{G}} e la tracciabilità dei dati;
    \item \textbf{Convalida degli schemi}: Assicura la validità e la compatibilità degli schemi dei dati;
    \item \textbf{Serializzazione e deserializzazione}: Supporta la serializzazione e la deserializzazione dei dati basati sugli schemi su reti distribuite.
\end{itemize}

\paragraph{Utilizzo nel progetto}
Nell'ambito del progetto didattico schema registry permette di validare i messaggi nell'ambito del topic kakfa di appartenenza definendo un contratto che i produttori, ovvero i sensori, dovranno rispettare nell'invio delle misurazioni.