\subsubsection{Python}
Linguaggio di programmazione ad alto livello, interpretato e multi-\textit{paradigma}\textsubscript{\textit{G}}.

\paragraph{Versione:}
Versione utilizzata: 3.9
\paragraph{Documentazione:}
\url{https://docs.python.org/release/3.9.0/} (Consultato: 19/03/2024).

\paragraph{Utilizzo nel progetto} 
\begin{itemize}
    \item Creazione delle simulazioni dei sensori, incluse le logiche di scrittura e invio dei dati registrati;
    \item Modello per il calcolo del punteggio di salute della città;
    \item Testing.
\end{itemize}

\paragraph{Librerie o framework}

\begin{itemize}
    \item \textbf{Confluent Kafka}
    \begin{itemize}
        \item \textbf{Documentazione:} \url{https://developer.confluent.io/get-started/python/} (Consultato: 19/03/2024);
        \item \textbf{Versione:} 2.3.0;
        \item Libreria \textit{Python}\textsubscript{\textit{G}} che fornisce un insieme completo di strumenti per agevolare la produzione e il consumo di messaggi da \textit{Apache Kafka}\textsubscript{\textit{G}}.
    \end{itemize}
    
    \item \textbf{Faust}
    \begin{itemize}
        \item \textbf{Documentazione:} \url{https://faust.readthedocs.io/en/latest/} (Consultato: 19/03/2024);
        \item \textbf{Versione:} 1.10.4;
        \item \textit{Framework}\textsubscript{\textit{G}} \textit{Python}\textsubscript{\textit{G}} per la creazione di applicazioni di data streaming in tempo reale. Fornisce un'\textit{API}\textsubscript{\textit{G}} dichiarativa e funzionale per definire i flussi di dati e le trasformazioni, consentendo agli sviluppatori di scrivere facilmente applicazioni scalabili e affidabili per il trattamento di grandi volumi di dati in tempo reale.
        
        Faust si integra nativamente con \textit{Apache Kafka}\textsubscript{\textit{G}} e offre funzionalità avanzate come il bilanciamento del carico, la gestione dello stato, la gestione delle \textit{query}\textsubscript{\textit{G}}, e la tolleranza ai guasti, rendendolo una scelta ottimale per lo sviluppo di sistemi di data streaming complessi e robusti.
    \end{itemize}
    
    \item \textbf{Pytest}
    \begin{itemize}
        \item \textbf{Documentazione:} \url{https://docs.pytest.org/en/7.1.x/contents.html} (Consultato: 19/03/2024);
        \item \textbf{Versione:} 8.0.2;
        \item \textit{Framework}\textsubscript{\textit{G}} di testing per \textit{Python}\textsubscript{\textit{G}}, noto per la sua semplicità. Consente agli sviluppatori di scrivere \textit{test}\textsubscript{\textit{G}} chiari e concisi utilizzando una sintassi intuitiva e flessibile.
        
        Pytest supporta una vasta gamma di funzionalità, tra cui \textit{test di unità}\textsubscript{\textit{G}}, \textit{integrazione}\textsubscript{\textit{G}} e accettazione, parametrizzazione dei \textit{test}\textsubscript{\textit{G}} e gestione delle \textit{fixture}\textsubscript{\textit{G}}.

        Merita menzione anche l'utilizzo di \textit{Pytest-asyncio} per testare codice asincrono e \textit{Pytest-cov} per la copertura del codice.
    \end{itemize}
    
    \item \textbf{Pylint}
    \begin{itemize}
        \item \textbf{Documentazione:} \url{https://pylint.readthedocs.io/en/stable/} (Consultato: 19/03/2024);
        \item \textbf{Versione:} 3.1.0;
        \item Strumento di analisi statica per il linguaggio di programmazione \textit{Python}\textsubscript{\textit{G}}. Esamina il codice sorgente per individuare potenziali errori, conformità alle linee guida stilistiche e altre possibili fonti di bug nel codice \textit{Python}\textsubscript{\textit{G}}. Inoltre, valuta anche la qualità del codice in termini di \textit{good practice} di programmazione.
        
        Pylint fornisce un punteggio di qualità del codice e suggerimenti per migliorare la leggibilità, la manutenibilità, sicurezza e la correttezza del codice \textit{Python}\textsubscript{\textit{G}}.
    \end{itemize}
    
    \item \textbf{Clickhouse-connect}
    \begin{itemize}
        \item \textbf{Documentazione:} \url{https://clickhouse.com/docs/en/integrations/python} (Consultato: 19/03/2024);
        \item \textbf{Versione:} 0.7.2;
        \item \textit{ClickHouse}\textsubscript{\textit{G}} Connect è una libreria open source sviluppata per semplificare l'interazione con il \textit{database}\textsubscript{\textit{G}} \textit{ClickHouse}\textsubscript{\textit{G}} tramite il linguaggio di programmazione \textit{Python}\textsubscript{\textit{G}}, viene utilizzata nei \textit{test}\textsubscript{\textit{G}}.
        
        Essa fornisce un'interfaccia per comunicare con \textit{ClickHouse}\textsubscript{\textit{G}}, consentendo agli sviluppatori di eseguire \textit{query}\textsubscript{\textit{G}}, inserire dati e gestire altri aspetti dell'interazione con il \textit{database}\textsubscript{\textit{G}} in modo efficiente e conveniente.
    \end{itemize}
\end{itemize}
