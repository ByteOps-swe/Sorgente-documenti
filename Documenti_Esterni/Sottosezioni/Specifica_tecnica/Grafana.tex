\subsubsection{Grafana}
\textit{Grafana}\textsubscript{\textit{G}} è una \textit{piattaforma}\textsubscript{\textit{G}} \textit{open-source}\textsubscript{\textit{G}} per la visualizzazione e l'analisi dei dati, utilizzata per creare \textit{dashboard}\textsubscript{\textit{G}} interattive e grafici da fonti di dati eterogenee. 

\paragraph{Versione}
Versione utilizzata: 10.4.0

\paragraph{Documentazione}
\url{https://grafana.com/docs/grafana/v10.4/}

\paragraph{Funzionalità e Vantaggi di Grafana}
\begin{itemize}
    \item \textbf{Dashboard interattive:} Creazione di \textit{dashboard}\textsubscript{\textit{G}} personalizzate e interattive per visualizzare dati provenienti da diverse fonti in un'unica interfaccia;
        
    \item \textbf{Ampia varietà di visualizzazioni:} Selezione di pannelli e visualizzazioni, tra cui grafici a linea, a barre, a torta, termometri, mappe geografiche e altro ancora, per adattarsi alle esigenze specifiche di visualizzazione dei dati;
    
    \item \textbf{Query e aggregazioni:} Esecuzione di \textit{query}\textsubscript{\textit{G}} e aggregazione dei dati in modi personalizzati per ottenere insight approfonditi dai dati;
    
    \item \textbf{Notifiche e allarmi:} Impostazione di avvisi in base a criteri predefiniti, come soglie di performance, e ricezione di notifiche tramite diversi canali di comunicazione;
    
    \item \textbf{Gestione degli accessi e dei permessi:} Controllo degli accessi e dei permessi degli utenti in modo granulare, gestendo chi può visualizzare, modificare o creare \textit{dashboard}\textsubscript{\textit{G}} e pannelli;
    
    \item \textbf{Integrazione con altre applicazioni e strumenti:} Integrazione con una vasta gamma di applicazioni e strumenti, tra cui sistemi di \textit{log}\textsubscript{\textit{G}} management, strumenti di monitoraggio delle prestazioni, sistemi di allerta e altro ancora.
  \end{itemize}
  
\paragraph{Casi d'uso di Grafana}
\begin{itemize}
    \item \textbf{Monitoraggio delle prestazioni:} Monitoraggio in tempo reale delle metriche di \textit{sistema}\textsubscript{\textit{G}} come CPU, memoria e \textit{rete}\textsubscript{\textit{G}} per identificare e risolvere rapidamente problemi di prestazioni;
    
    \item \textbf{Analisi dei log:} Analisi e visualizzazione dei \textit{log}\textsubscript{\textit{G}} delle applicazioni e dell'infrastruttura per individuare \textit{pattern}\textsubscript{\textit{G}} e risolvere problemi operativi;
    
    \item \textbf{DevOps e CI/CD:} Monitoraggio dei \textit{processi}\textsubscript{\textit{G}} di sviluppo, \textit{test}\textsubscript{\textit{G}} e distribuzione del \textit{software}\textsubscript{\textit{G}} per migliorare la collaborazione e l'efficienza del team;
    
    \item \textbf{Monitoraggio di dispositivi IoT:} Monitoraggio dei dispositivi IoT per raccogliere e visualizzare dati di sensori e dispositivi connessi, consentendo una gestione efficiente degli ambienti IoT.
\end{itemize}

\paragraph{Utilizzo nel progetto}
Nel contesto del nostro progetto che coinvolge la visualizzazione e l'analisi di grandi quantità di misurazioni, \textit{Grafana}\textsubscript{\textit{G}} viene utilizzato principalmente per:

\begin{itemize}
  \item \textbf{Visualizzazione dei dati:} \textit{Grafana}\textsubscript{\textit{G}} consente agli utenti di visualizzare tramite \textit{dashboard}\textsubscript{\textit{G}} grafici interattivi che mostrano i dati provenienti dai sensori IoT in modo chiaro e comprensibile. Questi grafici consentendo agli utenti di monitorare facilmente le prestazioni dei sensori e rilevare eventuali \textit{pattern}\textsubscript{\textit{G}} o anomalie nei dati.
  
  \item \textbf{Analisi dei dati:} \textit{Grafana}\textsubscript{\textit{G}} offre agli utenti la possibilità di analizzare i dati visualizzati in modo approfondito fornendo opzioni di filtraggio spaziale e temporale e aggregazioni temporali.
  
  \item \textbf{Monitoraggio in tempo reale:} \textit{Grafana}\textsubscript{\textit{G}} supporta il monitoraggio in tempo reale dei dati, consentendo agli utenti di visualizzare aggiornamenti istantanei sui valori dei sensori e le metriche correlate. Ciò è particolarmente utile per la rilevazione immediata di problemi o anomalie nei dati dei sensori.
  
  \item \textbf{Allerta e notifica:} \textit{Grafana}\textsubscript{\textit{G}} permette agli utenti di ricevere avvisi basati su condizioni specifiche delle misurazioni. In particolare, nel nostro caso, invia una notifica tramite discord quando un determinato \textit{sensore}\textsubscript{\textit{G}} supera una soglia prestabilita o quando si verifica un'anomalia nei dati.
\end{itemize} 