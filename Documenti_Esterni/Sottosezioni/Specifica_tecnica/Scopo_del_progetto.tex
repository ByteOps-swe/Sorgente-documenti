\subsection{Scopo del progetto}
Sviluppare una \textit{piattaforma}\textsubscript{\textit{G}} di monitoraggio di una "\textit{Smart City}\textsubscript{\textit{G}}" che consenta di avere sotto controllo lo stato di salute della città in modo tale da prendere decisioni veloci, efficaci ed analizzare poi gli effetti conseguenti.

A tale scopo il \textit{proponente}\textsubscript{\textit{G}} richiede di simulare dei sensori posti in diverse aree per reperire informazioni relative alle condizioni della città come, ad esempio, temperatura, umidità, quantità di polveri sottili nell’aria, livelli dell'acqua in punti strategici, stato di riempimento delle isole ecologiche, guasti elettrici e stato di occupazione delle colonnine di ricarica per macchine elettriche.

I dati trasmessi in tempo reale dai sensori devono poter essere memorizzati in un \textit{database}\textsubscript{\textit{G}} in modo tale da renderli disponibili per la visualizzazione tramite una \textit{dashboard}\textsubscript{\textit{G}}, composta da \textit{widget}\textsubscript{\textit{G}}, quali mappe, tabelle e grafici, per una visione d’insieme delle condizioni della città in tempo reale.

L’applicativo potrà consentire alle autorità locali di prendere decisioni informate e tempestive sulla gestione delle risorse e sull’implementazione di servizi.

L’implementazione di una città monitorata da sensori rappresenta un approccio promettente nell’ottica di ottimizzare l’efficienza e la qualità della vita urbana. Tale \textit{sistema}\textsubscript{\textit{G}} consente una raccolta continua di dati e informazioni cruciali, fornendo una base solida per l’ottimizzazione
dei servizi pubblici, la gestione del traffico, la sicurezza e la sostenibilità ambientale.
