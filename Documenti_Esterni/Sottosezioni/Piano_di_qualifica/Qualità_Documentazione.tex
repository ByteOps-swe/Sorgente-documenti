\subsection{Qualità di processo - Documentazione}

\vspace{0.3cm}

\subsubsection{Errori Ortografici}

\vspace{0.3cm}

\begin{figure}[H]
    \centering
    \includegraphics[width=1\textwidth]{../Images/PianoDiQualifica/errori_ortografici.png}
    \caption{Resoconto errori ortografici}
    \label{fig:Errori ortografici}
\end{figure}

\vspace{0.2cm}

\textbf{RTB} \\
Il grafico mostra l'andamento degli errori ortografici rilevati nei documenti. Si nota come il numero di errori ortografici sia inizialmente alto, ma tenda a diminuire con l'avanzare del progetto.
Questo è dovuto al fatto che il gruppo ha iniziato a prestare maggiore attenzione alla scrittura dei documenti raggiungendo l'ottimo nell'ultimo periodo.

\vspace{0.3cm}

\textbf{PB} \\
Il grafico mostra un calo degli errori nella seconda parte del progetto, ma l'ottimo è stato raggiunto solo nell'ultimo periodo. Questo indica che il gruppo ha migliorato la qualità della scrittura dei documenti attraverso un'attenta ricerca e verifica di eventuali errori, soprattutto grazie all'introduzione di tool per la segnalazione automatica di errori ortografici. Tuttavia, in alcuni casi sono sfuggiti degli errori nonostante gli sforzi profusi.

\subsubsection{Indice di Gulpease}

\vspace{0.3cm}

\begin{figure}[H]
    \centering
    \includegraphics[width=1\textwidth]{../Images/PianoDiQualifica/Gulpease.png}
    \caption{Resoconto indice di Gulpease}
    \label{fig:Indice di Gulpease}
\end{figure}

\vspace{0.2cm}

\textbf{RTB} \\
Dalla valutazione del grafico si nota un tendenza generale di crescita e/o mantenimento dell'indice per ogni documento durante i vari periodi considerati. \\
Si osserva che il glossario presenta un indice di Gulpease molto basso, il che è attribuibile alla sua natura tecnica e alla conseguente impossibilità di aumentare tale indice. \\
Gli altri documenti, invece, mostrano un indice di Gulpease elevato, in parte dovuto al loro contenuto meno tecnico e più accessibile.

\vspace{0.3cm}

\textbf{PB} \\
Dall’analisi del grafico, si osserva che, nei periodi post revisione RTB, i documenti preesistenti hanno conservato un indice di Gulpease pressoché immutato. Questo indica che il grado di leggibilità del loro contenuto è rimasto stabile nel tempo, senza variazioni significative nella complessità. \\
I documenti aggiuntivi includono il \textit{Manuale utente v2.0.0} e la \textit{Specifica tecnica v2.0.0}. Il primo mostra un indice di Gulpease superiore, mentre il secondo un indice inferiore. Questo è attribuibile alla natura più accessibile del primo documento e alla natura più tecnica del secondo documento. \\
La totalità dei documenti è comprensibile da chiunque abbia la licenza media.