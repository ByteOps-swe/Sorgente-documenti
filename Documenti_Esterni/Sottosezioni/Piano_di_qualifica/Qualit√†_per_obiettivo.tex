\subsection{Qualità per obiettivo}
Le metriche menzionate in precedenza vengono ora categorizzate secondo la struttura delineata nello \textit{standard}\textsubscript{\textit{G}} ISO/IEC 12207:1995, che le suddivide nei \textit{processi}\textsubscript{\textit{G}} primari, di supporto e organizzativi. Questo adattamento semplificato è stato realizzato per allineare le metriche alle specifiche esigenze del progetto.

\subsubsection{Processi primari}
\paragraph{Analisi dei requisiti}
L'Analisi dei Requisiti coinvolge la raccolta, l'analisi e la definizione dei requisiti del \textit{sistema}\textsubscript{\textit{G}} che si intende sviluppare.

Coinvolge l'interazione con gli \textit{stakeholder}\textsubscript{\textit{G}} per comprendere le loro esigenze e tradurle in requisiti dettagliati e comprensibili per il team di sviluppo.

Un'\textit{analisi dei requisiti}\textsubscript{\textit{G}} efficace è cruciale per garantire che il \textit{software}\textsubscript{\textit{G}} soddisfi le aspettative degli utenti finali.

\vspace{0.4cm}

\begin{longtable}{|C{1.9cm}|L{4cm}|L{2.2cm}|L{2.1cm}|}
    \hline
    \textbf{Metrica} & \centering{\textbf{Nome}} & \textbf{Valore di\linebreak accettazione} & {\textbf{Valore \linebreak preferibile}} \\
    \hline \hline
    
    \textbf{M18PROS} & Percentuale di requisiti obbligatori soddisfatti & $ 100\%$  & $ 100\%$ \\
    \hline

    \textbf{M19PRDS} & Percentuale di requisiti desiderabili soddisfatti & $\geq 0\%$ & $100\%$ \\
    \hline

    \textbf{M20PRPS} & Percentuale di Requisiti oPzionali Soddisfatti & $\geq 0\%$ & $100\%$ \\
    \hline

    \caption{Analisi dei requisiti - Metriche e indici di qualità.}
    \label{tab:analisi_requisiti_progetto}
\end{longtable}

\vspace{0.5cm}

\paragraph{Progettazione}
La Progettazione è un processo in cui vengono definite le specifiche tecniche e architetturali del \textit{software}\textsubscript{\textit{G}} che si intende sviluppare. Questo processo traduce i requisiti raccolti durante la fase di acquisizione in un piano strutturato e dettagliato per la creazione del \textit{software}\textsubscript{\textit{G}}.

\vspace{0.4cm}

\begin{longtable}{|C{1.9cm}|L{4cm}|L{2.2cm}|L{2.1cm}|}
    \hline
    \textbf{Metrica} & \centering{\textbf{Nome}} & \textbf{Valore di\linebreak accettazione} & {\textbf{Valore \linebreak preferibile}} \\
    \hline \hline

    \textbf{M25ATC} & Accoppiamento tra classi & $\leq 4$  & $\leq 2$ \\
    \hline

    \textbf{M30PG} & Profondità delle gerarchie & $\leq 5$  & $\leq 3$ \\
    \hline

    \textbf{M32FU} & Facilità di utilizzo & $\leq 7$ \textit{click}  & $\leq 5$ \textit{click} \\
    \hline

    \textbf{M33TA} & Tempo di apprendimento & $\leq 15$ \textit{minuti}  & $\leq 10$ \textit{minuti} \\
    \hline

    \caption{Progettazione - Metriche e indici di qualità.}
    \label{tab:progettazione_progetto}
\end{longtable}

\pagebreak

\paragraph{Fornitura}
La Fornitura è un processo che consiste nel decidere procedure e risorse adatte allo sviluppo del progetto.

\begin{longtable}{|C{1.9cm}|L{4cm}|L{2.2cm}|L{2.1cm}|}
    \hline
    \textbf{Metrica} & \centering{\textbf{Nome}} & \textbf{Valore di\linebreak accettazione} & {\textbf{Valore \linebreak preferibile}} \\
    \hline \hline

    \textbf{M2EAC} & Estimated at completion & $\pm 5\%$ rispetto al preventivo & Pari al preventivo \\
    \hline

    \textbf{M3CPI} & Cost performance index & $\pm 10\%$ & $0\%$ \\
    \hline

    \textbf{M5AC} & Actual cost & $\geq 0 $ & $ \leq$ \textit{EAC}\textsubscript{\textit{G}}  \\
    \hline

    \textbf{M7EV} & Earned value & $\geq 0 $ & $\leq$ \textit{EAC}\textsubscript{\textit{G}}  \\
    \hline

    \textbf{M8PV} & Planned value & $\geq 0  $ & $ \leq$ BAC  \\
    \hline

    \textbf{M9ETC} & Estimate to complete & $\geq 0  $ & $ \leq$ \textit{EAC}\textsubscript{\textit{G}}  \\
    \hline

    \caption{Fornitura - Metriche e indici di qualità.}
    \label{tab:controllo_progetto}
\end{longtable}

\vspace{0.5cm}

\paragraph{Codifica}
La fase di codifica è essenziale in quanto trasforma il progetto e le specifiche del \textit{software}\textsubscript{\textit{G}} in istruzioni comprensibili dalla macchina, permettendo al prodotto \textit{software}\textsubscript{\textit{G}} di prendere vita e funzionare effettivamente.

\vspace{0.4cm}

\begin{longtable}{|C{1.9cm}|L{4cm}|L{2.2cm}|L{2.1cm}|}
    \hline
    \textbf{Metrica} & \centering{\textbf{Nome}} & \textbf{Valore di\linebreak accettazione} & {\textbf{Valore \linebreak preferibile}} \\
    \hline \hline

    \textbf{M26MCCM} & Complessità ciclomatica per metodo & $\leq 5$ & $\leq 3$ \\
    \hline

    \textbf{M27PM} & Parametri per metodo & $\leq 6$ & $\leq 5$ \\
    \hline

    \textbf{M28APC} & Attributi per classe & $\leq 6$ & $\leq 4$ \\
    \hline

    \textbf{M29LCM} & Linee di codice per metodo & $\leq 30$ & $\leq 20$ \\
    \hline

    \textbf{M31TMR} & Tempo medio di risposta & $\leq 10$ \textit{secondi}  & $\leq 4$ \textit{secondi} \\
    \hline

    \textbf{M34VBS} & Versioni dei browser supportate & $\geq 80\%$ & $100\%$ \\
    \hline

    \caption{Codifica - Metriche e indici di qualità.}
    \label{tab:metriche}
\end{longtable}

\pagebreak

\subsubsection{Processi di supporto}

\paragraph{Documentazione}
La Documentazione è un processo essenziale che coinvolge la creazione e la gestione di documenti correlati allo sviluppo del \textit{software}\textsubscript{\textit{G}}. Una documentazione accurata e completa è fondamentale per comprendere, mantenere e supportare il \textit{software}\textsubscript{\textit{G}} nel tempo.

\begin{longtable}{|C{1.9cm}|L{4cm}|L{2.2cm}|L{2.1cm}|}
    \hline
    \textbf{Metrica} & \centering{\textbf{Nome}} & \textbf{Valore di\linebreak accettazione} & {\textbf{Valore \linebreak preferibile}} \\
    \hline \hline

    \textbf{M22CO} & Correttezza ortografica & $0$ & $0$ \\
    \hline

    \textbf{M23IG} & Indice gulpease & $\geq 40$ & $\geq 60$ \\
    \hline

    \caption{Documentazione - Metriche e indici di qualità.}
    \label{tab:metriche_testo}
\end{longtable}

\vspace{0.5cm}

\paragraph{Verifica}
La Verifica è un processo che assicura che i prodotti del \textit{software}\textsubscript{\textit{G}} siano conformi ai requisiti specificati e agli \textit{standard}\textsubscript{\textit{G}} stabiliti. Coinvolge l'analisi, l'esecuzione di \textit{test}\textsubscript{\textit{G}} e l'ispezione dei prodotti \textit{software}\textsubscript{\textit{G}} per identificare e correggere eventuali difetti o discrepanze.

\vspace{0.4cm}

\begin{longtable}{|C{1.9cm}|L{4cm}|L{2.2cm}|L{2.1cm}|}
    \hline
    \textbf{Metrica} & \centering{\textbf{Nome}} & \textbf{Valore di\linebreak accettazione} & {\textbf{Valore \linebreak preferibile}} \\
    \hline \hline

    \textbf{M15SC} & Statement coverage & $\geq 80\%$ & $100\%$ \\
    \hline

    \textbf{M16BC} & Branch coverage & $\geq 80\%$ & $100\%$ \\
    \hline

    \textbf{M17CNC} & CoNdition coverage & $\geq 80\%$ & $100\%$ \\
    \hline

    \textbf{M13PCTS} & Percentuale di casi di \textit{test}\textsubscript{\textit{G}} superati & $\geq 80\%$ & $100\%$ \\
    \hline

    \textbf{M14PCTF} & Percentuale di casi di \textit{test}\textsubscript{\textit{G}} falliti & $\leq 20\%$ & $0\%$ \\
    \hline

\caption{Verifica - Metriche e indici di qualità.}
\label{tab:metriche_verifica}
\end{longtable}

\vspace{0.5cm}

\paragraph{Gestione dei rischi}
Questo processo implica l'identificazione, l'analisi, la valutazione e il controllo dei rischi associati allo sviluppo del \textit{software}\textsubscript{\textit{G}}. 

\vspace{0.4cm}

\begin{longtable}{|C{1.9cm}|L{4cm}|L{2.2cm}|L{2.1cm}|}
    \hline
    \textbf{Metrica} & \centering{\textbf{Nome}} & \textbf{Valore di\linebreak accettazione} & {\textbf{Valore \linebreak preferibile}} \\
    \hline \hline

    \textbf{M11RNP} & Rischi non previsti & $\leq 5$ & $0$ \\
    \hline

    \caption{Gestione dei rischi - Metriche e indici di qualità.}
    \label{tab:tabella2}
\end{longtable}

\vspace{0.5cm}

\paragraph{Gestione della qualità}
Questo processo riguarda l'implementazione di \textit{standard}\textsubscript{\textit{G}}, procedure e metodologie atte a garantire che il \textit{software}\textsubscript{\textit{G}} soddisfi i requisiti di qualità stabiliti.

\vspace{0.4cm}

\begin{longtable}{|C{1.9cm}|L{4cm}|L{2.2cm}|L{2.1cm}|}
    \hline
    \textbf{Metrica} & \centering{\textbf{Nome}} & \textbf{Valore di\linebreak accettazione} & {\textbf{Valore \linebreak preferibile}} \\
    \hline \hline

    \textbf{M1PMS} & Percentuale di metriche soddisfatte & $\geq 80\%$ & $100\%$ \\
    \hline

    \caption{Gestione della qualità - Metriche e indici di qualità.}
    \label{tab:gestione_metriche_testo}
\end{longtable}

\vspace{0.5cm}

\subsubsection{Processi organizzativi}

\paragraph{Pianificazione}
La Pianificazione organizza obiettivi, risorse e tempistiche per guidare il successo di un progetto.

\vspace{0.4cm}

\begin{longtable}{|C{1.9cm}|L{4cm}|L{2.2cm}|L{2.1cm}|}
    \hline
    \textbf{Metrica} & \centering{\textbf{Nome}} & \textbf{Valore di\linebreak accettazione} & {\textbf{Valore \linebreak preferibile}} \\
    \hline \hline

    \textbf{M6SV} & Schedule variance & $\geq -10\%$ & $0\%$ \\
    \hline

    \textbf{M4BV} & Budget variance & $\geq -10\%$ & $0\%$ \\
    \hline

    \textbf{M12VR} & Variazione dei requisiti & $\leq 3$ & $0$ \\
    \hline

    \textbf{M21IF} & Implementazione delle funzionalità & $ 100\%$ & $ 100\%$ \\
    \hline

    \caption{Pianificazione - Metriche e indici di qualità.}
    \label{tab:metriche_pianificazione}
\end{longtable}

\vspace{0.5cm}

\paragraph{Miglioramento}
Il processo di miglioramento mira a identificare le aree che possono essere ottimizzate o migliorate.

\vspace{0.4cm}

\begin{longtable}{|C{1.9cm}|L{4cm}|L{2.2cm}|L{2.1cm}|}
    \hline
    \textbf{Metrica} & \centering{\textbf{Nome}} & \textbf{Valore di\linebreak accettazione} & {\textbf{Valore \linebreak preferibile}} \\
    \hline \hline

    \textbf{M24DE} & Densità errori & $\leq 10\%$ & $ 0\%$ \\
    \hline

    \caption{Miglioramento - Metriche e indici di qualità.}
    \label{tab:metriche_miglioramento}
\end{longtable}

\pagebreak