\subsection{Pianificazione}\label{subsec:Pianificazione}
    In conformità con la filosofia di sviluppo moderna e dinamica, abbiamo scelto di adottare il modello Agile, con un focus specifico sul \textit{framework}\textsubscript{\textit{G}} Scrum. \\
    Il \textit{framework}\textsubscript{\textit{G}} Scrum, con le sue pratiche iterative e collaborative, offre una risposta efficace alle sfide e alle mutevoli esigenze dello sviluppo \textit{software}\textsubscript{\textit{G}}.\\
    Attraverso l’implementazione dello Scrum, il nostro team mira a ottenere numerosi benefici positivi che influenzeranno in modo significativo il successo del progetto.
    
\subsubsection{Vantaggi del Modello Agile e Scrum}
    L'adozione del modello Agile, e in particolare del \textit{framework}\textsubscript{\textit{G}} Scrum, introduce una serie di lati positivi che contribuiranno al raggiungimento dei nostri obiettivi di progetto.
    Alcuni dei principali vantaggi che ci aspettiamo di acquisire includono:

\begin{itemize}
    \item \textbf{Flessibilità e Adattabilità:}
        il \textit{framework}\textsubscript{\textit{G}} Scrum consente una rapida risposta ai cambiamenti nei requisiti del cliente, garantendo una maggiore flessibilità durante tutto il ciclo di sviluppo;
    \item \textbf{Collaborazione e Comunicazione:}
        la struttura collaborativa del \textit{framework}\textsubscript{\textit{G}} Scrum promuove una comunicazione aperta e continua tra i membri del team e le parti interessate, migliorando la comprensione reciproca e la condivisione di conoscenze;
        \begin{itemize}
            \item In particolare con l'azienda \textit{proponente}\textsubscript{\textit{G}} sono fissati \textit{SAL}\textsubscript{\textit{G}} \textit{(Stato Avanzamento Lavori)} ogni due settimane. \\
            Successivamente alla revisione \textit{RTB}\textsubscript{\textit{G}} si è concordato con l'azienda \textit{proponente}\textsubscript{\textit{G}} di effettuare un \textit{SAL}\textsubscript{\textit{G}} a settimana anziché due.
        \end{itemize}
    \item \textbf{Consegna Incrementale:}
        attraverso la pratica di rilasci incrementali, il \textit{framework}\textsubscript{\textit{G}} Scrum consente la distribuzione graduale delle funzionalità, fornendo valore al cliente fin dalle prime fasi del progetto;
    \item \textbf{Miglioramento Continuo:}
        le retrospettive regolari incoraggiano il miglioramento continuo del processo, permettendo al team di identificare e risolvere eventuali problematiche in modo tempestivo.
\end{itemize}

La scelta di adottare il \textit{framework}\textsubscript{\textit{G}} Scrum riflette la nostra dedizione a fornire un prodotto di qualità, rispondendo in modo efficiente ai cambiamenti e alle esigenze del cliente.

\pagebreak

\subsubsection{Gestione e monitoraggio dell'avanzamento del progetto}
In accordo con il \textit{proponente}\textsubscript{\textit{G}}, si è concordato di organizzare l'avanzamento del progetto in periodi di durata prefissata seguendo un approccio simile agli sprint relativi al \textit{framework}\textsubscript{\textit{G}} Scrum.

Durante ciascun periodo, in collaborazione con l'azienda e i membri del team, verranno selezionate le \textit{attività}\textsubscript{\textit{G}} da svolgere.

La scelta dei task da svolgere per ogni periodo si baserà sulla loro importanza strategica e sulla fattibilità di completarle entro la durata del periodo di riferimento. Nel caso in cui alcune \textit{attività}\textsubscript{\textit{G}} non vengano portate a termine entro il periodo determinato, verranno riportate nel consuntivo di periodo e proseguiranno nel periodo successivo.

Ogni periodo sarà documentato attraverso una tabella esaustiva in cui saranno identificate le task relative a ciascun ruolo. Per ogni \textit{attività}\textsubscript{\textit{G}} verrà indicato lo stato di completamento, i tempi previsti ed effettivi, e i costi associati.

Al termine di ciascun periodo, sarà calcolato il costo totale del progetto fino a quel momento, fornendo una chiara visione del progresso complessivo.

Inoltre ogni periodo conterrà una discussione sui rischi occorsi e sull'esito della loro mitigazione seguendo quanto definito nella \textit{sezione \ref{sec:AnalisiRischi}}.

I dati riportati per ciascun periodo rappresentano un riepilogo delle informazioni inserite durante la fase di pianificazione e di preventivazione da parte del responsabile, nonché delle registrazioni orarie effettuate autonomamente dai membri del team tramite il foglio Google condiviso, appositamente utilizzato per questo scopo.

\subsubsection{Durata dei periodi}
\begin{itemize}
    \item \textbf{Requirements and Technology Baseline} \\
    Assieme alla proponente si è concordato di adottare periodi di durata bisettimanale, considerando l'inesperienza del team nella gestione di un progetto di tale complessità. Questa scelta è stata motivata anche dalla necessità di assicurare la possibilità di riuscire a realizzare un'adeguata quantità di materiale da poter valutare durante ciascun SAL.
    \item \textbf{Productt Baseline} \\
    In seguito alla conclusione della sessione di esami invernale, data l'alta disponibilità del team, è stata presa la decisione di organizzare i periodi del progetto in periodi settimanali anziché bisettimanali. Questo nuovo approccio permetterà di massimizzare l'efficienza e la flessibilità delle attività, consentendo di adattarsi più prontamente alle eventuali variazioni e di mantenere un ritmo di lavoro costante e dinamico.
\end{itemize}

\paragraph{Descrizione tabella dei periodi}\label{sec:DescrTabella}

Di seguito è presentata la struttura della tabella che verrà utilizzata per ogni periodo, contenente la pianificazione delle \textit{attività}\textsubscript{\textit{G}}. Nella colonna 'Avanzamento atteso' sono presenti le \textit{attività}\textsubscript{\textit{G}} pianificate suddivise per ruoli e ambiti, indicando il preventivo delle ore e dei costi per ciascuna \textit{attività}\textsubscript{\textit{G}}, oltre al consuntivo che indica se l'\textit{attività}\textsubscript{\textit{G}} è stata completata, con le ore e i costi effettivamente sostenuti.

\vspace{0.2cm}

Ogni \textit{attività}\textsubscript{\textit{G}} contiene le informazioni appena esposte sia per la task, ovvero l'effettivo compito da svolgere,  sia per la verifica che richiede tale task. \\
La tabella, accessibile a tutto il team come foglio Google condiviso, viene compilata dal responsabile nella sezione relativa alla pianificazione delle \textit{attività}\textsubscript{\textit{G}} e ai preventivi all'inizio del periodo, mentre la parte riguardante il consuntivo viene compilata autonomamente dai membri del team.

\vspace{0.2cm}

Le \textit{attività}\textsubscript{\textit{G}} elencate nella colonna 'Avanzamento atteso' non sono destinate a essere il principale punto di riferimento per i membri del team riguardo ai compiti da svolgere. A tale scopo infatti, vengono generate \textit{issue}\textsubscript{\textit{G}} nell'Issue Tracking System (ITS), le quali sono più esplicative, dettagliate e assegnate ad un unico membro. \\
La colonna 'Avanzamento atteso' funge da riferimento generico per le \textit{attività}\textsubscript{\textit{G}} pianificate, permettendo di identificarle per poter allegare i preventivi e i consuntivi associati e comprendere l'incremento apportato da ciascuna di esse.

\vspace{0.5cm}

\begin{figure}[H]
    \centering
    \includegraphics[width=0.9\textwidth]{../Images/spiegazioneTabella.png}
    \caption{Descrizione tabella} 
    \label{fig:spiegazioneTabella} 
\end{figure}