\subsection{Presentazione della struttura espositiva dei periodi}
Ogni periodo di avanzamento verrà esposto in seguito nel seguente formato:
\begin{enumerate}
    \item \textbf{Considerazioni:} considerazioni retrospettive e consuntive sul periodo effettuate una volta terminato;

    \pagebreak
    
    \item \textbf{Gestione dei rischi:}
            \begin{itemize}
                \item Rischi attesi e occorsi;
                \item Rischi attesi ma non occorsi;
                \item Rischi non attesi ma occorsi.
            \end{itemize}
        Nel caso in cui i rischi si verifichino essi conterranno considerazioni relative a:
        \begin{itemize}
            \item \textbf{Esito mitigazione:} considerazioni sulla validità della mitigazione pianificata;
            \item \textbf{Impatto:} impatto avuto nelle \textit{attività}\textsubscript{\textit{G}} pianificate.
        \end{itemize}
    \item \textbf{Definizione ruoli:} esposizione dei ruoli occupati dai membri del team nel periodo;
    \item \textbf{Pianificazione attività divise per ruoli con consuntivo e preventivo orario e dei costi:}
    la tabella, descritta in dettaglio nella \textit{sezione~\S~\ref{sec:DescrTabella}}, svolge simultaneamente il ruolo di pianificazione e stima delle risorse durante la compilazione iniziale da parte del responsabile, nonché quello di rendicontazione delle risorse e di monitoraggio dell'avanzamento effettivo. L'obiettivo è fornire una visione complessiva che rappresenti efficacemente l'esito del periodo in esame.
    Al di sotto della tabella, considerando i dati presentati, saranno incluse le osservazioni del responsabile riguardanti il totale speso fino al periodo in questione, la percentuale di \textit{attività}\textsubscript{\textit{G}} svolte rispetto a quelle pianificate per il periodo, nonché il nuovo preventivo a finire rivalutato al termine delle \textit{attività}\textsubscript{\textit{G}}. Inoltre, verrà considerata la necessità di rivalutare le \textit{attività}\textsubscript{\textit{G}} successive al termine di questo periodo;
    \item \textbf{Riferimento diagrammi di Gantt:} attraverso un click sul link "Vai al Diagramma di Gantt", è possibile raggiungere la parte riguardante i diagrammi di Gantt di \textit{GitHub}\textsubscript{\textit{G}} che il team ha creato. Una volta entrati, se la view non è già impostata correttamente, bisognerà cliccare in alto a destra su "Date fields" e impostare come "Start date" -> "Inizio" e come "Target date" -> "Scadenza". Successivamente, bisognerà cliccare in alto a sinistra la freccetta diretta verso il basso vicino alla scritta "Diagrammi di Gantt". Una volta che il menù a tendina si sarà aperto, cliccare prima su "Group by" e poi su "Milestone". In questo modo si arriverà ad avere la visualizzazione voluta dal nostro team;
    \item \textbf{Grafico a torta dello stato avanzamento dei lavori};
    \item \textbf{Preventivo orario:} espone le informazioni quali le ore preventivate svolte dai membri, nei ruoli che la tabella descritta nella \textit{sezione~\S~\ref{sec:DescrTabella}} non contiene; 
    \item \textbf{Consuntivo orario:} espone le informazioni quali le ore consuntivate svolte dai membri, nei ruoli che la tabella descritta nella \textit{sezione~\S~\ref{sec:DescrTabella}} non contiene. 
\end{enumerate}
