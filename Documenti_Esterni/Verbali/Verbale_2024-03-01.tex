\documentclass{article}
\usepackage[utf8]{inputenc}
\usepackage[absolute]{textpos}
\usepackage[default]{raleway}
\usepackage{titlesec, comment, tabularx, makecell, listings, array, setspace, geometry, graphicx, xcolor, xparse, fancyvrb, relsize, fancyhdr, booktabs, multirow, hyperref}
\usepackage{colortbl}
\usepackage{todonotes}
\renewcommand{\headrulewidth}{0pt}

\definecolor{light-gray}{gray}{0.92}

\lstdefinestyle{code}{
    frame=single,
    framesep=1mm,
    rulecolor=\color{light-gray},
    backgroundcolor=\color{light-gray},
    basicstyle=\ttfamily,
}

% ----------------------------- Definizione tabella ---------------------------

\newcolumntype{C}[1]{>{\centering\arraybackslash}m{#1}}

% ------------------------------Metadati indice --------------------------------
\title{\textbf{\fontsize{28}{6}\selectfont Indice}}
\author{\fontsize{14}{6}\selectfont ByteOps}
\date{Marzo 1, 2024}


% -----------------------------Creazione footer --------------------------------

\pagestyle{fancy}
\fancyhf{}
\renewcommand{\footrulewidth}{0.4pt}
\lfoot{
    \parbox[c]{2cm}{\includegraphics[width=2cm]{../../Images/logo.png}}
    \textcolor[RGB]{120, 120, 120}{$\cdot$ Verbale Esterno}
}
\rfoot{\thepage}

% --------------------------Modifica formato hyperlinks ------------------------

\hypersetup{
    colorlinks=true,
    linkcolor=black,
    filecolor=black,      
    pdftitle={Verbale Esterno 01/03/2024},  %inserisci data verbale
    pdfpagemode=FullScreen,
}

% ------------------------------- Valore sotto-paragrafi indice --------------------------------------

\setcounter{secnumdepth}{4}
\setcounter{tocdepth}{4}

\titleformat{\section}
{\normalfont\huge\bfseries}{\thesection}{0.2cm}{}
\titlespacing*{\paragraph}{0pt}{0.5cm}{0.1cm}

\titleformat{\subsection}
{\normalfont\Large\bfseries}{\thesubsection}{0.2cm}{}
\titlespacing*{\paragraph}{0pt}{0.5cm}{0.1cm}

\titleformat{\subsubsection}
{\normalfont\large\bfseries}{\thesubsubsection}{0.2cm}{}
\titlespacing*{\paragraph}{0pt}{0.5cm}{0.1cm}

\titleformat{\paragraph}
{\normalfont\normalsize\bfseries}{\theparagraph}{0.2cm}{}
\titlespacing*{\paragraph}{0pt}{0.5cm}{0.1cm}

% ------------------------------- Front Page ---------------------------------------

\begin{document}

% --------------------------Aggiunta firma finale ------------------------
\begin{textblock*}{\textwidth}(0.85\textwidth, 1.16\textheight)
    Il responsabile: L. Skenderi
\end{textblock*}
% ------------------------------------------------------------------------

\pagestyle{fancy}
\begin{center}
\includegraphics[width = 0.7\textwidth]{../../Images/logo.png} \\
\vspace{0.2cm}
\textcolor[RGB]{60, 60, 60}{\textit{ByteOps.swe@gmail.com}} \\
\vspace{1cm}
\fontsize{16}{6}\selectfont Verbale Esterno $\cdot$ Data: 01/03/2024 \\
\vspace{0.5cm}
\end{center}

\section*{Informazioni documento}
\def\arraystretch{1.2}
\begin{tabular}{>{\raggedleft\arraybackslash}p{0.3\textwidth}|>{\raggedright\arraybackslash}p{0.6\textwidth}c}
\hline
\addlinespace
\textbf{Luogo} & Google Meet \vspace{10pt} \\
\textbf{Orario} & 16:30 - 17:15 \vspace{10pt} \\
\textbf{Redattore} & F. Pozza \vspace{10pt} \\
\textbf{Verificatore} & E. Hysa \vspace{10pt} \\
\textbf{Amministratore} & F. Pozza \vspace{10pt} \\
\textbf{Destinatari} & T. Vardanega \\ & R. Cardin \vspace{10pt} \\
\multirow[t]{7}{*}{\textbf{Partecipanti interni}} & A. Barutta \\ & E. Hysa \\ & R. Smanio \\ & D. Diotto \\ & F. Pozza \\ & L. Skenderi \\ & N. Preto \vspace{10pt} \\
\multirow[t]{3}{*}{\textbf{Partecipanti esterni}} & A. Dorigo \\ & D. Zorzi \\ & F. Pallaro \\ 
\end{tabular}
\pagebreak 

% ------------------------- Changelog ----------------------------

\section*{Registro delle modifiche}

\begin{tabular}{|C{2.5cm}|C{2.5cm}|C{2.5cm}|C{2.5cm}|C{2.5cm}|}
    \hline
    \textbf{Versione} & \textbf{Data} & \textbf{Autore} & \textbf{Verificatore} & \textbf{Dettaglio} \\
    \hline \hline
    0.0.1 & 02/03/2024 & F. Pozza & E. Hysa & Redazione documento \\
    \hline
\end{tabular}
\pagebreak

% ------------------------- Generazione automatica indice ----------------------
\setstretch{1.5}
\maketitle
\thispagestyle{fancy}
\tableofcontents
\setstretch{1.2}
\pagebreak

% ------------------------ INIZIO DOCUMENTO ----------------------
\flushleft

\section{Revisione del periodo precedente}
Durante la prima parte del meeting, è stato presentato all'azienda \textit{proponente}\textsubscript{\textit{G}} l'\textit{MVP}\textsubscript{\textit{G}} per discutere dello stato di avanzamento e delle modifiche apportate rispetto al precedente \textit{SAL}\textsubscript{\textit{G}}. È stata illustrata la nuova struttura del \textit{database}\textsubscript{\textit{G}}, semplificata su suggerimento dell'azienda stessa per garantire una maggiore manutenibilità rispetto alla configurazione precedente. Inoltre, è stata esaminata in dettaglio la \textit{dashboard}\textsubscript{\textit{G}} definitiva per l'\textit{MVP}\textsubscript{\textit{G}} e tutte le sue varie funzionalità. \\
In conclusione, l'azienda \textit{proponente}\textsubscript{\textit{G}} si è dimostrata soddisfatta del lavoro svolto e della velocità di progressione, dovuta soprattutto alla qualità del proof of concept, che ha fornito una solida base di partenza per l'\textit{MVP}\textsubscript{\textit{G}}.

\section{Ordine del giorno}
    \subsection{Miglioramento dei grafici}
        Durante la presentazione della \textit{dashboard}\textsubscript{\textit{G}}, la \textit{proponente}\textsubscript{\textit{G}} ci ha fornito alcuni feedback per migliorare la leggibilità dei grafici. È emerso il suggerimento di includere, accanto al titolo del \textit{widget}\textsubscript{\textit{G}}, l'unità di misura delle misurazioni rappresentate dal grafico. Questo permetterebbe di specificare l'unità di misura una sola volta, anziché ripeterla per ogni valore sull'asse delle ordinate, rendendo i grafici più chiari e facili da comprendere.
    \subsection{Design pattern}
        Durante la discussione riguardante le scelte di design nel processo di progettazione dell'\textit{MVP}\textsubscript{\textit{G}}, è emerso un dibattito sui design \textit{pattern}\textsubscript{\textit{G}} adottati. Dopo un'attenta analisi dei \textit{pattern}\textsubscript{\textit{G}} utilizzati, si è discusso sull'eventuale necessità di implementarne di ulteriori. La \textit{proponente}\textsubscript{\textit{G}} ha espresso soddisfazione riguardo alla progettazione dei componenti e ai \textit{pattern}\textsubscript{\textit{G}} adottati, e si è concordato sul fatto che non sia necessario introdurne di ulteriori.
    \subsection{Integration Testing}
        Il team ha richiesto all'azienda consulenza su come eseguire con successo i \textit{test}\textsubscript{\textit{G}} di \textit{integrazione}\textsubscript{\textit{G}} tra il producer e il server \textit{Kafka}\textsubscript{\textit{G}}. La discussione è emersa a seguito delle difficoltà incontrate nel tentativo di realizzare tali \textit{test}\textsubscript{\textit{G}} utilizzando diverse metodologie e strumenti senza ottenere risultati positivi. A tal proposito, la \textit{proponente}\textsubscript{\textit{G}} si è offerta di assistere nella risoluzione dei problemi riscontrati. \\
        Inoltre, per evitare di inquinare il server \textit{Kafka}\textsubscript{\textit{G}} e il \textit{database}\textsubscript{\textit{G}} con i dati di \textit{test}\textsubscript{\textit{G}}, è stato proposto di creare una coda e una tabella specifiche, separate dal flusso di dati principale. È essenziale che sia la coda che la tabella dedicate ai \textit{test}\textsubscript{\textit{G}} mantengano la stessa struttura e il medesimo comportamento dell'ambiente di produzione. Ciò consentirebbe l'esecuzione dei \textit{test}\textsubscript{\textit{G}} senza la necessità di interventi manuali sulle tabelle del \textit{database}\textsubscript{\textit{G}} e sulla coda per eliminare i dati generati durante le prove.

\section{Attività da svolgere}
    \begin{itemize}
        \item Miglioramento grafici attraverso l'aggiunta di unità di misura;
        \item Creazione coda e tabelle ad hoc per Integration Test \textit{Kafka}\textsubscript{\textit{G}}-\textit{Clickhouse}\textsubscript{\textit{G}};
        \item Avanzamento \textit{test}\textsubscript{\textit{G}} di \textit{integrazione}\textsubscript{\textit{G}}.
    \end{itemize}

% ------------------------ Firma Azienda ----------------------
\begin{textblock*}{\textwidth}(0.22\textwidth, 1.08\textheight)
    Luogo e Data: Padova, 01/03/2024
\end{textblock*}

\begin{textblock*}{\textwidth}(0.90\textwidth, 1.08\textheight)
        Firma referente Sync Lab:
\end{textblock*}
% -------------------------------------------------------------
\end{document}
