\documentclass{article}
\usepackage[utf8]{inputenc}
\usepackage[absolute]{textpos}
\usepackage[default]{raleway}
\usepackage{titlesec, comment, tabularx, makecell, listings, array, setspace, geometry, graphicx, xcolor, xparse, fancyvrb, relsize, fancyhdr, booktabs, multirow, hyperref}
\usepackage{colortbl}
%\geometry{a4paper, left=2cm, right=2cm, top=2cm, bottom=2.5cm}
\renewcommand{\headrulewidth}{0pt}

% Definisci uno stile per i comandi git
\definecolor{light-gray}{gray}{0.92}

\lstdefinestyle{code}{
    frame=single,
    framesep=1mm,
    rulecolor=\color{light-gray},
    backgroundcolor=\color{light-gray},
    basicstyle=\ttfamily,
}

% ----------------------------- Definizione tabella ---------------------------

\newcolumntype{C}[1]{>{\centering\arraybackslash}m{#1}}

%\setcellgapes{2ex} % Imposta l'altezza dell'header (2ex)


% ------------------------------Metadati indice --------------------------------
\title{\textbf{\fontsize{28}{6}\selectfont Indice}}
\author{\fontsize{14}{6}\selectfont ByteOps}
\date{Febbraio 23, 2024}


% -----------------------------Creazione footer --------------------------------

\pagestyle{fancy}
\fancyhf{}
\renewcommand{\footrulewidth}{0.4pt}
\lfoot{
    \parbox[c]{2cm}{\includegraphics[width=2cm]{../../Images/logo.png}}
    \textcolor[RGB]{120, 120, 120}{$\cdot$ Verbale Esterno}
}
\rfoot{\thepage}

% --------------------------Modifica formato hyperlinks ------------------------

\hypersetup{
    colorlinks=true,
    linkcolor=black,
    filecolor=black,      
    pdftitle={Verbale Esterno 23/02/2024},  %inserisci data verbale
    pdfpagemode=FullScreen,
}

% ------------------------------- Valore sotto-paragrafi indice --------------------------------------

\setcounter{secnumdepth}{4}
\setcounter{tocdepth}{4}

\titleformat{\section}
{\normalfont\huge\bfseries}{\thesection}{0.2cm}{}
\titlespacing*{\paragraph}{0pt}{0.5cm}{0.1cm}

\titleformat{\subsection}
{\normalfont\Large\bfseries}{\thesubsection}{0.2cm}{}
\titlespacing*{\paragraph}{0pt}{0.5cm}{0.1cm}

\titleformat{\subsubsection}
{\normalfont\large\bfseries}{\thesubsubsection}{0.2cm}{}
\titlespacing*{\paragraph}{0pt}{0.5cm}{0.1cm}

\titleformat{\paragraph}
{\normalfont\normalsize\bfseries}{\theparagraph}{0.2cm}{}
\titlespacing*{\paragraph}{0pt}{0.5cm}{0.1cm}

% ------------------------------- Front Page ---------------------------------------

\begin{document}

% --------------------------Aggiunta firma finale ------------------------
\begin{textblock*}{\textwidth}(0.85\textwidth, 1.16\textheight)
    Il responsabile: Lisien Skenderi
\end{textblock*}
% ------------------------------------------------------------------------

\pagestyle{fancy}
\begin{center}
\includegraphics[width = 0.7\textwidth]{../../Images/logo.png} \\
\vspace{0.2cm}
\textcolor[RGB]{60, 60, 60}{\textit{ByteOps.swe@gmail.com}} \\
\vspace{1cm}
\fontsize{16}{6}\selectfont Verbale Esterno $\cdot$ Data: 23/02/2024 \\
\vspace{0.5cm}
\end{center}

\section*{Informazioni documento}
\def\arraystretch{1.2}
\begin{tabular}{>{\raggedleft\arraybackslash}p{0.3\textwidth}|>{\raggedright\arraybackslash}p{0.6\textwidth}c}
\hline
\addlinespace
\textbf{Luogo} & Google Meet \vspace{10pt} \\
\textbf{Orario} & 16:30 - 17:00 \vspace{10pt} \\
\textbf{Redattore} & N. Preto \vspace{10pt} \\
\textbf{Verificatore} & E. Hysa \vspace{10pt} \\
\textbf{Amministratore} & F. Pozza \vspace{10pt} \\
\textbf{Destinatari} & T. Vardanega \\ & R. Cardin \vspace{10pt} \\
\multirow[t]{7}{*}{\textbf{Partecipanti interni}} & A. Barutta \\ & E. Hysa \\ & R. Smanio \\ & D. Diotto \\ & F. Pozza \\ & L. Skenderi \\ & N. Preto \vspace{10pt} \\
\multirow[t]{3}{*}{\textbf{Partecipanti esterni}} & A. Dorigo \\ 
\end{tabular}
\pagebreak 

% ------------------------- Changelog ----------------------------

\section*{Registro delle modifiche}

\begin{tabular}{|C{2.5cm}|C{2.5cm}|C{2.5cm}|C{2.5cm}|C{2.5cm}|}
    \hline
    \textbf{Versione} & \textbf{Data} & \textbf{Autore} & \textbf{Verificatore} & \textbf{Dettaglio} \\
    \hline \hline
    0.0.1 & 23/02/2024 & N. Preto & R. Smanio & Redazione documento \\
    \hline
\end{tabular}
\pagebreak

% ------------------------- Generazione automatica indice ----------------------
\setstretch{1.5}
\maketitle
\thispagestyle{fancy}
\tableofcontents
\setstretch{1.2}
\pagebreak

% ------------------------ INIZIO DOCUMENTO ----------------------
\flushleft

\section{Revisione del periodo precedente}
In data 20/02/2024 si è tenuto l'incontro di revisione \textit{RTB}\textsubscript{\textit{G}} con il Professor Vardanega, il quale ha espresso un giudizio positivo sul lavoro svolto fino a quel momento. \\
Tuttavia, rispetto all'ultimo \textit{SAL}\textsubscript{\textit{G}}, non sono state avviate nuove \textit{attività}\textsubscript{\textit{G}} a causa dei precedenti impegni accademici, come l'esame di Ingegneria del \textit{Software}\textsubscript{\textit{G}}, che diversi membri del gruppo hanno dovuto affrontare il 22 febbraio. I componenti del team che non erano coinvolti nell'esame hanno continuato a lavorare sul progetto, iniziando l'implementazione delle correzioni indicate dal Professor Vardanega e conducendo sessioni di brainstorming per pianificare le future \textit{attività}\textsubscript{\textit{G}} relative alla \textit{PB}\textsubscript{\textit{G}}.

\section{Ordine del giorno}
    \subsection{Ottimizzazione database e query}
    Si è discusso sulle modalità per ottimizzare il \textit{database}\textsubscript{\textit{G}} e le \textit{query}\textsubscript{\textit{G}} per migliorare le prestazioni del \textit{sistema}\textsubscript{\textit{G}}. \\
    È emerso che le materialized view attualmente adottate, sebbene potenzialmente utili in un contesto ipotetico con grandi volumi di dati, potrebbero risultare complesse da mantenere ed eccessivamente sofisticate per il nostro contesto specifico. \\
    Di conseguenza, la \textit{Proponente}\textsubscript{\textit{G}} ha suggerito di valutare l'opportunità di utilizzare il \textit{database}\textsubscript{\textit{G}} in modo più tradizionale e di effettuare le aggregazioni di dati tramite \textit{query}\textsubscript{\textit{G}}, evitando così di dover calcolare le aggregazioni incrementalmente. Questo approccio potrebbe consentire di ottenere gli stessi risultati senza aggiungere eccessiva complessità alla configurazione del \textit{database}\textsubscript{\textit{G}}. È stato inoltre stabilito che successivamente ai \textit{test}\textsubscript{\textit{G}} di performance, verrà valutata l’opportunità di utilizzare o meno
    funzionalità più avanzate.
    \subsection{Pianificazione test da implementare}
    Si sono analizzati possibili \textit{test}\textsubscript{\textit{G}} da effettuare prestando particolare attenzione ai \textit{Test di integrazione}\textsubscript{\textit{G}} sul \textit{database}\textsubscript{\textit{G}} e ai \textit{test}\textsubscript{\textit{G}} sull'integrità dei dati. \\
    Uno dei \textit{test}\textsubscript{\textit{G}} pianificati riguarda la verifica della coerenza tra i dati generati dai \textit{sensori}\textsubscript{\textit{G}} e quelli memorizzati sia su \textit{Apache Kafka}\textsubscript{\textit{G}} che nel \textit{database}\textsubscript{\textit{G}} \textit{Clickhouse}\textsubscript{\textit{G}}. Poiché non sono disponibili \textit{librerie}\textsubscript{\textit{G}} predefinite che consentano di eseguire compiti così specifici, l'azienda \textit{proponente}\textsubscript{\textit{G}} ha suggerito lo sviluppo di una suite di \textit{test}\textsubscript{\textit{G}} di confronto personalizzata. Questo strumento è essenziale per garantire l'integrità dei dati lungo l'intero percorso di trasmissione e memorizzazione.\\
    Per quanto riguarda gli stress \textit{test}\textsubscript{\textit{G}}, la decisione è stata quella di generare considerevoli volumi di dati al secondo, e successivamente verificare le prestazioni in seguito ad operazioni e \textit{query}\textsubscript{\textit{G}} su tali dati. L'obiettivo è identificare il miglior approccio per la gestione ottimale del \textit{database}\textsubscript{\textit{G}} in situazioni di elevato carico. 
    \subsection{Verosimiglianza delle misurazioni}
    È stato trattato il quesito riguardante la necessità di rendere plausibili i dati generati, arrivando alla conclusione che l'attenzione dovrebbe essere rivolta alla correttezza dei dati piuttosto che alla loro verosimiglianza.\\
    Poiché il focus del progetto non verte sulla simulazione di dati realistici, bensì sulla gestione di dati in un contesto di \textit{Big Data}\textsubscript{\textit{G}}, la verosimiglianza dei dati non costituisce un requisito fondamentale; al contrario, sarebbe addirittura interessante poter gestire dati non realistici al fine di testare la capacità del \textit{sistema}\textsubscript{\textit{G}} nell'affrontare dati errati.
\section{Attività da svolgere}
    \begin{itemize}
        \item Predisporre l'ambiente per essere nelle condizioni di effettuare le diverse tipologie di \textit{test}\textsubscript{\textit{G}};
        \item Eseguire stress \textit{test}\textsubscript{\textit{G}} sul \textit{database}\textsubscript{\textit{G}} per valutare la sua efficienza;
        \item Eseguire \textit{test}\textsubscript{\textit{G}} di verifica sui dati per verificarne coerenza tra le varie componenti del \textit{sistema}\textsubscript{\textit{G}};
        \item Sviluppo dello strumento di verifica della coerenza tra i dati in uscita ed in entrata tra le componenti del \textit{sistema}\textsubscript{\textit{G}};
        \item Implementazione calcolo di punteggio della salute della città (opzionale).
    \end{itemize}

% ------------------------ Firma Azienda ----------------------
\begin{textblock*}{\textwidth}(0.22\textwidth, 1.08\textheight)
    Luogo e Data: Padova, 23/02/2024
\end{textblock*}

\begin{textblock*}{\textwidth}(0.90\textwidth, 1.08\textheight)
        Firma referente Sync Lab:
\end{textblock*}
% -------------------------------------------------------------
\end{document}
