\documentclass{article}
\usepackage[utf8]{inputenc}
\usepackage[absolute]{textpos}
\usepackage[default]{raleway}
\usepackage{titlesec, comment, tabularx, makecell, listings, array, setspace, geometry, graphicx, xcolor, xparse, fancyvrb, relsize, fancyhdr, booktabs, multirow, hyperref, todonotes}
\usepackage{colortbl}
\usepackage{todonotes}
\renewcommand{\headrulewidth}{0pt}

\definecolor{light-gray}{gray}{0.92}

\lstdefinestyle{code}{
    frame=single,
    framesep=1mm,
    rulecolor=\color{light-gray},
    backgroundcolor=\color{light-gray},
    basicstyle=\ttfamily,
}

% ----------------------------- Definizione tabella ---------------------------

\newcolumntype{C}[1]{>{\centering\arraybackslash}m{#1}}

% ------------------------------Metadati indice --------------------------------
\title{\textbf{\fontsize{28}{6}\selectfont Indice}}
\author{\fontsize{14}{6}\selectfont ByteOps}
\date{Marzo 22, 2024}


% -----------------------------Creazione footer --------------------------------

\pagestyle{fancy}
\fancyhf{}
\renewcommand{\footrulewidth}{0.4pt}
\lfoot{
    \parbox[c]{2cm}{\includegraphics[width=2cm]{../../Images/logo.png}}
    \textcolor[RGB]{120, 120, 120}{$\cdot$ Verbale Esterno}
}
\rfoot{\thepage}

% --------------------------Modifica formato hyperlinks ------------------------

\hypersetup{
    colorlinks=true,
    linkcolor=black,
    filecolor=black,      
    pdftitle={Verbale Esterno 22/03/2024},  %inserisci data verbale
    pdfpagemode=FullScreen,
}

% ------------------------------- Valore sotto-paragrafi indice --------------------------------------

\setcounter{secnumdepth}{4}
\setcounter{tocdepth}{4}

\titleformat{\section}
{\normalfont\huge\bfseries}{\thesection}{0.2cm}{}
\titlespacing*{\paragraph}{0pt}{0.5cm}{0.1cm}

\titleformat{\subsection}
{\normalfont\Large\bfseries}{\thesubsection}{0.2cm}{}
\titlespacing*{\paragraph}{0pt}{0.5cm}{0.1cm}

\titleformat{\subsubsection}
{\normalfont\large\bfseries}{\thesubsubsection}{0.2cm}{}
\titlespacing*{\paragraph}{0pt}{0.5cm}{0.1cm}

\titleformat{\paragraph}
{\normalfont\normalsize\bfseries}{\theparagraph}{0.2cm}{}
\titlespacing*{\paragraph}{0pt}{0.5cm}{0.1cm}

% ------------------------------- Front Page ---------------------------------------

\begin{document}

% --------------------------Aggiunta firma finale ------------------------
\begin{textblock*}{\textwidth}(0.85\textwidth, 1.16\textheight)
    Il responsabile: A. Barutta
\end{textblock*}
% ------------------------------------------------------------------------

\pagestyle{fancy}
\begin{center}
\includegraphics[width = 0.7\textwidth]{../../Images/logo.png} \\
\vspace{0.2cm}
\textcolor[RGB]{60, 60, 60}{\textit{ByteOps.swe@gmail.com}} \\
\vspace{1cm}
\fontsize{16}{6}\selectfont Verbale Esterno $\cdot$ Data: 22/03/2024 \\
\vspace{0.5cm}
\end{center}

\section*{Informazioni documento}
\def\arraystretch{1.2}
\begin{tabular}{>{\raggedleft\arraybackslash}p{0.3\textwidth}|>{\raggedright\arraybackslash}p{0.6\textwidth}c}
\hline
\addlinespace
\textbf{Luogo} & SyncLab srl, Galleria Spagna, 28, 35127 Padova PD \vspace{10pt} \\
\textbf{Orario} & 16:30 - 17:30 \vspace{10pt} \\
\textbf{Redattore} & F. Pozza \vspace{10pt} \\
\textbf{Verificatore} & D. Diotto \vspace{10pt} \\
\textbf{Amministratore} & R. Smanio \vspace{10pt} \\
\textbf{Destinatari} & T. Vardanega \\ & R. Cardin \vspace{10pt} \\
\multirow[t]{7}{*}{\textbf{Partecipanti interni}} & A. Barutta \\ & E. Hysa \\ & R. Smanio \\ & D. Diotto \\ & F. Pozza \\ & L. Skenderi \\ & N. Preto \vspace{10pt} \\
\multirow[t]{3}{*}{\textbf{Partecipanti esterni}} & D. Zorzi \\ & F. Pallaro \\ 
\end{tabular}
\pagebreak 

% ------------------------- Changelog ----------------------------

\section*{Registro delle modifiche}

\begin{tabular}{|C{2.5cm}|C{2.5cm}|C{2.5cm}|C{2.5cm}|C{2.5cm}|}
    \hline
    \textbf{Versione} & \textbf{Data} & \textbf{Autore} & \textbf{Verificatore} & \textbf{Dettaglio} \\
    \hline \hline
    0.0.1 & 22/03/2024 & F. Pozza & D. Diotto & Redazione documento \\
    \hline
\end{tabular}
\pagebreak

% ------------------------- Generazione automatica indice ----------------------
\setstretch{1.5}
\maketitle
\thispagestyle{fancy}
\tableofcontents
\setstretch{1.2}
\pagebreak

% ------------------------ INIZIO DOCUMENTO ----------------------
\flushleft

\section{Revisione del periodo precedente}
Lo scopo del meeting odierno è stato quello di valutare e validare il \textit{MVP}\textsubscript{\textit{G}} (Minimum Viable Product) del progetto, per cui non si è ritenuto necessario effettuare una revisione del periodo precedente.


\section{Ordine del giorno}
Come accennato nella sezione precedente, l'incontro si è focalizzato sulla validazione del \textit{MVP}\textsubscript{\textit{G}} del progetto. 
    \subsection{Presentazione MVP}

    Durante l'incontro, il team di sviluppo ha presentato il prodotto alla \textit{proponente}\textsubscript{\textit{G}}, il quale ha avuto la possibilità di visionare e testare il \textit{software}\textsubscript{\textit{G}}. La presentazione ha avuto inizio con l'introduzione della \textit{dashboard}\textsubscript{\textit{G}} principale, progettata per fornire una panoramica generale sullo stato di salute della città. Attraverso grafici intuitivi e dati aggregati, questa \textit{dashboard}\textsubscript{\textit{G}} offre una visione immediata delle principali metriche e tendenze.

    Successivamente, il team ha illustrato la \textit{dashboard}\textsubscript{\textit{G}} dedicata, progettata per consentire un'analisi dettagliata dei dati relativi a specifiche tipologie di misurazioni. Questa sezione del \textit{software}\textsubscript{\textit{G}} offre agli utenti la possibilità di esplorare i dati in profondità, consentendo loro di individuare correlazioni, tendenze e anomalie che potrebbero sfuggire a una visione più superficiale.

    Durante la presentazione, sono state evidenziate le funzionalità di filtraggio, che permettono agli utenti di personalizzare la visualizzazione dei dati in base alle proprie esigenze e preferenze. Inoltre, è stato presentato il \textit{sistema}\textsubscript{\textit{G}} di allerta integrato, che invia notifiche tempestive agli utenti tramite \textit{Discord}\textsubscript{\textit{G}} in caso di eventi significativi o situazioni di emergenza.

    \subsection{Validazione MVP}
    Al termine della presentazione, l'azienda \textit{proponente}\textsubscript{\textit{G}} ha espresso piena soddisfazione per il lavoro svolto, sottolineando come il prodotto risponda pienamente alle aspettative e ai requisiti prefissati.
    
    La \textit{proponente}\textsubscript{\textit{G}} ha inoltre apprezzato l'interfaccia intuitiva e user-friendly del \textit{software}\textsubscript{\textit{G}}, che consente una navigazione agevole e una comprensione immediata dei dati. Inoltre, ha gradito la completezza e la precisione delle funzionalità implementate, sottolineando come il prodotto risulti già maturo.

    Entrambe le parti hanno manifestato piena soddisfazione per il lavoro svolto e per l'eccellente collaborazione instaurata durante l'intero processo di realizzazione del progetto.

    In conclusione, è stato concordato che non saranno necessari ulteriori incontri o revisioni, poiché il prodotto rispecchia appieno le aspettative e i requisiti stabiliti.

\section{Attività da svolgere}
    \begin{itemize}
        \item Preparazione lettera di candidatura \textit{PB}\textsubscript{\textit{G}};
        \item Preparazione della presentazione per la prima fase della revisione \textit{PB}\textsubscript{\textit{G}};
        \item Revisione finale della documentazione.
\end{itemize}
% ------------------------ Firma Azienda ----------------------
\begin{textblock*}{\textwidth}(0.22\textwidth, 1.08\textheight)
    Luogo e Data: Padova, 22/03/2024
\end{textblock*}

\begin{textblock*}{\textwidth}(0.90\textwidth, 1.08\textheight)
        Firma referente Sync Lab:
\end{textblock*}
% -------------------------------------------------------------
\end{document}
