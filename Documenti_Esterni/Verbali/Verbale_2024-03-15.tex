\documentclass{article}
\usepackage[utf8]{inputenc}
\usepackage[absolute]{textpos}
\usepackage[default]{raleway}
\usepackage{titlesec, comment, tabularx, makecell, listings, array, setspace, geometry, graphicx, xcolor, xparse, fancyvrb, relsize, fancyhdr, booktabs, multirow, hyperref, todonotes}
\usepackage{colortbl}
\usepackage{todonotes}
\renewcommand{\headrulewidth}{0pt}

\definecolor{light-gray}{gray}{0.92}

\lstdefinestyle{code}{
    frame=single,
    framesep=1mm,
    rulecolor=\color{light-gray},
    backgroundcolor=\color{light-gray},
    basicstyle=\ttfamily,
}

% ----------------------------- Definizione tabella ---------------------------

\newcolumntype{C}[1]{>{\centering\arraybackslash}m{#1}}

% ------------------------------Metadati indice --------------------------------
\title{\textbf{\fontsize{28}{6}\selectfont Indice}}
\author{\fontsize{14}{6}\selectfont ByteOps}
\date{Marzo 15, 2024}


% -----------------------------Creazione footer --------------------------------

\pagestyle{fancy}
\fancyhf{}
\renewcommand{\footrulewidth}{0.4pt}
\lfoot{
    \parbox[c]{2cm}{\includegraphics[width=2cm]{../../Images/logo.png}}
    \textcolor[RGB]{120, 120, 120}{$\cdot$ Verbale Esterno}
}
\rfoot{\thepage}

% --------------------------Modifica formato hyperlinks ------------------------

\hypersetup{
    colorlinks=true,
    linkcolor=black,
    filecolor=black,      
    pdftitle={Verbale Esterno 15/03/2024},  %inserisci data verbale
    pdfpagemode=FullScreen,
}

% ------------------------------- Valore sotto-paragrafi indice --------------------------------------

\setcounter{secnumdepth}{4}
\setcounter{tocdepth}{4}

\titleformat{\section}
{\normalfont\huge\bfseries}{\thesection}{0.2cm}{}
\titlespacing*{\paragraph}{0pt}{0.5cm}{0.1cm}

\titleformat{\subsection}
{\normalfont\Large\bfseries}{\thesubsection}{0.2cm}{}
\titlespacing*{\paragraph}{0pt}{0.5cm}{0.1cm}

\titleformat{\subsubsection}
{\normalfont\large\bfseries}{\thesubsubsection}{0.2cm}{}
\titlespacing*{\paragraph}{0pt}{0.5cm}{0.1cm}

\titleformat{\paragraph}
{\normalfont\normalsize\bfseries}{\theparagraph}{0.2cm}{}
\titlespacing*{\paragraph}{0pt}{0.5cm}{0.1cm}

% ------------------------------- Front Page ---------------------------------------

\begin{document}

% --------------------------Aggiunta firma finale ------------------------
\begin{textblock*}{\textwidth}(0.85\textwidth, 1.16\textheight)
    Il responsabile: A. Barutta
\end{textblock*}
% ------------------------------------------------------------------------

\pagestyle{fancy}
\begin{center}
\includegraphics[width = 0.7\textwidth]{../../Images/logo.png} \\
\vspace{0.2cm}
\textcolor[RGB]{60, 60, 60}{\textit{ByteOps.swe@gmail.com}} \\
\vspace{1cm}
\fontsize{16}{6}\selectfont Verbale Esterno $\cdot$ Data: 15/03/2024 \\
\vspace{0.5cm}
\end{center}

\section*{Informazioni documento}
\def\arraystretch{1.2}
\begin{tabular}{>{\raggedleft\arraybackslash}p{0.3\textwidth}|>{\raggedright\arraybackslash}p{0.6\textwidth}c}
\hline
\addlinespace
\textbf{Luogo} & Google Meet \vspace{10pt} \\
\textbf{Orario} & 16:30 - 17:00 \vspace{10pt} \\
\textbf{Redattore} & D. Diotto \vspace{10pt} \\
\textbf{Verificatore} & A. Barutta \vspace{10pt} \\
\textbf{Amministratore} & R. Smanio \vspace{10pt} \\
\textbf{Destinatari} & T. Vardanega \\ & R. Cardin \vspace{10pt} \\
\multirow[t]{7}{*}{\textbf{Partecipanti interni}} & A. Barutta \\ & E. Hysa \\ & R. Smanio \\ & D. Diotto \\ & F. Pozza \\ & L. Skenderi \\ & N. Preto \vspace{10pt} \\
\multirow[t]{3}{*}{\textbf{Partecipanti esterni}} & A. Dorigo \\ & D. Zorzi \\ & F. Pallaro \\ 
\end{tabular}
\pagebreak 

% ------------------------- Changelog ----------------------------

\section*{Registro delle modifiche}

\begin{tabular}{|C{2.5cm}|C{2.5cm}|C{2.5cm}|C{2.5cm}|C{2.5cm}|}
    \hline
    \textbf{Versione} & \textbf{Data} & \textbf{Autore} & \textbf{Verificatore} & \textbf{Dettaglio} \\
    \hline \hline
    0.0.1 & 15/03/2024 & D. Diotto & A. Barutta & Redazione documento \\
    \hline
\end{tabular}
\pagebreak

% ------------------------- Generazione automatica indice ----------------------
\setstretch{1.5}
\maketitle
\thispagestyle{fancy}
\tableofcontents
\setstretch{1.2}
\pagebreak

% ------------------------ INIZIO DOCUMENTO ----------------------
\flushleft

\section{Revisione del periodo precedente}
In apertura di incontro, è stata informata l'azienda proponente che l'MVP non ha subito modifiche sostanziali rispetto al periodo precedente. Lo stato del prodotto è infatti già completo e definitivo, come illustrato all'azienda proponente nel precedente SAL. \\
È stato inoltre comunicato l'avanzamento del documento "\textit{Specifica Tecnica}" e la finalizzazione di tutti i test. Relativamente a tali attività non sono emersi problemi e non sono state riscontrate difficoltà.

\section{Ordine del giorno}

    \subsection{Parentesi su Docker e sulle unità di deploy}
    Durante la discussione, il gruppo ha sollevato il dubbio se fosse appropriato parlare di deploy in Docker. \\
    La proponente ha confermato che è del tutto possibile e sensato utilizzare Docker anche per scopi di deploy. \\
    Docker è infatti una soluzione efficace e ampiamente utilizzata per distribuire applicazioni in modo efficiente e scalabile. In combinazione con strumenti di orchestrazione come Kubernetes, Docker permette di creare un ambiente di sviluppo e produzione altamente efficiente e scalabile. 

    \subsection{Organizzazione validazione MVP}
    Durante l'incontro è stato esposto il desiderio del gruppo di completare il progetto entro i tempi previsti. È stato inoltre deciso di organizzare un incontro il giorno 22/03/2024 per la validazione da parte della proponente del Minimum Viable Product (MVP). \\
    Le modalità di svolgimento dell'incontro devono ancora essere definite, tuttavia entrambe le parti hanno manifestato la volontà di svolgere l'incontro in presenza, qualora le condizioni lo consentano.

\section{Attività da svolgere}
    \begin{itemize}
        \item Organizzare l'incontro per la validazione del MVP;
        \item Completare la documentazione relativa al progetto;
        \item Revisione e correzione del codice;
        \item Revisione e correzione della documentazione.
\end{itemize}
% ------------------------ Firma Azienda ----------------------
\begin{textblock*}{\textwidth}(0.22\textwidth, 1.08\textheight)
    Luogo e Data: Padova, 15/03/2024
\end{textblock*}

\begin{textblock*}{\textwidth}(0.90\textwidth, 1.08\textheight)
        Firma referente Sync Lab:
\end{textblock*}
% -------------------------------------------------------------
\end{document}
