\section{Specifica dei test}
L'esecuzione dei \textit{test}\textsubscript{\textit{G}} è un passaggio imprescindibile per confermare che il prodotto, nel suo insieme, rispecchi fedelmente e adempia pienamente a tutti i requisiti espressi e definiti all'interno del documento \textit{Analisi dei Requisiti v2.0.0 - sez. requisiti (\S~4)}. \\
Nelle successive tre sottosezioni vengono riportati i test suddivisi per categoria, esponendo per ciascuno un codice identificativo, una descrizione e lo stato. \\
Come specificato nel documento \textit{Norme di progetto v2.0.0 - sez. Stato dei test (\S~3.2.4.8)}, vengono utilizzate le seguenti abbreviazioni per identificare lo stato dei test:
\begin{itemize}
    \item \textbf{NI}: Non implementato;
    \item \textbf{S}: Superato;
    \item \textbf{NS}: Non Superato.
\end{itemize}

\input{Sottosezioni/Piano_di_qualifica/Test_di_unità.tex}

\subsection{Test di sistema}
Questa sezione illustra i test di sistema, i quali mirano a dimostrare la copertura completa dei requisiti identificati nel documento di Analisi dei Requisiti. Di seguito è fornito l'elenco di questi test:
\\
\begin{longtable}{|c|p{5cm}|>{\raggedright}p{2cm}|c|}
        \hline
        Codice Test & Descrizione & Requisito & Stato Test \\
        \hline
        TS01 & Verificare che l'accesso al sistema non richieda alcuna procedura di login e che sia immediatamente accessibile all'utente. & RF1 & NI \\
        \hline
        TS02 & Verificare che il prodotto non abbia alcuna sezione o funzionalità di amministrazione o gestione riservata. & RF2 & NI \\
        \hline
        TS03 & Verificare che i simulatori integrati producano dati di misurazione coerenti con l'ambito del sensore simulato. & RF3 & NI \\
        \hline
        TS04 & Verificare che ogni misurazione inviata dal simulatore contenga l'id del sensore, un timestamp e la misurazione stessa. & RF4 & NI \\
        \hline
        TS05 & Verificare che il sistema sia in grado di simulare almeno un sensore che rilevi la temperatura, espressa in gradi Celsius. & RF5 & NI \\
        \hline
        TS06 & Verificare che il sistema sia in grado di simulare almeno un sensore che misuri l'umidità, espressa in percentuale di umidità nell'aria.
        & RF6 & NI \\
        \hline
        TS07 & Verificare che il sistema sia in grado di simulare almeno un sensore per rilevare le particelle di polveri sottili nell'aria, esprimendole in microgrammi per metro cubo. & RF7 & NI \\
        \hline
        TS08 & Verificare che il sistema includa almeno un sensore per individuare guasti elettrici, segnalando le interruzioni nella fornitura di energia tramite un bit binario. & RF8 & NI \\
        \hline
        TS09 & Verificare che il sistema sia in grado di simulare almeno un sensore per monitorare lo stato di riempimento dei contenitori nelle isole ecologiche, segnalando con un bit binario se il contenitore è pieno. & RF9 & NI \\
        \hline
        TS10 & Verificare che il sistema includa almeno un sensore per le colonnine di ricarica, indicando tramite un bit binario se la colonnina è occupata o libera. & RF10 & NI \\
        \hline
        TS59 & Verificare che il sistema contenga almeno un sensore per il livello dell'acqua, indicando con un bit binario se il sensore rileva liquidi o meno. & RF59 & NI \\
        \hline
        TS11 & Verificare che ogni dato generato dai simulatori dei sensori sia strettamente correlato al dato successivo, garantendo una transizione realistica tra le misurazioni. & RF11 & NI \\
        \hline
        TS12 & Verificare che il sistema memorizzi in modo sicuro e efficiente i dati generati dai sensori, registrando accuratamente ogni misurazione per assicurare l'integrità e la coerenza dei dati. & RF12 & NI \\
        \hline
        TS13 & Verificare che la piattaforma supporti la visualizzazione di dati provenienti da diversi tipi di sensori, permettendo una rappresentazione corretta e coerente. & RF13 & NI \\
        \hline
        TS14 & Verificare che l'utente possa visualizzare una dashboard completa dello stato della città tramite l'uso di widget rappresentanti le misurazioni dei sensori. & RF14 & NI \\
        \hline
        TS15 & Verificare che l'utente possa vedere le misurazioni all'interno dei widget dedicati alla rappresentazione delle rilevazioni dei sensori in un formato grafico, facilitando la comprensione dei dati. & RF15 & NI \\
        \hline
        TS16 & Verificare che l'utente possa vedere le misurazioni all'interno dei widget dedicati alla rappresentazione delle rilevazioni dei sensori in un formato testuale. & RF16 & NI \\
        \hline
        TS17 & Verificare che la visualizzazione delle misurazioni in formato testuale segua il formato richiesto: IDSensore, TIMESTAMP, Dato. & RF17 & NI \\
        \hline
        TS18 & Verificare che la dashboard si aggiorni quasi istantaneamente per riflettere i dati provenienti dai sensori entro un massimo di 10 secondi. & RF18 & NI \\
        \hline
        TS19 & Verificare che la dashboard mostri widget distinti per ciascun tipo di sensore attivo che trasmette dati al sistema, contenenti le misurazioni in formato grafico. & RF19 & NI \\
        \hline
        TS21 & Verificare che ogni widget che visualizza le misurazioni includa informazioni sull'identificativo dei sensori che hanno contribuito a quelle misurazioni. & RF21 & NI \\
        \hline
        TS62 & Verificare che il widget destinato alla rappresentazione temporale delle misurazioni effettuate dai sensori di temperatura offra all'utente la possibilità di visualizzare tali dati in un formato grafico a linee, con una linea corrispondente a ciascun sensore coinvolto. & RF62 & NI \\
        \hline
        TS23 & Verificare che la dashboard includa un widget dedicato alle misurazioni dei sensori di umidità. & RF23 & NI \\
        \hline
        TS63 & Verificare che il widget destinato alla rappresentazione temporale delle misurazioni effettuate dai sensori di umidità offra all'utente la possibilità di visualizzare tali dati in un formato grafico a linee, con una linea corrispondente a ciascun sensore coinvolto. & RF63 & NI \\
        \hline
        TS64 & Verificare che il widget dedicato alla rappresentazione temporale delle misurazioni dei sensori di polveri sottili offra all'utente la possibilità di visualizzare tali dati in un formato grafico a linee, con una linea corrispondente a ciascun sensore coinvolto. & RF64 & NI \\
        \hline
        TS25 & Verificare che la dashboard includa un widget dedicato alle misurazioni dei sensori dei guasti elettrici. & RF25 & NI \\
        \hline
        TS65 & Verificare che il widget dedicato alla rappresentazione temporale delle misurazioni dei sensori dei guasti elettrici offra all'utente la possibilità di visualizzare tali dati con un grafico a linee per ciascun sensore coinvolto. & RF65 & NI \\
        \hline
        TS26 & Verificare che la dashboard includa un widget dedicato alle misurazioni dei sensori di soglia delle isole ecologiche. & RF26 & NI \\
        \hline
        TS66 & Verificare che il widget destinato alla rappresentazione temporale delle misurazioni dei sensori di soglia delle isole ecologiche offra all'utente la possibilità di visualizzare tali dati con un grafico a linee per ciascun sensore coinvolto. & RF66 & NI \\
        \hline
        TS27 & Verificare che la dashboard includa un widget dedicato alle misurazioni dei sensori delle colonnine di ricarica. & RF27 & NI \\
        \hline
        TS67 & Verificare che il widget destinato alla rappresentazione temporale delle misurazioni dei sensori delle colonnine di ricarica offra all'utente la possibilità di visualizzare tali dati con un grafico a linee per ciascun sensore coinvolto. & RF67 & NI \\
        \hline
        TS60 & Verificare che la dashboard includa un widget dedicato alle misurazioni dei sensori del livello dell'acqua. & RF60 & NI \\
        \hline
        TS68 & Verificare che il widget destinato alla rappresentazione temporale delle misurazioni dei sensori del livello dell'acqua offra all'utente la possibilità di visualizzare tali dati con un grafico a linee per ciascun sensore coinvolto. & RF68 & NI \\
        \hline
        TS28 & Verificare che la dashboard della città includa una mappa interattiva che mostri la posizione dei diversi sensori nella città. & RF28 & NI \\
        \hline
        TS29 & Verificare che i sensori sulla mappa siano etichettati in modo chiaro e distinguibile, permettendo il riconoscimento della loro tipologia. & RF29 & NI \\
        \hline
        TS61 & Verificare che i sensori posizionati sulla mappa mostrino l'ultimo valore registrato quando il puntatore del mouse è posizionato sopra di essi. & RF61 & NI \\
        \hline
        TS30 & Verificare che la dashboard fornisca un widget con il punteggio di salute relativo alla città basato sui dati aggregati provenienti dai sensori. & RF30 & NI \\
        \hline
        TS31 & Verificare che l'utente possa selezionare una cella specifica della città e visualizzare una dashboard dedicata contenente esclusivamente sensori, misurazioni e punteggio di salute correlati a essa. & RF31 & NI \\
        \hline
        TS32 & Verificare che l'utente possa filtrare la visualizzazione delle misurazioni di una specifica tipologia di sensori inserendo uno specifico intervallo temporale. & RF32 & NI \\
        \hline
        TS33 & Verificare che il sistema verifichi la validità dell'intervallo temporale inserito dall'utente. & RF33 & NI \\
        \hline
        TS34 & Verificare che, in caso di inserimento di un intervallo temporale non valido, il sistema generi una notifica di errore. & RF34 & NI \\
        \hline
        TS35 & Verificare che la notifica di errore relativa all'inserimento di un intervallo temporale non valido richieda all'utente di reinserire date valide. & RF35 & NI \\
        \hline
        TS36 & Verificare che la notifica di errore relativa all'inserimento di un intervallo temporale non valido sia chiara e informativa, indicando il motivo specifico dell'invalidità dell'intervallo temporale. & RF36 & NI \\
        \hline
        TS37 & Verificare che l'utente possa selezionare l'intervallo temporale desiderato (secondo, minuto, ora, giorno, mese, anno) per aggregare le misurazioni in base al periodo di registrazione corrispondente. & RF37 & NI \\
        \hline
        TS38 & Verificare che il sistema adatti dinamicamente la rappresentazione delle misurazioni secondo l'intervallo temporale di aggregazione selezionato dall'utente. & RF38 & NI \\
        \hline
        TS40 & Verificare che il sistema verifichi la validità dell'intervallo di rilevamento inserito dall'utente. & RF40 & NI \\
        \hline
        TS41 & Verificare che, in caso di inserimento di un intervallo di rilevamento non valido, il sistema generi una notifica di errore. & RF41 & NI \\
        \hline
        TS42 & Verificare che la notifica di errore relativa all'inserimento di un intervallo di rilevamento non valido richieda all'utente di reinserire valori validi. & RF42 & NI \\
        \hline
        TS43 & Verificare che la notifica generata in caso di inserimento di un intervallo di rilevamento non valido sia chiara e informativa, indicando specificamente il motivo dell'invalidità (ad esempio, data fine precedente a data inizio, arco temporale precedente o antecedente all’inizio della trasmissione dati). & RF43 & NI \\
        \hline
        TS44 & Verificare che l'utente possa filtrare le misurazioni selezionando uno o più sensori di una specifica categoria e visualizzare solo le misurazioni corrispondenti. & RF44 & NI \\
        \hline
        TS45 & Verificare che l'utente possa filtrare la visualizzazione delle misurazioni di una tipologia di sensori selezionando una o più specifiche celle come criterio di filtro. & RF45 & NI \\
        \hline
        TS46 & Verificare che l'utente possa applicare più filtri simultaneamente per la visualizzazione delle misurazioni di una specifica tipologia di sensori. & RF46 & NI \\
        \hline
        TS47 & Verificare che l'utente possa rimuovere i filtri applicati e ripristinare la visualizzazione senza tali filtri. & RF47 & NI \\
        \hline
        TS48 & Verificare che l'utente possa salvare una misurazione trasmessa da un sensore in una lista di misurazioni rilevanti. & RF48 Opzionale & NI \\
        \hline
        TS49 & Verificare che il sistema effettui una verifica per assicurarsi che la misurazione non sia già presente nella lista delle misurazioni rilevanti prima di salvarla. & RF49 Opzionale & NI \\
        \hline
        TS50 & Verificare che l'utente possa visualizzare la lista delle misurazioni rilevanti. & RF50 Opzionale & NI \\
        \hline
        TS69 & Verificare che ogni misurazione nella lista dei rilevanti fornisca correttamente l'identificativo del sensore. & RF69 Opzionale & NI \\
        \hline
        TS70 & Verificare che ogni misurazione nella lista dei rilevanti fornisca correttamente la tipologia del sensore. & RF70 Opzionale & NI \\
        \hline
        TS71 & Verificare che ogni misurazione nella lista dei rilevanti fornisca correttamente l'orario e la data di misurazione. & RF71 Opzionale & NI \\
        \hline
        TS72 & Verificare che ogni misurazione nella lista dei rilevanti fornisca correttamente il valore misurato e l'unità di misura corrispondente. & RF72 Opzionale & NI \\
        \hline
        TS51 & Verificare che l'utente possa rimuovere una misurazione specifica dalla lista delle misurazioni rilevanti. & RF51 Opzionale & NI \\
        \hline
        TS52 & Verificare che l'utente riceva notifiche quando i sensori superano determinate soglie di sicurezza. & RF52 & NI \\
        \hline
        TS53 & Verificare che l'utente possa visualizzare correttamente le informazioni richieste per i sensori. & RF53 & NI \\
        \hline
        TS54 & Verificare che l'utente possa visualizzare correttamente l'ID del sensore. & RF54 & NI \\
        \hline
        TS55 & Verificare che l'utente possa visualizzare correttamente il tipo di sensore. & RF55 & NI \\
        \hline
        TS56 & Verificare che l'utente possa visualizzare correttamente la posizione in coordinate dei sensori. & RF56 & NI \\
        \hline
        TS57 & Verificare che l'utente possa visualizzare correttamente la cella di installazione del sensore. & RF57 & NI \\
        \hline
        TS58 & Verificare che l'utente possa visualizzare correttamente la data di installazione del sensore. & RF58 & NI \\
        \hline
        TS59 & Verificare che l'utente possa visualizzare correttamente l'unità di misura associata al sensore. & RF59 & NI \\
        \hline
    \caption{Tabella test di sistema}
    \label{tab:testsSistema}
\end{longtable}

\subsection{Test di accettazione}
Nella sezione in questione, sono illustrati i \textit{test}\textsubscript{\textit{G}} di accettazione del prodotto, condotti sia dai membri del team che dal \textit{proponente}\textsubscript{\textit{G}} con il supporto del team di sviluppo.

L'obiettivo finale di tali \textit{test}\textsubscript{\textit{G}} è concludere il processo di validazione del prodotto.

\vspace{0.4cm}

\begin{longtable}{|C{1.9cm}|L{5cm}|C{2cm}|}
    \hline
    \textbf{Codice Test} & \centering{\textbf{Descrizione}} & \textbf{Stato Test} \\
    \hline \hline
    
    TA01 & Verificare che all'apertura del \textit{sistema}\textsubscript{\textit{G}} sia visualizzabile dall'utente la \textit{dashboard}\textsubscript{\textit{G}} riportante lo stato di salute della città. & S \\
    \hline
    
    TA01.1 & Verificare che tutti i \textit{widget}\textsubscript{\textit{G}} relativi alle diverse tipologie di sensori siano visibili sulla \textit{dashboard}\textsubscript{\textit{G}}. & S \\
    \hline
    
    TA01.2 & Verificare che la mappa dei sensori si carichi correttamente e permetta interazioni fluide. & S \\
    \hline
    
    TA01.3 & Verificare che il \textit{widget}\textsubscript{\textit{G}} relativo al punteggio di salute sia visibile e aggiornato. & S \\
    \hline
    
    TA02 & Verificare che il filtro permetta la corretta visualizzazione della \textit{dashboard}\textsubscript{\textit{G}} per una specifica cella. & S \\
    \hline
    
    TA01.1.1 & Verificare che le informazioni di un \textit{sensore}\textsubscript{\textit{G}} specifico siano visualizzate correttamente quando selezionate dalla \textit{dashboard}\textsubscript{\textit{G}}. & S \\
    \hline
    
    TA01.1.2 & Verificare che il \textit{sistema}\textsubscript{\textit{G}} consenta agli utenti di visualizzare correttamente le misurazioni dei sensori nel tempo. & S \\
    \hline
    
    TA04 & Verificare che ci sia la possibilità di visualizzare correttamente le misurazioni associate a uno specifico \textit{widget}\textsubscript{\textit{G}} nel formato testuale.  & S \\
    \hline
    
    TA04.1 & Verifica della gestione corretta degli errori nel caso in cui i dati dei sensori non siano disponibili o siano incompleti all'interno della visualizzazione testuale. & S \\
    \hline
    
    TA05 & Verificare che ci sia la possibilità di visualizzare correttamente le misurazioni associate a uno specifico \textit{widget}\textsubscript{\textit{G}} nel formato grafico. & S \\
    \hline
    
    TA05.1 & Verifica della gestione corretta degli errori nel caso in cui i dati dei sensori non siano disponibili o siano incompleti all'interno della visualizzazione grafica. & S \\
    \hline
    
    TA06 & Verificare ci sia l'opportunità di visualizzare correttamente il \textit{widget}\textsubscript{\textit{G}} contenente le misurazioni dei sensori di temperatura. & S \\
    \hline
    
    TA06.1 & Verificare l'accuratezza e la completezza delle opzioni di interazione offerte dall'interfaccia del \textit{widget}\textsubscript{\textit{G}} per esaminare i dati di temperatura. & X \\
    \hline
    
    TA07 & Verificare ci sia l'opportunità di visualizzare correttamente il \textit{widget}\textsubscript{\textit{G}} contenente le misurazioni dei sensori di umidità. & S \\
    \hline
    
    TA07.1 & Verificare l'accuratezza e la completezza delle opzioni di interazione offerte dall'interfaccia del \textit{widget}\textsubscript{\textit{G}} per esaminare i dati di umidità. & S \\
    \hline
    
    TA08 & Verificare ci sia l'opportunità di visualizzare correttamente il \textit{widget}\textsubscript{\textit{G}} contenente le misurazioni dei sensori riguardanti le polveri sottili nell'aria. & S \\
    \hline
    
    TA08.1 & Verificare l'accuratezza e la completezza delle opzioni di interazione offerte dall'interfaccia del \textit{widget}\textsubscript{\textit{G}} per esaminare i dati delle polveri sottili nell'aria. & S \\
    \hline
    
    TA09 & Verificare ci sia l'opportunità di visualizzare correttamente il \textit{widget}\textsubscript{\textit{G}} contenente le misurazioni dei sensori riguardanti i guasti elettrici. & S \\
    \hline
    
    TA09.1 & Verificare l'accuratezza e la completezza delle opzioni di interazione offerte dall'interfaccia del \textit{widget}\textsubscript{\textit{G}} per esaminare i dati dei sensori di guasti elettrici. & S \\
    \hline
    
    TA10 & Verificare ci sia l'opportunità di visualizzare correttamente il \textit{widget}\textsubscript{\textit{G}} contenente le misurazioni dei sensori riguardanti le isole ecologiche. & S \\
    \hline
    
    TA10.1 & Verificare l'accuratezza e la completezza delle opzioni di interazione offerte dall'interfaccia del \textit{widget}\textsubscript{\textit{G}} per esaminare i dati sulle isole ecologiche. & S \\
    \hline
    
    TA11 & Verificare ci sia l'opportunità di visualizzare correttamente il \textit{widget}\textsubscript{\textit{G}} contenente le misurazioni dei sensori riguardanti le colonnine di ricarica. & S \\
    \hline
    
    TA11 & Verificare l'accuratezza e la completezza delle opzioni di interazione offerte dall'interfaccia del \textit{widget}\textsubscript{\textit{G}} per esaminare i dati sulle colonnine di ricarica. & S \\
    \hline
    
    TA33 & Verificare ci sia l'opportunità di visualizzare correttamente il \textit{widget}\textsubscript{\textit{G}} contenente le misurazioni dei sensori riguardanti il livello dell'acqua. & S \\
    \hline
    
    TA33.1 & Verificare l'accuratezza e la completezza delle opzioni di interazione offerte dall'interfaccia del \textit{widget}\textsubscript{\textit{G}} per esaminare i dati sul livello dell'acqua. & S \\
    \hline
    
    TA12 & Verificare che si possa applicare con successo i filtri per la visualizzazione delle misurazioni e che solo le misurazioni che soddisfano i criteri di filtraggio vengano mostrate. & S \\
    \hline
    
    TA12.1 & Verificare si possa filtrare correttamente le misurazioni dei sensori in un intervallo temporale definito. & S \\
    \hline
    
    TA12.2 & Verificare che si possa filtrare correttamente le misurazioni visualizzate in base a valori di intervallo specifici. & S \\
    \hline
    
    TA12.3 & Verificare che si possa filtrare correttamente la visualizzazione delle misurazioni basate su specifiche celle urbane. & S \\
    \hline
    
    TA12.4 & Verificare si possa filtrare correttamente la visualizzazione delle misurazioni in base a specifici sensori selezionati. & S \\
    \hline
    
    TA30 & Verificare che il \textit{sistema}\textsubscript{\textit{G}} riconosca e notifichi in modo appropriato quando viene inserito un intervallo temporale non valido o incoerente. & S \\
    \hline
    
    TA13 & Verificare che si possa personalizzare con successo l'intervallo temporale di aggregazione delle misurazioni e che il \textit{sistema}\textsubscript{\textit{G}} aggiorni correttamente la visualizzazione in base a tale intervallo. & S \\
    \hline
    
    TA31 & Verificare che si possa rimuovere correttamente i filtri attivi dalla visualizzazione delle misurazioni dei sensori. & S \\
    \hline
    
    TA18 & Verificare che si possa visualizzare correttamente le informazioni dettagliate di uno specifico \textit{sensore}\textsubscript{\textit{G}} sulla \textit{dashboard}\textsubscript{\textit{G}}. & S \\
    \hline
    
    TA19 & Verificare che si possa inserire correttamente una misurazione nella lista delle misurazioni rilevanti. & NI \\
    \hline
    
    TA20 & Verificare che si possa visualizzare correttamente la lista delle misurazioni rilevanti. & NI \\
    \hline
    
    TA21 & Verificare che si possa rimuovere correttamente una o più misurazioni dalla lista delle misurazioni rilevanti. & NI \\
    \hline
    
    TA22 & Verificare che si riceva correttamente una notifica in caso di superamento delle soglie impostate per le misurazioni. & S \\
    \hline
    
    \caption{Tabella test di accettazione}
\end{longtable}

\pagebreak