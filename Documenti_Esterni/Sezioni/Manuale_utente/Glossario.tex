\section{Glossario}\label{sec:glossario}

\section*{A}


\vspace{0.4cm}

\textbf{Account}
\index{A}
\phantomsection

\vspace{0.1cm}

Un \textit{Account}\textsubscript{\textit{G}} è un'entità associata a un \textit{sistema}\textsubscript{\textit{G}} informatico che identifica e rappresenta un utente, un dispositivo o un'entità autorizzata a accedere e utilizzare determinate risorse, servizi o funzionalità all'interno del \textit{sistema}\textsubscript{\textit{G}}. L'\textit{Account}\textsubscript{\textit{G}} è caratterizzato da un insieme di credenziali uniche, come un nome utente e una password, che vengono utilizzate per l'autenticazione dell'entità associata e per garantire l'accesso autorizzato alle risorse del \textit{sistema}\textsubscript{\textit{G}}.

\vspace{0.4cm}

\textbf{Apache kafka}

\vspace{0.1cm}

\textit{Apache Kafka}\textsubscript{\textit{G}} è una \textit{piattaforma}\textsubscript{\textit{G}} di streaming distribuito \textit{open-source}\textsubscript{\textit{G}} utilizzata per la gestione e l'elaborazione di feed di dati in tempo reale. Progettato per affrontare problemi di ingestione, archiviazione e trasmissione di grandi volumi di dati in modo scalabile, \textit{Apache Kafka}\textsubscript{\textit{G}} è ampiamente utilizzato nell'ambito del \textit{data streaming}\textsubscript{\textit{G}} e dell'elaborazione degli eventi.

\pagebreak
\section*{B}
\index{B}
\phantomsection

\vspace{0.4cm}

\textbf{Browser}

\vspace{0.1cm}

Un \textit{Browser}\textsubscript{\textit{G}} è un'applicazione \textit{software}\textsubscript{\textit{G}} utilizzata per accedere e visualizzare le risorse su Internet. Funge da intermediario tra l'utente e i contenuti web, consentendo la navigazione attraverso i diversi siti web utilizzando un'interfaccia grafica intuitiva. 

\pagebreak
\section*{C}
\index{C}
\phantomsection

\textbf{Clickhouse}

\vspace{0.1cm}

\textit{ClickHouse}\textsubscript{\textit{G}} è un \textit{database}\textsubscript{\textit{G}} \textit{open-source}\textsubscript{\textit{G}} di analisi e archiviazione progettato per l'elaborazione rapida di grandi quantità di dati in modo scalabile ed efficiente. Questo \textit{sistema}\textsubscript{\textit{G}} di gestione dei dati utilizza un modello di \textit{architettura}\textsubscript{\textit{G}} di tipo column-oriented, in cui i dati vengono organizzati e memorizzati per colonne anziché per righe. Tale struttura ottimizza le prestazioni delle \textit{query}\textsubscript{\textit{G}} analitiche e l'efficienza della compressione dei dati, consentendo operazioni di lettura ad alta velocità su grandi volumi di informazioni. 

\vspace{0.4cm}

\textbf{Clock Rate}

\vspace{0.1cm}

Il \textit{Clock Rate}\textsubscript{\textit{G}}, anche conosciuto come frequenza di Clock, rappresenta la velocità alla quale avviene il ciclo di Clock in un dispositivo digitale, espressa in cicli per secondo (Hz) o Megahertz (MHz), Gigahertz (GHz) nel caso di velocità elevate. Indica la frequenza alla quale un dispositivo, come un processore o un circuito integrato, esegue le operazioni di base, come l'elaborazione delle istruzioni o il trasferimento dei dati. 

\pagebreak
\section*{D}
\index{D}
\phantomsection

\vspace{0.4cm}

\textbf{Dashboard}

\vspace{0.1cm}

Una Dashboard è un'interfaccia utente grafica che fornisce una panoramica visiva delle informazioni più importanti, dei dati o delle metriche pertinenti a un utente o a un processo specifico. Le Dashboard sono progettate per semplificare la visualizzazione dei dati complessi e offrire una rapida panoramica delle prestazioni, delle tendenze o delle metriche chiave. Le Dashboard possono essere personalizzate per soddisfare le esigenze specifiche di un utente o di un'organizzazione e spesso includono grafici, tabelle, grafici e altri elementi visivi che rappresentano i dati in modo chiaro e comprensibile.

\vspace{0.4cm}

\textbf{Discord}

\vspace{0.1cm}

\textit{Discord}\textsubscript{\textit{G}} fornisce canali di chat testuale e vocali, consentendo agli utenti di comunicare tramite messaggi di testo o chiamate vocali, sia uno a uno che in gruppo.

\vspace{0.4cm}

\textbf{Dual-Core CPU}

\vspace{0.1cm}

Una CPU Dual-Core è un tipo di processore che incorpora due unità di elaborazione centrali (CPU) all'interno di un singolo chip. Ogni Core funziona indipendentemente dall'altro, consentendo al processore di eseguire più operazioni contemporaneamente. Questo aumenta la capacità di elaborazione complessiva del processore, consentendo di gestire carichi di lavoro più pesanti e migliorando le prestazioni dei sistemi informatici.

\pagebreak
\section*{G}
\index{G}
\phantomsection

\vspace{0.4cm}

\textbf{Grafana}

\vspace{0.1cm}

\textit{Grafana}\textsubscript{\textit{G}} è una \textit{piattaforma}\textsubscript{\textit{G}} \textit{open-source}\textsubscript{\textit{G}} per la visualizzazione e l'analisi di dati metrici e \textit{log}\textsubscript{\textit{G}}. Essa fornisce strumenti potenti per la creazione di \textit{dashboard}\textsubscript{\textit{G}} interattive, che consentono agli utenti di monitorare e esplorare dati provenienti da una varietà di fonti. \textit{Grafana}\textsubscript{\textit{G}} è ampiamente utilizzato nel monitoraggio di sistemi, nell'analisi dei dati e nella creazione di \textit{dashboard}\textsubscript{\textit{G}} per presentare informazioni in modo chiaro e comprensibile.

\pagebreak
\section*{I}
\index{I}
\phantomsection

\vspace{0.4cm}

\textbf{IP}

\vspace{0.1cm}

Indirizzo \textit{IP}\textsubscript{\textit{G}} (Indirizzo Protocollo Internet) è una sequenza numerica univoca assegnata a ciascun dispositivo collegato a una \textit{rete}\textsubscript{\textit{G}} che utilizza il protocollo Internet Protocol (\textit{IP}\textsubscript{\textit{G}}) per la comunicazione dati. Questo indirizzo è utilizzato per identificare e localizzare un dispositivo all'interno della \textit{rete}\textsubscript{\textit{G}}, consentendo la trasmissione e la ricezione di dati tra i dispositivi collegati attraverso la \textit{rete}\textsubscript{\textit{G}}. Gli indirizzi \textit{IP}\textsubscript{\textit{G}} possono essere di tipo IPv4 (versione 4) o IPv6 (versione 6) e sono espressi come una serie di numeri separati da punti (per IPv4) o da due punti (per IPv6).

\pagebreak
\section*{J}
\index{J}
\phantomsection

\vspace{0.4cm}

\textbf{JSON}

\vspace{0.1cm}

JSON (JavaScript Object Notation) è un formato di scrittura per lo scambio di dati. È basato su un sottoinsieme del linguaggio JavaScript, ma è un formato di testo indipendente dal linguaggio di programmazione, ampiamente utilizzato per la trasmissione di dati strutturati tra un server e un client web o tra diverse componenti \textit{software}\textsubscript{\textit{G}}.

\pagebreak
\section*{L}
\index{L}
\phantomsection

\vspace{0.4cm}

\textbf{Link}

\vspace{0.1cm}

Un \textit{Link}\textsubscript{\textit{G}} è un elemento ipertestuale che consente di stabilire un collegamento tra due risorse all'interno di un \textit{sistema}\textsubscript{\textit{G}} informatico o su Internet. È costituito da un indirizzo, noto come URL (Uniform Resource Locator), che identifica la risorsa di destinazione, e può essere visualizzato come testo, immagine o altro elemento cliccabile.

\vspace{0.4cm}

\textbf{Linux}

\vspace{0.1cm}

\textit{Linux}\textsubscript{\textit{G}} è un \textit{sistema}\textsubscript{\textit{G}} operativo basato su kernel, originariamente sviluppato da Linus Torvalds nel 1991 e distribuito sotto i termini della GNU General Public License (GPL). Il kernel \textit{Linux}\textsubscript{\textit{G}} è il componente centrale del \textit{sistema}\textsubscript{\textit{G}} operativo, gestisce le risorse hardware del computer e fornisce un'interfaccia per le applicazioni \textit{software}\textsubscript{\textit{G}}. \textit{Linux}\textsubscript{\textit{G}} è noto per la sua stabilità, flessibilità e sicurezza ed è ampiamente utilizzato in una vasta gamma di dispositivi, inclusi server, computer desktop, dispositivi embedded e supercomputer. 

\vspace{0.4cm}

\textbf{Login}

\vspace{0.1cm}

Il \textit{Login}\textsubscript{\textit{G}} è un processo di autenticazione in un \textit{sistema}\textsubscript{\textit{G}} informatico mediante il quale un utente fornisce le proprie credenziali identificative, come il nome utente e la password, al fine di accedere a risorse protette o funzionalità specifiche del \textit{sistema}\textsubscript{\textit{G}}. Durante il processo di \textit{Login}\textsubscript{\textit{G}}, il \textit{sistema}\textsubscript{\textit{G}} verifica le credenziali fornite dall'utente confrontandole con quelle memorizzate nel \textit{sistema}\textsubscript{\textit{G}}. Se le credenziali fornite sono valide e corrispondono a un \textit{account}\textsubscript{\textit{G}} registrato nel \textit{sistema}\textsubscript{\textit{G}}, l'utente ottiene l'accesso autorizzato alle risorse o funzionalità desiderate. In caso contrario, se le credenziali sono errate o non corrispondono a un \textit{account}\textsubscript{\textit{G}} valido, il \textit{sistema}\textsubscript{\textit{G}} nega l'accesso e può fornire un messaggio di errore all'utente.

\vspace{0.4cm}

\textbf{Logout}

\vspace{0.1cm}

Il \textit{Logout}\textsubscript{\textit{G}} è un'operazione informatica che rappresenta il processo attraverso il quale un utente termina la sua sessione di accesso a un \textit{sistema}\textsubscript{\textit{G}} o a un'applicazione. Durante il \textit{logout}\textsubscript{\textit{G}}, il \textit{sistema}\textsubscript{\textit{G}} revoca l'accesso dell'utente alle risorse e alle funzionalità del \textit{sistema}\textsubscript{\textit{G}} o dell'applicazione, chiudendo la sessione attiva dell'utente e invalidando eventuali token di autenticazione o credenziali di accesso. L'utente viene quindi reindirizzato alla pagina di accesso o a una pagina di conferma di \textit{Logout}\textsubscript{\textit{G}}.

\pagebreak
\section*{M}
\index{M}
\phantomsection

\vspace{0.4cm}

\textbf{MacOS}

\vspace{0.1cm}

\textit{MacOS}\textsubscript{\textit{G}} è un \textit{sistema}\textsubscript{\textit{G}} operativo proprietario sviluppato da Apple Inc. per i computer Macintosh. Caratterizzato da un'interfaccia utente intuitiva e basata su grafica, \textit{MacOS}\textsubscript{\textit{G}} offre una vasta gamma di funzionalità, tra cui la gestione dei file, la connettività di \textit{rete}\textsubscript{\textit{G}}, la sicurezza informatica e il supporto per un'ampia varietà di applicazioni \textit{software}\textsubscript{\textit{G}}.

\pagebreak
\section*{P}
\index{P}
\phantomsection

\vspace{0.4cm}

\textbf{Piattaforma}

\vspace{0.1cm}

Il termine Piattaforma si riferisce a un ambiente hardware e/o \textit{software}\textsubscript{\textit{G}} che fornisce le risorse e i servizi necessari per lo sviluppo, l'esecuzione e la gestione di applicazioni \textit{software}\textsubscript{\textit{G}}.

\vspace{0.4cm}

\textbf{Proponente}

\vspace{0.1cm}

In ambito di sviluppo \textit{software}\textsubscript{\textit{G}}, il termine Proponente si riferisce generalmente a chi propone o suggerisce un progetto, un'idea o una soluzione.

\pagebreak
\section*{Q}
\index{Q}
\phantomsection

\vspace{0.4cm}

\textbf{Query}

\vspace{0.1cm}

Una \textit{Query}\textsubscript{\textit{G}} è un'istruzione o una richiesta formale utilizzata in sistemi di gestione dei \textit{database}\textsubscript{\textit{G}} o in altri contesti informatici per recuperare, modificare o manipolare i dati. Essa è espressa in un linguaggio specifico come SQL (Structured \textit{Query}\textsubscript{\textit{G}} Language) nei \textit{database}\textsubscript{\textit{G}} relazionali. Una \textit{Query}\textsubscript{\textit{G}} specifica le condizioni e i criteri che devono essere soddisfatti affinché venga restituito un insieme di dati o venga eseguita un'operazione specifica sui dati. Una \textit{Query}\textsubscript{\textit{G}} può essere utilizzata per selezionare dati da una tabella, aggiornare o eliminare dati esistenti, o per eseguire operazioni di aggregazione e analisi dei dati.

\pagebreak
\section*{S}
\index{S}
\phantomsection

\vspace{0.4cm}

\textbf{Servizio}

\vspace{0.1cm}

Un Servizio si riferisce a una funzionalità o a un'opzione offerta da un \textit{sistema}\textsubscript{\textit{G}}.

\vspace{0.4cm}

\textbf{Sistema}

\vspace{0.1cm}

Il termine Sistema si riferisce a un insieme di componenti interconnessi che lavorano insieme per raggiungere uno scopo comune.

\vspace{0.4cm}

\textbf{Snapshot}

\vspace{0.1cm}

Una \textit{Snapshot}\textsubscript{\textit{G}} è una copia istantanea dello stato o dei dati di un \textit{sistema}\textsubscript{\textit{G}} o di una risorsa in un determinato momento nel tempo. 

\vspace{0.4cm}

\textbf{Software}

\vspace{0.1cm}

Un \textit{Software}\textsubscript{\textit{G}} è una collezione di istruzioni, programmi, dati e documentazione che permette al computer di eseguire determinate funzioni o compiti.

\pagebreak
\section*{W}
\index{W}
\phantomsection

\vspace{0.4cm}

\textbf{Widget}

\vspace{0.1cm}

Un Widget è un componente grafico o un oggetto interattivo che può essere incorporato in un'interfaccia utente di un'applicazione \textit{software}\textsubscript{\textit{G}} o di una pagina web. I Widget sono progettati per fornire funzionalità specifiche e interazioni dirette con gli utenti, migliorando l'esperienza utente complessiva.



