\subsection{Scopo del capitolato}
    Il Capitolato C6 affidato al gruppo, si prefigge come obiettivo la realizzazione di una \textit{piattaforma}\textsubscript{\textit{G}} di monitoraggio di una \textit{"Smart City"} che consenta di avere sotto controllo lo stato di salute della città in modo tale da prendere decisioni veloci, efficaci ed analizzare poi gli effetti conseguenti.
    A tale scopo il \textit{proponente}\textsubscript{\textit{G}} richiede di simulare dei sensori posti in diverse aree per reperire informazioni relative alle condizioni della città.  
    I dati trasmessi in tempo reale dai sensori devono poter essere memorizzati in modo tale da renderli disponibili per la visualizzazione tramite una \textit{dashboard}\textsubscript{\textit{G}}, composta da \textit{widget}\textsubscript{\textit{G}}, per una visione d'insieme delle condizioni della città in tempo reale.  
    L'applicativo potrà consentire alle autorità locali di prendere decisioni informate e tempestive sulla gestione delle risorse e sull'implementazione di servizi e, inoltre, si potrebbe rivelare uno strumento essenziale per coinvolgere i cittadini nella gestione e nel miglioramento della città.  
