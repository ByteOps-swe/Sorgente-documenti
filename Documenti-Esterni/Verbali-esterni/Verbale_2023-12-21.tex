\documentclass{article}
\usepackage[utf8]{inputenc}
\usepackage[absolute]{textpos}
\usepackage[default]{raleway}
\usepackage{titlesec, comment, tabularx, makecell, listings, array, setspace, geometry, graphicx, xcolor, xparse, fancyvrb, relsize, fancyhdr, booktabs, hyperref}
\usepackage{colortbl}
%\geometry{a4paper, left=2cm, right=2cm, top=2cm, bottom=2.5cm}
\renewcommand{\headrulewidth}{0pt}

% Definisci uno stile per i comandi git
\definecolor{light-gray}{gray}{0.92}

\lstdefinestyle{code}{
    frame=single,
    framesep=1mm,
    rulecolor=\color{light-gray},
    backgroundcolor=\color{light-gray},
    basicstyle=\ttfamily,
}

% ----------------------------- Definizione tabella ---------------------------

\newcolumntype{C}[1]{>{\centering\arraybackslash}m{#1}}

%\setcellgapes{2ex} % Imposta l'altezza dell'header (2ex)


% ------------------------------Metadati indice --------------------------------
\title{\textbf{\fontsize{28}{6}\selectfont Indice}}
\author{\fontsize{14}{6}\selectfont ByteOps}
\date{Dicembre 21, 2023}


% -----------------------------Creazione footer --------------------------------

\pagestyle{fancy}
\fancyhf{}
\renewcommand{\footrulewidth}{0.4pt}
\lfoot{
    \parbox[c]{2cm}{\includegraphics[width=2cm]{../../Images/logo.png}}
    \textcolor[RGB]{120, 120, 120}{$\cdot$ Verbale Esterno}
}
\rfoot{\thepage}

% --------------------------Modifica formato hyperlinks ------------------------

\hypersetup{
    colorlinks=true,
    linkcolor=black,
    filecolor=black,      
    pdftitle={Verbale Esterno 21/12/2023},  %inserisci data verbale
    pdfpagemode=FullScreen,
}

% ------------------------------- Valore sotto-paragrafi indice --------------------------------------

\setcounter{secnumdepth}{4}
\setcounter{tocdepth}{4}

\titleformat{\section}
{\normalfont\huge\bfseries}{\thesection}{0.2cm}{}
\titlespacing*{\paragraph}{0pt}{0.5cm}{0.1cm}

\titleformat{\subsection}
{\normalfont\Large\bfseries}{\thesubsection}{0.2cm}{}
\titlespacing*{\paragraph}{0pt}{0.5cm}{0.1cm}

\titleformat{\subsubsection}
{\normalfont\large\bfseries}{\thesubsubsection}{0.2cm}{}
\titlespacing*{\paragraph}{0pt}{0.5cm}{0.1cm}

\titleformat{\paragraph}
{\normalfont\normalsize\bfseries}{\theparagraph}{0.2cm}{}
\titlespacing*{\paragraph}{0pt}{0.5cm}{0.1cm}

% ------------------------------- Front Page ---------------------------------------

\begin{document}

% --------------------------Aggiunta firma finale ------------------------
\begin{textblock*}{\textwidth}(0.85\textwidth, 1.16\textheight)
    Il responsabile: Lisien Skenderi
\end{textblock*}
% ------------------------------------------------------------------------

\pagestyle{fancy}
\begin{center}
\includegraphics[width = 0.7\textwidth]{../../Images/logo.png} \\
\vspace{0.2cm}
\textcolor[RGB]{60, 60, 60}{\textit{ByteOps.swe@gmail.com}} \\
\vspace{1cm}
\fontsize{16}{6}\selectfont Verbale Esterno $\cdot$ Data: 21/12/2023 \\
\vspace{0.5cm}
\end{center}

\section*{Informazioni documento}
\def\arraystretch{1.2}
\begin{tabular}{>{\raggedleft\arraybackslash}p{0.2\textwidth}|>{\raggedright\arraybackslash}p{0.6\textwidth}c}
\hline
\addlinespace
\textbf{Luogo} & Google Meet \vspace{10pt} \\
\textbf{Orario} & 16:30 - 17:30 \vspace{10pt} \\
\textbf{Redattore} & A. Barutta \vspace{10pt} \\
\textbf{Verificatore} & N.Preto \vspace{10pt} \\
\textbf{Amministratore} & D. Diotto \vspace{10pt} \\
\textbf{Destinatari} & T. Vardanega \\ & R. Cardin \vspace{10pt} \\
\textbf{Partecipanti} & A. Barutta \\ & E. Hysa \\ & R. Smanio \\ & D. Diotto \\ & F. Pozza \\ & L. Skenderi \\ & N. Preto \\ & D. Zorzi \\ & A. Dorigo \vspace{10pt}
\end{tabular}
\pagebreak 

% ------------------------- Changelog ----------------------------

\section*{Registro delle modifiche}

\begin{tabular}{|C{2.5cm}|C{2.5cm}|C{2.5cm}|C{2.5cm}|C{2.5cm}|}
    \hline
    \textbf{Versione} & \textbf{Data} & \textbf{Autore} & \textbf{Verificatore} & \textbf{Dettaglio} \\
    \hline \hline
    0.1.0 & 21/12/2023 & A. Barutta & N. Preto & Redazione documento \\
    \hline
\end{tabular}
\pagebreak

% ------------------------- Generazione automatica indice ----------------------
\setstretch{1.5}
\maketitle
\thispagestyle{fancy}
\tableofcontents
\setstretch{1.2}
\pagebreak

% ------------------------ INIZIO DOCUMENTO ----------------------
\flushleft

\section{Revisione del periodo precedente}
Tutte le \textit{attività}\textsubscript{\textit{G}} pianificate per il corrente \textit{SAL}\textsubscript{\textit{G}} sono state completate con successo, generando un alto livello di soddisfazione sia all'interno del team che presso l'azienda \textit{proponente}\textsubscript{\textit{G}}. Quest'ultima si mostra entusiasta dello stato attuale del Proof of Concept (PoC): le migliorie implementate nell'interfaccia utente hanno ricevuto apprezzamento e i problemi segnalati sono stati prontamente risolti.

In risposta alla richiesta dell'azienda \textit{proponente}\textsubscript{\textit{G}} riguardante le principali difficoltà nell'approccio lavorativo, si è ponderato sulla questione e si è constatato che la rotazione ciclica dei ruoli costituisce la sfida principale. Affrontare \textit{attività}\textsubscript{\textit{G}} precedentemente avviate da altri membri del gruppo richiede un'attenta valutazione preliminare del lavoro svolto fino a quel momento per mantenere coerenza ed integrità e ciò può richiedere un notevole dispendio di tempo.

Nonostante ciò, l'attuale approccio lavorativo è considerato valido e non si sono presentate circostanze che ne giustifichino la modifica.

\section{Ordine del giorno}
    \subsection{Inserimento vista misurazioni in forma tabellare}
    Il \textit{proponente}\textsubscript{\textit{G}} ha consigliato l'\textit{integrazione}\textsubscript{\textit{G}} nella \textit{dashboard}\textsubscript{\textit{G}} di alcuni \textit{widget}\textsubscript{\textit{G}} per la visualizzazione delle misurazioni in formato tabellare, questo ci consentirà di arricchire il Proof of Concept, fornendo diverse rappresentazioni delle misurazioni provenienti da una specifica categoria di sensori.

    \subsection{Spunti su alcuni test da effettuare in futuro}
    Alcuni dei \textit{test}\textsubscript{\textit{G}} da effettuare comprendono:
    \begin{itemize}
        \item Prove di carico ovvero simulare come reagisce il \textit{sistema}\textsubscript{\textit{G}} in seguito all'elaborazione di una grande quantità di dati trasmessi dai sensori. Questo può tornare utile anche per inserire dei requisiti sulla performance;
        \item Verifica che un dato sia visibile in ciascuno dei singoli componenti dell'\textit{architettura}\textsubscript{\textit{G}};
        \item Verifica che, dato un set di dati, il \textit{sistema}\textsubscript{\textit{G}} calcoli correttamente una media o una qualche funzione di aggregazione dei dati.
    \end{itemize}

    \subsection{Revisione presentazione RTB}
    È stata programmata una riunione con l'azienda \textit{proponente}\textsubscript{\textit{G}} il 12/01/2024 per esaminare la presentazione dedicata alla revisione \textit{RTB}\textsubscript{\textit{G}}. Durante l'incontro, avremo l'opportunità di chiarire eventuali dubbi e ottenere una valutazione diretta da parte del \textit{proponente}\textsubscript{\textit{G}}.

% ------------------------ Firma Azienda ----------------------
\begin{textblock*}{\textwidth}(0.35\textwidth, 1.08\textheight)
    Padova, 21/12/2023
\end{textblock*}

\begin{textblock*}{\textwidth}(0.80\textwidth, 1.08\textheight)
        Firma referente Sync Lab:
\end{textblock*}
% -------------------------------------------------------------
\end{document}
