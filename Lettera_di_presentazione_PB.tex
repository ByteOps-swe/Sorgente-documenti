\documentclass{article}
\usepackage[utf8]{inputenc}
\usepackage[absolute]{textpos}
\usepackage[default]{raleway}
\usepackage{pgf-pie, multirow, longtable, tikz, titlesec, comment, tabularx, makecell, float, listings, array, setspace, geometry, graphicx, xcolor, xparse, fancyvrb, relsize, fancyhdr, booktabs, hyperref, eurosym, fancybox, cleveref, todonotes}
\usepackage{colortbl}
\renewcommand{\headrulewidth}{0pt}

% Definisci uno stile per i comandi git
\definecolor{light-gray}{gray}{0.92}
\definecolor{darkgreen}{rgb}{0.0, 0.5, 0.0}
\definecolor{navyblue}{rgb}{0.0, 0.0, 0.8}

\lstdefinestyle{code}{
    frame=single,
    framesep=1mm,
    rulecolor=\color{light-gray},
    backgroundcolor=\color{light-gray},
    basicstyle=\ttfamily,
}
 
% ----------------------------- Definizione tabella ---------------------------

\newcolumntype{C}[1]{>{\centering\arraybackslash}m{#1}}

% ------------------------------Metadati indice --------------------------------
\title{\textbf{\fontsize{28}{6}\selectfont Indice}}
\author{\fontsize{14}{6}\selectfont ByteOps}
\date{Marzo 08, 2024}
 

% -----------------------------Creazione footer --------------------------------

\pagestyle{fancy}
\fancyhf{}
\renewcommand{\footrulewidth}{0.4pt}
\lfoot{
    \parbox[c]{2cm}{\includegraphics[width=2cm]{Images/logo.png}}
    \textcolor[RGB]{120, 120, 120}{$\cdot$ Candidatura PB}
}
\rfoot{\thepage}

% --------------------------Modifica formato hyperlinks ------------------------

\hypersetup{
    colorlinks=true,
    linkcolor=darkgreen,
    urlcolor=navyblue,
    filecolor=black,      
    pdftitle={Candidatura PB},  %inserisci data verbale
    pdfpagemode=FullScreen,
}

% ------------------------------- Valore sotto-paragrafi indice --------------------------------------

\setcounter{secnumdepth}{4}
\setcounter{tocdepth}{4}

\titleformat{\section}
{\normalfont\huge\bfseries}{\thesection}{0.2cm}{}
\titlespacing*{\paragraph}{0pt}{0.5cm}{0.1cm}

\titleformat{\subsection}
{\normalfont\Large\bfseries}{\thesubsection}{0.2cm}{}
\titlespacing*{\paragraph}{0pt}{0.5cm}{0.1cm}

\titleformat{\subsubsection}
{\normalfont\large\bfseries}{\thesubsubsection}{0.2cm}{}
\titlespacing*{\paragraph}{0pt}{0.5cm}{0.1cm}

\titleformat{\paragraph}
{\normalfont\normalsize\bfseries}{\theparagraph}{0.2cm}{}
\titlespacing*{\paragraph}{0pt}{0.5cm}{0.1cm}

% ------------------------------- Front Page ---------------------------------------

\begin{document}

% --------------------------Aggiunta firma finale ------------------------
\begin{textblock*}{\textwidth}(0.85\textwidth, 1.16\textheight)
    Il responsabile: F. Pozza
\end{textblock*}
% ------------------------------------------------------------------------

\pagestyle{fancy}
\begin{center}
\includegraphics[width = 0.7\textwidth]{Images/logo.png} \\
\vspace{0.2cm}
\textcolor[RGB]{60, 60, 60}{\textit{ByteOps.swe@gmail.com}} \\
\vspace{1cm}
\fontsize{16}{6}\selectfont Candidatura PB \\
\vspace{0.5cm}
\end{center}

\section*{Informazioni documento}
\def\arraystretch{1.2}
\begin{tabular}{>{\raggedleft\arraybackslash}p{0.2\textwidth}|>{\raggedright\arraybackslash}p{0.6\textwidth}c}
\hline
\addlinespace
\textbf{Data} & 08/03/2024 \vspace{10pt} \\
\textbf{Redattore} & N. Preto \vspace{10pt} \\
\textbf{Verificatore} & R. Smanio \vspace{10pt} \\
\textbf{Amministratore} & D. Diotto \vspace{10pt} \\
\textbf{Destinatari} & T. Vardanega \\ & R. Cardin \vspace{10pt} \\
\end{tabular}
\pagebreak 

% ------------------------- Generazione automatica indice ----------------------
\setstretch{1.5}
\maketitle
\thispagestyle{fancy}
{
    \hypersetup{linkcolor=black}
    \tableofcontents
}
\setstretch{1.2}
\pagebreak

% ------------------------ INIZIO DOCUMENTO ----------------------
\flushleft

\section{Introduzione}
Con la presente lettera di candidatura il gruppo \textit{ByteOps} comunica ufficialmente la volontà di richiedere un colloquio per la revisione \textit{PB}\textsubscript{\textit{G}} \textit{(Product Baseline)} riguardante il progetto \textit{InnovaCity: A smart city monitoring platform} proposto dall'azienda \textit{Sync Lab}.

\section{Documenti}
Tutta la documentazione è reperibile nella \textit{repository}\textsubscript{\textit{G}} pubblica del team \textit{ByteOps} accedendo al sito attraverso il seguente link:

\begin{center}
    \href{https://byteops-swe.github.io}{https://byteops-swe.github.io}
\end{center}

Nello specifico è presente una release 2.0 \textit{PB}\textsubscript{\textit{G}} nella quale saranno disponibili:
\vspace{0.2cm}

\textbf{Documenti esterni:}
\begin{itemize}
    \item Analisi dei requisiti \textit{v2.0.0};
    \item Piano di progetto \textit{v2.0.0};
    \item Piano di qualifica \textit{v2.0.0};
    \item Manuale utente \textit{v1.0.0};
    \item Specifica tecnica \textit{v1.0.0};
    \item Preventivo costi e assunzione impegni;
    \item Verbali esterni.
\end{itemize}
\vspace{0.2cm}

\textbf{Documenti interni:}
\begin{itemize}
    \item Norme di progetto \textit{v2.0.0};
    \item Glossario \textit{v2.0.0};
    \item Valutazione capitolati;
    \item Verbali interni.
\end{itemize}

\section{Validazione MVP}

La certificazione di validità del Minimum Viable Product (MVP) è stata rilasciata dall'azienda \textit{Sync Lab} in data 22/03/2024 e si può trovare nel relativo verbale esterno.

\section{Aggiornamento preventivo}
Rispetto al preventivo di €12.425, il team \textit{ByteOps} è stato in grado di rispettare i costi preventivati, mantenendo il budget iniziale invariato. \\
Per quanto concerne le ore di attività, ciascun membro del gruppo ha rispettato le ore preventivate, totalizzando 91 ore produttive per componente e un totale complessivo di 637 ore.\\
Di seguito è riportata una tabella riassuntiva in merito alle ore di lavoro impiegate da ciascun membro del team:
\begin{center}
    \begin{tabular}{|C{4cm}|C{2cm}|}
    \hline
        \textbf{Nominativo} & \textbf{Ore produttive} \\
        \hline\hline
        Andrea Barutta  & 91 \\
        \hline
        Davide Diotto   & 91 \\ 
        \hline
        Endi Hysa       & 91 \\ 
        \hline
        Francesco Pozza & 91 \\ 
        \hline
        Nicola Preto    & 91 \\ 
        \hline
        Lisien Skenderi & 91 \\ 
        \hline
        Riccardo Smanio & 91 \\ 
        \hline\hline
        \textbf{Totale} & \textbf{637} \\ 
        \hline
        \textbf{Ore rimanenti} & \textbf{0} \\
        \hline
    \end{tabular}
\end{center}

\section{Aggiornamento impegni}
Come preventivato inizialmente, il gruppo ByteOps ritiene concluso il progetto una volta completata la Product Baseline, pertanto rinuncia alla possibilità di proseguire con la Customer Acceptance.

\section{Il Team}
Di seguito vengono elencati i membri del gruppo ByteOps:
\vspace{1cm}

\begin{center}
    \begin{tabular}{|C{4cm}|C{2cm}|}
    \hline
        \textbf{Nominativo} & \textbf{Matricola} \\
        \hline\hline
        Andrea Barutta  & 2042355 \\
        \hline
        Davide Diotto   & 2042334 \\ 
        \hline
        Endi Hysa       & 2046424 \\ 
        \hline
        Francesco Pozza & 1174610 \\ 
        \hline
        Nicola Preto    & 2042371 \\ 
        \hline
        Lisien Skenderi & 2023461 \\ 
        \hline
        Riccardo Smanio & 1126491 \\ 
        \hline
    \end{tabular}
\end{center}


\vspace{1cm}
Nell'attesa di una sua risposta le rivolgiamo cordiali saluti,

\vspace{0.3cm}

\textit{ByteOps}.

\end{document}