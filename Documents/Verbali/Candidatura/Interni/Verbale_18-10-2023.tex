\documentclass{article}
\usepackage[utf8]{inputenc}
\usepackage[default]{raleway}
\usepackage{titlesec, array, setspace, geometry, graphicx, xcolor, relsize, fancyhdr, booktabs, hyperref}
%\geometry{a4paper, left=2cm, right=2cm, top=2cm, bottom=2.5cm}
\renewcommand{\headrulewidth}{0pt}

% ------------------------------Metadati indice --------------------------------
\title{\textbf{\fontsize{30}{6}\selectfont Indice}}
\author{\fontsize{14}{6}\selectfont ByteOps}
\date{\today}

% -----------------------------Creazione footer --------------------------------
\pagestyle{fancy}
\fancyhf{}
\renewcommand{\footrulewidth}{0.4pt}
\lfoot{
    \parbox[c]{2cm}{\includegraphics[width=2cm]{../../../Images/logo.png}}
    \textcolor[RGB]{120, 120, 120}{$\cdot$ Verbale}
}
\rfoot{\thepage}

% --------------------------Modifica formato hyperlinks ------------------------
\hypersetup{
    colorlinks=true,
    linkcolor=black,
    filecolor=black,      
    pdftitle={Verbale 18-10-2023},
    pdfpagemode=FullScreen,
}

\begin{document}
\pagestyle{fancy}
\begin{center}
\includegraphics[width = 0.7\textwidth]{../../../Images/logo.png} \\
\vspace{0.2cm}
\textcolor[RGB]{60, 60, 60}{\textit{ByteOps.swe@gmail.com}} \\
\vspace{1cm}
\fontsize{16}{6}\selectfont Verbale Interno $\cdot$ Data: 18/10/2023 \\
\vspace{0.5cm}
\end{center}

\section*{Informazioni documento}
\def\arraystretch{1.2}
\begin{tabular}{>{\raggedleft\arraybackslash}p{0.2\textwidth}|>{\raggedright\arraybackslash}p{0.6\textwidth}c}
\hline
\addlinespace
    \textbf{Luogo} & Discord \vspace{10pt} \\
    \textbf{Orario} & 15:00 - 17:30 \vspace{10pt} \\
    \textbf{Redattori} & A. Barutta \\ & R. Smanio \\ & N. Preto \vspace{10pt} \\
    \textbf{Verificatori} & E. Hysa \\ & L. Skenderi \\ & D. Diotto \vspace{10pt} \\
    \textbf{Amministratore} & F. Pozza \vspace{10pt} \\
    \textbf{Destinatari} & T. Vardanega \\ & R. Cardin \vspace{10pt} \\
    \textbf{Partecipanti} & A. Barutta \\ & E. Hysa \\ & R. Smanio \\ & D. Diotto \\ & F. Pozza \\ & L. Skenderi \\ & N. Preto \vspace{10pt} \\
\end{tabular}
\pagebreak 

% --------------------------- inizio documento ----------------------------------

\begin{itemize}
    \item \textbf{\fontsize{12}{6}\selectfont Creazione repository Git} 
    
    È stato deciso di creare un repository \textit{Git} per il versionamento dei file relativi al progetto e si è deciso di utilizzare \textit{Github} come piattaforma di hosting del repository.
    \vspace{0.5cm}
    
    \item \textbf{\fontsize{12}{6}\selectfont Gestione dei documenti relativi al progetto} 
    
    Si è scelto di creare una cartella condivisa su \textit{Google Drive} per la gestione dei documenti in fase di elaborazione. Si è scelto inoltre di utilizzare \textit{Google Docs} per poter apportare modifiche ai documenti sia in modalità sincrona che asincrona. Ogni documento dopo essere stato completato e revisionato viene redatto con \LaTeX\ e caricato nell’apposita cartella all’interno della repository dai verificatori.
    \vspace{0.5cm}

    \item \textbf{\fontsize{12}{6}\selectfont Gestione delle attività} 
    
    Abbiamo scelto di utilizzare \textit{Jira} come strumento per la suddivisione ed il tracciamento delle attività relative al progetto, così da associare ad ogni attività dettagli come, ad esempio, assegnatario/i, scadenza, priorità e stima ore richieste. \\
    In questo modo è possibile monitorare l'avanzamento del progetto ed organizzare al meglio il lavoro di gruppo permettendo ad ogni componente di avere una chiara visione delle attività svolte, delle attività in corso e delle attività da svolgere.
    Ogni qualvolta ci siano attività da aggiungere, l’amministratore del gruppo provvederà a farlo specificando i dettagli sopracitati.
    \vspace{0.5cm}

    \item \textbf{\fontsize{12}{6}\selectfont Programmazione riunioni settimanali} 
    
    Si è deciso di svolgere dei meeting di gruppo a cadenza settimanale, (salvo necessità straordinarie) in cui analizzeremo il lavoro svolto, discuteremo di eventuali dubbi e/o problemi sorti durante la settimana e definiremo le successive attività da svolgere.
    I meeting si terranno ogni lunedì su \textit{Discord}. \\
    Essendo che la maggior parte dei diari di bordo viene svolta il lunedì ci sarà modo di discutere anche di eventuali problemi riscontrati e di pianificare delle azioni di miglioramento.
    \vspace{0.1cm}

    \begin{samepage}
    \item \textbf{\fontsize{12}{6}\selectfont Stesura di una traccia per i primi documenti} 
    \begin{itemize}
        \item Valutazione dei capitolati
        \item Lettera di presentazione
        \item Preventivi costi
    \end{itemize}
    \end{samepage}
    \vspace{0.4cm}

    \item \textbf{\fontsize{12}{6}\selectfont Fissato meeting con proponente capitolato C6} 
    
    Data: 20/10/2023 \\
	Ora: 11:30 - 12:00 \\ 
    
    In merito a ciò, abbiamo concordato le interrogazioni da porre e le abbiamo documentate in un apposito file, gli esiti saranno visionabili nel relativo verbale esterno che verrà redatto successivamente all'incontro.
\end{itemize}


\end{document}