\documentclass{article}
\usepackage[utf8]{inputenc}
\usepackage[default]{raleway}
\usepackage{titlesec, array, setspace, geometry, graphicx, xcolor, relsize, fancyhdr, booktabs, hyperref, atbegshi}
%\geometry{a4paper, left=2cm, right=2cm, top=2cm, bottom=2.5cm}
\renewcommand{\headrulewidth}{0pt}

% ------------------------------Metadati indice --------------------------------
\title{\textbf{\fontsize{30}{6}\selectfont Indice}}
\author{\fontsize{14}{6}\selectfont ByteOps}
\date{\today}

% -----------------------------Creazione footer --------------------------------
\pagestyle{fancy}
\fancyhf{}
\renewcommand{\footrulewidth}{0.4pt}
\lfoot{
    \parbox[c]{2cm}{\includegraphics[width=2cm]{../../../Images/logo.png}}
    \textcolor[RGB]{120, 120, 120}{$\cdot$ Verbale}
}


% --------------------------Modifica formato hyperlinks ------------------------
\hypersetup{
    colorlinks=true,
    linkcolor=black,
    filecolor=black,      
    pdftitle={Verbale 20-10-2023}, % nome visualizzato in alto a sx nel programma 
    pdfpagemode=FullScreen,
}

% --------------------------Aggiunta firma finale ------------------------
\AtEndDocument{%
  \begin{center}
    \vfill
    \parbox{\textwidth}{%
      \centering
      \vspace{-1.5cm}
      \hspace{7cm}
      Il Responsabile: Davide Diotto
    }
  \end{center}
}


\begin{document}
\pagestyle{fancy}
\begin{center}
\includegraphics[width = 0.7\textwidth]{../../../Images/logo.png} \\
\vspace{0.2cm}
\textcolor[RGB]{60, 60, 60}{\textit{ByteOps.swe@gmail.com}} \\
\vspace{1cm}
\fontsize{16}{6}\selectfont Primo incontro conoscitivo - Sync Lab 
\\ Verbale Esterno $\cdot$ Data: 20/10/2023 \\
\vspace{0.5cm}
\end{center}

\section*{Informazioni documento}
\def\arraystretch{1.2}
\begin{tabular}{>{\raggedleft\arraybackslash}p{0.2\textwidth}|>{\raggedright\arraybackslash}p{0.6\textwidth}c}
\hline
\addlinespace
    \textbf{Luogo} & Goole Meet \vspace{10pt} \\
    \textbf{Orario} & 11:30 - 12:00 \vspace{10pt} \\
    \textbf{Redattori} & A. Barutta \\ & R. Smanio \\ & N. Preto \vspace{10pt} \\
    \textbf{Verificatori} & E. Hysa \\ & L. Skenderi \\ & D. Diotto \vspace{10pt} \\
    \textbf{Amministratore} & F. Pozza \vspace{10pt} \\
    \textbf{Destinatari} & T. Vardanega \\ & R. Cardin \vspace{10pt} \\
    \textbf{Partecipanti} & A. Barutta \\ & E. Hysa \\ & R. Smanio \\ & D. Diotto \\ & F. Pozza \\ & L. Skenderi \\ & N. Preto \vspace{10pt} \\
\end{tabular}
\pagebreak 

\section*{\textbf{Sintesi e risultati dell’incontro}}

Nel corso del meeting, sono state rivolte diverse domande all'azienda proponente Sync Lab in merito ad alcuni dubbi relativi alle specifiche presentate nel capitolato.
Di seguito, presentiamo una sintesi di quanto è emerso durante questa riunione.

\begin{itemize}
\item \textbf{\fontsize{12}{6}\selectfont Modalità di comunicazione proponente/fornitore}

Verrà fornito
da Sync Lab un invito per entrare all’interno di un server Discord.
Per le videoconferenze si utilizzerà Google Meet.
Verrà pubblicato un calendario condiviso tramite Google Drive per essere a conoscenza in anticipo dei vari incontri con l’azienda proponente. Un’alternativa proposta è utilizzare anche Google Calendar per essere in grado di monitorare meglio i vari gruppi.
\vspace{0.5cm} 

\item \textbf{\fontsize{12}{6}\selectfont SAL periodici}

Saranno organizzati, con probabile, ma non ancora definita cadenza settimanale, degli incontri periodici per monitorare e supportare il corretto avanzamento del progetto.
Gli obiettivi tra un SAL e il successivo saranno ben definiti e saranno determinati con il supporto di Sync Lab.
\vspace{0.5cm} 

\item \textbf{\fontsize{12}{6}\selectfont Approfondimento sulla provenienza dei dati generati}

Il dubbio nasce dalla richiesta di fornire dati al sistema non solo da sensori da noi simulati con script, ma anche attraverso altre fonti esterne non specificate.
I dati, oltre ai sensori, possono provenire da diverse fonti come ad esempio dispositivi indossabili e auto.
Inoltre è stato precisato che il focus non è sul simulatore poiché serve solo per testare l’efficacia del prodotto.
Sarà possibile discutere su come generare la fonte dei dati. L’azienda è disponibile a fornirci un set di dati da cui prendere spunto.
Non viene consigliato l’uso di API per il recupero di dati forniti da sensori realmente in uso perché si necessita di avere un elevato numero di dati, cosa che l’API non garantisce.
\vspace{0.5cm} 

\item \textbf{\fontsize{12}{6}\selectfont Pulizia di eventuali dati errati}

Nella realtà i dati raccolti dai sensori vengono filtrati. Non è un requisito obbligatorio, ma molto gradito, sul quale ci potrebbe essere una futura contrattazione. 
\vspace{0.5cm} 

\item \textbf{\fontsize{12}{6}\selectfont Chiarimenti a riguardo: “Strumenti proposti nel capitolato”}

Verranno organizzati dei meeting su Google Meet (o anche in presenza nella loro sede) per spiegare i concetti fondamentali su cui si basano le tecnologie suggerite. 
\vspace{0.5cm} 

\item \textbf{\fontsize{12}{6}\selectfont Cosa si intende per test end to end}

Si intende che l’applicazione deve essere testata nella sua interezza,”dall’inizio alla fine”, simulando l’interazione da parte del cliente/utente finale. Ciò permette di verificare che l’applicazione si comporti come ci si aspetta.
\vspace{0.5cm} 

\item \textbf{\fontsize{12}{6}\selectfont Chiarimenti a riguardo: “script python generano stream di messaggi, i tracciati devono essere proposti dal fornitore e validati dal proponente”}

Con “tracciati” si intende un file in cui vengono inseriti i dati raccolti dai sensori (formato JSON consigliato). Viene consigliato Python perché considerato più adatto e semplice per la realizzazione degli script.
\vspace{0.5cm} 

\item \textbf{\fontsize{12}{6}\selectfont Way of working proposto dal proponente}

L’azienda fissa un calendario in anticipo con dei SAL e degli obiettivi da raggiungere (di solito su base settimanale).
Ogni SAL permette di verificare lo stato di avanzamento del progetto.
È molto importante non “buttarsi” subito nello sviluppo, ma è necessario prestare particolare attenzione all’organizzazione e alla pianificazione delle attività per perseguire efficienza ed efficacia.

\vspace{0.5cm} 


\end{itemize}

\end{document}