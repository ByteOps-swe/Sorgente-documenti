\documentclass{article}
\usepackage[utf8]{inputenc}
\usepackage[default]{raleway}
\usepackage{titlesec, comment, tabularx, makecell, listings, array, setspace, geometry, graphicx, xcolor, xparse, fancyvrb, relsize, fancyhdr, booktabs, hyperref}
%\geometry{a4paper, left=2cm, right=2cm, top=2cm, bottom=2.5cm}
\renewcommand{\headrulewidth}{0pt}

% ----------------------------- Definizione tabella ---------------------------

\newcolumntype{C}[1]{>{\centering\arraybackslash}m{#1}}

% ------------------------------Metadati indice --------------------------------
\title{\textbf{\fontsize{30}{6}\selectfont Indice}}
\author{\fontsize{14}{6}\selectfont ByteOps}
\date{\today}

% -----------------------------Creazione footer --------------------------------
\pagestyle{fancy} 
\fancyhf{}
\renewcommand{\footrulewidth}{0.4pt} 
\lfoot{ 
    \parbox[c]{2cm}{\includegraphics[width=2cm]{../Images/logo.png}}
    \textcolor[RGB]{120, 120, 120}{$\cdot$ Piano di progetto}
}
\rfoot{\thepage}

% --------------------------Modifica formato hyperlinks ------------------------
\hypersetup{
    colorlinks=true,
    linkcolor=black,
    filecolor=black,      
    pdftitle={Piano di progetto},
    pdfpagemode=FullScreen,
}




\begin{document}
\pagestyle{fancy}
\begin{center}
\includegraphics[width = 0.7\textwidth]{../Images/logo.png} \\
\vspace{0.2cm}
\textcolor[RGB]{60, 60, 60}{\textit{ByteOps.swe@gmail.com}} \\
\vspace{1cm}
\fontsize{16}{6}\selectfont Piano di progetto \\
\vspace{0.5cm}
\end{center}

\section*{Informazioni documento}
\def\arraystretch{1.2}
\begin{tabular}{>{\raggedleft\arraybackslash}p{0.2\textwidth}|>{\raggedright\arraybackslash}p{0.6\textwidth}c}
\hline
\addlinespace
    \textbf{Redattori} & A. Barutta \\ & R. Smanio \\ & N. Preto \vspace{10pt} \\
    \textbf{Verificatori} & E. Hysa \\ & L. Skenderi \\ & D. Diotto \vspace{10pt} \\
    \textbf{Amministratore} & F. Pozza \vspace{10pt} \\
    \textbf{Destinatari} & T. Vardanega \\ & R. Cardin \vspace{10pt} \\
    \textbf{Partecipanti} & A. Barutta \\ & E. Hysa \\ & R. Smanio \\ & D. Diotto \\ & F. Pozza \\ & L. Skenderi \\ & N. Preto \vspace{10pt} \\
\end{tabular}
\pagebreak 

% ------------------------- Changelog ----------------------------

\begin{tabular}{|C{2.5cm}|C{2.5cm}|C{2.5cm}|C{2.5cm}|C{2.5cm}|}
    \hline
    \textbf{Versione} & \textbf{Data} & \textbf{Autore} & \textbf{Verificatore} & \textbf{Dettaglio} \\
    \hline
    \label{Git_Action_Version} 0.0.2 & 05/11/2023 & Nome autore & Nome verificatore & Aggiunto rischio sezione 2 \\
    \hline 
    0.0.1 & 03/11/2023 & Nome autore & Nome verificatore & Scrittura delle sezioni 1 e 2 \\
    \hline 


    
\end{tabular}

\pagebreak

% ------------------------- Generazione automatica indice ----------------------
\setstretch{1.5}
\maketitle
\thispagestyle{fancy}
\tableofcontents
\setstretch{1.2}
\pagebreak

% ---------------------------- Inizio documento -------------------------------

\section{Introduzione}
\subsection{Scopo del documento}
Il presente documento ha lo scopo di identificare la metodologia di pianificazione e illustrare le modalità con cui il gruppo \textit{ByteOps} sta sviluppando il progetto assegnato, al fine di garantire efficacia ed efficienza nel processo di sviluppo.\\
I contenuti che vengono trattati sono:
\begin{itemize}
    \item Analisi dei rischi
    \item Assegnazione ruoli dei membri del gruppo
    \item Descrizione del modello adottato con relative motivazioni della scelta
\end{itemize}


\subsection{Scopo del capitolato}
Il Capitolato C6 affidato al gruppo, si prefigge come obiettivo la realizzazione di una piattaforma di monitoraggio di una \textit{"Smart City"} che consenta di avere sotto controllo lo stato di salute della città in modo tale da prendere decisioni veloci, efficaci ed analizzare poi gli effetti conseguenti.
A tale scopo il proponente richiede di simulare dei sensori posti in diverse aree per reperire informazioni relative alle condizioni della città. 
I dati trasmessi in tempo reale dai sensori devono poter essere memorizzati in modo tale da renderli disponibili per la visualizzazione tramite una dashboard, composta anche da widget e grafici, per una visione d'insieme delle condizioni della città in tempo reale. 
L'applicativo potrà consentire alle autorità locali di prendere decisioni informate e tempestive sulla gestione delle risorse e sull'implementazione di servizi e, inoltre, si potrebbe rivelare uno strumento essenziale per coinvolgere i cittadini nella gestione e nel miglioramento della città.

\subsection{Riferimenti}
\subsubsection{Riferimenti normativi}
\begin{itemize}
\item Norme di progetto
\item \href {https://www.math.unipd.it/~tullio/IS-1/2023/Progetto/C6.pdf} {Capitolato d'appalto C6 - InnovaCity}
\item \href {https://www.math.unipd.it/~tullio/IS-1/2023/Dispense/PD2.pdf} {Regolamento del progetto didattico}
\end{itemize}

%--Il seguente documento è stato redatto con l'obiettivo di valutare e disaminare i potenziali rischi che il gruppo potrebbe affrontare nel corso delle attività di sviluppo del capitolato. In questo documento veranno esposti i vari rischi identificati, accompagnati da una relativa previsione di impatto che ciascuno di essi potrebbe avere e una possibile soluzione.

\section{Analisi dei rischi}
\subsection{Rallentamento delle attività}
Uno dei principali rischi che potrebbero manifestarsi è l'armonizzazione tra le attività personali con quelle attività progettuali. Tale rischio
potrebbe causare un rallentamento nell'esecuzione dei compiti assegnati ai membri del gruppo, con conseguente rallentamento del processo di sviluppo. 
Si prevede che questo rischio si intensificherà durante il periodo di sessione invernale 2023-2024, a causa degli impegni legati agli esami.\\ 
Per quanto riguarda le possibili soluzioni per attenuare il problema, si propone: 
\begin{itemize}
    \item Una buona organizzazione tra i membri del gruppo per garantire una distribuzione equa del lavoro da svolgere.
    \item La creazione di un ambiente di lavoro asincrono, in modo da permettere lo svolgimento dei compiti assegnati da parte di ciascun membro del gruppo secondo le proprie tempistiche.
\end{itemize}

\subsection{Apprendimento ed utilizzo delle nuove tecnologie}
L’apprendimento e l'implementazione delle tecnologie proposte possono rappresentare un rischio considerevole per lo 
sviluppo di un progetto, in quanto esista la possibilità che lo studio accurato di queste tecnologie richieda più tempo del previsto. \\
Per quanto riguarda le possibili soluzioni per attenuare il problema, si propone:
\begin{itemize}
    \item Pianificazione accurata del tempo necessario per l’assimilazione e l’implementazione \\delle nuove tecnologie.
    \item Sfruttare le competenze individuali di ciascun membro del gruppo, al fine di accelerare \\ il processo di apprendimento.
    \item Approfittare, ove possibile, delle opportunità di formazione offerte dall’azienda proponente, in modo da risolvere eventuali dubbi e consolidare i concetti appresi.
\end{itemize}
\subsection{Inefficacia della comunicazione} 
L’inefficacia della comunicazione tra i componenti del gruppo potrebbe condurre a ritardi significativi nello sviluppo del progetto. Questo rischio assume una rilevanza particolare considerando la natura collaborativa del lavoro di gruppo, che richiede l’adesione a norme concordate collettivamente.
\\Per quanto riguarda le possibili soluzioni per attenuare il problema, si propone:
   \begin{itemize} 
    \item Impiego di strumenti di comunicazione efficaci e l’istituzione di regole chiare per le riunioni e le discussioni tra i componenti del gruppo.
    \item Incentivare un ambiente di lavoro che favorisca la collaborazione e sia aperto all'espres-sione di idee o feedback da parte dei membri del gruppo.
\end{itemize}








\end{document} 