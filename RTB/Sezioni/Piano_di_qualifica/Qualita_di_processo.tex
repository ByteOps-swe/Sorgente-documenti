\section{Qualità di processo}
\subsection{Introduzione}
Durante l'implementazione del progetto, i processi adottano criteri di qualità al fine di perseguire un miglioramento continuo finalizzato a soddisfare appieno tali criteri.
 Nel contesto di questo progetto, si è deliberatamente scelto di utilizzare il metodo PDCA (Plan-Do-Check-Act) e lo standard ISO/IEC 15504.
\\
Ciò consente di garantire un'implementazione dei processi che, basandosi sull'esperienza accumulata, si orienta al miglioramento costante e assicura al cliente il raggiungimento di un prodotto di elevata qualità. Nella presente sezione, vengono delineati i livelli di qualità accettabili e ottimali in base alle metriche definite nel documento NORME DI PROGETTO V4.0.0D.

\subsection{Gestione delle risorse}
Il processo di gestione delle risorse è responsabile della supervisione dell'utilizzo ottimale delle risorse disponibili e del monitoraggio del progresso delle attività pianificate nel contesto del documento "Piano di progetto".
Le metriche utilizzate possono essere visionate all’interno del documento "Norme di progetto".
\subsubsection*{Metriche}
\begin{itemize}
    \item MQ13CPS Costo pianificato sostenuto;
    \item MQ14CAS Costo attuale sostenuto;
    \item MQ15CVS Costo preventivato sostenuto;
    \item MQ16VP Varazione programmazione;
    \item MQ17VC Varazione costi.
\end{itemize}
