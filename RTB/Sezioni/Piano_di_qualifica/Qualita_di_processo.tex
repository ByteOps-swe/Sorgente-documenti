\section{Qualità di processo}
\subsection{Introduzione}
Durante l'implementazione del progetto, i processi adottano criteri di qualità al fine di perseguire un miglioramento continuo finalizzato a soddisfare appieno tali criteri.
 Nel contesto di questo progetto, si è scelto di utilizzare il metodo PDCA (Plan-Do-Check-Act) e lo standard ISO/IEC 15504 noto anche come SPICE (Software Process Improvement and Capability Determination).
\\
Ciò consente di garantire un'implementazione dei processi che, basandosi sull'esperienza accumulata, si orienta al miglioramento costante e assicura al cliente il raggiungimento di un prodotto di elevata qualità. Nella presente sezione, vengono delineati i livelli di qualità accettabili e ottimali in base alle metriche definite nel documento "Norme di progetto".

\subsection{Processi primari}
\subsubsection{Fornitura}
Processo che consiste nel decidere procedure e risorse
adatte allo sviluppo del progetto.

\paragraph{Metriche}
\paragraph*{Parametri}
\begin{itemize}
    \item \textbf{BAC (Budget at Completion):}
    Il Budget at Completion rappresenta il budget totale previsto per completare l'intero progetto. In altre parole, è il costo massimo che si prevede di sostenere per portare a termine tutte le attività pianificate;
    \item \textbf{EAC (Estimate at Completion):}
    Stima previsionale del costo totale per completare un progetto. In termini pratici, l'EAC tiene conto dei costi effettivi già sostenuti nel progetto e stima quanto sarà necessario spendere per portare a termine il lavoro. Definita 
    come: (BAC / CPI);
    \item \textbf{EV (Earned Value):}
    Valore del lavoro effettivamente svolto fino a quel periodo;
    \item \textbf{AC (Actual Cost):} Misura i costi effettivamente sostenuti dall’inizio del progetto fino all’attualità;
    \item \textbf{PV (Planned Value):} Stima la somma dei costi realizzativi delle attività imminenti periodo per periodo;
    \item \textbf{SV (Schedule Variance):}Indica in percentuale quanto si è in anticipo o in ritardo con le attività;
    pianificate. Definita come: (EV - PV);
    \item \textbf{CPI (Cost Performance Index):} Misura il rapporto tra il valore del lavoro effettivamente svolto ed il 
    costo reale del lavoro fino al periodo di riferimento. E definita come: (EV / AC);
    \item \textbf{BV (Budget Variance):} Misura la differenza percentuale di budget tra quanto previsto nella
    pianificazione di un periodo e l’effettiva realizzazione. Definita come: (EV - AC);
    \item  \textbf{ETC (Estimate to Complete):} Stima i costi realizzativi fino alla fine del progetto. Definita come: 
    (EAC - AC).
    
\end{itemize} \hspace{1pt}



\begin{table}[H]
    \centering
    \begin{tabular}{|p{1.5cm}|p{3cm}|p{4cm}|p{3cm}|p{3cm}|}
        \hline
        \textbf{Metrica} & \textbf{Nome} & \textbf{Descrizione} & \textbf{Valore di accettazione} & \textbf{Valore preferibile} \\
        \hline
        \textbf{M\arabic{metriccounter}EAC} & Estimated at Completion &  Misura il costo realizzativo stimato per terminare il progetto.  & $\pm 5\%$ rispetto al preventivo & Pari al preventivo \\
        \hline
        \stepcounter{metriccounter}\textbf{M\arabic{metriccounter}CPI} & Cost Performance Index & Misura il rapporto tra il valore del lavoro effettivamente svolto ed il 
        costo reale del lavoro fino al periodo di riferimento. & $\pm 10\%$ & $0\%$ \\
        \hline
        \stepcounter{metriccounter}\textbf{M\arabic{metriccounter}BV} & Budget Variance & misura la differenza percentuale di budget tra quanto previsto nella 
        pianificazione di un periodo e l’effettiva realizzazione. & $\geq -10\%$ & $0\%$ \\
        \hline
        \stepcounter{metriccounter}\textbf{M\arabic{metriccounter}AC} & Actual Cost & Misura i costi effettivamente sostenuti dall’inizio del progetto fino 
        all’attualità.
         & $\geq 0 $ & $ \leq$ EAC  \\
        \hline
        \stepcounter{metriccounter}\textbf{M\arabic{metriccounter}SV} & Schedule Variance & Indica in percentuale quanto si è in anticipo o in ritardo con le attività
        pianificate. & $\geq -10\%$ & $0\%$ \\
        \hline
        \stepcounter{metriccounter}\textbf{M\arabic{metriccounter}EV} & Earned Value & Valore del lavoro effettivamente svolto fino a quel periodo.
        & $\geq 0 $ & $\leq$ EAC  \\
        \hline
        \stepcounter{metriccounter}\textbf{M\arabic{metriccounter}PV} & Planned Value & Stima la somma dei costi realizzativi delle attività imminenti periodo 
        per periodo. & $\geq 0  $ & $ \leq$ BAC  \\
        \hline
        \stepcounter{metriccounter}\textbf{M\arabic{metriccounter}ETC} & Estimate to Complete &  Stima i costi realizzativi fino alla fine del progetto. & $\geq 0  $ & $ \leq$ EAC  \\
        \hline
    \end{tabular}
    \caption{Metriche di Controllo di Progetto}
    \label{tab:controllo_progetto}
\end{table}


\subsubsection{Codifica}
\paragraph{Metriche} 
\hspace{1pt}
\begin{table}[H]
    \centering
    \begin{tabular}{|p{1.5cm}|p{3cm}|p{4cm}|p{3cm}|p{3cm}|}
        \hline
        \textbf{Metrica} & \textbf{Nome} & \textbf{Descrizione} & \textbf{Valore di accettazione} & \textbf{Valore preferibile} \\
        \hline
        \stepcounter{metriccounter}\textbf{M\arabic{metriccounter}CCM} & Complessità Ciclomatica per Metodo & Rappresenta la complessità di un metodo in base ai percorsi possibili & $\leq 5$ & $\leq 3$ \\
        \hline
        \stepcounter{metriccounter}\textbf{M\arabic{metriccounter}CC} & Code Coverage & Percentuale del codice sorgente coperto dai test & $\geq 80\%$ & $100\%$ \\
        \hline
        \stepcounter{metriccounter}\textbf{M\arabic{metriccounter}SC} & Statement Coverage & Percentuale degli statement del codice coperti dai test & $\geq 70\%$ & $\geq 85\%$ \\
        \hline
        \stepcounter{metriccounter}\textbf{M\arabic{metriccounter}BC} & Branch Coverage & Percentuale dei rami decisionali del codice coperti dai test & $\geq 50\%$ & $\geq 75\%$ \\
        \hline
        \textbf{M\arabic{metriccounter}PM} & Parametri per Metodo & Numero massimo di parametri per metodo & $\leq 6$ & $\leq 5$ \\
        \hline
        \stepcounter{metriccounter} \textbf{M\arabic{metriccounter}LCM} & Linee di Codice per Metodo & Limite massimo di linee di codice per metodo & $\leq 30$ & $\leq 20$ \\
        \hline
    \end{tabular}
    \caption{Codifica - Metriche e indici di qualità}
    \label{tab:metriche}
\end{table}

\subsection{Processi di supporto}
\paragraph{Documentazione}
\paragraph*{Metriche}
\hspace{1pt}
\begin{table}[H]
    \centering
    \begin{tabular}{|p{1.5cm}|p{3cm}|p{4cm}|p{3cm}|p{3cm}|}
        \hline
        \textbf{Metrica} & \textbf{Nome} & \textbf{Descrizione} & \textbf{Valore di accettazione} & \textbf{Valore preferibile} \\
        \hline
        \stepcounter{metriccounter} \textbf{M\arabic{metriccounter}CO} & Correttezza Ortografica & Misura la presenza di errori ortografici nei documenti & $0$ & $0$ \\
        \hline
        \stepcounter{metriccounter}\textbf{M\arabic{metriccounter}IG} & Indice Gulpease & Misura la leggibilità di un testo in base alla lunghezza delle parole e delle frasi & $\geq 40$ & $\geq 60$ \\
        \hline
    \end{tabular}
    \caption{Documentazione - Metriche e indici di qualità}
    \label{tab:metriche_testo}
\end{table}


\paragraph{Gestione dei rischi}
\paragraph*{Metriche}
\hspace{1pt}
\begin{table}[H]
    \centering
    \begin{tabular}{|p{1.5cm}|p{3cm}|p{4cm}|p{3cm}|p{3cm}|}
      \hline
      \textbf{Metrica} &  \textbf{Nome} &  \textbf{Descrizione} & \textbf{Valore di accettazione} & \textbf{Valore preferibile} \\
      \hline
      \stepcounter{metriccounter}\stepcounter{metriccounter}\textbf{M\arabic{metriccounter}RNP}    & Rischi non previsti   & Misura il numero di rischi non previsti nel corso del progetto. & $\leq 5$ &   $0$ \\
      \hline
    \end{tabular}
    \caption{Gestione dei rischi - Metriche e indici di qualità}
    \label{tab:tabella2}
\end{table}


\paragraph{Gestione della qualità}
\paragraph*{Metriche}
\hspace{1pt}
\begin{table}[H]
    \centering
    \begin{tabular}{|p{1.5cm}|p{3cm}|p{4cm}|p{3cm}|p{3cm}|}
        \hline
        \textbf{Metrica} & \textbf{Nome} & \textbf{Descrizione} & \textbf{Valore di accettazione} & \textbf{Valore preferibile} \\
        \hline
        \stepcounter{metriccounter}\textbf{M\arabic{metriccounter}QMS} & Quality Metrics Satisfied & Misura la percentuale di metriche del progetto che sono soddisfatte rispetto a quelle definite nelle norme di progetto. & $\geq 80\%$ & $ 100\% $ \\
        \hline
    \end{tabular}
    \caption{Gestione della qualità - Metriche e indici di qualità}
    \label{tab:metriche_testo}
\end{table}

