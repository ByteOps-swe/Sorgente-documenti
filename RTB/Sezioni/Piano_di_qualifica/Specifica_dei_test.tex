\section{Specifica dei test}
L'esecuzione dei test è un passaggio imprescindibile per confermare che il prodotto, nel suo insieme, rispecchi fedelmente e adempia pienamente a tutti i requisiti espressi e definiti all'interno del documento di Analisi dei Requisiti. I test utili all'interno di un progetto sono:
\begin{itemize}
    \item \textbf{Test di unità}.
    \item \textbf{Test di integrazione}
    \item \textbf{Test di sistema}
    \item \textbf{Test di regressione}
    \item \textbf{Test di accettazione}
\end{itemize}
\subsection{Test di unità}
Sono test mirati a verificare singole parti (unità) del codice, come funzioni, classi o metodi. L'obiettivo è assicurarsi che ciascuna unità funzioni correttamente in isolamento.
\\
\begin{table}[htbp]
    \centering
    \begin{tabular}{|c|p{3cm}|p{5cm}|c|}
        \hline
        Codice Test & Descrizione & Stato Test \\
        \hline
        TU01 & Esempio. & N-I \\
        \hline
        TU02 & Esempio. & N-I \\
        \hline
    \end{tabular}
    \caption{Tabella test di unità}
    \label{tab:testsUnità}
\end{table}

\subsection{Test di integrazione}
Sono test finalizzati a verificare la corretta interazione tra le varie unità di codice o moduli. Si eseguono dopo i test di unità per assicurare che le unità, una volta combinate, lavorino insieme correttamente.
\\
\begin{table}[htbp]
    \centering
    \begin{tabular}{|c|p{3cm}|p{5cm}|c|}
        \hline
        Codice Test & Descrizione & Stato Test \\
        \hline
        TU01 & Esempio. & N-I \\
        \hline
        TU02 & Esempio. & N-I \\
        \hline
    \end{tabular}
    \caption{Tabella test di integrazione}
    \label{tab:testsIntegrazione}
\end{table}

\subsection{Test di sistema}Questa sezione illustra i test di sistema, i quali mirano a dimostrare la copertura completa dei requisiti identificati nel documento di Analisi dei Requisiti. Di seguito è fornito l'elenco di questi test.
\\
\begin{table}[htbp]
    \centering
    \begin{tabular}{|c|p{3cm}|p{5cm}|c|}
        \hline
        Codice Test & Descrizione & Stato Test \\
        \hline
        TF01 & Esempio. & N-I \\
        \hline
        TF02 & Esempio. & N-I \\
        \hline
    \end{tabular}
    \caption{Tabella test di sistema}
    \label{tab:testsSistema}
\end{table}

\subsection{Test di regressione}
Sono test che vengono eseguiti per assicurarsi che le modifiche apportate al codice non abbiano introdotto nuovi errori nelle funzionalità già testate in precedenza. Si eseguono dopo ogni modifica al codice per garantire che le funzionalità esistenti continuino a funzionare come previsto.
\\
\begin{table}[htbp]
    \centering
    \begin{tabular}{|c|p{3cm}|p{5cm}|c|}
        \hline
        Codice Test & Descrizione & Stato Test \\
        \hline
        TU01 & Esempio. & N-I \\
        \hline
        TU02 & Esempio. & N-I \\
        \hline
    \end{tabular}
    \caption{Tabella test di regressione}
    \label{tab:testsRegressione}
\end{table}

\subsection{Test di accettazione}
Nella sezione in questione, sono illustrati i test di accettazione del prodotto, condotti sia dai membri del team che dal proponente con il supporto del team di sviluppo. L'obiettivo finale di tali test è concludere il processo di validazione del prodotto.
\\
\begin{table}[htbp]
    \centering
    \begin{tabular}{|c|p{3cm}|p{5cm}|c|}
        \hline
        Codice Test & Descrizione & Stato Test \\
        \hline
        TF01 & Esempio. & N-I \\
        \hline
        TF02 & Esempio. & N-I \\
        \hline
    \end{tabular}
    \caption{Tabella test funzionalità}
    \label{tab:testsAccettazione}
\end{table}



