\section{Specifica dei test}

In questa sezione del documento viene presentato il piano di testing, atto a garantire la correttezza del prodotto finale. Riflettendo quanto detto nelle Norme di Progetto v.1.0.0, il piano segue il modello a V, dove ad ogni fase di sviluppo viene fatta corrispondere una specifica tipologia di test da adottare. Questi ultimi sono suddivisi in:

\begin{itemize}
    \item \textbf{Test di Unità:} Si verifica il corretto funzionamento delle singole unità (atomiche e indipendenti) che compongono il sistema;
    \item \textbf{Test di Integrazione:} Si verifica il corretto funzionamento di più unità che cooperano tra di loro al fine di svolgere uno specifico compito. Vengono svolti dopo i test di unità;
    \item \textbf{Test di Sistema:} Si verifica il corretto funzionamento dell'intero sistema. I requisiti funzionali obbligatori, di vincolo, di prestazione e di qualità che devono essere soddisfatti per intero;
    \item \textbf{Test di Accettazione:} Si verifica, assieme al committente, il corretto funzionamento del software. Il loro superamento permette di procedere con il rilascio del prodotto finale.
\end{itemize}

\subsection{Test di unità}


\subsection{Test di integrazione}


\subsection{Test di Sistema}


\subsection{Test di Accettazione}
