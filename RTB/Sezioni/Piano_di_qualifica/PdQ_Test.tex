\section{Strategie di testing}
La presente sezione del documento è dedicata alla presentazione del piano di testing, concepito per garantire l'integrità e la precisione del prodotto finale. Seguendo le direttive stabilite nelle Norme di Progetto, tale piano si adatta al modello a V, nel quale ciascuna fase di sviluppo è accompagnata da test specifici da implementare. Tali test sono classificati secondo:
\begin{itemize}
    \item \textbf{Test funzionalità:} Si tratta di serie di test finalizzati a verificare l'accurata implementazione delle funzionalità fondamentali richieste dal prodotto, garantendo un livello di qualità adeguato.
    \item \textbf{Test di sistema:} Questa fase mira a verificare il corretto funzionamento dell'intero sistema. Affinché sia superata con successo, è essenziale garantire il completo soddisfacimento dei requisiti funzionali, vincolanti, di prestazione e di qualità concordati nel contratto con il committente.
    \item \textbf{Test di accettazione} Questi test, condotti in collaborazione con il committente, attestano la corretta operatività del software. Il superamento di tali test costituisce il presupposto per il rilascio definitivo del prodotto.
\end{itemize}
\subsection{Test funzionalità}
Nella seguente sezione, saranno esposti i test delle funzionalità previsti per essere eseguiti sul prodotto dopo il completamento della fase di progettazione.
\\
\begin{table}[htbp]
    \centering
    \begin{tabular}{|c|p{3cm}|p{5cm}|c|}
        \hhline{----}
        Codice Test & Descrizione & Stato Test \\
        \hhline{---}
        TF01 & Esempio. & Verificato \\
        \hhline{----}
        TF02 & Esempio. & Verificato \\
        \hhline{----}
    \end{tabular}
    \caption{Tabella test funzionalità}
    \label{tab:testsFunzionalità}
\end{table}
\\
\subsection{Tracciamento test funzionalità}
\begin{table}[htbp]
    \centering
    \begin{tabular}{|c|c|}
        \hhline{--}
        Codice Test & Codice Caso d'uso \\
        \hhline{--}
        TF01 & UC 4 \\
        \hhline{--}
        TF02 & UC 4.1, UC 4.1.1, UC 4.1.2, UC 4.2, UC 4.2.1, UC 4.2.2 \\
        \hhline{--}
    \end{tabular}
    \caption{Tabella dei Test}
    \label{tab:testsTracciamentoFunz}
\end{table}
\\
\subsection{Test di sistema}Questa sezione illustra i test di sistema, i quali mirano a dimostrare la copertura completa dei requisiti identificati nel documento di Analisi dei Requisiti. Di seguito è fornito l'elenco di questi test.
\\
\begin{table}[htbp]
    \centering
    \begin{tabular}{|c|p{3cm}|p{5cm}|c|}
        \hhline{----}
        Codice Test & Descrizione & Stato Test \\
        \hhline{---}
        TF01 & Esempio. & N-I \\
        \hhline{----}
        TF02 & Esempio. & N-I \\
        \hhline{----}
    \end{tabular}
    \caption{Tabella test di sistema}
    \label{tab:testsSistema}
\end{table}
\\
\subsection{Tracciamento dei test di sistema}
\begin{table}[htbp]
    \centering
    \begin{tabular}{|c|c|}
        \hhline{--}
        Codice Test & Codice Caso d'uso \\
        \hhline{--}
        TF01 & UC 4 \\
        \hhline{--}
        TF02 & UC 4.1, UC 4.1.1, UC 4.1.2, UC 4.2, UC 4.2.1, UC 4.2.2 \\
        \hhline{--}
    \end{tabular}
    \caption{Tabella tracciamento test di sistema}
    \label{tab:testsTracciamentoSist}
\end{table}

\subsection{Test di accettazione}
Nella sezione in questione, sono illustrati i test di accettazione del prodotto, condotti sia dai membri del team che dal proponente con il supporto del team di sviluppo. L'obiettivo finale di tali test è concludere il processo di validazione del prodotto.
\\
\begin{table}[htbp]
    \centering
    \begin{tabular}{|c|p{3cm}|p{5cm}|c|}
        \hhline{----}
        Codice Test & Descrizione & Stato Test \\
        \hhline{---}
        TF01 & Esempio. & N-I \\
        \hhline{----}
        TF02 & Esempio. & N-I \\
        \hhline{----}
    \end{tabular}
    \caption{Tabella test funzionalità}
    \label{tab:testsAccettazione}
\end{table}

\subsection{Tracciamento dei test di accettazione}
\begin{table}[htbp]
    \centering
    \begin{tabular}{|c|c|}
        \hhline{--}
        Codice Test & Codice Caso d'uso \\
        \hhline{--}
        TF01 & UC 4 \\
        \hhline{--}
        TF02 & UC 4.1, UC 4.1.1, UC 4.1.2, UC 4.2, UC 4.2.1, UC 4.2.2 \\
        \hhline{--}
    \end{tabular}
    \caption{Tabella tracciamento test di accettazione}
    \label{tab:testsTracciamentoAcc}
\end{table}

