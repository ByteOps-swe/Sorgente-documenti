\section{Metriche di qualità}
Le metriche di qualità dei prodotti software e dei processi sono strumenti fondamentali per valutare e migliorare l'efficacia e l'efficienza nello sviluppo del software. Queste metriche forniscono indicatori oggettivi e misurabili che consentono di valutare la conformità agli standard, identificare aree di miglioramento e monitorare la salute complessiva del processo di sviluppo

\subsection{Metriche per la qualità di processo}
\begin{itemize}
    \item \textbf{Metrica M1PMS:}
           \begin{itemize}
            \item \textbf{Nome:} Percentuale di Metriche Soddisfatte (PMS)
            \item \textbf{Descrizione:} Misura che valuta quante metriche che sono state definite sono state effettivamente adottate o soddisfatte.
            \item \textbf{Formula:} $\frac{metriche \ soddisfatte}{metriche \ totali}\times 100$
           \end{itemize}

    \item \textbf{Metrica M2EAC:}
          \begin{itemize}
              \item \textbf{Nome:} Estimated at Completion (EAC)
              \item \textbf{Descrizione:} Misura il costo realizzativo stimato per terminare il progetto.
              \item \textbf{Formula:} $EAC = AC + ETC$
          \end{itemize}

    \item \textbf{Metrica M3CPI:}
          \begin{itemize}
              \item \textbf{Nome:} Cost Performance Index (CPI)
              \item \textbf{Descrizione:} Misura il rapporto tra il valore del lavoro effettivamente svolto ed il costo reale del lavoro fino al periodo di riferimento.
              \item \textbf{Formula:} $CPI = \frac{EV}{AC}$
          \end{itemize}

    \item \textbf{Metrica M4BV:}
          \begin{itemize}
              \item \textbf{Nome:} Budget Variance (BV)
              \item \textbf{Descrizione:} Misura la differenza percentuale di budget tra quanto previsto nella pianificazione di un periodo e l’effettiva realizzazione.
              \item \textbf{Formula:} $BV = AC - PV \times 100\% $
          \end{itemize}

    \item \textbf{Metrica M5AC:}
          \begin{itemize}
              \item \textbf{Nome:} Actual Cost (AC)
              \item \textbf{Descrizione:} Misura i costi effettivamente sostenuti dall’inizio del progetto fino all’attualità.
              \item \textbf{Formula:} Dato disponibile e aggiornato in "Piano di progetto" per ogni periodo.
          \end{itemize}

    \item \textbf{Metrica M6SV:}
          \begin{itemize}
              \item \textbf{Nome:} Schedule Variance (SV)
              \item \textbf{Descrizione:} Indica in percentuale quanto si è in anticipo o in ritardo con le attività pianificate.
              \item \textbf{Formula:} $SV = (FP - IP) - (FC - IC)$
              \\con \begin{itemize}
                \item $FP$: giorno pianificato di fine attività;
                \item $IP$: giorno pianificato di inizio attività;
                \item $FC$: giorno consuntivato di fine attività;
                \item $IC$: giorno consuntivato di inizio attività.
            \end{itemize}
            
            Il risultato se:
            \begin{itemize}
                \item $> 0$: indica un anticipo rispetto alla previsione;
                \item $= 0$: indica se si è in linea rispetto alla previsione;
                \item $< 0$: indica se si è in ritardo rispetto alla previsione.
            \end{itemize}
          \end{itemize}

    \item \textbf{Metrica M7EV:}
          \begin{itemize}
              \item \textbf{Nome:} Earned Value (EV)
              \item \textbf{Descrizione:} Valore del lavoro effettivamente svolto fino a quel periodo.
              \item \textbf{Formula:} $EV = \% lavoro svolto \times EAC$
          \end{itemize}

    \item \textbf{Metrica M8PV:}
          \begin{itemize}
              \item \textbf{Nome:} Planned Value
              \item \textbf{Descrizione:} Stima la somma dei costi realizzativi delle attività imminenti periodo per periodo.
              \item \textbf{Formula:} $PV = \% lavoro svolto \times BAC$
          \end{itemize}

    \item \textbf{Metrica M9ETC:}
          \begin{itemize}
              \item \textbf{Nome:} Estimate to Complete (ETC)
              \item \textbf{Descrizione:} Stima i costi realizzativi fino alla fine del progetto.
              \item \textbf{Formula:} $ETC = EAC - AC$
          \end{itemize}

    \item \textbf{Metrica M11RNP:}
          \begin{itemize}
              \item \textbf{Nome:} Rischi Non Previsti (RNP)
              \item \textbf{Descrizione:} Misura il numero di rischi non previsti nel corso del progetto.
          \end{itemize}

    \item \textbf{Metrica M12VR:}
          \begin{itemize}
              \item \textbf{Nome:} Variazione dei Requisiti (VR)
              \item \textbf{Descrizione:} Misura la variazione nei requisiti dal momento della pianificazione.
              \item \textbf{Formula:} \textit{NRA + NRR + NRM}, dove:\begin{itemize}
                \item \textit{NRA}(Numero Requisiti Aggiunti) è la quantità di requisiti aggiunti dall'ultimo incremento;
                \item \textit{NRR}(Numero Requisiti Rimossi) è la quantità di requisiti rimossi dall'ultimo incremento.
                \item \textit{Numero Requisiti Modificati}(Numero Requisiti Modificati) è la quantità di requisiti modificati dall'ultimo incremento.
              \end{itemize}
          \end{itemize}

    \item \textbf{Metrica M13CC:}
          \begin{itemize}
           \item \textbf{Nome:} Code Coverage (PMS)
           \item \textbf{Descrizione:} Percentuale che rappresenta le linee di codice percorse dai test durante la loro esecuzione.
           \item \textbf{Formula:} $\frac{linee \ codice \ percorse}{linee \ codice \ totali}\times 100$
          \end{itemize}
    
    \item \textbf{Metrica M14PCTS:}
          \begin{itemize}
           \item \textbf{Nome:} Percentuale di Casi di Test Superati (PCTS)
           \item \textbf{Descrizione:} Percentuale di casi di test superati.
           \item \textbf{Formula:} $\frac{numero \ di \ casi \ di \ test \ superati}{numero \ totale \ di \ casi \ di \ test}\times 100$
          \end{itemize}

    \item \textbf{Metrica M14PCTF:}
          \begin{itemize}
           \item \textbf{Nome:} Percentuale di Casi di Test Falliti (PCTF)
           \item \textbf{Descrizione:} Percentuale di casi di test non superati.
           \item \textbf{Formula:} $\frac{numero \ di \ casi \ di \ test \ non \ superati}{numero \ totale \ di \ casi \ di \ test}\times 100$
          \end{itemize}
\end{itemize}


\subsection{Metriche per la qualità di prodotto}
\begin{itemize}
    
    \item \textbf{Metrica M16PROS:}
    \begin{itemize}
     \item \textbf{Nome:} Percentuale di Requisiti Obbligatori Soddisfatti (PROS)
     \item \textbf{Descrizione:} Metrica che valuta quanto del lavoro svolto durante lo sviluppo corrisponda ai requisiti essenziali o obbligatori definiti in fase di analisi dei requisiti.
     \item \textbf{Formula:} $\frac{requisiti \ obbligatori \ soddisfatti}{requisiti \ obbligatori \ totali}\times 100$
    \end{itemize}

    \item \textbf{Metrica M17PRDS:}
    \begin{itemize}
     \item \textbf{Nome:} Percentuale di Requisiti Desiderati Soddisfatti (PRDS)
     \item \textbf{Descrizione:} Metrica usata per valutare quanti di quei requisiti, che se integrati arricchirebbero l'esperienza dell'utente o fornirebbero vantaggi aggiuntivi non strettamente necessari, sono stati implementati o soddisfatti nel prodotto.
     \item \textbf{Formula:} $\frac{requisiti \ desiderabili \ soddisfatti}{requisiti \ desiderabili \ totali}\times 100$
    \end{itemize}

    \item \textbf{Metrica M18PRPS:}
    \begin{itemize}
     \item \textbf{Nome:} Percentuale di Requisiti oPzionali Soddisfatti (PRPS)
     \item \textbf{Descrizione:} Metrica per valutare quanti dei requisiti aggiuntivi, non essenziali o di bassa priorità, sono stati implementati o soddisfatti nel prodotto.
     \item \textbf{Formula:} $\frac{requisiti \ opzionali \ soddisfatti}{requisiti \ opzionali \ totali}\times 100$
    \end{itemize}

    \item \textbf{Metrica M19CO:}
          \begin{itemize}
              \item \textbf{Nome:} Correttezza Ortografica (CO)
              \item \textbf{Descrizione:} Rappresenta il numero di errori grammaticali ed
              ortografici all'interno di un documento;
          \end{itemize}

            \item \textbf{Metrica M20IG:}
            \begin{itemize}
                \item \textbf{Nome:} Indice Gulpease (MIG)
                \item \textbf{Descrizione:} Indice di leggibilità di un testo tarato sulla lingua italiana, che utilizza la lunghezza delle parole in lettere anziché in sillabe, semplificandone il calcolo automatico.
                \item \textbf{Formula:} $IG = 89 + \frac{300 \cdot N_f - 10 \cdot N_l}{N_p}$
                \item \textbf{Dove:}
                      \begin{itemize}
                          \item $N_f$: numero di frasi;
                          \item $N_l$: numero di lettere;
                          \item $N_p$: numero di parole.
                      \end{itemize}
                \item I risultati sono compresi tra 0 e 100, dove il valore "100" indica la leggibilità più alta e "0" la leggibilità più bassa. In generale risulta che i testi con un indice:
                      \begin{itemize}
                          \item $< 80$: sono difficili da leggere per chi ha la licenza elementare;
                          \item $< 60$: sono difficili da leggere per chi ha la licenza media;
                          \item $< 40$: sono difficili da leggere per chi ha un diploma superiore.
                      \end{itemize}
            \end{itemize}

            \item \textbf{Metrica M21DE:}
                  \begin{itemize}
                      \item \textbf{Nome:} Densità degli Errori (DE)
                      \item \textbf{Descrizione:} Misura la percentuale di errori presenti nel prodotto rispetto al totale del codice.
                      \item \textbf{Formula:} $DE = \frac{\textit{Numero di errori}}{\textit{Totale delle linee di codice}} \times 100$
                  \end{itemize}

            \item \textbf{Metrica M22IF:}
                  \begin{itemize}
                      \item \textbf{Nome:} Implementazione delle Funzionalità (IF)
                      \item \textbf{Descrizione:} Misura qual è la quantità di funzionalità pianificate che sono state implementate.
                      \item \textbf{Formula:}$(1 - \frac{F_{NL}}{F_L}) \times 100$
                  \end{itemize}


            \item \textbf{Metrica M24ATC:}
                  \begin{itemize}
                      \item \textbf{Nome:} Accoppiamento tra Classi (ATC)
                      \item \textbf{Descrizione:} Misura il livello di accoppiamento tra le classi del sistema.
                  \end{itemize}


            \item \textbf{Metrica M26TMR:}
                  \begin{itemize}
                      \item \textbf{Nome:} Tempo Medio di Risposta (TMR)
                      \item \textbf{Descrizione:} Metrica che misura quanto è efficiente e reattivo un sistema software.
                      \item \textbf{Formula:}$\frac{\textit{Somma dei tempi di risposta}}{\textit{Numero totale di misurazioni}}$


            \item \textbf{Metrica M27FU:}
                  \begin{itemize}
                      \item \textbf{Nome:} Facilità di Utilizzo (FU)
                      \item \textbf{Descrizione:} Metrica che misura l'usabilità di un sistema software.


            \item \textbf{Metrica M28CCM}
            \begin{itemize}
            \item \textbf{Nome:} Complessità Ciclomatica per Metodo (CCM)
            \item \textbf{Descrizione:} Misura il numero di cammini linearmente indipendenti attraverso il grafo di controllo di flusso del metodo.
            \item \textbf{Formula:} $MCCM = e - n + 2$
            \item \textbf{Dove:}\textit{e}: numero di archi del grafo del flusso di esecuzione del metodo;\\ \textit{n}: numero di vertici del grafo del flusso di esecuzione del metodo.
        \end{itemize}
                                    

    \item \textbf{Metrica M29PM:}
        \begin{itemize}
            \item \textbf{Nome:} Parametri per Metodo (M12PM)
            \item \textbf{Descrizione:} Numero massimo di parametri per metodo.
          \end{itemize}
    
    \item \textbf{Metrica M30APC:}
          \begin{itemize}
              \item \textbf{Nome:} Attributi per Classe (ATC)
              \item \textbf{Descrizione:} Misura il numero massimo di attributi per classe.
          \end{itemize} 
          
    \item \textbf{Metrica M31LCM:}
          \begin{itemize}
              \item \textbf{Nome:} Linee di Codice per Metodo (LCM)
              \item \textbf{Descrizione:} Limite massimo di linee di codice per metodo.
          \end{itemize} 
\end{itemize}