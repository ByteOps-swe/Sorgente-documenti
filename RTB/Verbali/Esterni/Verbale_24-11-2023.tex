\documentclass{article}
\usepackage[utf8]{inputenc}
\usepackage[absolute]{textpos}
\usepackage[default]{raleway}
\usepackage{titlesec, comment, tabularx, makecell, listings, array, setspace, geometry, graphicx, xcolor, xparse, fancyvrb, relsize, fancyhdr, booktabs, hyperref}
\usepackage{colortbl}
%\geometry{a4paper, left=2cm, right=2cm, top=2cm, bottom=2.5cm}
\renewcommand{\headrulewidth}{0pt}

% Definisci uno stile per i comandi git
\definecolor{light-gray}{gray}{0.92}

\lstdefinestyle{code}{
    frame=single,
    framesep=1mm,
    rulecolor=\color{light-gray},
    backgroundcolor=\color{light-gray},
    basicstyle=\ttfamily,
}

% ----------------------------- Definizione tabella ---------------------------

\newcolumntype{C}[1]{>{\centering\arraybackslash}m{#1}}

%\setcellgapes{2ex} % Imposta l'altezza dell'header (2ex)


% ------------------------------Metadati indice --------------------------------
\title{\textbf{\fontsize{28}{6}\selectfont Indice}}
\author{\fontsize{14}{6}\selectfont ByteOps}
\date{Mese Giorno, Anno}


% -----------------------------Creazione footer --------------------------------

\pagestyle{fancy}
\fancyhf{}
\renewcommand{\footrulewidth}{0.4pt}
\lfoot{
    \parbox[c]{2cm}{\includegraphics[width=2cm]{../../../Images/logo.png}}
    \textcolor[RGB]{120, 120, 120}{$\cdot$ Verbale Esterno}
}
\rfoot{\thepage}

% --------------------------Modifica formato hyperlinks ------------------------

\hypersetup{
    colorlinks=true,
    linkcolor=black,
    filecolor=black,      
    pdftitle={Verbale Esterno dd/mm/yyyy},  %inserisci data verbale
    pdfpagemode=FullScreen,
}

% ------------------------------- Valore sotto-paragrafi indice --------------------------------------

\setcounter{secnumdepth}{4}
\setcounter{tocdepth}{4}

\titleformat{\section}
{\normalfont\huge\bfseries}{\thesection}{0.2cm}{}
\titlespacing*{\paragraph}{0pt}{0.5cm}{0.1cm}

\titleformat{\subsection}
{\normalfont\Large\bfseries}{\thesubsection}{0.2cm}{}
\titlespacing*{\paragraph}{0pt}{0.5cm}{0.1cm}

\titleformat{\subsubsection}
{\normalfont\large\bfseries}{\thesubsubsection}{0.2cm}{}
\titlespacing*{\paragraph}{0pt}{0.5cm}{0.1cm}

\titleformat{\paragraph}
{\normalfont\normalsize\bfseries}{\theparagraph}{0.2cm}{}
\titlespacing*{\paragraph}{0pt}{0.5cm}{0.1cm}

% ------------------------------- Front Page ---------------------------------------

\begin{document}

% --------------------------Aggiunta firma finale ------------------------
\begin{textblock*}{\textwidth}(0.85\textwidth, 1.16\textheight)
    Il responsabile: Nome Cognome
\end{textblock*}
% ------------------------------------------------------------------------

\pagestyle{fancy}
\begin{center}
\includegraphics[width = 0.7\textwidth]{../../../Images/logo.png} \\
\vspace{0.2cm}
\textcolor[RGB]{60, 60, 60}{\textit{ByteOps.swe@gmail.com}} \\
\vspace{1cm}
\fontsize{16}{6}\selectfont Verbale Esterno $\cdot$ Data: 24/11/2023 \\
\vspace{0.5cm}
\end{center}

\section*{Informazioni documento}
\def\arraystretch{1.2}
\begin{tabular}{>{\raggedleft\arraybackslash}p{0.2\textwidth}|>{\raggedright\arraybackslash}p{0.6\textwidth}c}
\hline
\addlinespace
\textbf{Luogo} & Luogo \vspace{10pt} \\
\textbf{Orario} & hh:mm - hh:mm \vspace{10pt} \\
\textbf{Redattore} & Nome \vspace{10pt} \\
\textbf{Verificatore} & Nome \vspace{10pt} \\
\textbf{Amministratore} & Nome \vspace{10pt} \\
\textbf{Destinatari} & T. Vardanega \\ & R. Cardin \vspace{10pt} \\
\textbf{Partecipanti} & A. Barutta \\ & E. Hysa \\ & R. Smanio \\ & D. Diotto \\ & F. Pozza \\ & L. Skenderi \\ & N. Preto \\ & A. Dorigo \\ & D. Zorzi \\ & F. Pallaro \vspace{10pt} \\
\end{tabular}
\pagebreak 

% ------------------------- Changelog ----------------------------

\section*{Registro delle modifiche}

\begin{tabular}{|C{2.5cm}|C{2.5cm}|C{2.5cm}|C{2.5cm}|C{2.5cm}|}
    \hline
    \textbf{Versione} & \textbf{Data} & \textbf{Autore} & \textbf{Verificatore} & \textbf{Dettaglio} \\
    \hline \hline
    0.0.1 & dd/mm/yyyy & Nome Cognome & Nome Cognome & Descrizione \\
    \hline
\end{tabular}
\pagebreak

% ------------------------- Generazione automatica indice ----------------------
\setstretch{1.5}
\maketitle
\thispagestyle{fancy}
\tableofcontents
\setstretch{1.2}
\pagebreak

% ------------------------ INIZIO DOCUMENTO ----------------------
\flushleft

\section{Revisione del periodo precedente}

\section{Ordine del giorno}
    \subsection{Misurazione rispetto alla call precedente}
    Viene dato un resoconto delle misurazioni effettuate rispetto alla call precedente e per eventuali problemi riscontrati effettuare delle misure correttive.
    Viene consigliato di condivere quanto fatto di modo da visualizzare i progressi fatti.
    \subsection{Visualizzazione del lavoro fatto}
    Viene mostrato il lavoro fatto fino a questo momento, con particolare attenzione alla scalabilità del codice prodotto. Viene consigliato di rivedere i metodi di connessione tra Kafka e Clickhouse poichè vengono resi disponibili 
    delle funzionalità già implementate da Kafka e Clickhouse. Viene mostrato che i sensori sono indipendenti tra di loro e che la generazione dei dati viene fatta seguendo una distribuzione basata sulla temperatura registrata precedentemente dal sensore per 
    garantire che i dati siano più realistici possibili.
    Per quanto riguarda in Docker, come immagine veiene consigliato quella originale di Apache.
    \subsection{Webinar su Docker}
    Viene introdotta la possibilità di partecipare ad un webinar, offerto dall'azienda proponente, su Docker per approfondire le conoscenze su questo strumento insieme all'altro gruppo.
    \subsection{Presentati i casi d'uso}
    Vengono presentati i vari casi d'uso e viene consiglaito di proporre casi d'uso solo se si ha un'idea di come implementarli.


% ------------------------ Firma Azienda ----------------------
\begin{textblock*}{\textwidth}(0.22\textwidth, 1.08\textheight)
    Luogo e Data: Padova, 24/11/2023
\end{textblock*}

\begin{textblock*}{\textwidth}(0.90\textwidth, 1.08\textheight)
        Firma referente Sync Lab:
\end{textblock*}
% -------------------------------------------------------------
\end{document}
