\documentclass{article}
\usepackage[utf8]{inputenc}
\usepackage[absolute]{textpos}
\usepackage[default]{raleway}
\usepackage{titlesec, comment, tabularx, makecell, listings, array, setspace, geometry, graphicx, xcolor, xparse, fancyvrb, relsize, fancyhdr, booktabs, hyperref} \usepackage{colortbl}
%\geometry{a4paper, left=2cm, right=2cm, top=2cm, bottom=2.5cm} 
\usepackage{multirow}
\renewcommand{\headrulewidth}{0pt}

% Definisci uno stile per i comandi git 
\definecolor{light-gray}{gray}{0.92}

\lstdefinestyle{code}{
    frame=single,
    framesep=1mm,
    rulecolor=\color{light-gray},
    backgroundcolor=\color{light-gray},
    basicstyle=\ttfamily,
}

% ----------------------------- Definizione tabella --------------------------- 

\newcolumntype{C}[1]{>{\centering\arraybackslash}m{#1}} 

%\setcellgapes{2ex} % Imposta l'altezza dell'header (2ex)


% ------------------------------Metadati indice -------------------------------- 
\title{\textbf{\fontsize{28}{6}\selectfont Indice}}
\author{\fontsize{14}{6}\selectfont ByteOps}
\date{Novembre 10, 2023}

% -----------------------------Creazione footer --------------------------------

\pagestyle{fancy}
\fancyhf{}
\renewcommand{\footrulewidth}{0.4pt}
\lfoot{
    \parbox[c]{2cm}{\includegraphics[width=2cm]{../../../Images/logo.png}}
    \textcolor[RGB]{120, 120, 120}{$\cdot$ Verbale Esterno}
}
\rfoot{\thepage}

% --------------------------Modifica formato hyperlinks ------------------------ 

\hypersetup{
    colorlinks=true,
    linkcolor=black,
    filecolor=black,
    pdftitle={Verbale Esterno 10/11/2023}, %inserisci data verbale
    pdfpagemode=FullScreen,
}

% ------------------------------- Valore sotto-paragrafi indice --------------------------------------

\setcounter{secnumdepth}{4}
\setcounter{tocdepth}{4}

\titleformat{\section}
{\normalfont\huge\bfseries}{\thesection}{0.2cm}{}
\titlespacing*{\paragraph}{0pt}{0.5cm}{0.1cm}

\titleformat{\subsection}
{\normalfont\Large\bfseries}{\thesubsection}{0.2cm}{}
\titlespacing*{\paragraph}{0pt}{0.5cm}{0.1cm}

\titleformat{\subsubsection}
{\normalfont\large\bfseries}{\thesubsubsection}{0.2cm}{}
\titlespacing*{\paragraph}{0pt}{0.5cm}{0.1cm}

\titleformat{\paragraph}
{\normalfont\normalsize\bfseries}{\theparagraph}{0.2cm}{}
\titlespacing*{\paragraph}{0pt}{0.5cm}{0.1cm}

% ------------------------------- Front Page --------------------------------------- 

\begin{document}

% --------------------------Aggiunta firma finale ------------------------ 

\begin{textblock*}{\textwidth}(0.85\textwidth, 1.08\textheight)
L'azienda: Sync Lab
\end{textblock*}

\begin{textblock*}{\textwidth}(0.85\textwidth, 1.16\textheight)
Il responsabile: Francesco Pozza
\end{textblock*}

% ------------------------------------------------------------------------

\pagestyle{fancy}
\begin{center}
\includegraphics[width = 0.7\textwidth]{../../../Images/logo.png} \\
\vspace{0.2cm}
\textcolor[RGB]{60, 60, 60}{\textit{ByteOps.swe@gmail.com}} \\
\vspace{1cm}
\fontsize{16}{6}\selectfont Verbale Esterno $\cdot$ Data: 10/11/2023 \\
\vspace{0.5cm}
\end{center}

\section*{Informazioni documento}
\def\arraystretch{1.2}
\begin{tabular}{>{\raggedleft\arraybackslash}p{0.2\textwidth}|>{\raggedright\arraybackslash}p {0.6\textwidth}c}
\hline
\addlinespace
\textbf{Luogo} & Google Meet \vspace{10pt} \\
\textbf{Orario} & 11:30 - 12:30 \vspace{10pt} \\
\textbf{Redattore} & R. Smanio \vspace{10pt} \\
\textbf{Verificatore} & E. Hysa \vspace{10pt} \\
\textbf{Amministratore} & L. Skenderi \vspace{10pt} \\
\textbf{Destinatari} & T. Vardanega \\ & R. Cardin \vspace{10pt} \\
\textbf{Partecipanti} & A. Barutta \\ & E. Hysa \\ & R. Smanio \\ & D. Diotto \\ & F. Pozza \\ & L. Skenderi \\ & N. Preto \\ & A. Dorigo \\ & D. Zorzi \\ & F. Pallaro \vspace{10pt} \\
\end{tabular}
\pagebreak

% ------------------------- Changelog ----------------------------

\section*{Registro delle modifiche}

\begin{tabular}{|C{2.5cm}|C{2.5cm}|C{2.5cm}|C{2.5cm}|C{2.5cm}|}
    \hline
    \textbf{Versione} & \textbf{Data} & \textbf{Autore} & \textbf{Verificatore} & \textbf{Dettaglio} \\
    \hline \hline
    0.0.1 & 10/11/2023 & R. Smanio & E. Hysa & Redazione documento \\
    \hline
\end{tabular}

\pagebreak

% ------------------------- Generazione automatica indice ---------------------- 

\setstretch{1.5}
\maketitle
\thispagestyle{fancy}
\tableofcontents
\setstretch{1.2}
\pagebreak

% ------------------------ INIZIO DOCUMENTO ----------------------
\flushleft

\section{Revisione del periodo precedente}
    Poiché questa riunione costituisce il primo incontro con l'azienda proponente da quando ci è stato assegnato il capitolato, non è stata necessaria una revisione del periodo precedente.  

\section{Ordine del giorno}

    \subsection{SAL Pianificati} 
        I referenti aziendali hanno proposto di programmare i SAL con una frequenza bisettimanale, poiché ritengono che un periodo di due settimane sia l'intervallo temporale ottimale per raggiungere gli obiettivi stabiliti in ciascun SAL. Durante ogni SAL, si procederà con un'analisi comparativa tra le aspettative definite e i risultati effettivamente conseguiti dal team.
        L'obiettivo delineato è la realizzazione di un POC entro la seconda metà di dicembre.  

    \subsection{Contatti con l'azienda} 
        È stato concordato che Element sarà il principale mezzo di comunicazione con i referenti aziendali. Verrà creato un canale dedicato in cui sarà possibile richiedere informazioni ed eventuali chiarimenti.  

    \subsection{Obiettivi per il primo sprint}
        \begin{itemize}
            \item Creazione di una versione semplificata di un sistema in grado di simulare la generazione dei dati di un sensore. La scelta del tipo di sensore è lasciata libera e i dati generati devono includere un ID, il tipo di sensore, il valore rilevato e il timestamp relativo alla rilevazione. È preferibile che il sensore emetta i dati in formato JSON. Non è necessario che il sensore generi grandi quantità di dati; inviare anche solo un dato al secondo è sufficiente, allo scopo di testare la ricezione di dati su Kafka.  
            \item Configurare una versione semplificata dell'ambiente di esecuzione mediante Docker Compose che permetta l'interconnessione delle diverse componenti del sistema. Inizialmente, è sufficiente verificare che i dati generati dal sensore vengano trasmessi con successo a Kafka, valutando eventualmente l'opzione di salvataggio su ClickHouse che altrimenti sarà svolta in futuro.
            \item Individuazione delle principali user stories relative all’intero progetto.  
        \end{itemize}

    \subsection{Domanda su Autorità e Cittadino, Use Case}
        L'obiettivo principale del progetto è sviluppare un sistema dedicato al monitoraggio dello stato di salute della città, al fine di agevolare la presa di decisioni informate e tempestive nella gestione delle risorse e nell'implementazione dei servizi. I principali destinatari del servizio saranno quindi figure istituzionali, quali autorità locali, enti comunali o organizzazioni affini.\\
        Successivamente, sarà possibile rendere accessibili al pubblico i dati monitorati attraverso portali online o app mobili. È importante notare, però, che i dati di interesse per i cittadini saranno diversi da quelli rilevanti per le autorità, delineando così obiettivi distinti per il progetto.  
    
    \subsection{Modalità di convalida verbali esterni}
        Dopo aver concluso le fasi di redazione e verifica, il documento verrà inviato all'azienda in formato PDF. Salvo eccezioni, l'azienda procederà con la lettura e la convalida mediante l'apposizione di una firma.  

\section{Attività da svolgere}  %le attività inserite servono solo a dare un'idea. non sono definitive, non sono scritte correttamente!
    \begin{center}
        \begin{tabular}{|C{7cm}|C{1,5cm}|C{3cm}|} 
            \hline
            \textbf{Titolo} & \textbf{\# Issue} & \textbf{Verificatore} \\ \hline\hline
            Aggiungi spazio firma azienda nel template verbale esterno & 21 & \multirow{6}{*}{E. Hysa}\\
            Identificare e definire le user stories & 22 & \\ 
            Progettare ed implementare un sensore tipo & 23 & \\ 
            Progettare ed implementare una versione semplificata dell'ambiente di esecuzione & 24 & \\ 
            Creazione repository dedicato al POC & 25 & \\ 
            \hline
        \end{tabular}
    \end{center}

\end{document}