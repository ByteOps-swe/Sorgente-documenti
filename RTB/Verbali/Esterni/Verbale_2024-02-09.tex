\documentclass{article}
\usepackage[utf8]{inputenc}
\usepackage[absolute]{textpos}
\usepackage[default]{raleway}
\usepackage{titlesec, comment, tabularx, makecell, listings, array, setspace, geometry, graphicx, xcolor, xparse, fancyvrb, relsize, fancyhdr, booktabs, hyperref}
\usepackage{colortbl}
%\geometry{a4paper, left=2cm, right=2cm, top=2cm, bottom=2.5cm}
\renewcommand{\headrulewidth}{0pt}

% Definisci uno stile per i comandi git
\definecolor{light-gray}{gray}{0.92}

\lstdefinestyle{code}{
    frame=single,
    framesep=1mm,
    rulecolor=\color{light-gray},
    backgroundcolor=\color{light-gray},
    basicstyle=\ttfamily,
}

% ----------------------------- Definizione tabella ---------------------------

\newcolumntype{C}[1]{>{\centering\arraybackslash}m{#1}}

%\setcellgapes{2ex} % Imposta l'altezza dell'header (2ex)


% ------------------------------Metadati indice --------------------------------
\title{\textbf{\fontsize{28}{6}\selectfont Indice}}
\author{\fontsize{14}{6}\selectfont ByteOps}
\date{Febbraio 09, 2024}


% -----------------------------Creazione footer --------------------------------

\pagestyle{fancy}
\fancyhf{}
\renewcommand{\footrulewidth}{0.4pt}
\lfoot{
    \parbox[c]{2cm}{\includegraphics[width=2cm]{../../../Images/logo.png}}
    \textcolor[RGB]{120, 120, 120}{$\cdot$ Verbale Esterno}
}
\rfoot{\thepage}

% --------------------------Modifica formato hyperlinks ------------------------

\hypersetup{
    colorlinks=true,
    linkcolor=black,
    filecolor=black,      
    pdftitle={Verbale Esterno 09/02/2024},  %inserisci data verbale
    pdfpagemode=FullScreen,
}

% ------------------------------- Valore sotto-paragrafi indice --------------------------------------

\setcounter{secnumdepth}{4}
\setcounter{tocdepth}{4}

\titleformat{\section}
{\normalfont\huge\bfseries}{\thesection}{0.2cm}{}
\titlespacing*{\paragraph}{0pt}{0.5cm}{0.1cm}

\titleformat{\subsection}
{\normalfont\Large\bfseries}{\thesubsection}{0.2cm}{}
\titlespacing*{\paragraph}{0pt}{0.5cm}{0.1cm}

\titleformat{\subsubsection}
{\normalfont\large\bfseries}{\thesubsubsection}{0.2cm}{}
\titlespacing*{\paragraph}{0pt}{0.5cm}{0.1cm}

\titleformat{\paragraph}
{\normalfont\normalsize\bfseries}{\theparagraph}{0.2cm}{}
\titlespacing*{\paragraph}{0pt}{0.5cm}{0.1cm}

% ------------------------------- Front Page ---------------------------------------

\begin{document}

% --------------------------Aggiunta firma finale ------------------------
\begin{textblock*}{\textwidth}(0.85\textwidth, 1.16\textheight)
    Il responsabile: Andrea Barutta
\end{textblock*}
% ------------------------------------------------------------------------

\pagestyle{fancy}
\begin{center}
\includegraphics[width = 0.7\textwidth]{../../../Images/logo.png} \\
\vspace{0.2cm}
\textcolor[RGB]{60, 60, 60}{\textit{ByteOps.swe@gmail.com}} \\
\vspace{1cm}
\fontsize{16}{6}\selectfont Verbale Esterno $\cdot$ Data: 09/02/2024 \\
\vspace{0.5cm}
\end{center}

\section*{Informazioni documento}
\def\arraystretch{1.2}
\begin{tabular}{>{\raggedleft\arraybackslash}p{0.2\textwidth}|>{\raggedright\arraybackslash}p{0.6\textwidth}c}
\hline
\addlinespace
\textbf{Luogo} & Google Meet \vspace{10pt} \\
\textbf{Orario} & 16:30 - 17:30 \vspace{10pt} \\
\textbf{Redattore} & N. Preto \vspace{10pt} \\
\textbf{Verificatore} & F. Pozza \vspace{10pt} \\
\textbf{Amministratore} & E. Hysa \vspace{10pt} \\
\textbf{Destinatari} & T. Vardanega \\ & R. Cardin \vspace{10pt} \\
\textbf{Partecipanti} & A. Barutta \\ & E. Hysa \\ & R. Smanio \\ & D. Diotto \\ & F. Pozza \\ & L. Skenderi \\ & N. Preto \\ & D. Zorzi \\ & A. Dorigo \vspace{10pt}
\end{tabular}
\pagebreak 

% ------------------------- Changelog ----------------------------

\section*{Registro delle modifiche}

\begin{tabular}{|C{2.5cm}|C{2.5cm}|C{2.5cm}|C{2.5cm}|C{2.5cm}|}
    \hline
    \textbf{Versione} & \textbf{Data} & \textbf{Autore} & \textbf{Verificatore} & \textbf{Dettaglio} \\
    \hline \hline
    0.1.0 & 09/02/2024 & N. Preto & F. Pozza & Redazione documento \\
    \hline
\end{tabular}
\pagebreak

% ------------------------- Generazione automatica indice ----------------------
\setstretch{1.5}
\maketitle
\thispagestyle{fancy}
\tableofcontents
\setstretch{1.2}
\pagebreak

% ------------------------ INIZIO DOCUMENTO ----------------------
\flushleft

\section{Revisione del periodo precedente}

Il 6/02/2024 si è tenuto l'incontro con il Professor Cardin per la presentazione del Proof of Concept. Nonostante siano emerse alcune perplessità da parte del docente, il suo giudizio complessivo è stato positivo. È stato comunicato che, a meno di sorprese nel documento di Analisi dei Requisiti, il gruppo potrà procedere con l'incontro successivo con il Professor Vardanega.

\section{Ordine del giorno}
    \subsection{Analisi della presentazione del Proof of Concept}

    Durante l'incontro con il \textit{proponente}\textsubscript{\textit{G}}, si sono affrontati i temi trattati durante la presentazione del Proof of Concept, concentrandoci soprattutto sui punti di perplessità emersi e sulle possibili soluzioni per affrontarli.
    
    \subsection{Spunti sugli approcci futuri nello sviluppo del progetto}
    Alcuni punti principali su cui è stato posto l'accento sono stati:
        \begin{enumerate}
        \item Utilità di uno \textit{script}\textsubscript{\textit{G}} consumer e connessione diretta tra \textit{Apache Kafka}\textsubscript{\textit{G}} e \textit{Clickhouse}\textsubscript{\textit{G}};
        \item Implementazione di \textit{framework}\textsubscript{\textit{G}} integrativi per il data streaming o in alternativa l'utilizzo di una connessione diretta;
        \item Test di carico e di performance del \textit{sistema}\textsubscript{\textit{G}} nella realtà distribuita.
        \end{enumerate}
    
        La risposta al primo e secondo punto è stata la decisione di rendere diretta la connessione tra Kafka e \textit{Clickhouse}\textsubscript{\textit{G}} rimuovendo l'intermediario da noi costruito, di cui la scalabilità sarebbe stata gestita internamente senza l'utilizzo di un \textit{framework}\textsubscript{\textit{G}} apposito comportando quindi evidenti rischi come casi particolari non gestiti. Per il calcolo del punteggio di salute, che si ricorda essere un requisito opzionale, si è deciso di utilizzare uno \textit{script}\textsubscript{\textit{G}} con un \textit{framework}\textsubscript{\textit{G}} o libreria appropriato come \textit{kafka stream} per \textit{Java} oppure \textit{Faust} per \textit{Python}, che reperisca i dati da Kafka e li reinserisca in una coda dedicata al punteggio di salute, cosicché i servizi interessati possano ottenere il dato.
        La decisione presa per affrontare il terzo punto implica la richiesta di un incontro con il Professore Cardin oppure il contatto via email, al fine di chiarire quali \textit{test}\textsubscript{\textit{G}} si prevede debbano essere condotti, tenendo conto dell'impossibilità di ricreare un ambiente realistico distribuito.
    
    \pagebreak
   
    \subsection{Obbiettivi per il prossimo sprint}
    
    In vista del prossimo \textit{SAL}\textsubscript{\textit{G}} pianificato per la data 15/02/2024 si è deciso di: 
    \begin{itemize}
        \item Studiare le possibili nuove tecnologie (\textit{kafka stream} per \textit{Java} oppure \textit{Faust} per \textit{Python}) analizzate durante l'incontro con il \textit{proponente}\textsubscript{\textit{G}}; 
        \item Organizzare e presentare al Professor Vardanega la documentazione relativa all'\textit{RTB}\textsubscript{\textit{G}}.
    \end{itemize}

% ------------------------ Firma Azienda ----------------------
\begin{textblock*}{\textwidth}(0.35\textwidth, 1.08\textheight)
    Padova, 09/02/2024
\end{textblock*}

\begin{textblock*}{\textwidth}(0.80\textwidth, 1.08\textheight)
        Firma referente Sync Lab:
\end{textblock*}
% -------------------------------------------------------------
\end{document}
