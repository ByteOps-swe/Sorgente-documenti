\documentclass{article}
\usepackage[utf8]{inputenc}
\usepackage[absolute]{textpos}
\usepackage[default]{raleway}
\usepackage{titlesec, comment, tabularx, makecell, listings, array, setspace, geometry, graphicx, xcolor, xparse, fancyvrb, relsize, fancyhdr, booktabs, hyperref}
\usepackage{colortbl}
%\geometry{a4paper, left=2cm, right=2cm, top=2cm, bottom=2.5cm}
\renewcommand{\headrulewidth}{0pt}

% Definisci uno stile per i comandi git
\definecolor{light-gray}{gray}{0.92}

\lstdefinestyle{code}{
    frame=single,
    framesep=1mm,
    rulecolor=\color{light-gray},
    backgroundcolor=\color{light-gray},
    basicstyle=\ttfamily,
}

% ----------------------------- Definizione tabella ---------------------------

\newcolumntype{C}[1]{>{\centering\arraybackslash}m{#1}}

%\setcellgapes{2ex} % Imposta l'altezza dell'header (2ex)


% ------------------------------Metadati indice --------------------------------
\title{\textbf{\fontsize{28}{6}\selectfont Indice}}
\author{\fontsize{14}{6}\selectfont ByteOps}
\date{Mese Giorno, Anno}


% -----------------------------Creazione footer --------------------------------

\pagestyle{fancy}
\fancyhf{}
\renewcommand{\footrulewidth}{0.4pt}
\lfoot{
    \parbox[c]{2cm}{\includegraphics[width=2cm]{../../../Images/logo.png}}
    \textcolor[RGB]{120, 120, 120}{$\cdot$ Verbale Esterno}
}
\rfoot{\thepage}

% --------------------------Modifica formato hyperlinks ------------------------

\hypersetup{
    colorlinks=true,
    linkcolor=black,
    filecolor=black,      
    pdftitle={Verbale Esterno 16/02/2024},  %inserisci data verbale
    pdfpagemode=FullScreen,
}

% ------------------------------- Valore sotto-paragrafi indice --------------------------------------

\setcounter{secnumdepth}{4}
\setcounter{tocdepth}{4}

\titleformat{\section}
{\normalfont\huge\bfseries}{\thesection}{0.2cm}{}
\titlespacing*{\paragraph}{0pt}{0.5cm}{0.1cm}

\titleformat{\subsection}
{\normalfont\Large\bfseries}{\thesubsection}{0.2cm}{}
\titlespacing*{\paragraph}{0pt}{0.5cm}{0.1cm}

\titleformat{\subsubsection}
{\normalfont\large\bfseries}{\thesubsubsection}{0.2cm}{}
\titlespacing*{\paragraph}{0pt}{0.5cm}{0.1cm}

\titleformat{\paragraph}
{\normalfont\normalsize\bfseries}{\theparagraph}{0.2cm}{}
\titlespacing*{\paragraph}{0pt}{0.5cm}{0.1cm}

% ------------------------------- Front Page ---------------------------------------

\begin{document}

% --------------------------Aggiunta firma finale ------------------------
\begin{textblock*}{\textwidth}(0.85\textwidth, 1.16\textheight)
    Il responsabile: Nome Cognome
\end{textblock*}
% ------------------------------------------------------------------------

\pagestyle{fancy}
\begin{center}
\includegraphics[width = 0.7\textwidth]{../../../Images/logo.png} \\
\vspace{0.2cm}
\textcolor[RGB]{60, 60, 60}{\textit{ByteOps.swe@gmail.com}} \\
\vspace{1cm}
\fontsize{16}{6}\selectfont Verbale Esterno $\cdot$ Data: dd/mm/yyyy \\
\vspace{0.5cm}
\end{center}

\section*{Informazioni documento}
\def\arraystretch{1.2}
\begin{tabular}{>{\raggedleft\arraybackslash}p{0.2\textwidth}|>{\raggedright\arraybackslash}p{0.6\textwidth}c}
\hline
\addlinespace
\textbf{Luogo} & Luogo \vspace{10pt} \\
\textbf{Orario} & 16:15 - 17:15 \vspace{10pt} \\
\textbf{Redattore} & R. Smanio \vspace{10pt} \\
\textbf{Verificatore} & A. Barutta \vspace{10pt} \\
\textbf{Amministratore} & Nome \vspace{10pt} \\
\textbf{Destinatari} & T. Vardanega \\ & R. Cardin \vspace{10pt} \\
\textbf{Partecipanti} & A. Barutta \\ & E. Hysa \\ & R. Smanio \\ & D. Diotto \\ & F. Pozza \\ & L. Skenderi \\ & N. Preto \\ & A. Dorigo \\ & D. Zorzi \\ & F. Pallaro \vspace{10pt} \\
\end{tabular}
\pagebreak 

% ------------------------- Changelog ----------------------------

\section*{Registro delle modifiche}

\begin{tabular}{|C{2.5cm}|C{2.5cm}|C{2.5cm}|C{2.5cm}|C{2.5cm}|}
    \hline
    \textbf{Versione} & \textbf{Data} & \textbf{Autore} & \textbf{Verificatore} & \textbf{Dettaglio} \\
    \hline \hline
    0.0.1 & 19/02/2024 & R. Smanio & A. Barutta & Redazione documento \\
    \hline
\end{tabular}
\pagebreak

% ------------------------- Generazione automatica indice ----------------------
\setstretch{1.5}
\maketitle
\thispagestyle{fancy}
\tableofcontents
\setstretch{1.2}
\pagebreak

% ------------------------ INIZIO DOCUMENTO ----------------------
\flushleft

\section{Revisione del periodo precedente}
Nel periodo precedente abbiamo riesaminato il lavoro svolto nel POC per apportarvi miglioramenti e utilizzarlo come fondamento per lo sviluppo dell'MVP. \\
Durante il meeting, nella fase di revisione di tale periodo, ci siamo dedicati all'analisi della struttura del database. In particolare, abbiamo discusso relativamente alla configurazione delle tabelle relative alle misurazioni trasmesse dai sensori, con particolare attenzione all'inclusione di dati "statici", come la posizione del sensore (latitudine e longitudine), in ciascuna tabella. È emerso che tale scelta comporta duplicazione dei dati e un utilizzo aggiuntivo di risorse di memoria. \\
Tuttavia, rispetto a una configurazione delle tabelle che miri a evitare la ridondanza, è stato riconosciuto che questo approccio offre un vantaggio significativo in termini di riduzione del costo computazionale, soprattutto durante operazioni di join su ampi set di dati. In questa prospettiva, il costo computazionale risulta essere più impattante rispetto all'eventuale costo di memoria aggiuntiva derivante dalla duplicazione dei dati. \\
È stato anche sollevato il punto che questa implementazione potrebbe rendere il database non conforme alla forma normale. Tuttavia, nel contesto di database non relazionali come quello adottato, tale caratteristica è considerata accettabile e non compromette l'efficienza complessiva del sistema.

\section{Ordine del giorno}

    \subsection{Manuale Utente}
    Procedura di installazione o no? 
    A quanto pare no, basta il link di accesso, non è necessario specificare procedure di installazione (Docker, Git Clone, ecc...) in quanto non è di pertinenza dell'utente.

    \subsection{Test di integrazione}
    Come svolgere test di integrazione nel nostro contesto.
    Come effettuare test di integrazione: automatizzare troppo complicato, test di unità solo sui sensori abbastanza inutili, test a mano relativamente a quello di integrazione. controlliamo che i dati scorrano tutta la catena, kafka, clickhouse e verifica visualizzazione dato a grafana.
    
    inviare set predeterminato di misurazioni così da indentificarle e verificare se proprio quelle misurazioni arrivano in Grafana     

\section{Niente richieste e chiarimenti?}

% ------------------------ Firma Azienda ----------------------
\begin{textblock*}{\textwidth}(0.22\textwidth, 1.08\textheight)
    Luogo e Data: Padova, gg/mm/yy
\end{textblock*}

\begin{textblock*}{\textwidth}(0.90\textwidth, 1.08\textheight)
        Firma referente Sync Lab:
\end{textblock*}
% -------------------------------------------------------------
\end{document}
