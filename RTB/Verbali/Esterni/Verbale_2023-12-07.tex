\documentclass{article}
\usepackage[utf8]{inputenc}
\usepackage[absolute]{textpos}
\usepackage[default]{raleway}
\usepackage{titlesec, comment, tabularx, makecell, listings, array, setspace, geometry, graphicx, xcolor, xparse, fancyvrb, relsize, fancyhdr, booktabs, hyperref}
\usepackage{colortbl}
%\geometry{a4paper, left=2cm, right=2cm, top=2cm, bottom=2.5cm}
\renewcommand{\headrulewidth}{0pt}

% Definisci uno stile per i comandi git
\definecolor{light-gray}{gray}{0.92}

\lstdefinestyle{code}{
    frame=single,
    framesep=1mm,
    rulecolor=\color{light-gray},
    backgroundcolor=\color{light-gray},
    basicstyle=\ttfamily,
}

% ----------------------------- Definizione tabella ---------------------------

\newcolumntype{C}[1]{>{\centering\arraybackslash}m{#1}}

%\setcellgapes{2ex} % Imposta l'altezza dell'header (2ex)


% ------------------------------Metadati indice --------------------------------
\title{\textbf{\fontsize{28}{6}\selectfont Indice}}
\author{\fontsize{14}{6}\selectfont ByteOps}
\date{Dicembre 7, 2023}


% -----------------------------Creazione footer --------------------------------

\pagestyle{fancy}
\fancyhf{}
\renewcommand{\footrulewidth}{0.4pt}
\lfoot{
    \parbox[c]{2cm}{\includegraphics[width=2cm]{../../../Images/logo.png}}
    \textcolor[RGB]{120, 120, 120}{$\cdot$ Verbale Esterno}
}
\rfoot{\thepage}

% --------------------------Modifica formato hyperlinks ------------------------

\hypersetup{
    colorlinks=true,
    linkcolor=black,
    filecolor=black,      
    pdftitle={Verbale Esterno 07/12/2023},  %inserisci data verbale
    pdfpagemode=FullScreen,
}

% ------------------------------- Valore sotto-paragrafi indice --------------------------------------

\setcounter{secnumdepth}{4}
\setcounter{tocdepth}{4}

\titleformat{\section}
{\normalfont\huge\bfseries}{\thesection}{0.2cm}{}
\titlespacing*{\paragraph}{0pt}{0.5cm}{0.1cm}

\titleformat{\subsection}
{\normalfont\Large\bfseries}{\thesubsection}{0.2cm}{}
\titlespacing*{\paragraph}{0pt}{0.5cm}{0.1cm}

\titleformat{\subsubsection}
{\normalfont\large\bfseries}{\thesubsubsection}{0.2cm}{}
\titlespacing*{\paragraph}{0pt}{0.5cm}{0.1cm}

\titleformat{\paragraph}
{\normalfont\normalsize\bfseries}{\theparagraph}{0.2cm}{}
\titlespacing*{\paragraph}{0pt}{0.5cm}{0.1cm}

% ------------------------------- Front Page ---------------------------------------

\begin{document}

% --------------------------Aggiunta firma finale ------------------------
\begin{textblock*}{\textwidth}(0.85\textwidth, 1.16\textheight)
    Il responsabile: Lisien Skenderi
\end{textblock*}
% ------------------------------------------------------------------------

\pagestyle{fancy}
\begin{center}
\includegraphics[width = 0.7\textwidth]{../../../Images/logo.png} \\
\vspace{0.2cm}
\textcolor[RGB]{60, 60, 60}{\textit{ByteOps.swe@gmail.com}} \\
\vspace{1cm}
\fontsize{16}{6}\selectfont Verbale Esterno $\cdot$ Data: 07/12/2023 \\
\vspace{0.5cm}
\end{center}

\section*{Informazioni documento}
\def\arraystretch{1.2}
\begin{tabular}{>{\raggedleft\arraybackslash}p{0.2\textwidth}|>{\raggedright\arraybackslash}p{0.6\textwidth}c}
\hline
\addlinespace
\textbf{Luogo} & Google Meet \vspace{10pt} \\
\textbf{Orario} & 16:30 - 17:30 \vspace{10pt} \\
\textbf{Redattore} & E. Hysa \vspace{10pt} \\
\textbf{Verificatore} & R. Smanio \vspace{10pt} \\
\textbf{Amministratore} & A. Barutta \vspace{10pt} \\
\textbf{Destinatari} & T. Vardanega \\ & R. Cardin \vspace{10pt} \\
\textbf{Partecipanti} & A. Barutta \\ & E. Hysa \\ & R. Smanio \\ & D. Diotto \\ & F. Pozza \\ & L. Skenderi \\ & N. Preto \\ & D. Zorzi \vspace{10pt}
\end{tabular}
\pagebreak 

% ------------------------- Changelog ----------------------------

\section*{Registro delle modifiche}

\begin{tabular}{|C{2.5cm}|C{2.5cm}|C{2.5cm}|C{2.5cm}|C{2.5cm}|}
    \hline
    \textbf{Versione} & \textbf{Data} & \textbf{Autore} & \textbf{Verificatore} & \textbf{Dettaglio} \\
    \hline \hline
    0.1.0 & 07/12/2023 & E.Hysa & R.Smanio & Redazione documento. \\
    \hline
\end{tabular}
\pagebreak

% ------------------------- Generazione automatica indice ----------------------
\setstretch{1.5}
\maketitle
\thispagestyle{fancy}
\tableofcontents
\setstretch{1.2}
\pagebreak

% ------------------------ INIZIO DOCUMENTO ----------------------
\flushleft

\section{Revisione del periodo precedente}
Tutte le \textit{attività}\textsubscript{\textit{G}} previste per questo \textit{SAL}\textsubscript{\textit{G}} sono state completate con successo, lasciando sia il team che l'azienda \textit{proponente}\textsubscript{\textit{G}} soddisfatti dello stato attuale del PoC.\\
Restano alcune migliorie da apportare all'interfaccia utente (UI) e la risoluzione di un bug che occasionalmente influisce sulla corretta visualizzazione dei dati nei grafici.\\
La pianificazione e l'esecuzione delle \textit{attività}\textsubscript{\textit{G}} sono state gestite in modo organizzato, rispettando i tempi stabiliti e aderendo alle nostre direttive interne, di conseguenza non sono emersi elementi significativi che richiedano modifiche nel nostro approccio lavorativo.

\section{Ordine del giorno}
    \subsection{Confronto sul lavoro svolto}
    È stato presentato all'azienda \textit{proponente}\textsubscript{\textit{G}} lo stato attuale del PoC, il quale ora, rispetto al \textit{SAL}\textsubscript{\textit{G}} precedente, include una \textit{dashboard}\textsubscript{\textit{G}} con vari \textit{widget}\textsubscript{\textit{G}} per visualizzare i dati provenienti dai sensori.\\
    Inoltre, abbiamo informato il \textit{proponente}\textsubscript{\textit{G}} riguardo a un bug che si manifesta in circostanze ancora non ben definite, il quale, in certi intervalli temporali, compromette la corretta visualizzazione dei grafici.\\
    Il \textit{proponente}\textsubscript{\textit{G}} ha approvato lo stato attuale del PoC e ha espresso l'intenzione di avanzare ulteriormente, integrando sensori aggiuntivi per ottenere una mappatura più completa e dettagliata.

    \subsection{Obiettivi del prossimo sprint}
    Per il prossimo \textit{SAL}\textsubscript{\textit{G}} è stato richiesto un miglioramento del \textit{front-end}\textsubscript{\textit{G}}, lavorando sull'ottimizzazione dell'UI e sulla diversificazione dei grafici.\\
    Inoltre, considerando l'ampio volume di dati generato dai sensori, si è suggerito di condurre alcuni \textit{test}\textsubscript{\textit{G}} per valutare la reattività del \textit{sistema}\textsubscript{\textit{G}} di fronte alla necessità di visualizzare grandi quantità di informazioni. Per effettuare questa valutazione, è essenziale far funzionare l'applicazione per un certo periodo di tempo al fine di raccogliere un insieme piuttosto vasto di misurazioni e condurre delle query su tali dati.\\
    Un'ulteriore richiesta è quella di implementare un nuovo \textit{sensore}\textsubscript{\textit{G}} per rappresentare nella mappa della città le colonnine di ricarica per auto elettriche, mostrando il loro stato, che può essere "attivo" o "non attivo", attraverso un controllo a intervalli di 30 minuti. 

\section{Richieste e chiarimenti}
    Durante la presentazione sono stati posti degli interrogativi riguardanti possibili miglioramenti nell'utilizzo di \textit{Docker}\textsubscript{\textit{G}}, nello specifico relativi all'implementazione dei volumes. \\
    Su suggerimento dell'azienda, è stato consigliato di riportare tali domande nel canale Element dedicato, in modo da poter ricevere risposte dai membri aziendali più esperti sull'argomento, alcuni dei quali assenti durante la riunione.  

% ------------------------ Firma Azienda ----------------------
\begin{textblock*}{\textwidth}(0.35\textwidth, 1.08\textheight)
    Padova, 07/12/2023
\end{textblock*}

\begin{textblock*}{\textwidth}(0.80\textwidth, 1.08\textheight)
        Firma referente Sync Lab:
\end{textblock*}
% -------------------------------------------------------------
\end{document}
