\documentclass{article}
\usepackage[utf8]{inputenc}
\usepackage[absolute]{textpos}
\usepackage[default]{raleway}
\usepackage{titlesec, comment, tabularx, makecell, listings, array, setspace, geometry, graphicx, xcolor, xparse, fancyvrb, relsize, fancyhdr, booktabs, hyperref}
\usepackage{colortbl}
%\geometry{a4paper, left=2cm, right=2cm, top=2cm, bottom=2.5cm}
\renewcommand{\headrulewidth}{0pt}

% Definisci uno stile per i comandi git
\definecolor{light-gray}{gray}{0.92}

\lstdefinestyle{code}{
    frame=single,
    framesep=1mm,
    rulecolor=\color{light-gray},
    backgroundcolor=\color{light-gray},
    basicstyle=\ttfamily,
}

% ----------------------------- Definizione tabella ---------------------------

\newcolumntype{C}[1]{>{\centering\arraybackslash}m{#1}}

%\setcellgapes{2ex} % Imposta l'altezza dell'header (2ex)


% ------------------------------Metadati indice --------------------------------
\title{\textbf{\fontsize{28}{6}\selectfont Indice}}
\author{\fontsize{14}{6}\selectfont ByteOps}
\date{Dicembre 7, 2023}


% -----------------------------Creazione footer --------------------------------

\pagestyle{fancy}
\fancyhf{}
\renewcommand{\footrulewidth}{0.4pt}
\lfoot{
    \parbox[c]{2cm}{\includegraphics[width=2cm]{../../../Images/logo.png}}
    \textcolor[RGB]{120, 120, 120}{$\cdot$ Verbale Esterno}
}
\rfoot{\thepage}

% --------------------------Modifica formato hyperlinks ------------------------

\hypersetup{
    colorlinks=true,
    linkcolor=black,
    filecolor=black,      
    pdftitle={Verbale Esterno 07/12/2023},  %inserisci data verbale
    pdfpagemode=FullScreen,
}

% ------------------------------- Valore sotto-paragrafi indice --------------------------------------

\setcounter{secnumdepth}{4}
\setcounter{tocdepth}{4}

\titleformat{\section}
{\normalfont\huge\bfseries}{\thesection}{0.2cm}{}
\titlespacing*{\paragraph}{0pt}{0.5cm}{0.1cm}

\titleformat{\subsection}
{\normalfont\Large\bfseries}{\thesubsection}{0.2cm}{}
\titlespacing*{\paragraph}{0pt}{0.5cm}{0.1cm}

\titleformat{\subsubsection}
{\normalfont\large\bfseries}{\thesubsubsection}{0.2cm}{}
\titlespacing*{\paragraph}{0pt}{0.5cm}{0.1cm}

\titleformat{\paragraph}
{\normalfont\normalsize\bfseries}{\theparagraph}{0.2cm}{}
\titlespacing*{\paragraph}{0pt}{0.5cm}{0.1cm}

% ------------------------------- Front Page ---------------------------------------

\begin{document}

% --------------------------Aggiunta firma finale ------------------------
\begin{textblock*}{\textwidth}(0.85\textwidth, 1.16\textheight)
    Il responsabile: Lisien Skenderi
\end{textblock*}
% ------------------------------------------------------------------------

\pagestyle{fancy}
\begin{center}
\includegraphics[width = 0.7\textwidth]{../../../Images/logo.png} \\
\vspace{0.2cm}
\textcolor[RGB]{60, 60, 60}{\textit{ByteOps.swe@gmail.com}} \\
\vspace{1cm}
\fontsize{16}{6}\selectfont Verbale Esterno $\cdot$ Data: 07/12/2023 \\
\vspace{0.5cm}
\end{center}

\section*{Informazioni documento}
\def\arraystretch{1.2}
\begin{tabular}{>{\raggedleft\arraybackslash}p{0.2\textwidth}|>{\raggedright\arraybackslash}p{0.6\textwidth}c}
\hline
\addlinespace
\textbf{Luogo} & Google Meet \vspace{10pt} \\
\textbf{Orario} & 16:30 - 17:30 \vspace{10pt} \\
\textbf{Redattore} & Nome \vspace{10pt} \\
\textbf{Verificatore} & Nome \vspace{10pt} \\
\textbf{Amministratore} & Nome \vspace{10pt} \\
\textbf{Destinatari} & T. Vardanega \\ & R. Cardin \vspace{10pt} \\
\textbf{Partecipanti} & A. Barutta \\ & E. Hysa \\ & R. Smanio \\ & D. Diotto \\ & F. Pozza \\ & L. Skenderi \\ & N. Preto \\ & D. Zorzi \vspace{10pt}
\end{tabular}
\pagebreak 

% ------------------------- Changelog ----------------------------

\section*{Registro delle modifiche}

\begin{tabular}{|C{2.5cm}|C{2.5cm}|C{2.5cm}|C{2.5cm}|C{2.5cm}|}
    \hline
    \textbf{Versione} & \textbf{Data} & \textbf{Autore} & \textbf{Verificatore} & \textbf{Dettaglio} \\
    \hline \hline
    0.1.0 & 07/12/2023 & E.Hysa & R.Smanio & Redazione documento \\
    \hline
\end{tabular}
\pagebreak

% ------------------------- Generazione automatica indice ----------------------
\setstretch{1.5}
\maketitle
\thispagestyle{fancy}
\tableofcontents
\setstretch{1.2}
\pagebreak

% ------------------------ INIZIO DOCUMENTO ----------------------
\flushleft

\section{Revisione del periodo precedente}
L'esecuzione delle attività è stata attentamente rivista e aggiornata per aderire pienamente alle direttive fornite dalla proponente nel precedente SAL. Tale revisione ha comportato l'attuarsi di misure corretive nel lavoro svolto al fine di garantire un allineamento completo e accurato con le indicazioni ricevute.


\section{Ordine del giorno}
    \subsection{Presentazione del lavoro svolto}
    La presentazione del lavoro eseguito è stata conforme con gli obiettivi da attuare definiti nel precedente SAL.
    Sono state esplicitate le difficoltà riscontrate nell'utilizzo di Grafana, compresi i bug associati, attribuibili alla limitata disponibilità di documentazione durante il periodo considerato.

    Inoltre, è stata presentata la realizzazione della rappresentazione visiva dei dati sulla mappa presente nella dashboard Grafana, inclusi i relativi dati di misurazione. Il proponente ha approvato il lavoro eseguito come PoC, prospettando l'ulteriore passo avanti con l'integrazione di sensori aggiuntivi per conseguire una mappatura più esaustiva e dettagliata in fasi successive.

    \subsection{Obiettivi per il prossimo SAL}
    In vista del prossimo SAL, è stato richiesto come obiettivo principale, concentrarsi sul miglioramento della componente front-end, con una particolare attenzione alla visualizzazione lato utente, focalizzata sulla diversificazione dei grafici, è stato richiesto di eseguire test di visualizzazione su intervalli di tempo specifici che richiedono una considerevole quantità di dati raccolti e di sincronizzare i dati visualizzati nelle mappe con i dati visualizzati nei grafici a linee.
    
    L'azienda ha richiesto inoltre di implementare un nuovo sensore per rappresentare nella mappa le colonnine di ricarica per auto elettriche, mostrando il loro stato che può essere "attivo" o "non attivo" stabilendo un controllo a intervalli di 30 minuti.

\section{Richieste e chiarimenti}
Durante la presentazione sono stati posti interrogativi riguardanti migliorie nell'utilizzo di Docker, nello specifico relativi all'implementazione dei volumes. L'azienda ha consigliato di riportare tali domande nel canale Element designato per poter ricevere risposta dai membri aziendali più esperti nell'argomento, assenti al momento della riunione.

% ------------------------ Firma Azienda ----------------------
\begin{textblock*}{\textwidth}(0.35\textwidth, 1.08\textheight)
    Padova, 07/12/2023
\end{textblock*}

\begin{textblock*}{\textwidth}(0.80\textwidth, 1.08\textheight)
        Firma referente Sync Lab:
\end{textblock*}
% -------------------------------------------------------------
\end{document}
