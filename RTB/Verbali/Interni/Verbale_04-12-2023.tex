\documentclass{article}
\usepackage[utf8]{inputenc}
\usepackage[absolute]{textpos}
\usepackage[default]{raleway}
\usepackage{titlesec, comment, tabularx, makecell, listings, array, setspace, geometry, graphicx, xcolor, xparse, fancyvrb, relsize, fancyhdr, booktabs, hyperref}
\usepackage{colortbl}
%\geometry{a4paper, left=2cm, right=2cm, top=2cm, bottom=2.5cm}
\renewcommand{\headrulewidth}{0pt}

% Definisci uno stile per i comandi git
\definecolor{light-gray}{gray}{0.92}

\lstdefinestyle{code}{
    frame=single,
    framesep=1mm,
    rulecolor=\color{light-gray},
    backgroundcolor=\color{light-gray},
    basicstyle=\ttfamily,
}

% ----------------------------- Definizione tabella ---------------------------

\newcolumntype{C}[1]{>{\centering\arraybackslash}m{#1}}

%\setcellgapes{2ex} % Imposta l'altezza dell'header (2ex)


% ------------------------------Metadati indice --------------------------------
\title{\textbf{\fontsize{28}{6}\selectfont Indice}}
\author{\fontsize{14}{6}\selectfont ByteOps}
\date{Dicembre 4, 2023}

% -----------------------------Creazione footer --------------------------------

\pagestyle{fancy}
\fancyhf{}
\renewcommand{\footrulewidth}{0.4pt}
\lfoot{
    \parbox[c]{2cm}{\includegraphics[width=2cm]{../../../Images/logo.png}}
    \textcolor[RGB]{120, 120, 120}{$\cdot$ Verbale Interno}
}
\rfoot{\thepage}

% --------------------------Modifica formato hyperlinks ------------------------

\hypersetup{
    colorlinks=true,
    linkcolor=black,
    filecolor=black,      
    pdftitle={Verbale Interno 04/12/2023},  %inserisci data verbale
    pdfpagemode=FullScreen,
}

% ------------------------------- Valore sotto-paragrafi indice --------------------------------------

\setcounter{secnumdepth}{4}
\setcounter{tocdepth}{4}

\titleformat{\section}
{\normalfont\huge\bfseries}{\thesection}{0.2cm}{}
\titlespacing*{\paragraph}{0pt}{0.5cm}{0.1cm}

\titleformat{\subsection}
{\normalfont\Large\bfseries}{\thesubsection}{0.2cm}{}
\titlespacing*{\paragraph}{0pt}{0.5cm}{0.1cm}

\titleformat{\subsubsection}
{\normalfont\large\bfseries}{\thesubsubsection}{0.2cm}{}
\titlespacing*{\paragraph}{0pt}{0.5cm}{0.1cm}

\titleformat{\paragraph}
{\normalfont\normalsize\bfseries}{\theparagraph}{0.2cm}{}
\titlespacing*{\paragraph}{0pt}{0.5cm}{0.1cm}

% ------------------------------- Front Page ---------------------------------------

\begin{document}

% --------------------------Aggiunta firma finale ------------------------
\begin{textblock*}{\textwidth}(0.85\textwidth, 1.16\textheight)
    Il responsabile: Lisien Skenderi
\end{textblock*}
% ------------------------------------------------------------------------

\pagestyle{fancy}
\begin{center}
\includegraphics[width = 0.7\textwidth]{../../../Images/logo.png} \\
\vspace{0.2cm}
\textcolor[RGB]{60, 60, 60}{\textit{ByteOps.swe@gmail.com}} \\
\vspace{1cm}
\fontsize{16}{6}\selectfont Verbale Interno $\cdot$ Data: 04/12/20232 \\
\vspace{0.5cm}
\end{center}

\section*{Informazioni documento}
\def\arraystretch{1.2}
\begin{tabular}{>{\raggedleft\arraybackslash}p{0.2\textwidth}|>{\raggedright\arraybackslash}p{0.6\textwidth}c}
\hline
\addlinespace
\textbf{Luogo} & Discord \vspace{10pt} \\
\textbf{Orario} & 14:30 - 15:30 \vspace{10pt} \\
\textbf{Redattore} & D. Diotto \vspace{10pt} \\
\textbf{Verificatore} & R. Smanio \vspace{10pt} \\
\textbf{Amministratore} & A. Barutta \vspace{10pt} \\
\textbf{Destinatari} & T. Vardanega \\ & R. Cardin \vspace{10pt} \\
\textbf{Partecipanti} & A. Barutta \\ & E. Hysa \\ & R. Smanio \\ & D. Diotto \\ & F. Pozza \\ & L. Skenderi \\ & N. Preto \vspace{10pt} \\
\end{tabular}
\pagebreak 

% ------------------------- Changelog ----------------------------

\section*{Registro delle modifiche}

\begin{tabular}{|C{2.5cm}|C{2.5cm}|C{2.5cm}|C{2.5cm}|C{2.5cm}|}
    \hline
    \textbf{Versione} & \textbf{Data} & \textbf{Autore} & \textbf{Verificatore} & \textbf{Dettaglio} \\
    \hline \hline
    0.0.1 & 04/12/2023 & Davide Diotto & Riccardo Smanio & Prima stesura \\
    \hline
\end{tabular}
\pagebreak

% ------------------------- Generazione automatica indice ----------------------
\setstretch{1.5}
\maketitle
\thispagestyle{fancy}
\tableofcontents
\setstretch{1.2}
\pagebreak

% ------------------------ INIZIO DOCUMENTO ----------------------
\flushleft

\section{Revisione del periodo precedente}
Nel corso dell'incontro odierno, i primi minuti sono stati dedicati a una revisione dettagliata delle attività svolte dal precedente incontro e delle modalità di realizzazione delle stesse. In particolare, i membri incaricati dello sviluppo del codice da presentare durante l'incontro SAL di giovedì 07/12/2023 hanno presentato i progressi ottenuti, focalizzandosi sulle funzionalità implementate in Grafana. Si è esaminata l'integrazione di diversi tipi di grafici che offrono una rappresentazione visuale dei dati generati dai simulatori dei sensori. È stata, altresì, illustrata la nuova porzione di codice sviluppata con l'obiettivo di migliorare la verosimiglianza dei valori prodotti casualmente dai sensori.\\
Successivamente alla presentazione del codice, si è avviata una discussione riguardante lo stato di avanzamento delle attività relative alle issue aperte nel repository 'Documents' su Github a partire dall'ultimo incontro. Si è voluto approfondire lo stato di queste problematiche, valutando le azioni intraprese e identificando eventuali nuove prospettive per il loro completamento.

\section{Ordine del giorno}
    \subsection{Modifica della sezione “Redazione” in "Norme di Progetto"}
    Durante la riunione, si è svolta un'approfondita discussione riguardo alle modifiche necessarie per adeguare la sezione "Revisione" all'interno del documento delle Norme di Progetto. Tale necessità è emersa in seguito al recente processo di refactoring dei file sorgenti in \LaTeX. È stato riconosciuto l'importante correlazione tra le modifiche apportate e la necessità di mantenere la sezione "Revisione" allineata alle decisioni prese e alle attuali direzioni del progetto. L'obiettivo è garantire che questa sezione rifletta accuratamente le evoluzioni introdotte attraverso il refactoring, mantenendo coerenza e chiarezza nel documento.
    
    \subsection{Inizio stesura metriche e piano di qualifica}
    Durante la riunione, si è posto un'enfasi particolare sull'importanza di definire metriche adeguate per valutare il nostro progresso. È stato concordato di avviare la redazione del documento relativo al Piano di Qualifica.

    \subsection{Rotazione ruoli}
    La rotazione dei ruoli è stato un altro tema importante trattato, infatti durante l'incontro li abbiamo sanciti. Coloro che avevano incertezze sulle responsabilità associate ai loro nuovi ruoli hanno avuto l'opportunità di porre domande e chiarire i dubbi con coloro che hanno già esperienza in quei ruoli. Questo approccio mira a garantire una transizione fluida e un'assunzione coerente delle responsabilità.
    
    \subsection{Problemi sull'orario SAL 04/12/2023}
     Oggetto di discussione è stato anche l'orario proposto da SyncLab per il prossimo incontro SAL di Giovedì 07/12/2023. Alcuni membri del team hanno evidenziato problematiche personali e accademiche legate all'orario inizialmente proposto. Di conseguenza, si è concordato di comunicare all'azienda le difficoltà riscontrate e proporre un orario alternativo, cercando di garantire la partecipazione della maggioranza dei membri del team.

\section{Attività da svolgere}
    \begin{center}
        \begin{tabular}{|C{7cm}|C{1,5cm}|C{3cm}|}
            \hline
            \textbf{Titolo} & \textbf{\# Issue} & \textbf{Verificatore} \\
            \hline\hline
            Modifiche struttura "volumes" in docker-compose.yml & 36 & Nicola Preto\\
            
            Rimozione invio socket in "volumes" in docker-compose.yml & 37 & Nicola Preto \\
            
            Avvio client Clickhouse tramite nome container & 38 & Nicola Preto \\
            
            Modifica sezione Redazionein Norme di Progetto & 43 & Francesco Pozza \\
            \hline
        \end{tabular}
    \end{center}

\end{document}