\documentclass{article}
\usepackage[utf8]{inputenc}
\usepackage[absolute]{textpos}
\usepackage[default]{raleway}
\usepackage{titlesec, comment, tabularx, makecell, listings, array, setspace, geometry, graphicx, xcolor, xparse, fancyvrb, relsize, fancyhdr, booktabs, hyperref}
\usepackage{colortbl}
%\geometry{a4paper, left=2cm, right=2cm, top=2cm, bottom=2.5cm}
\renewcommand{\headrulewidth}{0pt}

% Definisci uno stile per i comandi git
\definecolor{light-gray}{gray}{0.92}

\lstdefinestyle{code}{
    frame=single,
    framesep=1mm,
    rulecolor=\color{light-gray},
    backgroundcolor=\color{light-gray},
    basicstyle=\ttfamily,
}

% ----------------------------- Definizione tabella ---------------------------

\newcolumntype{C}[1]{>{\centering\arraybackslash}m{#1}}

%\setcellgapes{2ex} % Imposta l'altezza dell'header (2ex)


% ------------------------------Metadati indice --------------------------------
\title{\textbf{\fontsize{28}{6}\selectfont Indice}}
\author{\fontsize{14}{6}\selectfont ByteOps}
\date{Dicembre 15, 2023}


% -----------------------------Creazione footer --------------------------------

\pagestyle{fancy}
\fancyhf{}
\renewcommand{\footrulewidth}{0.4pt}
\lfoot{
    \parbox[c]{2cm}{\includegraphics[width=2cm]{../../../Images/logo.png}}
    \textcolor[RGB]{120, 120, 120}{$\cdot$ Verbale Interno}
}
\rfoot{\thepage}

% --------------------------Modifica formato hyperlinks ------------------------

\hypersetup{
    colorlinks=true,
    linkcolor=black,
    filecolor=black,      
    pdftitle={Verbale Interno 15/12/2023},  %inserisci data verbale
    pdfpagemode=FullScreen,
}

% ------------------------------- Valore sotto-paragrafi indice --------------------------------------

\setcounter{secnumdepth}{4}
\setcounter{tocdepth}{4}

\titleformat{\section}
{\normalfont\huge\bfseries}{\thesection}{0.2cm}{}
\titlespacing*{\paragraph}{0pt}{0.5cm}{0.1cm}

\titleformat{\subsection}
{\normalfont\Large\bfseries}{\thesubsection}{0.2cm}{}
\titlespacing*{\paragraph}{0pt}{0.5cm}{0.1cm}

\titleformat{\subsubsection}
{\normalfont\large\bfseries}{\thesubsubsection}{0.2cm}{}
\titlespacing*{\paragraph}{0pt}{0.5cm}{0.1cm}

\titleformat{\paragraph}
{\normalfont\normalsize\bfseries}{\theparagraph}{0.2cm}{}
\titlespacing*{\paragraph}{0pt}{0.5cm}{0.1cm}

% ------------------------------- Front Page ---------------------------------------

\begin{document}

% --------------------------Aggiunta firma finale ------------------------
\begin{textblock*}{\textwidth}(0.85\textwidth, 1.16\textheight)
    Il responsabile: Riccardo Smanio
\end{textblock*}
% ------------------------------------------------------------------------

\pagestyle{fancy}
\begin{center}
\includegraphics[width = 0.7\textwidth]{../../../Images/logo.png} \\
\vspace{0.2cm}
\textcolor[RGB]{60, 60, 60}{\textit{ByteOps.swe@gmail.com}} \\
\vspace{1cm}
\fontsize{16}{6}\selectfont Verbale Interno $\cdot$ Data: 15/12/2023 \\
\vspace{0.5cm}
\end{center}

\section*{Informazioni documento}
\def\arraystretch{1.2}
\begin{tabular}{>{\raggedleft\arraybackslash}p{0.2\textwidth}|>{\raggedright\arraybackslash}p{0.6\textwidth}c}
\hline
\addlinespace
\textbf{Luogo} & Discord \vspace{10pt} \\
\textbf{Orario} & 14:30 - 15:30 \vspace{10pt} \\
\textbf{Redattore} & N. Preto \vspace{10pt} \\
\textbf{Verificatore} & D. Diotto \vspace{10pt} \\
\textbf{Amministratore} & D. Diotto \vspace{10pt} \\
\textbf{Destinatari} & T. Vardanega \\ & R. Cardin \vspace{10pt} \\
\textbf{Partecipanti} & A. Barutta \\ & E. Hysa \\ & R. Smanio \\ & D. Diotto \\ & F. Pozza \\ & L. Skenderi \\ & N. Preto \vspace{10pt} \\
\end{tabular}
\pagebreak 

% ------------------------- Changelog ----------------------------

\section*{Registro delle modifiche}

\begin{tabular}{|C{2.5cm}|C{2.5cm}|C{2.5cm}|C{2.5cm}|C{2.5cm}|}
    \hline
    \textbf{Versione} & \textbf{Data} & \textbf{Autore} & \textbf{Verificatore} & \textbf{Dettaglio} \\
    \hline \hline
    0.0.1 & 17/12/2023 & N. Preto & D. Diotto & Redazione documento \\
    \hline
\end{tabular}
\pagebreak

% ------------------------- Generazione automatica indice ----------------------
\setstretch{1.5}
\maketitle
\thispagestyle{fancy}
\tableofcontents
\setstretch{1.2}
\pagebreak

% ------------------------ INIZIO DOCUMENTO ----------------------
\flushleft

\section{Revisione del periodo precedente}
Durante la riunione, è stato esaminato lo stato di avanzamento delle attività assegnate nella riunione interna precedente, valutando quali issue sono state completate e quali necessitano di ulteriore tempo per essere portate a termine. \\
La stesura del documento di Analisi dei Requisiti è quasi completa: tutte le sezioni sono state redatte ad eccezione di quella relativa ai requisiti di vincolo e ai requisiti qualitativi. Inoltre, rimangono alcuni dettagli da perfezionare. \\
Il documento "Norme di Progetto" è stato completato, come precedentemente indicato nel verbale del meeting precedente, e attualmente è in corso una revisione finale per verificare la coerenza e la coesione di tutte le sezioni. \\
Il Piano di Qualifica è attualmente in fase di redazione. La maggior parte delle metriche sono già state redatte, ma alcune sezioni sono ancora in fase di lavorazione e devono essere completate. \\
Riguardo al nostro approccio lavorativo, si è notato che finora tutte le issue venivano create e gestite esclusivamente nella repository 'Documents'. Tuttavia, per assicurare una gestione più coerente, abbiamo deciso di distribuire le issue nelle relative repository: quelle concernenti la documentazione saranno create e gestite nella repository 'Documents', mentre le issue relative al Proof of Concept nella repository 'proof-of-concept'. Inoltre, sfruttando la funzionalità di Github che permette di collegare una stessa dashboard tra diverse repository, potremo avere una visione complessiva di tutte le issue e del loro stato, indipendentemente dalla repository di provenienza. \\
In conclusione, la pianificazione e l'esecuzione delle attività sono state condotte in modo metodico, rispettando le tempistiche stabilite e conformandosi alle direttive interne.

\section{Ordine del giorno}
    \subsection{Miglioramento POC}
    Attraverso una ricerca sulla possibilità di automatizzare il collegamento della sorgente dati a Grafana e sulla possibilità di caricare uno specifico set di widget all'apertura della schermata principale, è stato individuato che ciò è realizzabile tramite l'impiego dei file di configurazione. Nel prossimo sprint, sarà necessario creare due file di configurazione distinti (uno per il data source e uno per la dashboard) per implementare questa soluzione.

\section{Attività da svolgere}
    \begin{center}
        \begin{tabular}{|C{7cm}|C{1,5cm}|C{3cm}|}
            \hline
            \textbf{Titolo} & \textbf{\# Issue} & \textbf{Verificatore} \\
            \hline\hline
            Inserire configurazione Datasources & 60 & N. Preto \\
            Inserire configurazione Dashboard & 61 & N. Preto \\
            Selezione query in base a variabili Grafana & 62 & N. Preto \\
            Rimuovere limite di visualizzazione dati nella query & 63 & N. Preto \\
            Redazione verbale 15/12/2023 & 72 & D. Diotto \\
            \hline
        \end{tabular}
    \end{center}

\end{document}