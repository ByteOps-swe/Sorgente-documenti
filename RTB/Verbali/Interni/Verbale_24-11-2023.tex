\documentclass{article}
\usepackage[utf8]{inputenc}
\usepackage[absolute]{textpos}
\usepackage[default]{raleway}
\usepackage{titlesec, comment, tabularx, makecell, listings, array, setspace, geometry, graphicx, xcolor, xparse, fancyvrb, relsize, fancyhdr, booktabs, hyperref}
\usepackage{colortbl}
%\geometry{a4paper, left=2cm, right=2cm, top=2cm, bottom=2.5cm}
\renewcommand{\headrulewidth}{0pt}

% ----------------- Definisci uno stile per i comandi git ----------------------
\definecolor{light-gray}{gray}{0.92}

\lstdefinestyle{code}{
    frame=single,
    framesep=1mm,
    rulecolor=\color{light-gray},
    backgroundcolor=\color{light-gray},
    basicstyle=\ttfamily,
}

% ----------------------------- Definizione tabella ---------------------------
\newcolumntype{C}[1]{>{\centering\arraybackslash}m{#1}}

% ------------------------------Metadati indice --------------------------------
\title{\textbf{\fontsize{28}{6}\selectfont Indice}}
\author{\fontsize{14}{6}\selectfont ByteOps}
\date{Novembre 24, 2023}


% -----------------------------Creazione footer --------------------------------

\pagestyle{fancy}
\fancyhf{}
\renewcommand{\footrulewidth}{0.4pt}
\lfoot{
    \parbox[c]{2cm}{\includegraphics[width=2cm]{../../../Images/logo.png}}
    \textcolor[RGB]{120, 120, 120}{$\cdot$ Verbale Interno}
}
\rfoot{\thepage}

% --------------------------Modifica formato hyperlinks ------------------------

\hypersetup{
    colorlinks=true,
    linkcolor=black,
    filecolor=black,      
    pdftitle={Verbale Interno 24/11/2023},  %inserisci data verbale
    pdfpagemode=FullScreen,
}

% ------------------------------- Valore sotto-paragrafi indice --------------------------------------

\setcounter{secnumdepth}{4}
\setcounter{tocdepth}{4}

\titleformat{\section}
{\normalfont\huge\bfseries}{\thesection}{0.2cm}{}
\titlespacing*{\paragraph}{0pt}{0.5cm}{0.1cm}

\titleformat{\subsection}
{\normalfont\Large\bfseries}{\thesubsection}{0.2cm}{}
\titlespacing*{\paragraph}{0pt}{0.5cm}{0.1cm}

\titleformat{\subsubsection}
{\normalfont\large\bfseries}{\thesubsubsection}{0.2cm}{}
\titlespacing*{\paragraph}{0pt}{0.5cm}{0.1cm}

\titleformat{\paragraph}
{\normalfont\normalsize\bfseries}{\theparagraph}{0.2cm}{}
\titlespacing*{\paragraph}{0pt}{0.5cm}{0.1cm}

% ------------------------------- Front Page ---------------------------------------

\begin{document}

% --------------------------Aggiunta firma finale ------------------------
\begin{textblock*}{\textwidth}(0.85\textwidth, 1.16\textheight)
    Il responsabile: Francesco Pozza
\end{textblock*}

% ------------------------------------------------------------------------

\pagestyle{fancy}
\begin{center}
\includegraphics[width = 0.7\textwidth]{../../../Images/logo.png} \\
\vspace{0.2cm}
\textcolor[RGB]{60, 60, 60}{\textit{ByteOps.swe@gmail.com}} \\
\vspace{1cm}
\fontsize{16}{6}\selectfont Verbale Interno $\cdot$ Data: 24/11/2023 \\
\vspace{0.5cm}
\end{center}

\section*{Informazioni documento}
\def\arraystretch{1.2}
\begin{tabular}{>{\raggedleft\arraybackslash}p{0.2\textwidth}|>{\raggedright\arraybackslash}p{0.6\textwidth}c}
\hline
\addlinespace
\textbf{Luogo} & Discord \vspace{10pt} \\
\textbf{Orario} &  14:00 - 15:00  \vspace{10pt} \\
\textbf{Redattore} & F. Pozza \vspace{10pt} \\
\textbf{Verificatore} & E. Hysa \vspace{10pt} \\
\textbf{Amministratore} & L. Skenderi \vspace{10pt} \\
\textbf{Destinatari} & T. Vardanega \\ & R. Cardin \vspace{10pt} \\
\textbf{Partecipanti} & A. Barutta \\ & E. Hysa \\ & R. Smanio \\ & D. Diotto \\ & F. Pozza \\ & L. Skenderi \\ & N. Preto \vspace{10pt} \\
\end{tabular}
\pagebreak 

% ------------------------- Changelog ----------------------------

\section*{Registro delle modifiche}

\begin{tabular}{|C{2.5cm}|C{2.5cm}|C{2.5cm}|C{2.5cm}|C{2.5cm}|}
    \hline
    \textbf{Versione} & \textbf{Data} & \textbf{Autore} & \textbf{Verificatore} & \textbf{Dettaglio} \\
    \hline \hline
    0.0.1 & 24/11/2023 & F. Pozza & E. Hysa & Redazione documento \\
    \hline
\end{tabular}
\pagebreak

% ------------------------- Generazione automatica indice ----------------------
\setstretch{1.5}
\maketitle
\thispagestyle{fancy}
\tableofcontents
\setstretch{1.2}
\pagebreak

% ------------------------ INIZIO DOCUMENTO ----------------------
\flushleft

\section{Revisione del periodo precedente}
    Durante la fase iniziale dell'incontro, il tema predominante è stato il resoconto dell'incontro con il proponente tenutosi nella mattinata. Durante tale discussione, sono stati esaminati i feedback ricevuti e individuati gli aspetti passibili di miglioramento.
    
    A seguire si è parlato dei compiti assegnati alla riunione precedente, delle difficoltà riscontrate, che sono risultate essere minime, e di come le nuove procedure adottate per quanto riguarda la stesura dei file \LaTeX\ avessero migliorato in modo positivo i problemi di merge precedentemente riscontrati.

    Terminata la fase di retrospettiva, è stato redatto un ordine del giorno in merito ai temi da affrontare per le attività previste nel prossimo periodo.

\section{Ordine del giorno}
    \subsection{Discussione incontro con proponente}
        Nella mattinata è stato affrontato lo stato di avanzamento dei lavori durante un incontro esterno, nel quale il proponente ha manifestato soddisfazione per il lavoro svolto. Non sono stati sollevati dubbi o critiche, eccezion fatta per alcuni feedback in merito a possibili miglioramenti al codice, volti a renderlo più estensibile e strutturalmente robusto. Tali suggerimenti sono stati successivamente riportati nella sezione "Attività da svolgere" attraverso la creazione di specifiche issue.

    \subsection{Fissata data ricevimento}
        Dopo aver redatto la prima bozza del documento "Analisi dei requisiti" è stato richiesto un ricevimento al Professor R. Cardin al fine di chiarire alcuni dubbi emersi durante la stesura di alcuni casi d'uso. \\
        La comunicazione è stata gestita tramite corrispondenza via e-mail, ed ha portato a fissare un incontro in data 1 dicembre 2023 alle ore 14:30.

    \subsection{Sezioni Norme di Progetto}
        Si è discusso su quale dovesse essere il contenuto delle sezioni "Verifica" e "Fornitura" in "Norme di progetto", si è poi definita una traccia dei contenuti da inserire nelle sezioni e successivamente sono state assegnate le issue riguardanti le stesse.
    
    \subsection{Discussione contenuto slide}
        Considerando l'anticipo della redazione del verbale interno dal lunedì pomeriggio, giorno originariamente pianificato, al venerdì pomeriggio, è stato possibile esaminare e discutere il contenuto delle slide destinate al prossimo diario di bordo, e conseguentemente redigere una lista con i punti da trattare.
        
\section{Attività da svolgere}
    \begin{center}
        \begin{tabular}{|C{7cm}|C{1,5cm}|C{3cm}|}
            \hline
            \textbf{Titolo} & \textbf{\# Issue} & \textbf{Verificatore} \\
            \hline\hline
            Stesura sezione "Fornitura" in NdP & 33 & L. Skenderi \\
            \hline
            Connessione ClickHouse - Grafana & 34 & E. Hysa \\
            \hline
            Modifiche struttura "volumes" in docker-compose.yml & 36 & L. Skenderi \\
            \hline
            Rimozione invio socket in "volumes" in docker-compose.yml & 37 & L. Skenderi \\
            \hline
            Avvio client ClickHouse tramite nome container & 38 & E. Hysa \\
            \hline
            Redazione verbale interno 24-11-2023 & 39 & E. Hysa \\
            \hline
            Stesura sezione "Verifica" in NdP & 40 & L. Skenderi \\
            \hline
        \end{tabular}
    \end{center}

\end{document}