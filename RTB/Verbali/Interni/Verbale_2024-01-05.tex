\documentclass{article}
\usepackage[utf8]{inputenc}
\usepackage[absolute]{textpos}
\usepackage[default]{raleway}
\usepackage{titlesec, comment, tabularx, makecell, listings, array, setspace, geometry, graphicx, xcolor, xparse, fancyvrb, relsize, fancyhdr, booktabs, hyperref}
\usepackage{colortbl}
%\geometry{a4paper, left=2cm, right=2cm, top=2cm, bottom=2.5cm}
\renewcommand{\headrulewidth}{0pt}

% Definisci uno stile per i comandi git
\definecolor{light-gray}{gray}{0.92}

\lstdefinestyle{code}{
    frame=single,
    framesep=1mm,
    rulecolor=\color{light-gray},
    backgroundcolor=\color{light-gray},
    basicstyle=\ttfamily,
}

% ----------------------------- Definizione tabella ---------------------------

\newcolumntype{C}[1]{>{\centering\arraybackslash}m{#1}}

%\setcellgapes{2ex} % Imposta l'altezza dell'header (2ex)


% ------------------------------Metadati indice --------------------------------
\title{\textbf{\fontsize{28}{6}\selectfont Indice}}
\author{\fontsize{14}{6}\selectfont ByteOps}
\date{Gennaio 05, 2024}


% -----------------------------Creazione footer --------------------------------

\pagestyle{fancy}
\fancyhf{}
\renewcommand{\footrulewidth}{0.4pt}
\lfoot{
    \parbox[c]{2cm}{\includegraphics[width=2cm]{../../../Images/logo.png}}
    \textcolor[RGB]{120, 120, 120}{$\cdot$ Verbale Interno}
}
\rfoot{\thepage}

% --------------------------Modifica formato hyperlinks ------------------------

\hypersetup{
    colorlinks=true,
    linkcolor=black,
    filecolor=black,      
    pdftitle={Verbale Interno 05/01/2024},  %inserisci data verbale
    pdfpagemode=FullScreen,
}

% ------------------------------- Valore sotto-paragrafi indice --------------------------------------

\setcounter{secnumdepth}{4}
\setcounter{tocdepth}{4}

\titleformat{\section}
{\normalfont\huge\bfseries}{\thesection}{0.2cm}{}
\titlespacing*{\paragraph}{0pt}{0.5cm}{0.1cm}

\titleformat{\subsection}
{\normalfont\Large\bfseries}{\thesubsection}{0.2cm}{}
\titlespacing*{\paragraph}{0pt}{0.5cm}{0.1cm}

\titleformat{\subsubsection}
{\normalfont\large\bfseries}{\thesubsubsection}{0.2cm}{}
\titlespacing*{\paragraph}{0pt}{0.5cm}{0.1cm}

\titleformat{\paragraph}
{\normalfont\normalsize\bfseries}{\theparagraph}{0.2cm}{}
\titlespacing*{\paragraph}{0pt}{0.5cm}{0.1cm}

% ------------------------------- Front Page ---------------------------------------

\begin{document}

% --------------------------Aggiunta firma finale ------------------------
\begin{textblock*}{\textwidth}(0.85\textwidth, 1.16\textheight)
    Il responsabile: Nicola Preto
\end{textblock*}
% ------------------------------------------------------------------------

\pagestyle{fancy}
\begin{center}
\includegraphics[width = 0.7\textwidth]{../../../Images/logo.png} \\
\vspace{0.2cm}
\textcolor[RGB]{60, 60, 60}{\textit{ByteOps.swe@gmail.com}} \\
\vspace{1cm}
\fontsize{16}{6}\selectfont Verbale Interno $\cdot$ Data: 05/01/2024 \\
\vspace{0.5cm}
\end{center}

\section*{Informazioni documento}
\def\arraystretch{1.2}
\begin{tabular}{>{\raggedleft\arraybackslash}p{0.2\textwidth}|>{\raggedright\arraybackslash}p{0.6\textwidth}c}
    \hline
    \addlinespace
    \textbf{Luogo} & Discord \vspace{10pt} \\
    \textbf{Orario} & 14:30 - 16:30 \vspace{10pt} \\
    \textbf{Redattore} & A. Barutta \vspace{10pt} \\
    \textbf{Verificatore} & L. Skenderi \vspace{10pt} \\
    \textbf{Amministratore} & D. Diotto \vspace{10pt} \\
    \textbf{Destinatari} & T. Vardanega \\ & R. Cardin \vspace{10pt} \\
    \textbf{Partecipanti} & A. Barutta \\ & E. Hysa \\ & R. Smanio \\ & D. Diotto \\ & F. Pozza \\ & L. Skenderi \\ & N. Preto \vspace{10pt} \\
\end{tabular}
\pagebreak 

% ------------------------- Changelog ----------------------------

\section*{Registro delle modifiche}

\begin{tabular}{|C{2.5cm}|C{2.5cm}|C{2.5cm}|C{2.5cm}|C{2.5cm}|}
    \hline
    \textbf{Versione} & \textbf{Data} & \textbf{Autore} & \textbf{Verificatore} & \textbf{Dettaglio} \\
    \hline \hline
    0.0.1 & 05/01/2024 & A. Barutta & L. Skenderi & Redazione verbale \\
    \hline
\end{tabular}
\pagebreak

% ------------------------- Generazione automatica indice ----------------------
\setstretch{1.5}
\maketitle
\thispagestyle{fancy}
\tableofcontents
\setstretch{1.2}
\pagebreak

% ------------------------ INIZIO DOCUMENTO ----------------------
\flushleft

\section{Revisione del periodo precedente}
In ultima analisi, tutte le funzionalità presentate nel Proof of Concept (PoC) sono state sottoposte a una revisione di gruppo, segnando così la conclusione del suo sviluppo. In aggiunta, la pagina web "gitHub.io" è stata valutata positivamente per la sua efficacia nella visualizzazione e navigazione della repository di gruppo.

Tutte le attività programmate durante il quarto periodo sono state portate a termine senza riscontri o commenti utili per il miglioramento dei processi.
Tutte le attività necessarie per la richiesta della revisione RTB sono state ufficialmente portate a termine.
\section{Ordine del giorno}
\subsection{Definizione piano revisione RTB}
Il team ha condotto una valutazione per determinare il momento più opportuno per richiedere la revisione RTB, considerando il completamento di tutte le attività ad essa associate e gli imminenti impegni accademici previsti per il corrente mese.
Dopo una riflessione, si è deciso di richiedere un colloquio per la valutazione del Proof of Concept (PoC) e dell'analisi dei requisiti. Successivamente, verrà valutato il momento più adatto per la richiesta della seconda fase di revisione.
La richiesta formale per il colloquio sarà inviata terminato il SAL con la proponente fissato il 12 gennaio 2024.
\subsection{Pianificazione presentazione revisione Poc}
La creazione della presentazione del PoC è stata formalizzata attraverso l'identificazione di tre punti chiave e la definizione delle relative tempistiche:

\begin{enumerate}
    \item \textbf{Spiegazione dello scopo del capitolato} (Massimo 2 minuti): In questa fase, si dedica il tempo necessario per illustrare in maniera chiara e concisa l'obiettivo principale del capitolato.
    
    \item \textbf{Definizione delle tecnologie utilizzate in stretta relazione alle necessità funzionali del progetto} (Massimo 8 minuti): Questo segmento è stato dedicato a delineare le tecnologie adottate, evidenziandone la stretta correlazione con le esigenze funzionali e prestazionali del progetto.
    
    \item \textbf{Dimostrazione visuale del proof of concept} (Massimo 10 minuti): La parte centrale della presentazione è stata riservata alla dimostrazione pratica del proof of concept, offrendo un'illustrazione visiva delle capacità e dei risultati ottenuti dallo sviluppo.
\end{enumerate}

Le diapositive relative a ciascuna di queste sezioni sono state create e sono disponibili sulla repository Google Drive del gruppo.

\subsection{Pianificazione presentazione revisione prof. Vardanega}
Infine, si è proceduto con la formalizzazione della struttura della presentazione destinata all'incontro con il professor Vardanega. Ciascuna diapositiva sarà dedicata all'analisi di un documento specifico, fornendo una breve esposizione sulla sua struttura interna e sui relativi contenuti. In conclusione, ogni diapositiva presenterà considerazioni emerse dall'applicazione dei processi conformemente alle norme stabilite, nonché dagli sforzi compiuti durante la redazione di tali documenti.
Anche queste diapositive sono state create e sono disponibili sulla repository Google Drive del gruppo.


\section{Attività da svolgere}
    \begin{center}
        \begin{tabular}{|C{7cm}|C{1,5cm}|C{3cm}|}
            \hline
            \textbf{Titolo} & \textbf{\# Issue} & \textbf{Verificatore} \\
            \hline\hline
            Redazione verbale 05/01/2024 & 78 & L. Skenderi \\
            \hline
        \end{tabular}
    \end{center}

\end{document}
