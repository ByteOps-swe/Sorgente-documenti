\documentclass{article}
\usepackage[utf8]{inputenc}
\usepackage[absolute]{textpos}
\usepackage[default]{raleway}
\usepackage{titlesec, comment, tabularx, makecell, listings, array, setspace, geometry, graphicx, xcolor, xparse, fancyvrb, relsize, fancyhdr, booktabs, hyperref}
\usepackage{colortbl}
%\geometry{a4paper, left=2cm, right=2cm, top=2cm, bottom=2.5cm}
\renewcommand{\headrulewidth}{0pt}

% Definisci uno stile per i comandi git
\definecolor{light-gray}{gray}{0.92}

\lstdefinestyle{code}{
    frame=single,
    framesep=1mm,
    rulecolor=\color{light-gray},
    backgroundcolor=\color{light-gray},
    basicstyle=\ttfamily,
} 

% ----------------------------- Definizione tabella ---------------------------

\newcolumntype{C}[1]{>{\centering\arraybackslash}m{#1}}

%\setcellgapes{2ex} % Imposta l'altezza dell'header (2ex)


% ------------------------------Metadati indice --------------------------------
\title{\textbf{\fontsize{28}{6}\selectfont Indice}}
\author{\fontsize{14}{6}\selectfont ByteOps}
\date{Dicembre 8, 2023}

% -----------------------------Creazione footer --------------------------------

\pagestyle{fancy}
\fancyhf{}
\renewcommand{\footrulewidth}{0.4pt}
\lfoot{
    \parbox[c]{2cm}{\includegraphics[width=2cm]{../../../Images/logo.png}}
    \textcolor[RGB]{120, 120, 120}{$\cdot$ Verbale Interno}
}
\rfoot{\thepage}

% --------------------------Modifica formato hyperlinks ------------------------

\hypersetup{
    colorlinks=true,
    linkcolor=black,
    filecolor=black,      
    pdftitle={Verbale Interno 04/12/2023},  %inserisci data verbale
    pdfpagemode=FullScreen,
}

% ------------------------------- Valore sotto-paragrafi indice --------------------------------------

\setcounter{secnumdepth}{4}
\setcounter{tocdepth}{4}

\titleformat{\section}
{\normalfont\huge\bfseries}{\thesection}{0.2cm}{}
\titlespacing*{\paragraph}{0pt}{0.5cm}{0.1cm}

\titleformat{\subsection}
{\normalfont\Large\bfseries}{\thesubsection}{0.2cm}{}
\titlespacing*{\paragraph}{0pt}{0.5cm}{0.1cm}

\titleformat{\subsubsection}
{\normalfont\large\bfseries}{\thesubsubsection}{0.2cm}{}
\titlespacing*{\paragraph}{0pt}{0.5cm}{0.1cm}

\titleformat{\paragraph}
{\normalfont\normalsize\bfseries}{\theparagraph}{0.2cm}{}
\titlespacing*{\paragraph}{0pt}{0.5cm}{0.1cm}

% ------------------------------- Front Page ---------------------------------------

\begin{document}

% --------------------------Aggiunta firma finale ------------------------
\begin{textblock*}{\textwidth}(0.85\textwidth, 1.16\textheight)
    Il responsabile: Riccardo Smanio
\end{textblock*}
% ------------------------------------------------------------------------

\pagestyle{fancy}
\begin{center}
\includegraphics[width = 0.7\textwidth]{../../../Images/logo.png} \\
\vspace{0.2cm}
\textcolor[RGB]{60, 60, 60}{\textit{ByteOps.swe@gmail.com}} \\
\vspace{1cm}
\fontsize{16}{6}\selectfont Verbale Interno $\cdot$ Data: 08/12/2023 \\
\vspace{0.5cm}
\end{center}

\section*{Informazioni documento}
\def\arraystretch{1.2}
\begin{tabular}{>{\raggedleft\arraybackslash}p{0.2\textwidth}|>{\raggedright\arraybackslash}p{0.6\textwidth}c}
\hline
\addlinespace
\textbf{Luogo} & Discord \vspace{10pt} \\
\textbf{Orario} & 10:30 - 11:30 \vspace{10pt} \\
\textbf{Redattore} & A. Barutta \vspace{10pt} \\
\textbf{Verificatore} & N. Preto \vspace{10pt} \\
\textbf{Amministratore} & D. Diotto \vspace{10pt} \\
\textbf{Destinatari} & T. Vardanega \\ & R. Cardin \vspace{10pt} \\
\textbf{Partecipanti} & A. Barutta \\ & E. Hysa \\ & R. Smanio \\ & D. Diotto \\ & F. Pozza \\ & L. Skenderi \\ & N. Preto \vspace{10pt} \\
\end{tabular}
\pagebreak 

% ------------------------- Changelog ----------------------------

\section*{Registro delle modifiche}

\begin{tabular}{|C{2.5cm}|C{2.5cm}|C{2.5cm}|C{2.5cm}|C{2.5cm}|}
    \hline
    \textbf{Versione} & \textbf{Data} & \textbf{Autore} & \textbf{Verificatore} & \textbf{Dettaglio} \\
    \hline \hline
    0.0.1 & 09/12/2023 & Andrea Barutta & N. Preto & Redazione completa \\
    \hline
\end{tabular}
\pagebreak

% ------------------------- Generazione automatica indice ----------------------
\setstretch{1.5}
\maketitle
\thispagestyle{fancy}
\tableofcontents
\setstretch{1.2}
\pagebreak

% ------------------------ INIZIO DOCUMENTO ----------------------
\flushleft

\section{Revisione del periodo precedente}
Durante la fase iniziale della riunione, abbiamo approfondito lo stato attuale delle attività in corso, esaminando attentamente la presenza di eventuali problematiche non ancora riscontrate. È importante notare che, in questa analisi, non sono emerse proposte di modifiche ai processi né sono sorte discussioni rilevanti.
\section{Ordine del giorno}
    \subsection{Cruscotto di valutazione della qualità - Piano di qualifica}
    Dopo aver inizialmente redatto le metriche di qualità di processo e prodotto, è essenziale condurre una revisione approfondita e sviluppare un cruscotto di valutazione per tracciare il valore di queste metriche durante i diversi periodi del progetto. Questo cruscotto consentirà di visualizzare l'andamento e facilitare valutazioni migliorative. La decisione presa è stata di creare un documento condiviso su Google Fogli, dove verranno registrati i valori delle metriche per ogni periodo e saranno generati grafici illustrativi degli andamenti.
    \subsection{Revisione completa - Norme di progetto}
    Dopo aver completato la stesura del documento "Norme di progetto", si è reputato indispensabile condurre una revisione completa al fine di verificare la coerenza e la consistenza delle diverse sezioni redatte, garantendo così che il file sia pronto per la revisione RTB.
    \subsection{Ottimizzazione POC}
    Dopo aver ricevuto un feedback positivo dal SAL del 06/12/2023, si è concordato con il proponente di considerare complete le funzionalità richieste per il POC. Di conseguenza, il focus dello sviluppo sarà ora sulla risoluzione di piccoli bug e sulla creazione di una versione stabile e affidabile da presentare durante la revisione RTB.
    \subsection{Ridefinizione Casi d'Uso - Analisi dei requisiti}
    La ridefinizione dei casi d'uso è stata completata dopo l'incontro con il Prof. Cardin. L'obiettivo entro il prossimo sprint è la creazione di una nuova versione del documento da sottoporre nuovamente al professore.
    \subsection{Rotazione ruoli}
    \begin{itemize}
        \item \textbf{Responsabile (Re):}
              \begin{itemize}
                  \item R. Smanio.
              \end{itemize}
        \item \textbf{Amministratore (Am):}
              \begin{itemize}
                  \item D. Diotto
              \end{itemize}
        \item \textbf{Analisti (An):}
              \begin{itemize}
                  \item L. Skenderi.
              \end{itemize}
        \item \textbf{Verificatore (Ve):}
              \begin{itemize}
                  \item N. Preto;
                  \item A. Barutta.
              \end{itemize}
        \item \textbf{Programmatori (Pr):}
              \begin{itemize}
                  \item F. Pozza;
                  \item E. Hysa.
              \end{itemize}
    \end{itemize}
\section{Attività da svolgere}
    \begin{center}
        \begin{tabular}{|C{7cm}|C{1,5cm}|C{3cm}|}
            \hline
            \textbf{Titolo} & \textbf{\# Issue} & \textbf{Verificatore} \\
            \hline\hline           
            Piano di qualifica - cruscotto valutazione e grafici per il miglioramento  & 55 & A. Barutta \\
            
            Revisione finale Norme di progetto & 56 & A. Barutta \\
            
            Ottimizzazione POC: Risoluzione Bug e Revisione RTB & 57 & N. Preto \\

            Ridefinizione Casi d'Uso e Creazione Nuova Versione Documento & 59 & A. Barutta \\

            \hline
        \end{tabular}
    \end{center}

\end{document}