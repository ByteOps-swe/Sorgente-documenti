\documentclass{article}
\usepackage[utf8]{inputenc}
\usepackage[absolute]{textpos}
\usepackage[default]{raleway}
\usepackage{titlesec, comment, tabularx, makecell, listings, array, setspace, geometry, graphicx, xcolor, xparse, fancyvrb, relsize, fancyhdr, booktabs, hyperref}
\usepackage{colortbl}
%\geometry{a4paper, left=2cm, right=2cm, top=2cm, bottom=2.5cm}
\renewcommand{\headrulewidth}{0pt}

% Definisci uno stile per i comandi git
\definecolor{light-gray}{gray}{0.92}

\lstdefinestyle{code}{
    frame=single, 
    framesep=1mm,
    rulecolor=\color{light-gray},    
    backgroundcolor=\color{light-gray},
    basicstyle=\ttfamily,
}

% ----------------------------- Definizione tabella ---------------------------

\newcolumntype{C}[1]{>{\centering\arraybackslash}m{#1}}

%\setcellgapes{2ex} % Imposta l'altezza dell'header (2ex)


% ------------------------------Metadati indice --------------------------------
\title{\textbf{\fontsize{28}{6}\selectfont Indice}}
\author{\fontsize{14}{6}\selectfont ByteOps}
\date{Novembre 6, 2023}


% -----------------------------Creazione footer --------------------------------

\pagestyle{fancy}
\fancyhf{}
\renewcommand{\footrulewidth}{0.4pt}
\lfoot{
    \parbox[c]{2cm}{\includegraphics[width=2cm]{../../../Images/logo.png}}
    \textcolor[RGB]{120, 120, 120}{$\cdot$ Verbale Interno}
}
\rfoot{\thepage}

% --------------------------Modifica formato hyperlinks ------------------------

\hypersetup{
    colorlinks=true,
    linkcolor=black,
    filecolor=black,      
    pdftitle={Verbale Interno 06/11/2023},  %inserisci data verbale
    pdfpagemode=FullScreen,
}

% ------------------------------- Valore sotto-paragrafi indice --------------------------------------

\setcounter{secnumdepth}{4}
\setcounter{tocdepth}{4}

\titleformat{\section}
{\normalfont\huge\bfseries}{\thesection}{0.2cm}{}
\titlespacing*{\paragraph}{0pt}{0.5cm}{0.1cm}

\titleformat{\subsection}
{\normalfont\Large\bfseries}{\thesubsection}{0.2cm}{}
\titlespacing*{\paragraph}{0pt}{0.5cm}{0.1cm}

\titleformat{\subsubsection}
{\normalfont\large\bfseries}{\thesubsubsection}{0.2cm}{}
\titlespacing*{\paragraph}{0pt}{0.5cm}{0.1cm}

\titleformat{\paragraph}
{\normalfont\normalsize\bfseries}{\theparagraph}{0.2cm}{}
\titlespacing*{\paragraph}{0pt}{0.5cm}{0.1cm}

% ------------------------------- Front Page ---------------------------------------

\begin{document}

% --------------------------Aggiunta firma finale ------------------------
\begin{textblock*}{\textwidth}(0.85\textwidth, 1.16\textheight)
    Il responsabile: Davide Diotto
\end{textblock*}
% ------------------------------------------------------------------------

\pagestyle{fancy}
\begin{center}
\includegraphics[width = 0.7\textwidth]{../../../Images/logo.png} \\
\vspace{0.2cm}
\textcolor[RGB]{60, 60, 60}{\textit{ByteOps.swe@gmail.com}} \\
\vspace{1cm}
\fontsize{16}{6}\selectfont Verbale Interno $\cdot$ Data: 06/11/2023 \\
\vspace{0.5cm}
\end{center}

\section*{Informazioni documento}
\def\arraystretch{1.2}
\begin{tabular}{>{\raggedleft\arraybackslash}p{0.2\textwidth}|>{\raggedright\arraybackslash}p{0.6\textwidth}c}
\hline
\addlinespace
\textbf{Luogo} & Discord \vspace{10pt} \\
\textbf{Orario} & 15:00 - 17:00 \vspace{10pt} \\
\textbf{Redattore} & R. Smanio \vspace{10pt} \\
\textbf{Verificatore} & A. Barutta \vspace{10pt} \\
\textbf{Amministratore} & F. Pozza \vspace{10pt} \\
\textbf{Destinatari} & T. Vardanega \\ & R. Cardin \vspace{10pt} \\
\textbf{Partecipanti} & A. Barutta \\ & E. Hysa \\ & R. Smanio \\ & D. Diotto \\ & F. Pozza \\ & L. Skenderi \\ & N. Preto \vspace{10pt} \\
\end{tabular}
\pagebreak 

% ------------------------- Changelog ----------------------------

\section*{Registro delle modifiche}

\begin{tabular}{|C{2.5cm}|C{2.5cm}|C{2.5cm}|C{2.5cm}|C{2.5cm}|}
    \hline
    \textbf{Versione} & \textbf{Data} & \textbf{Autore} & \textbf{Verificatore} & \textbf{Dettaglio} \\
    \hline \hline
    0.0.1 & 08/11/2023 & R.Smanio & A. Barutta & Redazione documento \\
    \hline
\end{tabular}
\pagebreak

% ------------------------- Generazione automatica indice ----------------------
\setstretch{1.5}
\maketitle
\thispagestyle{fancy}
\tableofcontents
\setstretch{1.2}
\pagebreak

% ------------------------ INIZIO DOCUMENTO ----------------------
\flushleft

\section{Revisione del periodo precedente}
    In seguito ai feedback ricevuti riguardo alla documentazione presentata per la candidatura al capitolato scelto, è emersa la necessità di rivedere la struttura del verbale, di introdurre un registro delle modifiche per tutti i documenti e di riorganizzare la struttura della repository.
    Osservando la struttura attuale del verbale, abbiamo constatato che essa risulta essere eccessivamente essenziale, mancante di dettagli necessari per una comprensione approfondita di quanto discusso durante il meeting.

\section{Ordine del giorno}
    
    \subsection{Definizione nuova struttura per i verbali}
        È stato deciso di implementare il registro delle modifiche e di introdurre una sezione di revisione del periodo precedente. Quest'ultima avrà il compito di sintetizzare le discussioni inerenti all'analisi delle attività svolte a partire dall'ultimo meeting. In aggiunta, è stata inserita una tabella per identificare le nuove attività definite e pianificate durante la riunione.\\
        La sezione "Ordine del giorno" è stata strutturata per elencare in modo puntato gli argomenti chiave trattati durante il meeting, fornendo una breve descrizione per ciascuno di essi.

    \subsection{Registro delle modifiche: definizione formato versione documento}
    La definizione di una convenzione per identificare la versione di un documento è stata formalizzata e può essere accuratamente consultata nella sezione 2.3.2 del documento intitolato "Norme di Progetto".

    \subsection{Programmazione incontro con azienda proponente}
        È stato programmato un meeting con l'azienda proponente per il 10/11/2023, dalle 11:30 alle 12:30.\\
        In occasione di questo incontro, verranno formulate delle domande riguardanti l'avvio delle prime fasi del progetto.\\
        Le discussioni e le decisioni emerse durante questa sessione verranno successivamente documentate nel relativo verbale.    
    
    \subsection{Definizione nuova struttura per le repository relative alla documentazione}
        Si è scelto di strutturare le repository relative alla documentazione sulla base delle revisioni di avanzamento del progetto:
        \begin{enumerate}
            \item Candidatura
            \item RTB: Requirements and Technology Baseline
            \item PB: Product Baseline
            \item CA: Customer Acceptante (opzionale)
        \end{enumerate}
        Ogni punto sopra citato identifica una specifica cartella, all'interno della quale sono collocati i documenti richiesti per quella specifica revisione di avanzamento.

    \subsection{Repository "Sorgente-documenti" da privata a pubblica}
        In seguito alla scelta di automatizzare il processo di compilazione dei file \LaTeX e la generazione dei corrispondenti PDF, è diventato necessario modificare la visibilità della repository "Sorgente-documenti" da privata a pubblica.

    \subsection{Linee guida per la stesura del documento "Norme di Progetto"}
        Abbiamo adottato lo standard ISO/IEC 12207-1995 come riferimento principale per la redazione del documento "Norme di Progetto". Questa scelta riflette l'impegno per l'aderenza a pratiche consolidate e riconosciute nell'ambito delle norme di sviluppo del software.
        Partendo da questa decisione, abbiamo definito una prima bozza per la sezione 2.3, che riguarda la Gestione della Configurazione.

\section{Attività da svolgere}
    \begin{center}
        \begin{tabular}{|C{7cm}|C{1,5cm}|C{3cm}|}
            \hline
            \textbf{Titolo} & \textbf{\# Issue} & \textbf{Verificatore} \\
            \hline\hline
            Creazione template verbali & 15 & L. Skenderi \\
            Redazione verbale odierno & 16 & A. Barutta \\
            Aggiornamento verbali candidatura & 17 & F. Pozza \\
            Norme di progetto: sez. Gestione della configurazione & 18 & E. Hysa \\
            Riorganizzazione repository & 19 & N. Preto \\
            \hline
        \end{tabular}
    \end{center}

\end{document}