\subsection{Gestione della configurazione}
\subsubsection{Introduzione}
La gestione della configurazione del progetto è un processo che norma il tracciamento e il controllo delle modifiche a documenti e prodotti del software detti Configuration Item (CI).

La gestione della configurazione viene applicata a qualunque categoria di “artefatti” coinvolti nel ciclo di vita del software. \\
Secondo lo standard \href{https://www.math.unipd.it/~tullio/IS-1/2009/Approfondimenti/ISO_12207-1995.pdf}{ISO/IEC 12207:1995}, la gestione della configurazione del software è un processo di identificazione, organizzazione e controllo delle modifiche apportate ai prodotti software durante il loro ciclo di vita. \\
Lo standard \href{https://standards.ieee.org/ieee/828/5367/}{IEEE 828-2012} definisce la gestione della configurazione del software come “un processo disciplinato per gestire l’evoluzione del software”.

\subsubsection{Numeri di versionamento}\label{subsubsec:versionamento}
La convenzione per identificare la versione di un documento è nel formato X.Y.Z con:
\begin{itemize}
    \item \textbf{X:} viene incrementato al raggiungimento di RTB, PB ed eventualmente CA;
    \item \textbf{Y:} viene incrementato quando vengono apportate modifiche significative al documento, come cambiamenti strutturali, nuove sezioni importanti o modifiche sostanziali nel contenuto;
    \item \textbf{Z:} viene incrementato per modifiche minori o aggiornamenti al documento. Questi potrebbero includere correzioni di errori, miglioramenti marginali o l'aggiunta di nuovi contenuti meno rilevanti.
\end{itemize}

L'incremento dei valori più significativi porta i meno significativi a zero. Ad esempio, se in un documento nella versione 0.6.4 vengono effettuate modifiche significative e viene conseguentemente incrementato il valore Y, si passa alla versione 0.7.0 e non alla versione 0.7.4. \\
Ogni variazione di versione deve essere presente nel registro delle modifiche.

\subsubsection{Repository}
Di seguito sono elencate le repository del team ByteOps con i relativi riferimenti:
\begin{itemize}
    \item \href{https://github.com/ByteOps-swe/Sorgente-documenti}{Sorgente-documenti:} repository per il versionamento, la gestione e lo sviluppo dei file sorgente relativi alla documentazione;
    \item \href{https://github.com/ByteOps-swe/Documents}{Documents:} repository destinata ai committenti/proponenti, dove vengono esclusivamente condivisi i file in formato PDF relativi alla documentazione, ottenuti dalla compilazione dei file sorgente presenti nella cartella "Sorgente-documenti".
    \item \href{https://github.com/ByteOps-swe/proof-of-concept}{proof-of-concept:} repository destinata al POC.
\end{itemize}
\paragraph{Struttura repository}
I repository destinati alla documentazione sono organizzati come segue:
\begin{itemize}
    \item \textbf{Candidatura:} contenente i documenti richiesti per la candidatura;
          \begin{itemize}
              \item \textbf{Verbali:} contiene tutti i verbali redatti nel corso del periodo di candidatura, distinti tra verbali esterni e interni;
              \item \textbf{Lettera di presentazione};
              \item \textbf{Valutazione dei costi e assunzione impegni};
              \item \textbf{Valutazione dei capitolati}.
          \end{itemize}
    \item \textbf{RTB:} contenente i documenti richiesti per la revisione omonima;
          \begin{itemize}
              \item \textbf{Verbali:} contiene tutti i verbali redatti nel corso del periodo della RTB, distinti tra verbali esterni e interni;
              \item \textbf{Piano di Qualifica};
              \item \textbf{Piano di Progetto};
              \item \textbf{Analisi dei Requisiti};
              \item \textbf{Glossario};
              \item \textbf{Norme di Progetto}.
          \end{itemize}
    \item \textbf{PB}.
\end{itemize}


\hypertarget{par:sincronizzazione&branching}{\paragraph{Sincronizzazione e Branching}}
\textbf{Documentazione} \\
Il processo operativo utilizzato per la redazione della documentazione, noto come "\href{https://www.atlassian.com/git/tutorials/comparing-workflows/feature-branch-workflow}{Feature Branch workflow}", implica la creazione di un ramo dedicato per ciascun documento o sezione da elaborare.
Tale metodologia permette una parallelizzazione agile dei lavori evitando sovrascritture indesiderate di altri lavori e permette l'adozione dei principi "Documentation as code" definiti nella sezione~\ref{sec:DocumentationAscode}. \\ 

\paragraph*{\hypertarget{par:convezioninomenclaturabranchdocumenti}{\textbf{Convenzioni per la nomenclatura dei branch relativi alle attività di redazione o modifica di documenti}}}

\begin{itemize}
    \item Il nome del branch deve presentare l'identificativo del documento che si vuole redarre o modificare. \\
    Ad ogni documento è associato un identificativo, come descritto nella seguente tabella:
\end{itemize}

\begin{table}[H]
    \centering
    \begin{tabular}{|c|c|}
        \hline
        Documento & ID \\
        \hline
        Verbale interno & VI \\
        Verbale esterno & VE \\
        Norme di Progetto & NdP \\
        Piano di Qualifica & PdQ \\
        Piano di Progetto & PdP \\ 
        Analisi dei Requisiti & AdR \\
        \hline
    \end{tabular}
    \caption{ID per la nomenclatura dei branch relativi alla documentazione}
\end{table}

\begin{itemize}
    \item Nel caso dei verbali, dopo l'identificativo del documento si aggiunge un underscore seguito dalla data: \\
    \texttt{IdDocumento\_dd-mm-yyyy} (ex. VI\_27-12-2023);
    \item Nel caso della redazione di una specifica sezione di un documento, il nome del branch deve avere il formato: \\
    \texttt{IdDocumento\_NomeSezione} (ex. NdP\_Documentazione);
    \item Nel caso di modifica di una specifica sezione di un documento, il nome del branch deve avere il formato: \\
    \texttt{IdDocumento\_ModNomeSezione} (ex. NdP\_ModDocumentazione).
\end{itemize}

\textbf{Sviluppo} \\
\href{https://www.atlassian.com/it/git/tutorials/comparing-workflows/gitflow-workflow}{Gitflow} è lo stile di flusso di lavoro Git che utilizza il team ByteOps per lo sviluppo.

\paragraph*{Flusso generale di Gitflow}

\begin{enumerate}
    \item \textbf{Branch \texttt{develop}:} viene creato a partire dal branch principale (\texttt{main}). È il punto di partenza per lo sviluppo di nuove funzionalità;

    \item \textbf{Branch \texttt{release}:} creato da \texttt{develop}, questo branch gestisce la preparazione del software per un rilascio. Durante questa fase, sono consentite solo correzioni di bug e miglioramenti minori;

    \item \textbf{Branch \texttt{feature}:} creati da \texttt{develop}, sono utilizzati per lo sviluppo di nuove funzionalità o miglioramenti; 

    \item \textbf{Merge di \texttt{feature} in \texttt{develop}:} quando una funzionalità è completa, il branch \texttt{feature} viene "fuso" nel branch \texttt{develop};

    \item \textbf{Merge di \texttt{release} in \texttt{develop} e \texttt{main}:} dopo il completamento del branch \texttt{release}, quest'ultimo viene unito sia in \texttt{develop} che in \texttt{main}, segnalando un nuovo rilascio stabile;

    \item \textbf{Branch \texttt{hotfix}:} creato da \texttt{main} in caso di problemi critici rilevati nell'ambiente di produzione;

    \item \textbf{Merge di \texttt{hotfix} in \texttt{develop} e \texttt{main}:} una volta risolto il problema, il branch \texttt{hotfix} viene unito sia in \texttt{develop} che in \texttt{main} per garantire coerenza tra le versioni.
\end{enumerate}

\subsubsection*{Comandi di comodo}
\paragraph*{Inizializzare Gitflow}

% Inizializzare Gitflow
\begin{lstlisting}[style=code]
    git flow init
    \end{lstlisting}
    
    % Sviluppo di una Nuova Funzionalità
    \paragraph*{Sviluppo di una Nuova Funzionalità}
    
    % Inizia una Feature
    \begin{lstlisting}[style=code]
    git flow feature start nome_feature
    \end{lstlisting}
    
    % Lavora sul Codice
    % Effettua le modifiche e i commit sulla branch della feature.
    
    % Completa la Feature
    \begin{lstlisting}[style=code]
    git flow feature finish nome_feature
    \end{lstlisting}
    
    % Risoluzione di Bug
    \paragraph*{Risoluzione di Bug}
    
    % Inizia una Hotfix
    \begin{lstlisting}[style=code]
    git flow hotfix start nome_hotfix
    \end{lstlisting}
    
    % Lavora sul Codice
    % Effettua le modifiche e i commit sulla branch del hotfix.
    
    % Completa il Hotfix
    \begin{lstlisting}[style=code]
    git flow hotfix finish nome_hotfix
    \end{lstlisting}
    
    % Rilascio di una Nuova Versione
    \paragraph*{Rilascio di una Nuova Versione}
    
    % Inizia una Release
    \begin{lstlisting}[style=code]
    git flow release start X.X.X
    \end{lstlisting}
    
    % Lavora sul Codice
    % Effettua eventuali aggiustamenti per la release e imposta il numero di versione nei file appropriati.
    
    % Completa la Release
    \begin{lstlisting}[style=code]
    git flow release finish X.X.X
    \end{lstlisting}
    
    % Pubblicazione delle Modifiche
    \paragraph*{Pubblicazione delle Modifiche}
    
    % Push delle Branch
    \begin{lstlisting}[style=code]
    git push origin develop
    git push origin master
    git push origin --tags
    \end{lstlisting}
    
    % Push delle Feature/Hotfix/Release
    \begin{lstlisting}[style=code]
    git push origin feature/nome_feature
    git push origin hotfix/nome_hotfix
    git push origin release/X.X.X
    \end{lstlisting}
\subsubsection{Controllo di configurazione}
\paragraph{Change request (Richiesta di modifica)}
Seguendo lo standard \href{https://www.math.unipd.it/~tullio/IS-1/2009/Approfondimenti/ISO_12207-1995.pdf}{ISO/IEC 12207:1995} per affrontare questo processo in modo strutturato le attività sono:
\begin{enumerate}
    \item \textbf{Identificazione e registrazione} \\
    Le change request vengono identificate, registrate e documentate accuratamente. Questo include informazioni come la natura della modifica richiesta, l'urgenza e l'impatto sul sistema esistente. \\
    L'identificazione avviene tramite la creazione di una issue nell'ITS con label: "Change request";
    \item \textbf{Valutazione e analisi} \\
    Le change request vengono valutate dal team per determinare la loro fattibilità, importanza e impatto sul progetto. Si analizzano i costi e i benefici associati all'implementazione della modifica;
    \item \textbf{Approvazione o rifiuto} \\
    Il responsabile valuta le informazioni raccolte e decide se approvare o respingere la change request. Questa decisione può essere basata su criteri come il budget, il tempo, le priorità degli stakeholder;
    \item \textbf{Pianificazione delle modifiche} \\
    Se una change request viene approvata, viene pianificata e integrata nel ciclo di sviluppo del software. Questo può richiedere, ad esempio, la rinegoziazione dei tempi di consegna e la revisione del piano di progetto.
    \item \textbf{Implementazione delle modifiche} \\
    Le modifiche vengono effettivamente implementate. Durante questo processo, è fondamentale mantenere una traccia accurata di ciò che viene fatto per consentire una corretta documentazione e, se necessario, la possibilità di un rollback;
    \item \textbf{Verifica e validazione} \\
    Le modifiche apportate vengono verificate per assicurarsi che abbiano raggiunto gli obiettivi previsti e non abbiano introdotto nuovi problemi o errori;
    \item \textbf{Documentazione} \\
    Tutti i passaggi del processo di gestione delle change request vengono documentati accuratamente per garantire la trasparenza e la tracciabilità. Questa documentazione è utile per futuri riferimenti e per l'apprendimento delle modifiche apportate;
    \item \textbf{Comunicazione agli stakeholder} \\
    Durante tutto il processo è importante comunicare in modo chiaro e tempestivo con gli stakeholder, per mantenere trasparenza e fiducia.
\end{enumerate}

\subsubsection{Configuration Status Accounting (Contabilità dello Stato di Configurazione)}
La Contabilità dello Stato di Configurazione è un'attività dedicata a registrare e monitorare lo stato di tutte le configurazioni di un sistema software durante il suo ciclo di vita. Questo processo è cruciale per mantenere la trasparenza e la tracciabilità nel ciclo di vita del software, aiutando a gestire le configurazioni in modo uniforme e a mantenere un registro accurato di tutte le attività e le modifiche relative agli elementi di configurazione.\\

\begin{itemize}
    \item \hypertarget{item:registrazioneconfigurazioni}{\textbf{Registrazione delle configurazioni}}: registrazione delle informazioni dettagliate su ogni Configuration Item;
          \begin{itemize}
              \item  \textbf{Documentazione:} le informazioni relative alla configurazione sono presenti nella prima pagina di ciascuno;
              \item  \textbf{Sviluppo:} le informazioni relative alla configurazione sono inserite come prime righe di ciascun file sotto forma di commento.
          \end{itemize}
    \item \textbf{Stato e cambiamenti}: tenere traccia dello stato attuale di ciascun elemento di configurazione e di tutti i cambiamenti che avvengono nel corso del tempo. \\
    Ciò include le versioni attuali, le revisioni, le modifiche e le baselines;
        \begin{itemize}
            \item \textbf{Registro delle modifiche:} per monitorare lo stato di ciascun Configuration Item si utilizza il registro delle modifiche incorporato in ognuno di essi, come specificato in \hyperlink{item:registrazioneconfigurazioni}{"registrazione delle configurazioni"};
            \item \textbf{Branching \& DashBoard:} per verificare se vi sono attività in corso su un Configuration Item, è sufficiente controllare se ci sono branch attivi correlati e, dato che ogni issue è associata a un CI tramite label, verificare le eventuali issue contrassegnate come "In Progress" nella colonna corrispondente della Dashboard del progetto.
        \end{itemize}
    \item \textbf{Supporto per la gestione delle change request}: registra e documenta le modifiche apportate agli elementi di configurazione in risposta alle richieste di modifica. \\
    Per gestire le richieste di modifica, si utilizza l'Issue Tracking System (ITS) di GitHub, creando una issue contrassegnata con l'etichetta "Change request".
\end{itemize}

Per approfondimenti su come creare issues e associare labels a una issue, consultare la sezione relativa al \hyperlink{par:ticketing}{\textit{Ticketing}}.

\subsubsection{Release management and delivery} \todo{non mi piace sta cosa, non so cosa inserirci}
Nello standard \href{https://www.math.unipd.it/~tullio/IS-1/2009/Approfondimenti/ISO_12207-1995.pdf}{ISO/IEC 12207:1995}, il processo "release management and delivery" è definito come un processo che si occupa della pianificazione, del coordinamento e del controllo delle attività necessarie per preparare e distribuire una versione di un prodotto software per l'uso operativo. Questo processo comprende la gestione delle release, la distribuzione del software, la preparazione della documentazione correlata e altre attività correlate al rilascio e alla consegna del prodotto software. \\
Le attività e le procedure legate al processo di rilascio e consegna sono strettamente correlate al workflow adottato dal team, il quale è descritto nel dettaglio alla sezione \hyperlink{par:sincronizzazione&branching}{"Sincronizzazione e Branching"}.
Dopo aver portato a termine le attività nel proprio branch, il responsabile del suo sviluppo è tenuto ad avviare una Pull Request per incorporare le modifiche effettuate nel ramo principale (main). La Pull Request viene approvata solo dopo che le modifiche sono state verificate sulla base di quanto descritto nella sezione~\ref{subsec:verifica}.

\hypertarget{par:creazionePR}{\paragraph*{Procedura per la creazione di Pull Request}} \todo{non sono sicuro vada all'interno del processo "Release management and delivery"}
Per creare una Pull Request eseguire i seguenti passaggi:
\begin{enumerate}
    \item Accedere al repository GitHub e cliccare sulla scheda "Pull requests";
    \item Cliccare il pulsante "New Pull Request";
    \item Selezionare il branch di partenza ed il branch target;
    \item Cliccare il pulsante "Create Pull Request";
    \item Aggiungere un titolo alla Pull Request;
    \item Aggiungere una descrizione;
    \item Selezionare i verificatori;
    \item Come convenzione, l'assegnatario è colui che richiede la Pull Request;
    \item Selezionare le labels;
    \item Selezionare il progetto;
    \item Selezionare le milestones;
    \item Clicca il pulsante "create Pull Request";
    \item Nel caso siano presenti conflitti seguire le istruzioni in Github per rimuovere tali conflitti;
\end{enumerate}

\subsubsection{Strumenti}
Le tecnologie adottate per la gestione dei Configuration Item sono:
\begin{itemize}
    \item \textbf{Git}: Version Control System distribuito utilizzato per il versionamento dei Configuration Item;
    \item \textbf{GitHub}: piattaforma web per il controllo di versione (tramite Git) dei Configuration Item e per il \hyperlink{par:ticketing}{Ticketing}. È impiegata per gestire le richieste di modifica tramite issue e label, oltre che per la contabilità dello stato di configurazione.
\end{itemize}
