\subsection{Gestione della configurazione}
\subsubsection{Introduzione}
La gestione della configurazione del progetto è una \textit{attività}\textsubscript{\textit{G}} che norma il tracciamento e il controllo delle modifiche a documenti e prodotti del \textit{software}\textsubscript{\textit{G}} detti Configuration Item (CI).

La gestione della configurazione può essere applicata a qualunque categoria di documenti o di “artefatti” che svolga un ruolo nello sviluppo \textit{software}\textsubscript{\textit{G}}.
Secondo lo \textit{standard}\textsubscript{\textit{G}} \textit{IEEE}\textsubscript{\textit{G}}, la gestione della configurazione del \textit{software}\textsubscript{\textit{G}} è un processo di identificazione, organizzazione e controllo delle modifiche apportate ai prodotti \textit{software}\textsubscript{\textit{G}} durante il loro ciclo di vita.
Lo \textit{standard}\textsubscript{\textit{G}} \textit{IEEE}\textsubscript{\textit{G}} 828-2012 definisce la gestione della configurazione del \textit{software}\textsubscript{\textit{G}} come “un processo disciplinato per gestire l’evoluzione del \textit{software}\textsubscript{\textit{G}}”.
\subsubsection{Numeri di versionamento}\label{sec:versionamento}
La convenzione per identificare la versione di un documento è: X.Y.Z
con :
\begin{itemize}
    \item \textbf{X: }Viene incrementato al raggiungimento di \textit{RTB}\textsubscript{\textit{G}}, \textit{PB}\textsubscript{\textit{G}} ed eventualmente CA;
    \item \textbf{Y: }Viene incrementato quando vengono apportate modifiche significative al documento, come cambiamenti strutturali, nuove sezioni importanti o modifiche sostanziali nel contenuto;
    \item \textbf{Z: }Viene incrementato per modifiche minori o aggiornamenti al documento. Questi potrebbero includere correzioni di errori, miglioramenti marginali o l'aggiunta di nuovi contenuti meno rilevanti.
\end{itemize}

L'incremento dei valori più significati  porta i meno significativi a zero. \\
Ogni variazione di versione deve essere presente nel registro delle modifiche.


\subsubsection{Repository}
Le \textit{repository}\textsubscript{\textit{G}} del team sono:
\begin{itemize}
    \item \href{https://github.com/ByteOps-swe/Sorgente-documenti}{Sorgente-documenti:} Repository per il versionamento, la gestione e lo sviluppo dei sorgenti della documentazione;
    \item \href{https://github.com/ByteOps-swe/Documents}{Documents:} Repository destinata ai commintenti/proponenti dove vengono condivisi solo i PDF dei file sorgenti revisionati.
    \item \href{https://github.com/ByteOps-swe/proof-of-concept}{proof-of-concept:} Repository destinata al \textit{POC}\textsubscript{\textit{G}}.
\end{itemize}
\paragraph{Struttura repository}
I \textit{repository}\textsubscript{\textit{G}} destinati alla documentazione sono organizzati come segue:
\begin{itemize}
    \item \textbf{Candidatura:} Contenente i documenti richiesti per la candidatura;
          \begin{itemize}
              \item \textbf{Verbali:} Contiene tutti i verbali redatti nel corso del periodo di candidatura, distinti tra verbali esterni e interni;
              \item Lettera di presentazione;
              \item Valutazione dei costi e assunzione impegni;
              \item Valutazione dei capitolati.
          \end{itemize}
    \item \textbf{RTB:} Contenente i documenti richiesti per la revisione omonima;
          \begin{itemize}
              \item \textbf{Verbali:} Contiene tutti i verbali redatti nel corso del periodo della \textit{RTB}\textsubscript{\textit{G}}, distinti tra verbali esterni e interni;
              \item Piano di Qualifica;
              \item Piano di Progetto;
              \item Analisi dei Requisiti;
              \item Glossario;
              \item Norme di Progetto.
          \end{itemize}
    \item \textbf{PB}.
\end{itemize}


\paragraph{Sincronizzazione e Branching}
\textbf{Documentazione} \\
Il processo operativo utilizzato per la redazione della documentazione, noto come "feature \textit{branch}\textsubscript{\textit{G}} \textit{workflow}\textsubscript{\textit{G}}", implica la creazione di un ramo dedicato per ciascun documento o sezione da elaborare.
Tale metodologia permette una parallelizzazione agile dei lavori evitando sovrascritture indesiderate di altri lavori e permette l'adozione dei principi "Documentation as code". \\ % ~\ref{sec:DocumentationAsCode} 

\textbf{Convenzioni per la nomenclatura dei branch relativi alle attività di redazione o modifica di documenti} \\

\begin{itemize}
    \item Il nome del \textit{branch}\textsubscript{\textit{G}} deve presentare l'identificativo del documento che si vuole redarre o modificare. \\
    Ad ogni documento è associato un identificativo, come descritto nella seguente tabella:
\end{itemize}

\begin{table}[H]
    \centering
    \begin{tabular}{|c|c|}
        \hline
        Documento & \textit{ID}\textsubscript{\textit{G}} \\
        \hline
        Verbale interno & VI \\
        Verbale esterno & VE \\
        Norme di Progetto & NdP \\
        Piano di Qualifica & PdQ \\
        Piano di Progetto & PdP \\ 
        Analisi dei Requisiti & AdR \\
        \hline
    \end{tabular}
    \caption{ID per la nomenclatura dei branch relativi alla documentazione}
\end{table}

\begin{itemize}
    \item Nel caso dei verbali, dopo l'identificativo del documento si aggiunge un underscore seguito dalla data: \\
    \texttt{IdDocumento\_dd-mm-yyyy} (ex. VI\_27-12-2023);
    \item Nel caso della redazione di una specifica sezione di un documento, il nome del \textit{branch}\textsubscript{\textit{G}} deve avere il formato: \\
    \texttt{IdDocumento\_NomeSezione} (ex. NdP\_Documentazione);
    \item Nel caso di modifica di una specifica sezione di un documento, il nome del \textit{branch}\textsubscript{\textit{G}} deve avere il formato: \\
    \texttt{IdDocumento\_ModNomeSezione} (ex. NdP\_ModDocumentazione).
\end{itemize}

\textbf{Sviluppo} \\
Gitflow è lo stile di flusso di lavoro Git che utilizza il team ByteOps per lo sviluppo.

\paragraph*{Flusso generale di Gitflow}

\begin{enumerate}
    \item \textbf{Branch \texttt{develop}:} Viene creato a partire dal \textit{branch}\textsubscript{\textit{G}} principale (\texttt{main}). È il punto di partenza per lo sviluppo di nuove funzionalità;

    \item \textbf{Branch \texttt{release}:} Creato da \texttt{develop}, questo \textit{branch}\textsubscript{\textit{G}} gestisce la preparazione del \textit{software}\textsubscript{\textit{G}} per un rilascio. Durante questa fase, sono consentite solo correzioni di bug e miglioramenti minori;

    \item \textbf{Branch \texttt{feature}:} Creati da \texttt{develop}, sono utilizzati per lo sviluppo di nuove funzionalità o miglioramenti. 

    \item \textbf{Merge di \texttt{feature} in \texttt{develop}:} Quando una funzionalità è completa, il \textit{branch}\textsubscript{\textit{G}} \texttt{feature} viene fuso nel \textit{branch}\textsubscript{\textit{G}} \texttt{develop};

    \item \textbf{Merge di \texttt{release} in \texttt{develop} e \texttt{main}:} Dopo il completamento del \textit{branch}\textsubscript{\textit{G}} \texttt{release}, viene unito sia in \texttt{develop} che in \texttt{main}, segnalando un nuovo rilascio stabile;

    \item \textbf{Branch \texttt{hotfix}:} Creato da \texttt{main} in caso di problemi critici rilevati nell'ambiente di produzione;

    \item \textbf{Merge di \texttt{hotfix} in \texttt{develop} e \texttt{main}:} Una volta risolto il problema, il \textit{branch}\textsubscript{\textit{G}} \texttt{hotfix} viene unito sia in \texttt{develop} che in \texttt{main} per garantire coerenza tra le versioni.
\end{enumerate}

\subsubsection*{Comandi di comodo}
\paragraph*{Inizializzare Gitflow}

% Inizializzare Gitflow
\begin{lstlisting}[style=code]
    git flow init
    \end{lstlisting}
    
    % Sviluppo di una Nuova Funzionalità
    \paragraph*{Sviluppo di una Nuova Funzionalità}
    
    % Inizia una Feature
    \begin{lstlisting}[style=code]
    git flow feature start nome_feature
    \end{lstlisting}
    
    % Lavora sul Codice
    % Effettua le modifiche e i commit sulla \textit{branch}\textsubscript{\textit{G}} della feature.
    
    % Completa la Feature
    \begin{lstlisting}[style=code]
    git flow feature finish nome_feature
    \end{lstlisting}
    
    % Risoluzione di Bug
    \paragraph*{Risoluzione di Bug}
    
    % Inizia una Hotfix
    \begin{lstlisting}[style=code]
    git flow hotfix start nome_hotfix
    \end{lstlisting}
    
    % Lavora sul Codice
    % Effettua le modifiche e i commit sulla \textit{branch}\textsubscript{\textit{G}} del hotfix.
    
    % Completa il Hotfix
    \begin{lstlisting}[style=code]
    git flow hotfix finish nome_hotfix
    \end{lstlisting}
    
    % Rilascio di una Nuova Versione
    \paragraph*{Rilascio di una Nuova Versione}
    
    % Inizia una Release
    \begin{lstlisting}[style=code]
    git flow release start X.X.X
    \end{lstlisting}
    
    % Lavora sul Codice
    % Effettua eventuali aggiustamenti per la release e imposta il numero di versione nei file appropriati.
    
    % Completa la Release
    \begin{lstlisting}[style=code]
    git flow release finish X.X.X
    \end{lstlisting}
    
    % Pubblicazione delle Modifiche
    \paragraph*{Pubblicazione delle Modifiche}
    
    % Push delle Branch
    \begin{lstlisting}[style=code]
    git push origin develop
    git push origin master
    git push origin --tags
    \end{lstlisting}
    
    % Push delle Feature/Hotfix/Release
    \begin{lstlisting}[style=code]
    git push origin feature/nome_feature
    git push origin hotfix/nome_hotfix
    git push origin release/X.X.X
    \end{lstlisting}
\subsubsection{Controllo di configurazione}
\paragraph{Change request (Richiesta di modifica)}
Seguendo lo \textit{standard}\textsubscript{\textit{G}}  ISO/IEC 12207 per affrontare questo processo in modo strutturato le \textit{attività}\textsubscript{\textit{G}} sono:
\begin{enumerate}
    \item \textbf{Identificazione e registrazione}: Le change request vengono identificate, registrate e documentate accuratamente. Questo include informazioni come la natura della modifica richiesta, l'urgenza, l'impatto sul \textit{sistema}\textsubscript{\textit{G}} esistente\\
          L'identificazione avviene tramite la creazione di un \textit{issue}\textsubscript{\textit{G}} nell'ITS con \textit{label}\textsubscript{\textit{G}}: "Change request";
    \item \textbf{Valutazione e analisi}: Le change request vengono valutate dal team per determinare la loro fattibilità, importanza e impatto sul progetto. Si analizzano i costi e i benefici associati all'implementazione della modifica;
    \item \textbf{Approvazione o rifiuto}:Il responsabile valuta le informazioni raccolte e decide se approvare o respingere la change request. Questa decisione può essere basata su criteri come il budget, il tempo, le priorità degli \textit{stakeholder}\textsubscript{\textit{G}};
    \item \textbf{Pianificazione delle modifiche}: Se una change request viene approvata, viene pianificata e integrata nel ciclo di sviluppo del \textit{software}\textsubscript{\textit{G}}. Questo può richiedere la rinegoziazione dei tempi di consegna, la revisione del piano di progetto, ecc;
    \item \textbf{Implementazione delle modifiche}: Le modifiche vengono effettivamente implementate. Durante questo processo, è fondamentale mantenere una traccia accurata di ciò che viene fatto per consentire una corretta documentazione e, se necessario, la possibilità di un \textit{rollback}\textsubscript{\textit{G}};
    \item \textbf{Verifica e validazione}: Le modifiche apportate vengono verificate per assicurarsi che abbiano raggiunto gli obiettivi previsti e non abbiano introdotto nuovi problemi o errori;
    \item \textbf{Documentazione}: Tutti i passaggi del processo di gestione delle change request vengono documentati accuratamente per garantire la trasparenza e la tracciabilità. Questa documentazione è utile per futuri riferimenti e per l'apprendimento dalle modifiche apportate;
    \item \textbf{Comunicazione agli interessati}: Durante tutto il processo, è importante comunicare in modo chiaro e tempestivo agli interessati, come il team di sviluppo, i clienti e altri \textit{stakeholder}\textsubscript{\textit{G}}, per mantenere la trasparenza e la fiducia.
\end{enumerate}

\subsubsection{Configuration Status Accounting (Contabilità dello Stato di Configurazione)}
Questo processo si occupa di registrare e tenere traccia dello stato di tutte le configurazioni di un \textit{sistema}\textsubscript{\textit{G}} \textit{software}\textsubscript{\textit{G}} durante il suo ciclo di vita.
\begin{itemize}
    \item \textbf{Registrazione delle configurazioni}: Registrazione delle informazioni dettagliate su ogni elemento di configurazione;
          \begin{itemize}
              \item  \textbf{Documentazione:} Le informazioni del CI sono presenti nella prima pagina di ciascuno.
              \item  \textbf{Sviulppo:} Le informazioni del CI sono inserite come prime righe del file sotto forma di commento.
          \end{itemize}
    \item \textbf{Stato e cambiamenti}: Tenere traccia dello stato attuale di ciascun elemento di configurazione e di tutti i cambiamenti che avvengono nel corso del tempo. Ciò include le versioni attuali, le revisioni, le modifiche e le baselines;
          \begin{itemize}
              \item  \textbf{Registro delle modifiche:} Per monitorare lo stato di ciascun CI, si utilizza il registro delle modifiche incorporato in ognuno di essi;
              \item  \textbf{Branching \& DashBoard:} Per visionare la presenza di lavori in atto su di un CI si utilizza a supporto la presenza di \textit{branch}\textsubscript{\textit{G}} attivi che lo riguardano e il posizionamento di \textit{issue}\textsubscript{\textit{G}} associate nella Dashboard di progetto.
                    \begin{itemize}
                        \item Ogni \textit{issue}\textsubscript{\textit{G}} è associato ad un CI tramite \textit{label}\textsubscript{\textit{G}}.  \label{sec:ticketing}
                    \end{itemize}
          \end{itemize}
    \item \textbf{Supporto per la gestione delle change request}: Registra e documenta le modifiche apportate agli elementi di configurazione in risposta alle richieste di modifica.
          \begin{itemize}
              \item  \textbf{ITS:} Per supportare i change request viene utilizzato l'ITS di \textit{github}\textsubscript{\textit{G}}, l'identificazione avviene tramite la creazione di un \textit{issue}\textsubscript{\textit{G}} nell'ITS con \textit{label}\textsubscript{\textit{G}}: "Change request".
          \end{itemize}
\end{itemize}
La Contabilità dello Stato di Configurazione è un processo fondamentale per mantenere la trasparenza e la tracciabilità nel ciclo di vita del \textit{software}\textsubscript{\textit{G}}, aiutando a gestire le configurazioni in modo coerente e a mantenere un registro accurato di tutte le \textit{attività}\textsubscript{\textit{G}} e le modifiche che coinvolgono gli elementi di configurazione.\\
\vspace{0.2cm}

Ad esempio, l'aggiunta di un elemento di configurazione al ramo principale (main) lo designa come la versione più recente e revisionata, mentre le versioni (X.0.0) con X>=1 indicano l'appartenenza del suddetto elemento di configurazione alla baseline, con la medesima versione.

\subsubsection{Release management and delivery}
Dopo aver portato a termine le \textit{attività}\textsubscript{\textit{G}} nel proprio \textit{branch}\textsubscript{\textit{G}}, il responsabile del suo sviluppo è tenuto a avviare una richiesta di pull per incorporare le modifiche nel ramo principale. La richiesta di pull può essere accettata solo se la verifica ha esito positivo.

\subsubsection{Strumenti}
Le tecnologie adottata per la gestione dei \textit{configuration item}\textsubscript{\textit{G}} sono:
\begin{itemize}
    \item \textbf{Git}: Version Control System distribuito utilizzato per il versionamento dei CI;
    \item \textbf{GitHub}: Piattaforma web che utilizza Git per il controllo di versione dei CI e per il Ticketing.
          Utilizzata per la gestione dei change request tramite \textit{issue}\textsubscript{\textit{G}} e \textit{label}\textsubscript{\textit{G}} e per la contabilità dello stato di configurazione (Branching, posizionamento nella \textit{repository}\textsubscript{\textit{G}} e DashBoard di progetto).
\end{itemize}

