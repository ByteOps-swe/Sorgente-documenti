\subsection{Metriche per la qualità di processo}
\begin{itemize}
    \vspace{0.4cm}
    \item \hypertarget{item:M1PMS}{\textbf{Metrica M1PMS:}}
    \vspace{0.2cm}

    \begin{minipage}[H]{0.9\textwidth}
        \begin{itemize}
        \item \textbf{Nome:} Percentuale di Metriche Soddisfatte (PMS);
        \item \textbf{Descrizione:} Misura che valuta quante metriche che sono state definite sono state effettivamente adottate o soddisfatte;
        \item \textbf{Formula:} $\frac{metriche \ soddisfatte}{metriche \ totali}\times 100$
        \end{itemize}
    \end{minipage}

    \vspace{0.4cm}
    \item \hypertarget{item:M2EAC}{\textbf{Metrica M2EAC:}}
    \vspace{0.2cm}

    \begin{minipage}[H]{0.9\textwidth}
        \begin{itemize}
            \item \textbf{Nome:} Estimated at Completion (\textit{EAC}\textsubscript{\textit{G}});
            \item \textbf{Descrizione:} Misura il costo realizzativo stimato per terminare il progetto;
            \item \textbf{Formula:} $\textit{EAC}\textsubscript{\textit{G}} = \frac{BAC}{CPI}$
        \end{itemize}
    \end{minipage}

    \vspace{0.4cm}
    \item \hypertarget{item:M3CPI}{\textbf{Metrica M3CPI:}}
    \vspace{0.2cm}

    \begin{minipage}[H]{0.9\textwidth}
        \begin{itemize}
            \item \textbf{Nome:} Cost Performance Index (\textit{CPI}\textsubscript{\textit{G}});
            \item \textbf{Descrizione:} Misura il rapporto tra il valore del lavoro effettivamente svolto ed il costo reale del lavoro fino al periodo di riferimento;
            \item \textbf{Formula:} $\textit{CPI}\textsubscript{\textit{G}} = \frac{EV}{AC}$
        \end{itemize}
    \end{minipage}

    \vspace{0.4cm}
    \item \hypertarget{item:M4BV}{\textbf{Metrica M4BV:}}
    \vspace{0.2cm}

    \begin{minipage}[H]{0.9\textwidth}
        \begin{itemize}
            \item \textbf{Nome:} Budget Variance (BV);
            \item \textbf{Descrizione:} Misura la differenza percentuale di budget tra quanto previsto nella pianificazione di un periodo e l’effettiva realizzazione;
            \item \textbf{Formula:} $BV = EV - AC $
        \end{itemize}
    \end{minipage}

    \vspace{0.4cm}
    \item \hypertarget{item:M5AC}{\textbf{Metrica M5AC:}}
    \vspace{0.2cm}

    \begin{minipage}[H]{0.9\textwidth}
        \begin{itemize}
            \item \textbf{Nome:} Actual Cost (AC);
            \item \textbf{Descrizione:} Misura i costi effettivamente sostenuti dall’inizio del progetto fino all’attualità; 
            \item \textbf{Formula:} Dato disponibile e aggiornato in "Piano di progetto" per ogni periodo.
        \end{itemize}
    \end{minipage}

    \vspace{0.4cm}  
    \item \hypertarget{item:M6SV}{\textbf{Metrica M6SV:}}
    \vspace{0.2cm}
    
    \begin{minipage}[H]{0.9\textwidth}
        \begin{itemize}
            \item \textbf{Nome:} Schedule Variance (SV);
            \item \textbf{Descrizione:} Indica in percentuale quanto si è in anticipo o in ritardo con le \textit{attività}\textsubscript{\textit{G}} pianificate;
            \item \textbf{Formula:} $SV = EV - \textit{PV}\textsubscript{\textit{G}}$
        \end{itemize}
    \end{minipage}

    \vspace{0.4cm}
    \item \hypertarget{item:M7EV}{\textbf{Metrica M7EV:}}
    \vspace{0.2cm}

    \begin{minipage}[H]{0.9\textwidth}
        \begin{itemize}
            \item \textbf{Nome:} Earned Value (EV);
            \item \textbf{Descrizione:} Valore del lavoro effettivamente svolto fino a quel periodo;
            \item \textbf{Formula:} $EV = \% lavoro \ svolto \times \textit{EAC}\textsubscript{\textit{G}}$
        \end{itemize}
    \end{minipage}

    \vspace{0.4cm}
    \item \hypertarget{item:M8PV}{\textbf{Metrica M8PV:}}
    \vspace{0.2cm}

    \begin{minipage}[H]{0.9\textwidth}
        \begin{itemize}
            \item \textbf{Nome:} Planned Value (\textit{PV}\textsubscript{\textit{G}});
            \item \textbf{Descrizione:} Stima la somma dei costi realizzativi delle \textit{attività}\textsubscript{\textit{G}} imminenti periodo per periodo;
            \item \textbf{Formula:} $\textit{PV}\textsubscript{\textit{G}} = \% lavoro \ svolto \times BAC$
        \end{itemize}
    \end{minipage}

    \vspace{0.4cm}
    \item \hypertarget{item:M9ETC}{\textbf{Metrica M9ETC:}}
    \vspace{0.2cm}

    \begin{minipage}[H]{0.9\textwidth}
        \begin{itemize}
            \item \textbf{Nome:} Estimate to Complete (\textit{ETC}\textsubscript{\textit{G}});
            \item \textbf{Descrizione:} Stima i costi realizzativi fino alla fine del progetto;
            \item \textbf{Formula:} $\textit{ETC}\textsubscript{\textit{G}} =\textit{EAC}\textsubscript{\textit{G}} - AC$
        \end{itemize}
    \end{minipage}

    \vspace{0.4cm}
    \item \hypertarget{item:M11RNP}{\textbf{Metrica M11RNP:}}
    \vspace{0.2cm}

    \begin{minipage}[H]{0.9\textwidth}
        \begin{itemize}
            \item \textbf{Nome:} Rischi Non Previsti (RNP);
            \item \textbf{Descrizione:} Misura il numero di rischi non previsti nel corso del progetto.
        \end{itemize}
    \end{minipage}

    \pagebreak
    \item \hypertarget{item:M12VR}{\textbf{Metrica M12VR:}}
    \vspace{0.2cm}

    \begin{minipage}[H]{0.8\textwidth}
        \begin{itemize}
            \item \textbf{Nome:} Variazione dei Requisiti (VR);
            \item \textbf{Descrizione:} Misura la variazione nei requisiti dal momento della pianificazione;
            \item \textbf{Formula:} \textit{NRA + NRR + NRM}, dove:\begin{itemize}
            \item \textit{NRA} (Numero Requisiti Aggiunti) è la quantità di requisiti aggiunti dall'ultimo incremento;
            \item \textit{NRR} (Numero Requisiti Rimossi) è la quantità di requisiti rimossi dall'ultimo incremento;
            \item \textit{NRM} (Numero Requisiti Modificati) è la quantità di requisiti modificati dall'ultimo incremento.
            \end{itemize}
        \end{itemize}
    \end{minipage}
    
    \vspace{0.4cm}
    \item \hypertarget{item:M13PCTS}{\textbf{Metrica M13PCTS:}}
    \vspace{0.2cm}
    
    \begin{minipage}[H]{0.9\textwidth}
        \begin{itemize}
            \item \textbf{Nome:} Percentuale di Casi di Test Superati (PCTS);
            \item \textbf{Descrizione:} Percentuale di casi di \textit{test}\textsubscript{\textit{G}} superati;
            \item \textbf{Formula:} $\frac{numero \ di \ casi \ di \ test \ superati}{numero \ totale \ di \ casi \ di \ test}\times 100$
        \end{itemize}
    \end{minipage}

    \vspace{0.4cm}
    \item \hypertarget{item:M14PCTF}{\textbf{Metrica M14PCTF:}}
    \vspace{0.2cm}

    \begin{minipage}[H]{0.9\textwidth}
        \begin{itemize}
           \item \textbf{Nome:} Percentuale di Casi di Test Falliti (PCTF);
           \item \textbf{Descrizione:} Percentuale di casi di \textit{test}\textsubscript{\textit{G}} non superati;
           \item \textbf{Formula:} $\frac{numero \ di \ casi \ di \ test \ non \ superati}{numero \ totale \ di \ casi \ di \ test}\times 100$
        \end{itemize}
    \end{minipage}

    \vspace{0.4cm}
    \item \hypertarget{item:M15SC}{\textbf{Metrica M15SC:}}
    \vspace{0.2cm}

    \begin{minipage}[H]{0.9\textwidth}
        \begin{itemize}
            \item \textbf{Nome:} Statement Coverage (\textit{MSC}\textsubscript{\textit{G}});
            \item \textbf{Descrizione:} Misura il numero di istruzioni che sono state eseguite almeno una volta;
            \item \textbf{Formula:} $\textit{MSC}\textsubscript{\textit{G}} = \frac{istruzioni \ eseguite}{istruzioni \ totali} \times 100$
        \end{itemize}
    \end{minipage}

    \pagebreak
    \item \hypertarget{item:M16BC}{\textbf{Metrica M16BC:}}
    \vspace{0.2cm}

    \begin{minipage}[H]{0.9\textwidth}
        \begin{itemize}
            \item \textbf{Nome:} Branch Coverage (MBC);
            \item \textbf{Descrizione:} Indice di quante diramazioni del codice vengono eseguite dai \textit{test}\textsubscript{\textit{G}}. Un \textit{branch}\textsubscript{\textit{G}} è uno dei possibili percorsi di esecuzione che il codice può seguire dopo che un'istruzione decisionale viene valutata;
            \item \textbf{Formula:} $MBC = \frac{flussi \ funzionali \ implementati \ e \ testati}{flussi \ condizionali \ riusciti \ e \ non} \times 100$
        \end{itemize}
    \end{minipage}

    \vspace{0.4cm}
    \item \hypertarget{item:M17CNC}{\textbf{Metrica M17CNC:}}
    \vspace{0.2cm}

    \begin{minipage}[H]{0.9\textwidth}
        \begin{itemize}
            \item \textbf{Nome:} CoNdition Coverage (CNC);
            \item \textbf{Descrizione:} Metrica di copertura del codice che indica la percentuale di condizioni logiche nel codice sorgente che sono state eseguite durante i \textit{test}\textsubscript{\textit{G}};
            \item \textbf{Formula:} $CNC = \frac{numero \ di \ operandi \ eseguiti}{numero \ totale \ di \ operandi \ eseguiti} \times 100$
        \end{itemize}
    \end{minipage}
\end{itemize}

\vspace{0.3cm}