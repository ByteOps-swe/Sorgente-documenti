\subsection{Miglioramento}
\subsubsection{Scopo}
Secondo lo \textit{standard}\textsubscript{\textit{G}} ISO/IEC 12297:1995, il processo di miglioramento nel ciclo di vita del \textit{software}\textsubscript{\textit{G}} è finalizzato a stabilire, misurare, controllare e migliorare i \textit{processi}\textsubscript{\textit{G}} che lo compongono.
L'\textit{attività}\textsubscript{\textit{G}} di miglioramento è composta da:  
 \begin{itemize}
    \item \textbf{Analisi:} Identificare le aree di miglioramento dei \textit{processi}\textsubscript{\textit{G}}; 
    \item  \textbf{Miglioramento:} Implementare le modifiche necessarie per migliorare i \textit{processi}\textsubscript{\textit{G}}. 
    di sviluppo del \textit{software}\textsubscript{\textit{G}};
 \end{itemize}
 \subsubsection{Analisi}
 Esame 
Questa operazione richiede di essere eseguita regolarmente e ad intervalli di tempo appropriati e costanti. L'analisi fornisce un ritorno sulla reale efficacia e correttezza dei \textit{processi}\textsubscript{\textit{G}} implementati, permettendo di identificare prontamente quelli che necessitano di miglioramenti.

\vspace*{0.1cm}

Durante ogni riunione, il team dedica inizialmente del tempo per condurre una retrospettiva sulle \textit{attività}\textsubscript{\textit{G}} svolte nell'ultimo periodo. Questa pratica implica una riflessione approfondita su ciò che è stato realizzato, coinvolgendo tutti i membri nella identificazione delle aree di successo e di possibili miglioramenti. L'obiettivo principale è formulare azioni correttive da implementare nel prossimo sprint, promuovendo così un costante feedback e un adattamento continuo per migliorare le prestazioni complessive del team nel corso del tempo. 
 \subsubsection{Miglioramento}
 Il team implementa le azioni correttive stabilite durante la retrospettiva, successivamente valuta la loro efficacia e le sottopone nuovamente a esame durante la retrospettiva successiva.\\
L'esito di ogni azione correttiva sarà documentato nella sezione "Revisione del periodo precedente" di ogni verbale. 
