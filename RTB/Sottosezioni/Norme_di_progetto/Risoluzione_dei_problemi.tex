\subsection{Risoluzione dei problemi}
Il processo di risoluzione dei problemi è un processo che mira ad analizzare e risolvere i problemi (incluse le non conformità) di qualunque natura o fonte, che vengono scoperti durante l'esecuzione di sviluppo, operazioni, manutenzione o altri processi.
L'obiettivo è fornire un mezzo tempestivo, responsabile e documentato per garantire che tutti i problemi scoperti siano analizzati e risolti e che siano riconosciute le tendenze.

\subsubsection{Gestione dei rischi}
Nel documento piano di progetto, nella sezione "Analisi dei rischi", vengono identificati dal responsabile tutti i potenziali rischi di progetto, inclusa la probabilità della loro occorrenza e le misure di mitigazione. Per ogni fase di avanzamento, è dedicata una sezione alla documentazione dei problemi riscontrati, con un'analisi del loro impatto e una valutazione dell'esito della mitigazione programmata nella sezione "Analisi dei rischi". Un esito negativo evidenzia una mitigazione inadeguata che richiede modifiche.

\paragraph{Codifica dei rischi}
La convenzione utilizzata per la codifica dei rischi è la seguente: 
\begin{center}
    \texttt{R[Tipologia]-[Probabilità][Priorità]-[Indice]} : \textit{Nome associato al rischio}
\end{center} 

\begin{flushleft}
    \textbf{Tipologia}: natura del rischio:
    \begin{itemize}
        \item \textbf{T}: Rischi legati all'utilizzo delle tecnologie
        \item \textbf{O}: Rischi legati all'organizzazione del gruppo
        \item \textbf{P}: Rischi legati agli impegni personali dei membri del gruppo
    \end{itemize}
    \textbf{Probabilità}: valore alfabetico che indica la probabilità di occorrenza del rischio:
    \begin{itemize}
        \item \textbf{1}: Alta
        \item \textbf{2}: Media
        \item \textbf{3}: Bassa
    \end{itemize}
    \textbf{Priorità}: valore numerico che indica la pericolosità del rischio:
    \begin{itemize}
        \item \textbf{A}: Alta
        \item \textbf{M}: Media
        \item \textbf{B}: Bassa
    \end{itemize}
    \textbf{Indice}: valore numerico incrementale che determina univocamente il rischio per ogni tipologia di rischio. 
\end{flushleft}

\paragraph{Metriche}
\begin{table}[H]
  \centering
  \begin{tabular}{|C{3cm}|C{3cm}|C{4cm}|}
  \hline
  Metrica & Nome & Riferimento \\
  \hline \hline
  M11RNP & Rischi non previsti (RNP) &  \hyperlink{item:M11RNP}{\textcolor{linkcolor}{M11RNP}}\\ 
  \hline
  \end{tabular}
  \caption{Metriche relative alla gestione dei processi}
\end{table}

\subsubsection{Identificazione dei problemi}
Nel caso in cui un membro del team identifichi un problema, è obbligatorio notificarlo immediatamente al gruppo, e contemporaneamente, deve essere aperta una segnalazione nel sistema di tracciamento degli issue (ITS) con l'etichetta "bug" e una descrizione completa del problema.

\subsubsection{Strumenti}
\begin{itemize}
    \item \textbf{GitHub ITS:} 
        per la segnalazione dei problemi.
\end{itemize}