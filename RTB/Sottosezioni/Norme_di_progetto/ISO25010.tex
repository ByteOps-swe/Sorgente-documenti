Il team ByteOps adotta per la valutazione della qualità del software lo standard ISO/IEC 25010.
\subsection{ISO/IEC 25010: Modello di Qualità del Software}

L'ISO/IEC 25010 è uno standard internazionale che definisce un modello di qualità del software dettagliato, fornendo una struttura completa per valutare e gestire la qualità del prodotto software. Questo modello è organizzato in due categorie principali di caratteristiche di qualità: le caratteristiche del sistema e le sottocaratteristiche.

\subsubsection{Caratteristiche del Sistema}

\paragraph{Funzionalità}
\begin{itemize}
    \item \textbf{Idoneità Funzionale:} La capacità del software di fornire funzionalità che soddisfano i requisiti specificati;
    \item \textbf{Accuratezza:} La precisione con cui il software esegue le sue funzioni;
    \item \textbf{Interoperabilità:} La capacità del software di interagire con altri sistemi.
\end{itemize}

\paragraph{Affidabilità}
\begin{itemize}
    \item \textbf{Maturità:} La capacità del software di evitare errori o malfunzionamenti;
    \item \textbf{Tolleranza agli Errori:} La capacità di mantenere un certo livello di prestazioni nonostante la presenza di errori;
    \item \textbf{Recuperabilità:} La capacità del software di ripristinarsi dopo un errore.
\end{itemize}

\paragraph{Usabilità}
\begin{itemize}
    \item \textbf{Comprensibilità:} La facilità con cui gli utenti possono comprendere il software;
    \item \textbf{Apprendibilità:} Il tempo e lo sforzo richiesti per apprendere a utilizzare il software;
    \item \textbf{Operabilità:} La facilità con cui gli utenti possono operare il software.
\end{itemize}

\paragraph{Efficienza}
\begin{itemize}
    \item \textbf{Tempo di Risposta:} Il tempo impiegato dal software per rispondere alle richieste dell'utente;
    \item \textbf{Utilizzo delle Risorse:} L'efficienza nell'uso delle risorse del sistema.
\end{itemize}

\paragraph{Manutenibilità}
\begin{itemize}
    \item \textbf{Analizzabilità:} La facilità con cui è possibile analizzare il codice per individuare errori o problemi;
    \item \textbf{Modificabilità:} La facilità con cui il software può essere modificato;
    \item \textbf{Stabilità:} La capacità di evitare effetti indesiderati durante o dopo le modifiche.
\end{itemize}

\paragraph{Portabilità}
\begin{itemize}
    \item \textbf{Adattabilità:} La facilità con cui il software può essere adattato a diversi ambienti;
    \item \textbf{Installabilità:} La facilità con cui il software può essere installato;
    \item \textbf{Conformità:} Il rispetto delle norme e degli standard relativi alla portabilità.
\end{itemize}


%\subsubsection{Processi e Criteri di Valutazione:}

%L'ISO/IEC 25010 fornisce anche un insieme di processi e criteri per valutare queste caratteristiche. Questi includono l'identificazione dei requisiti di qualità, la pianificazione delle attività di valutazione, l'implementazione di misure di valutazione e la formulazione di conclusioni sulla qualità del software. Inoltre, vengono forniti modelli di riferimento per aiutare le organizzazioni a migliorare la qualità del loro software attraverso un approccio strutturato e coerente.

%In sintesi, l'ISO/IEC 25010 rappresenta uno strumento completo per valutare e gestire la qualità del software, coprendo una vasta gamma di aspetti che contribuiscono al successo di un prodotto software.
