\subsection{Metriche per la qualità di prodotto}
\begin{itemize}
    
    \item \textbf{Metrica M18PROS:}
    \begin{itemize}
     \item \textbf{Nome:} Percentuale di Requisiti Obbligatori Soddisfatti (PROS)
     \item \textbf{Descrizione:} Metrica che valuta quanto del lavoro svolto durante lo sviluppo corrisponda ai requisiti essenziali o obbligatori definiti in fase di analisi dei requisiti.
     \item \textbf{Formula:} $\frac{requisiti \ obbligatori \ soddisfatti}{requisiti \ obbligatori \ totali}\times 100$
     \item \textbf{Caratteristica di qualità:} Funzionalità.
    \end{itemize}

    \item \textbf{Metrica M19PRDS:}
    \begin{itemize}
     \item \textbf{Nome:} Percentuale di Requisiti Desiderati Soddisfatti (PRDS)
     \item \textbf{Descrizione:} Metrica usata per valutare quanti di quei requisiti, che se integrati arricchirebbero l'esperienza dell'utente o fornirebbero vantaggi aggiuntivi non strettamente necessari, sono stati implementati o soddisfatti nel prodotto.
     \item \textbf{Formula:} $\frac{requisiti \ desiderabili \ soddisfatti}{requisiti \ desiderabili \ totali}\times 100$
     \item \textbf{Caratteristica di qualità:} Funzionalità.
    \end{itemize}

    \item \textbf{Metrica M20PRPS:}
    \begin{itemize}
     \item \textbf{Nome:} Percentuale di Requisiti oPzionali Soddisfatti (PRPS)
     \item \textbf{Descrizione:} Metrica per valutare quanti dei requisiti aggiuntivi, non essenziali o di bassa priorità, sono stati implementati o soddisfatti nel prodotto.
     \item \textbf{Formula:} $\frac{requisiti \ opzionali \ soddisfatti}{requisiti \ opzionali \ totali}\times 100$
     \item \textbf{Caratteristica di qualità::} Funzionalità.
    \end{itemize}

    \item \textbf{Metrica M21IF:}
                  \begin{itemize}
                      \item \textbf{Nome:} Implementazione delle Funzionalità (IF)
                      \item \textbf{Descrizione:} Misura qual è la quantità di funzionalità pianificate che sono state implementate.
                      \item \textbf{Formula:}$(1 - \frac{F_{NL}}{F_L}) \times 100$
                      \item \textbf{Caratteristica di qualità:} Funzionalità.
                  \end{itemize}

    \item \textbf{Metrica M22CO:}
          \begin{itemize}
              \item \textbf{Nome:} Correttezza Ortografica (CO)
              \item \textbf{Descrizione:} Rappresenta il numero di errori grammaticali ed
              ortografici all'interno di un documento;
              \item \textbf{Caratteristica di qualità:} Affidabilità.
          \end{itemize}

    \item \textbf{Metrica M23IG:}
            \begin{itemize}
                \item \textbf{Nome:} Indice Gulpease (MIG)
                \item \textbf{Descrizione:} Indice di leggibilità di un testo tarato sulla lingua italiana, che utilizza la lunghezza delle parole in lettere anziché in sillabe, semplificandone il calcolo automatico.
                \item \textbf{Formula:} $IG = 89 + \frac{300 \cdot N_f - 10 \cdot N_l}{N_p}$
                \item \textbf{Dove:}
                      \begin{itemize}
                          \item $N_f$: numero di frasi;
                          \item $N_l$: numero di lettere;
                          \item $N_p$: numero di parole.
                      \end{itemize}
                \item I risultati sono compresi tra 0 e 100, dove il valore "100" indica la leggibilità più alta e "0" la leggibilità più bassa. In generale risulta che i testi con un indice:
                      \begin{itemize}
                          \item $< 80$: sono difficili da leggere per chi ha la licenza elementare;
                          \item $< 60$: sono difficili da leggere per chi ha la licenza media;
                          \item $< 40$: sono difficili da leggere per chi ha un diploma superiore.
                      \end{itemize}
                \item \textbf{Caratteristica di qualità:} Affidabilità.
            \end{itemize}

        \item \textbf{Metrica M24DE:}
                  \begin{itemize}
                      \item \textbf{Nome:} Densità degli Errori (DE)
                      \item \textbf{Descrizione:} Misura la percentuale di errori presenti nel prodotto rispetto al totale del codice.
                      \item \textbf{Formula:} $DE = \frac{numero \ di \ errori}{totale \ delle \ linee \ di \ codice} \times 100$
                      \item \textbf{Caratteristica di qualità:} Affidabilità.
                  \end{itemize}


            \item \textbf{Metrica M25ATC:}
                  \begin{itemize}
                      \item \textbf{Nome:} Accoppiamento tra Classi (ATC)
                      \item \textbf{Descrizione:} Misura il livello di accoppiamento tra le classi del sistema.
                      \item \textbf{Caratteristica di qualità:} Manutenibilità.
                  \end{itemize}

        \item \textbf{Metrica M26MCCM}
            \begin{itemize}
            \item \textbf{Nome:} Complessità Ciclomatica per Metodo (MCCM)
            \item \textbf{Descrizione:} Misura il numero di cammini linearmente indipendenti attraverso il grafo di controllo di flusso del metodo.
            \item \textbf{Formula:} $MCCM = e - n + 2$
            \item \textbf{Dove:}
            \begin{itemize}
                        \item \textit{e}: Numero di archi del grafo del flusso di esecuzione del metodo.
                        \item \textit{n}: Numero di vertici del grafo del flusso di esecuzione del metodo.
            \end{itemize}
            \item \textbf{Caratteristica di qualità:} Manutenibilità.
        \end{itemize}

        \item \textbf{Metrica M27PM:}
        \begin{itemize}
            \item \textbf{Nome:} Parametri per Metodo (M12PM)
            \item \textbf{Descrizione:} Numero massimo di parametri per metodo.
            \item \textbf{Caratteristica di qualità:} Manutenibilità.
          \end{itemize}

        \item \textbf{Metrica M28APC:}
          \begin{itemize}
              \item \textbf{Nome:} Attributi Per Classe (APC)
              \item \textbf{Descrizione:} Misura il numero massimo di attributi per classe.
              \item \textbf{Caratteristica di qualità:} Manutenibilità.
          \end{itemize} 

          \item \textbf{Metrica M29LCM:}
          \begin{itemize}
              \item \textbf{Nome:} Linee di Codice per Metodo (LCM)
              \item \textbf{Descrizione:} Limite massimo di linee di codice per metodo.
              \item \textbf{Caratteristica di qualità:} Manutenibilità.
          \end{itemize} 

          \item \textbf{Metrica M30PG:}
          \begin{itemize}
              \item \textbf{Nome:} Profondità delle Gerarchie (PG)
              \item \textbf{Descrizione:} Metrica che misura il numero di livelli tra una classe base (super-classe) e le sue sotto-classi (classi derivate).
              \item \textbf{Caratteristica di qualità:} Manutenibilità.
          \end{itemize} 

            \item \textbf{Metrica M31TMR:}
                  \begin{itemize}
                      \item \textbf{Nome:} Tempo Medio di Risposta (TMR)
                      \item \textbf{Descrizione:} Metrica che misura quanto è efficiente e reattivo un sistema software.
                      \item \textbf{Formula:}$\frac{somma \ dei \ tempi \ di \ risposta}{numero \ totale \ di \ misurazioni}$
                      \item \textbf{Caratteristica di qualità:} Efficienza.
                    \end{itemize}  


            \item \textbf{Metrica M32FU:}
                  \begin{itemize}
                      \item \textbf{Nome:} Facilità di Utilizzo (FU)
                      \item \textbf{Descrizione:} Metrica che misura l'usabilità di un sistema software.
                      \item \textbf{Caratteristica di qualità:} Usabilità.
                    \end{itemize}

            \item \textbf{Metrica M33TA:}
                  \begin{itemize}
                      \item \textbf{Nome:} Tempo di Apprendimento (TA)
                      \item \textbf{Descrizione:} Misura il tempo massimo richiesto per apprendere l'utilizzo del prodotto.
                      \item \textbf{Caratteristica di qualità:} Usabilità.
                    \end{itemize}

            \item \textbf{Metrica M34VBS:}
                    \begin{itemize}
                        \item \textbf{Nome:} Versioni dei Browser Supportate (VBS)
                        \item \textbf{Descrizione:} Metrica che misura la percentuale delle versioni di browser supportate rispetto al totale delle versioni disponibili.
                        \item \textbf{Caratteristica di qualità:} Portabilità.
                      \end{itemize}

\end{itemize}