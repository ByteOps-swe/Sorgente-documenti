\subsection{Gestione della qualità}
\subsubsection{Introduzione}
Questa sezione è dedicata alle direttive del team per gestire la qualità. Il processo mira a garantire la qualità del flusso operativo adottato dal \textit{fornitore}\textsubscript{\textit{G}} e dei prodotti sviluppati, al fine di soddisfare le aspettative del cliente e del \textit{proponente}\textsubscript{\textit{G}}.

La gestione della qualità costituisce un approccio \textit{olistico}\textsubscript{\textit{G}} che abbraccia l'intero ciclo di vita del \textit{software}\textsubscript{\textit{G}}, dall'atto del concepimento all'implementazione e oltre, al fine di assicurare che il prodotto finale sia conforme agli \textit{standard}\textsubscript{\textit{G}} di qualità prestabiliti e consenta un miglioramento costante dei \textit{processi}\textsubscript{\textit{G}}.

\subsubsection{Attività}
Le \textit{attività}\textsubscript{\textit{G}} che il team si impegna a svolgere per assicurare qualità dei \textit{processi}\textsubscript{\textit{G}} e di conseguenza dei prodotti sono:
\begin{enumerate}
    \item \textbf{Definizione degli Standard di qualità:}
        La gestione della qualità inizia con la definizione chiara degli \textit{standard}\textsubscript{\textit{G}} di qualità che il \textit{software}\textsubscript{\textit{G}} dovrebbe raggiungere. Questi \textit{standard}\textsubscript{\textit{G}} possono includere requisiti funzionali e non funzionali;

    \item \textbf{Pianificazione della qualità:}
        Viene sviluppato un piano di qualità che identifica \textit{attività}\textsubscript{\textit{G}} specifiche, risorse e tempistiche per garantire la qualità del \textit{software}\textsubscript{\textit{G}} durante l'intero ciclo di vita del progetto;

    \item \textbf{Assicurazione della qualità:}
        La fase di assicurazione della qualità coinvolge \textit{attività}\textsubscript{\textit{G}} continue di monitoraggio e valutazione per garantire che i \textit{processi}\textsubscript{\textit{G}} di sviluppo siano conformi agli \textit{standard}\textsubscript{\textit{G}} di qualità stabiliti;

    \item \textbf{Controllo della qualità:}
        Il controllo della qualità include l'esecuzione di \textit{test}\textsubscript{\textit{G}} e verifiche per garantire che il \textit{software}\textsubscript{\textit{G}} soddisfi gli \textit{standard}\textsubscript{\textit{G}} di qualità e risponda alle specifiche richieste;

    \item \textbf{Gestione delle modifiche:}
        Un \textit{sistema}\textsubscript{\textit{G}} di gestione delle modifiche è implementato per gestire e controllare le modifiche al \textit{software}\textsubscript{\textit{G}}. Questo assicura che ogni modifica venga valutata in termini di impatto sulla qualità complessiva del prodotto;

    \item \textbf{Miglioramento continuo e correzione:}
        La gestione della qualità promuove il miglioramento continuo dei \textit{processi}\textsubscript{\textit{G}}. Attraverso la raccolta di feedback, l'analisi delle prestazioni e l'implementazione di \textit{best practice}\textsubscript{\textit{G}}, il team cerca costantemente di ottimizzare la qualità del \textit{software}\textsubscript{\textit{G}};

    \item \textbf{Coinvolgimento degli stakeholder:}
        Gli \textit{stakeholder}\textsubscript{\textit{G}}, inclusi clienti e utenti finali, sono coinvolti nel processo di gestione della qualità. I loro feedback sono preziosi per garantire che il \textit{software}\textsubscript{\textit{G}} risponda alle esigenze e alle aspettative;

    \item \textbf{Formazione e competenza del team:}
        La formazione continua del team è essenziale per mantenere elevate competenze e conoscenze. Un team ben addestrato è in grado di produrre \textit{software}\textsubscript{\textit{G}} di alta qualità.
\end{enumerate}

\subsubsection{Strumenti}
Gli strumenti impiegati per la gestione della qualità sono rappresentati dalle metriche.

\subsubsection{Struttura e identificazione metriche}
\begin{itemize}
    \item \textbf{Codice:} Identificativo della metrica nel formato: 
        \begin{center}
            \textbf{M [numero] [abbreviazione]}
        \end{center}
        dove:
        \begin{itemize}
            \item \texttt{M}: sta per metrica;
            \item \texttt{[numero]}: è un numero progressivo univoco per ogni metrica; 
            \item \texttt{[abbreviazione]}: è una abbreviazione composta dalle iniziali del nome della metrica.
        \end{itemize}
    \item \textbf{Nome:} Specifica il nome della metrica;
    \item \textbf{Descrizione:} Breve descrizione della funzionalità della metrica adottata;
    \item \textbf{Scopo:}
        Il motivo per cui è importante tale misura al fine del progetto;

\end{itemize}
    \vspace{0.2cm}
    Eventualmente anche:
\begin{itemize}
    \item \textbf{Formula:} Come viene calcolata; 
    \item \textbf{Strumento:} Lo strumento che viene usato per calcolarla;
\end{itemize}

\subsubsection{Criteri di accettazione}
Per ciascuna metrica nel documento "Piano di Qualifica" vengono definiti in formato tabellare:
\begin{itemize}
    \item \textbf{Valore accettabile:}
        Valore considerato accettabile una volta assunto dalla metrica;
    \item \textbf{Valore preferibile:}
        Valore ideale che dovrebbe essere assunto dalla metrica; 
\end{itemize}

\subsubsection{Metriche}
\begin{table}[H]
    \centering
    \begin{tabular}{|C{3cm}|C{3cm}|C{4cm}|}
    \hline
    Metrica & Nome & Riferimento\\
    \hline \hline
    M1PMS & Percentuale di Metriche Soddisfatte (PMS) &  \hyperlink{item:M1PMS}{\textcolor{linkcolor}{M1PMS}} \\
    M24DE & Densità degli Errori (DE) &  \hyperlink{item:M24DE}{\textcolor{linkcolor}{M24DE}} \\
    \hline
    \end{tabular}
    \caption{Metriche relative alla gestione della qualità}
\end{table}

