\subsection{Gestione della qualità}
\subsubsection{Introduzione}
Le attività correlate al processo di gestione della qualità mirano a garantire la qualità del flusso operativo adottato dal fornitore sui prodotti sviluppati, al fine di soddisfare pienamente le aspettative del cliente e del proponente, nonché di rispettare i requisiti di qualità specificati. Questo include la definizione degli obiettivi di qualità, l'identificazione delle metriche e dei criteri di qualità, la pianificazione e l'esecuzione delle attività di controllo della qualità, e la verifica della qualità attraverso revisioni, ispezioni e test. L'obiettivo è garantire che il prodotto software sia conforme alle aspettative degli utenti e ai requisiti del progetto. \\
La gestione della qualità rappresenta quindi un approccio olistico che abbraccia l'intero ciclo di vita del software, dal concepimento all'implementazione e oltre, con l'obiettivo di assicurare che il prodotto finale rispetti gli standard di qualità predefiniti e consenta un continuo miglioramento dei processi.

\subsubsection{Attività}
Le attività che il team si impegna a svolgere per assicurare qualità dei processi e di conseguenza dei prodotti sono:
\begin{enumerate}
    \item \textbf{Definizione degli Standard di qualità:}
        la gestione della qualità inizia con la definizione chiara degli standard di qualità che il software dovrebbe raggiungere. Questi standard possono includere requisiti funzionali e non funzionali;

    \item \textbf{Pianificazione della qualità:}
        viene sviluppato un piano di qualità che identifica attività specifiche, risorse e tempistiche per garantire la qualità del software durante l'intero ciclo di vita del progetto;

    \item \textbf{Assicurazione della qualità:}
        la fase di assicurazione della qualità coinvolge attività continue di monitoraggio e valutazione per garantire che i processi di sviluppo siano conformi agli standard di qualità stabiliti;

    \item \textbf{Controllo della qualità:}
        il controllo della qualità include l'esecuzione di test e verifiche per garantire che il software soddisfi gli standard di qualità e risponda alle specifiche richieste;

    \item \textbf{Gestione delle modifiche:}
        un sistema di gestione delle modifiche è implementato per gestire e controllare le modifiche al software. Questo assicura che ogni modifica venga valutata in termini di impatto sulla qualità complessiva del prodotto;

    \item \textbf{Miglioramento continuo e correzione:}
        la gestione della qualità promuove il miglioramento continuo dei processi. Attraverso la raccolta di feedback, l'analisi delle prestazioni e l'implementazione di best practice, il team cerca costantemente di ottimizzare la qualità del software;

    \item \textbf{Coinvolgimento degli stakeholder:}
        gli stakeholder, inclusi clienti e utenti finali, sono coinvolti nel processo di gestione della qualità. I loro feedback sono preziosi per garantire che il software risponda alle esigenze e alle aspettative;

    \item \textbf{Formazione e competenza del team:}
        la formazione continua del team è essenziale per mantenere elevate competenze e conoscenze. Un team ben addestrato è in grado di produrre software di alta qualità.
\end{enumerate}

\subsubsection{Strumenti}
Gli strumenti impiegati per la gestione della qualità sono rappresentati dalle metriche.

\subsubsection{Struttura e identificazione metriche}
\begin{itemize}
    \item \textbf{Codice:} \\
    identificativo della metrica nel formato:
        \begin{center}
            \textbf{M [numero] [abbreviazione]}
        \end{center}
        dove:
        \begin{itemize}
            \item \texttt{M}: sta per metrica;
            \item \texttt{[numero]}: numero progressivo univoco per ogni metrica;
            \item \texttt{[abbreviazione]}: abbreviazione composta dalle iniziali del nome della metrica.
        \end{itemize}
    \item \textbf{Nome:} specifica il nome della metrica;
    \item \textbf{Descrizione:} breve descrizione della funzionalità della metrica adottata;
    \item \textbf{Scopo:} il motivo per cui è importante tale misura al fine del progetto.
\end{itemize}
    \vspace{0.2cm}
Eventualmente anche:
\begin{itemize}
    \item \textbf{Formula:} come viene calcolata la metrica;
    \item \textbf{Strumento:} lo strumento che viene usato per calcolare la metrica;
\end{itemize}

\subsubsection{Criteri di accettazione}
Per ciascuna metrica nel documento \textit{Piano di Qualifica} vengono definiti in formato tabellare:
\begin{itemize}
    \item \textbf{Valore accettabile:} valore che la metrica deve raggiungere per essere considerata soddisfacente o confrome agli standard stabiliti;
    \item \textbf{Valore preferibile:} valore ideale che dovrebbe essere assunto dalla metrica.
\end{itemize}
