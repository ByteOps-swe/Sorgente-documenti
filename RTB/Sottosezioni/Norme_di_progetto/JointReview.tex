\subsection{Joint review} 

\subsubsection{Introduzione}
Il processo di revisione congiunta costituisce un metodo formale per valutare lo stato e i risultati di un'attività all'interno di un progetto, coinvolgendo sia il livello gestionale che tecnico. Tale procedura viene attuata durante l'intero periodo contrattuale. Può essere attivata da due entità qualsiasi, di cui una (la parte recensita) esamina criticamente l'altra parte (la parte recensente).

Nel nostro caso i recensori sono gli stakeholder: committente, proponente mentre noi fornitori siamo i recensiti.

Il processo consiste nelle seguenti attività:

\subsubsection{Implementazione del processo}
Questa attività comprende i seguenti compiti:

\paragraph{Revisioni periodiche}
Saranno condotte revisioni periodiche in corrispondenza di milestone prestabilite come specificato nel documento piano di progetto.

\paragraph{SAL}
Ogni due settimane, viene eseguita una revisione SAL \textit{(Stato Avanzamento Lavori)} tra fornitore e proponente, al fine di valutare il lavoro svolto nelle due settimane trascorse dal SAL precedente e definire le prossime scadenze delle attività.

\paragraph{Revisioni ad hoc}
Revisioni ad hoc sono convocate quando ritenute necessarie da una qualsiasi delle parti.
Vengono richieste dal fornitore tramite responsabile per questioni ritenute da lui valide.

\paragraph{Risorse per le revisioni}
Tutte le risorse necessarie per condurre le revisioni sono concordate tra le parti. Queste risorse includono personale, località, strutture, hardware, software e strumenti.

\paragraph{Elementi da concordare}
Le parti concordano i seguenti elementi in ciascuna revisione:
\begin{itemize}
    \item 
        agenda della riunione;
    \item 
        prodotti software (risultati di un'attività) e relativi problemi da esaminare;
    \item 
        ambito e procedure;
    \item 
        criteri di ingresso e uscita per la revisione.
\end{itemize}

%\paragraph{Registrazione dei problemi}
%I problemi rilevati durante le revisioni devono essere registrati ed inseriti nel Processo di Risoluzione dei Problemi (6.8) come richiesto.

\paragraph{Documentazione e distribuzione dei risultati}
I risultati della revisione devono essere documentati e distribuiti in verbali esterni.

La parte recensente riconoscerà alla parte recensita l'adeguatezza (ad esempio approvazione, disapprovazione o approvazione condizionale) dei risultati della revisione.

\subsubsection{Project management reviews}

\paragraph{Stato del Progetto}
Lo stato del progetto deve essere valutato in relazione ai piani di progetto, agli schemi temporali, agli standard e alle linee guida applicabili.

L'esito della revisione è discusso tra le due parti e prevede quanto segue:

\begin{enumerate}
    \item
        garantire che le attività progrediscano secondo i piani, basandosi su una valutazione dello stato dell'attività e/o del prodotto software;
    \item
        mantenere il controllo globale del progetto attraverso l'allocazione adeguata delle risorse;
    \item
        modificare la direzione del progetto o determinare la necessità di pianificazioni alternative;
    \item
        valutare e gestire le questioni legate al rischio che potrebbero compromettere il successo del progetto.
\end{enumerate}

\subsubsection{Recensioni Tecniche}
Le recensioni tecniche devono essere condotte per valutare i prodotti o servizi software presi in considerazione e fornire evidenze che:

\begin{enumerate}
    \item
        siano completi;
    \item
        siano conformi agli standard e alle specifiche previsti;
    \item
        le modifiche ad essi siano correttamente implementate e influiscano solo sulle aree identificate dalla change request;
    \item
        siano conformi agli schemi temporali applicabili;
    \item
        siano pronti per la successiva attività;
    \item
        lo sviluppo, l'operatività o la manutenzione siano condotti secondo i piani, gli schemi temporali, gli standard e le linee guida del progetto.
\end{enumerate}

\subsubsection{Strumenti}

\begin{itemize}
    \item \textbf{Zoom:} 
        per revisioni tecniche con i committenti;
    \item \textbf{Google meet:} 
        per SAL con la proponente; 
    \item \textbf{Element:} 
        per richieste di revisioni ad hoc alla proponente.
\end{itemize}