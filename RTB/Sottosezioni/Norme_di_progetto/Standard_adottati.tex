Nel corso dell'analisi e della valutazione della qualità dei \textit{processi}\textsubscript{\textit{G}} e del \textit{software}\textsubscript{\textit{G}}, adotteremo \textit{standard}\textsubscript{\textit{G}} internazionali ben definiti per garantire una valutazione rigorosa e conforme agli \textit{standard}\textsubscript{\textit{G}} globali. In particolare, l'utilizzo dello \textit{standard}\textsubscript{\textit{G}} ISO/IEC 9126 fornirà una solida struttura per valutare la qualità del \textit{software}\textsubscript{\textit{G}}, concentrandosi su attributi quali la funzionalità, l'affidabilità, l'usabilità, l'efficienza, la manutenibilità e la portabilità. Questo \textit{framework}\textsubscript{\textit{G}} ci consentirà di misurare in modo accurato e completo la qualità del prodotto \textit{software}\textsubscript{\textit{G}}. Parallelamente, la suddivisione dei \textit{processi}\textsubscript{\textit{G}} in primari, di supporto e organizzativi sarà guidata dall'adozione dello \textit{standard}\textsubscript{\textit{G}} \href{https://www.math.unipd.it/~tullio/IS-1/2009/Approfondimenti/ISO_12207-1995.pdf}{ISO/IEC 12207:1995}. Infine, l'adozione dello \textit{standard}\textsubscript{\textit{G}} ISO/IEC 25010 ci fornirà un quadro completo per la definizione e la suddivisione delle metriche di qualità del \textit{software}\textsubscript{\textit{G}}.
L'utilizzo congiunto di questi \textit{standard}\textsubscript{\textit{G}} consentirà un approccio completo e strutturato alla valutazione della qualità dei \textit{processi}\textsubscript{\textit{G}} e del \textit{software}\textsubscript{\textit{G}}, assicurando un'elevata coerenza, affidabilità e conformità agli \textit{standard}\textsubscript{\textit{G}} riconosciuti a livello internazionale.

\subsubsection{Caratteristiche del Sistema, Standard ISO/IEC 25010}

\paragraph{Funzionalità}
\begin{itemize}
    \item \textbf{Idoneità Funzionale:} la capacità del \textit{software}\textsubscript{\textit{G}} di fornire funzionalità che soddisfano i requisiti specificati;
    \item \textbf{Accuratezza:} la precisione con cui il \textit{software}\textsubscript{\textit{G}} esegue le sue funzioni;
    \item \textbf{Interoperabilità:} la capacità del \textit{software}\textsubscript{\textit{G}} di interagire con altri sistemi.
\end{itemize}

\paragraph{Affidabilità}
\begin{itemize}
    \item \textbf{Maturità:} la capacità del \textit{software}\textsubscript{\textit{G}} di evitare errori o malfunzionamenti;
    \item \textbf{Tolleranza agli Errori:} la capacità di mantenere un certo livello di prestazioni nonostante la presenza di errori;
    \item \textbf{Recuperabilità:} la capacità del \textit{software}\textsubscript{\textit{G}} di ripristinarsi dopo un errore.
\end{itemize}

\paragraph{Usabilità}
\begin{itemize}
    \item \textbf{Comprensibilità:} la facilità con cui gli utenti possono comprendere il \textit{software}\textsubscript{\textit{G}};
    \item \textbf{Apprendibilità:} il tempo e lo sforzo richiesti per apprendere a utilizzare il \textit{software}\textsubscript{\textit{G}};
    \item \textbf{Operabilità:} la facilità con cui gli utenti possono operare il \textit{software}\textsubscript{\textit{G}}.
\end{itemize}

\paragraph{Efficienza}
\begin{itemize}
    \item \textbf{Tempo di Risposta:} il tempo impiegato dal \textit{software}\textsubscript{\textit{G}} per rispondere alle richieste dell'utente;
    \item \textbf{Utilizzo delle Risorse:} l'efficienza nell'uso delle risorse del \textit{sistema}\textsubscript{\textit{G}}.
\end{itemize}

\paragraph{Manutenibilità}
\begin{itemize}
    \item \textbf{Analizzabilità:} la facilità con cui è possibile analizzare il codice per individuare errori o problemi;
    \item \textbf{Modificabilità:} la facilità con cui il \textit{software}\textsubscript{\textit{G}} può essere modificato;
    \item \textbf{Stabilità:} la capacità di evitare effetti indesiderati durante o dopo le modifiche.
\end{itemize}

\paragraph{Portabilità}
\begin{itemize}
    \item \textbf{Adattabilità:} la facilità con cui il \textit{software}\textsubscript{\textit{G}} può essere adattato a diversi ambienti;
    \item \textbf{Installabilità:} la facilità con cui il \textit{software}\textsubscript{\textit{G}} può essere installato;
    \item \textbf{Conformità:} il rispetto delle norme e degli \textit{standard}\textsubscript{\textit{G}} relativi alla portabilità.
\end{itemize}


\subsubsection{Suddivisione secondo Standard \href{https://www.math.unipd.it/~tullio/IS-1/2009/Approfondimenti/ISO_12207-1995.pdf}{ISO/IEC 12207:1995}}

\paragraph{Processi primari}
Sono gli elementi centrali e fondamentali delle \textit{attività}\textsubscript{\textit{G}} di sviluppo del \textit{software}\textsubscript{\textit{G}}. Questi \textit{processi}\textsubscript{\textit{G}} sono direttamente coinvolti nella produzione del prodotto \textit{software}\textsubscript{\textit{G}}, dalla sua concezione alla sua realizzazione effettiva. Essi comprendono le \textit{attività}\textsubscript{\textit{G}} principali di acquisizione dei requisiti, progettazione, implementazione, \textit{test}\textsubscript{\textit{G}} e manutenzione del \textit{software}\textsubscript{\textit{G}}.

\paragraph{Processi di supporto}
Sono i \textit{processi}\textsubscript{\textit{G}} che forniscono supporto ai \textit{processi}\textsubscript{\textit{G}} primari durante tutto il ciclo di vita del \textit{software}\textsubscript{\textit{G}}. Questi \textit{processi}\textsubscript{\textit{G}} comprendono \textit{attività}\textsubscript{\textit{G}} come la gestione della configurazione, la garanzia della qualità, la gestione dei documenti, la formazione, la pianificazione e il controllo dei progetti. Il loro obiettivo principale è garantire un ambiente efficiente e ben organizzato per il corretto svolgimento dei \textit{processi}\textsubscript{\textit{G}} primari.

\paragraph{Processi organizzativi}
Rappresentano i \textit{processi}\textsubscript{\textit{G}} che definiscono la struttura e l'ambiente organizzativo all'interno del quale avvengono le \textit{attività}\textsubscript{\textit{G}} di sviluppo del \textit{software}\textsubscript{\textit{G}}. 
