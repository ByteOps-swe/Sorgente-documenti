Nel corso dell'analisi e della valutazione della qualità dei processi e del software, adotteremo standard internazionali ben definiti per garantire una valutazione rigorosa e conforme agli standard globali. In particolare, l'utilizzo dello standard ISO/IEC 9126 fornirà una solida struttura per valutare la qualità del software, concentrandosi su attributi quali la funzionalità, l'affidabilità, l'usabilità, l'efficienza, la manutenibilità e la portabilità. Questo framework ci consentirà di misurare in modo accurato e completo la qualità del prodotto software. Parallelamente, la suddivisione dei processi in primari, di supporto e organizzativi sarà guidata dall'adozione dello standard ISO/IEC 12207:1995. Infine, l'adozione dello standard ISO/IEC 25010 ci fornirà un quadro completo per la definizione e la suddivisione delle metriche di qualità del software.
L'utilizzo congiunto di questi standard consentirà un approccio completo e strutturato alla valutazione della qualità dei processi e del software, assicurando un'elevata coerenza, affidabilità e conformità agli standard riconosciuti a livello internazionale.

\subsubsection{Caratteristiche del Sistema, Standard ISO/IEC 25010}

\paragraph{Funzionalità}
\begin{itemize}
    \item \textbf{Idoneità Funzionale:} La capacità del software di fornire funzionalità che soddisfano i requisiti specificati;
    \item \textbf{Accuratezza:} La precisione con cui il software esegue le sue funzioni;
    \item \textbf{Interoperabilità:} La capacità del software di interagire con altri sistemi.
\end{itemize}

\paragraph{Affidabilità}
\begin{itemize}
    \item \textbf{Maturità:} La capacità del software di evitare errori o malfunzionamenti;
    \item \textbf{Tolleranza agli Errori:} La capacità di mantenere un certo livello di prestazioni nonostante la presenza di errori;
    \item \textbf{Recuperabilità:} La capacità del software di ripristinarsi dopo un errore.
\end{itemize}

\paragraph{Usabilità}
\begin{itemize}
    \item \textbf{Comprensibilità:} La facilità con cui gli utenti possono comprendere il software;
    \item \textbf{Apprendibilità:} Il tempo e lo sforzo richiesti per apprendere a utilizzare il software;
    \item \textbf{Operabilità:} La facilità con cui gli utenti possono operare il software.
\end{itemize}

\paragraph{Efficienza}
\begin{itemize}
    \item \textbf{Tempo di Risposta:} Il tempo impiegato dal software per rispondere alle richieste dell'utente;
    \item \textbf{Utilizzo delle Risorse:} L'efficienza nell'uso delle risorse del sistema.
\end{itemize}

\paragraph{Manutenibilità}
\begin{itemize}
    \item \textbf{Analizzabilità:} La facilità con cui è possibile analizzare il codice per individuare errori o problemi;
    \item \textbf{Modificabilità:} La facilità con cui il software può essere modificato;
    \item \textbf{Stabilità:} La capacità di evitare effetti indesiderati durante o dopo le modifiche.
\end{itemize}

\paragraph{Portabilità}
\begin{itemize}
    \item \textbf{Adattabilità:} La facilità con cui il software può essere adattato a diversi ambienti;
    \item \textbf{Installabilità:} La facilità con cui il software può essere installato;
    \item \textbf{Conformità:} Il rispetto delle norme e degli standard relativi alla portabilità.
\end{itemize}


\subsubsection{Suddivisione secondo Standard ISO/IEC 12207:1995}

\paragraph{Processi primari}
Sono gli elementi centrali e fondamentali delle attività di sviluppo del software. Questi processi sono direttamente coinvolti nella produzione del prodotto software, dalla sua concezione alla sua realizzazione effettiva. Essi comprendono le attività principali di acquisizione dei requisiti, progettazione, implementazione, test e manutenzione del software.

\paragraph{Processi di supporto}
Sono i processi che forniscono supporto ai processi primari durante tutto il ciclo di vita del software. Questi processi comprendono attività come la gestione della configurazione, la garanzia della qualità, la gestione dei documenti, la formazione, la pianificazione e il controllo dei progetti. Il loro obiettivo principale è garantire un ambiente efficiente e ben organizzato per il corretto svolgimento dei processi primari.

\paragraph{Processi organizzativi}
Rappresentano i processi che definiscono la struttura e l'ambiente organizzativo all'interno del quale avvengono le attività di sviluppo del software. 
