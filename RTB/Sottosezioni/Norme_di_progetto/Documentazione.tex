\subsection{Documentazione}

\subsubsection{Introduzione}
La documentazione costituisce l'insieme di informazioni scritte che accompagnano un prodotto software, fornendo dettagli utili a sviluppatori, distributori e utenti.\\
Tra gli scopi della documentazione troviamo:
\begin{itemize}
    \item Comprensione del prodotto senza supporto umano, usandola come mezzo di comunicazione;
    \item Segnare il confine tra creatività e disciplina;
    \item Assicurare che i processi produttivi si svolgano con la qualità attesa.
\end{itemize}
L'obiettivo della sezione è:
\begin{itemize}
    \item  Definire delle procedure ripetibili che permettano di uniformare la documentazione prodotta dal gruppo ed il metodo di lavoro;
    \item  Raccogliere ed organizzare le norme che i membri del team devono seguire così da semplificare l'operazione di scrittura dei documenti.
\end{itemize}
Tali norme dovranno essere applicate da tutti i membri del team, ed i sorgenti \LaTeX\ che contengono tale documentazione verranno inseriti nel repository disponibile all'indirizzo:
\href{https://github.com/ByteOps-swe/Sorgente-documenti}{https://github.com/ByteOps-swe/Sorgente-documenti}

\begin{comment} \paragraph*{Primi approcci alla redazione di documenti e problematiche riscontrate}
Per la composizione iniziale dei documenti richiesti per la candidatura, è stato sperimentato un'approccio che impiegava gli strumenti di Google Drive. Tale metodologia consentiva ai redattori di redigere agevolmente i documenti senza la necessità di padroneggiare la sintassi LaTeX, con l'intenzione di trasporre successivamente il contenuto in LaTeX una volta che fosse stato validato dai verificatori. Tuttavia, questo approccio ha suscitato problematiche, tra cui:
\begin{itemize}
    \item Rischio di incoerenza tra il contenuto presente negli strumenti di Google Drive.
    \item Prolungato impiego di tempo per la riscrittura in LaTeX, dovuto alla necessità di un passaggio aggiuntivo.
\end{itemize}
Per tali ragioni si è presa la decisione di adottare un nuovo approccio.
\end{comment}
\paragraph{Documentation as Code}\label{sec:DocumentationAscode}
L'approccio che si intende adottare è quello di "Documentation as Code" (Documentazione come Codice) che consiste nel trattare la documentazione di un progetto software allo stesso modo in cui si tratta il codice sorgente. Questo approccio è incentrato sull'utilizzo di pratiche e strumenti tipici dello sviluppo software per creare, gestire e distribuire la documentazione.
Alcuni aspetti chiave di "Documentation as Code" includono:
\begin{itemize}
    \item \textbf{Versionamento}
    \item \textbf{Scrittura in Formato Testuale}
    \item \textbf{Automazione}
    \item \textbf{Collaborazione}
    \item \textbf{Integrazione Continua}
    \item \textbf{Distribuzione}
\end{itemize}
Questo approccio porta diversi vantaggi, tra cui una maggiore coerenza, una migliore tracciabilità delle modifiche e facilità di manutenzione. Inoltre, il concetto di "Documentation as Code" si allinea con la filosofia DevOps, dove la collaborazione e l'automazione sono valori chiave.\\

\subsubsection{Sorgente documenti}
Poiché i documenti vengono redatti in LaTeX, per favorire una migliore collaborazione tra diversi autori, si è scelto di creare un file sorgente per ciascuna sezione e sottosezione di ogni documento. Questo approccio consente a ciascun autore di lavorare sulle singole sezioni o sottosezioni, evitando di apportare modifiche a parti del documento non pertinenti al proprio ambito. Infine, si creerà un file principale che prenderà il nome del documento, nel quale verranno collegati tutti i file delle varie sezioni e sottosezioni mediante l'uso del comando \verb|\input{Sezione.tex}| o, nel caso delle sottosezioni, \verb|\input{Sottosezione.tex}|.
Nel caso dei Verbali, essendo documenti composti da sezioni molto brevi, non si utilizza questo approccio.

\subsubsection{Ciclo di vita dei documenti}
Il ciclo di vita dei documenti è una sequenza di stati e attività:
\begin{enumerate}
    \item \textbf{Stato: Necessità}
          \begin{enumerate}
              \item Nasce la necessità di una documentazione, per obbligo o per opportunità;
              \item Pianicazione della sua stesura;
              \item Suddivisione in sezioni;
              \item Durante le riunioni, si procede alla discussione e alla definizione collettiva di una traccia per il contenuto;
              \item Assegnazione delle sezioni ai redattori tramite task su ITS.
          \end{enumerate}
    \item \textbf{Stato: Redazione}
          \begin{enumerate}
              \item  Ogni tipo di documento viene creato secondo la struttura specificata nei prossimi paragrafi;
              \item Il team realizza il documento redigendone il contenuto rispettando le norme definite;
          \end{enumerate}
    \item \textbf{Stato: Verifica}
          \begin{enumerate}
              \item Quando completato il documento viene revisionato dai verificatori;
              \item Il documento viene compilato e il PDF generato viene inserito nella repository all'indirizzo: \href{https://github.com/ByteOps-swe/Documents} {https://github.com/ByteOps-swe/Documents} ;
          \end{enumerate}
    \item \textbf{Stato: Manutenzione}
          \begin{enumerate}
              \item Manutenzione: ogni documento sotto configuration management deve essere modificato in accordo alle norme di versionamento e di change management.
              \item La richiesta di modifica nasce da una nuova necessità sul contenuto del documento e riporta il documento allo stato omonimo.
          \end{enumerate}
\end{enumerate}

%\textbf{Procedimento in sintesi}\\
%   Per rispettare i principi  di "Documentation as Code" l'approccio attualmente adottato richiede la competenza e l'utilizzo di LaTeX per la stesura dei documenti e una repository ad accesso limitato Git chiamata "Sorgente documenti" dove versionarli. I redattori, seguendo la struttura dettagliata nei prossimi paragrafi, redigeranno il contenuto dei documenti utilizzando il linguaggio \LaTeX\ su un ramo dedicato nella repository "Sorgente documenti". \\
%  Al termine del loro lavoro, segnaleranno il completamento posizionando l'Issue relativa al compito di redazione nella colonna "Da revisionare" della \href{https://github.com/orgs/ByteOps-swe/projects/1/views/1}{DashBoard documentazione} di GitHub (unica e condivisa sia dalla repository pubblica "\href{https://github.com/ByteOps-swe/Documents}{Documents}" che dalla repository "Sorgente documenti") e generando una pull request . Questa colonna conterrà i compiti relativi ai documenti completati dai redattori, che necessitano ora della validazione da parte dei verificatori. \\

% I file sorgenti LaTeX saranno quindi gestiti e versionati nella repository con accesso limitato denominata "Sorgente documenti", strutturata in modo identico alla repository pubblica chiamata "\href{https://github.com/ByteOps-swe/Documents}{Documents}" ma con file \LaTeX\ al posto dei PDF. \\
%   I verificatori quindi troveranno il documento da validare in questa repository quando riceveranno una notifica mail generata dalla pull request.\\
%  Quando i verificatori avranno convalidato il contenuto, dovranno seguire le istruzioni al paragrafo ~\ref{sec:verificatori} per formalizzare la verifica.


\paragraph{I redattori}

Il redattore responsabile della redazione di un documento o di una sua sezione deve seguire lo stesso approccio richiesto per la codifica del software, adottando il workflow noto come "feature branch".\\
\vspace{0.1cm}
\textbf{Caso redazione nuovo documento, sezione o modifica}\\
Dalla repository  "Sorgente documenti" contenente i sorgenti \LaTeX, il redattore dovrà creare un nuovo branch Git in locale e passare ad esso mediante l'utilizzo del comando:
\begin{lstlisting}[style=code]
    git checkout main
    git checkout -b branch_identificativoTask 
\end{lstlisting}

Il nome del branch destinato alla redazione o modifica del documento o di una sua sezione deve essere descrittivo al fine di consentire un'identificazione immediata e precisa del documento o della sezione su cui si sta lavorando. Pertanto, relativamente alla redazione dei documenti, devonono essere adottatate le \textbf{convenzioni per la nomenclatura dei branch} esposte all'interno della sezione 3.4.3.2.

A questo punto, dopo aver creato il documento,la sezione o la modifca assegnata e avviato la stesura, affinché gli altri redattori possano continuare il lavoro, è necessario rendere il branch accessibile anche nella repository remota, seguendo i seguenti passaggi:
\begin{lstlisting}[style=code]
    git add .
    git commit -m "Descrizione del commit"
    git push origin branch_identificativoTask
        \end{lstlisting}



%\begin{itemize}
%   \item \textbf{Verbali interni}: \verb|branch_VerbaleIn_GG_MM_AAAA|
%  \item \textbf{Verbali esterni}: \verb|branch_VerbaleEx_GG_MM_AAAA|
% \item \textbf{Norme di progetto}: \verb|Norme-di-progetto|
%\end{itemize}

\textbf{Caso modifica documento in stato di redazione}\\
Qualora il redattore intenda continuare la stesura di un documento (o sezione) iniziato da un altro redattore,sono necessari i comandi:

\begin{lstlisting}[style=code]
        git pull
        git checkout branch_identificativoTask

        \end{lstlisting}

\textbf{Salvataggio e condivisone progressi di task non completate}\\
Alla fine di una sessione di modifiche di un file, nel caso si desideri rendere accessibile ai membri il lavoro non ancora completato e, pertanto, non pronto alla verifica, è necessario:
\begin{enumerate}
    \item Eseguire il push\textsubscript{\textit{G}}  delle modifiche fatte nel branch
          \begin{lstlisting}[style=code]
    git add .
    git commit -m "Descrizione del commit"
    git push origin branch_identificativoDocumento
        \end{lstlisting}

    \item Nel caso di problemi con il punto 1:
          \begin{lstlisting}[style=code]
   git pull origin branch_identificativoDocumento
        \end{lstlisting}
    \item Risolvi i conflitti e ripeti punto 1.
\end{enumerate}
\vspace{0.3cm}
\textbf{Completamento compito di redazione}\\
Al termine del loro lavoro, i redattori:
\begin{enumerate}
    \item Segnalano il completamento dell'attività a loro assegnata (essa sia la stesura completa di un documento o di una sua parte) posizionando l'issue relativa nella colonna "Da revisionare" della  \href{https://github.com/orgs/ByteOps-swe/projects/1/views/1}{DashBoard documentazione}.
    \item Attuano una \textbf{pull request:}
          \begin{enumerate}
              \item Aggiornare il registro delle modifiche inserendo i dati richiesti in  una nuova riga e incrementando la versione (Il verificatore è definito all'assegnazione della task, presente nella descrizione della Issue del'ITS).
              \item Eseguire i passaggi dettagliati nel caso "\textit{\textbf{Salvataggio e condivisone progressi di task non completate}}"
              \item Vai sul repository su GitHub >> Apri la sezione "pull request" >> Clicca "New pull request"
              \item Seleziona come branch di destinazione "main" e come branch sorgente il ramo utilizzato per la redazione del documento (o sezione) da validare
              \item Clicca "Create pull request"
              \item Dai un titolo e una breve descrizione alla pull request >> Seleziona i verificatori su "Reviewers" >> Clicca "Create pull request"
          \end{enumerate}
\end{enumerate}


Se i verificatori non convalidano il documento (o sezione),i redattori riceveranno feedback allegati alla pull request relativi ai problemi identificati.
\paragraph{I verificatori}
I compiti e le procedure dei verificatori sono dettagliate al paragrafo ~\ref{sec:verificatori}.
\paragraph{Il responsabile}
Nel processo di redazione dei documenti, il compito del responsabile consiste nel:
\begin{itemize}
    \item \textbf{Identificare i documenti o le sezioni da redigere. }
    \item \textbf{Stabilire le scadenze entro cui devono essere completati.}
    \item \textbf{Assegnare i redattori e i verificatori ai task.}
    \item \textbf{Approvazione:} Prima della conclusione del suo mandato, il responsabile si riserva il diritto di approvare il lavoro o richiedere eventuali ulteriori modifiche su tutti i documenti redatti e verificati nel periodo in cui ha esercitato la propria carica.
\end{itemize}

\paragraph{L'amministratore}
Nel processo di redazione dei documenti, il compito dell'amministratore consiste nel:
\begin{itemize}
    \item \textbf{Inserire nel ITS le attività specificate dal responsabile:} \begin{itemize}
              \item Redattori: assegnatari della issue;
              \item Verificatori: specificati nella descrizione della issue;
              \item Scadenza.
          \end{itemize}
\end{itemize}
\subsubsection{Struttura del documento}
I documenti ufficiali seguono un rigoroso e uniforme schema strutturale, il quale richiede un'osservanza scrupolosa.
\paragraph*{Prima pagina}
Nella prima pagina di ogni documento deve essere presente:
\begin{itemize}
    \item Nome e mail del gruppo
    \item Nome del documento
    \item Redattori
    \item Verificatori
    \item Destinatari
\end{itemize}
\paragraph*{Indice}
Tutti i documenti generici devono essere provvisti di un indice dove saranno elencate le varie sezioni e sottosezioni, con la possibilità di raggiungerle direttamente tramite click.
In caso di presenza di figure nel documento, sarà presente anche un indice relativo.
\paragraph*{Pié di pagina}
In ogni pié di pagina deve essere presente:
\begin{itemize}
    \item Nome del gruppo
    \item Nome del documento
    \item Numero di pagina
\end{itemize}
\paragraph*{Registro delle modifiche}\label{sec:RegistroModifiche}
Tutti i documenti devono essere provvisti di un registro delle modifiche in formato tabellare che contiene un riassunto
delle modifiche apportate al documento nel corso del tempo.
La tabella  deve essere inserita nella
sezione registro delle modifiche subito prima dell’indice del documento.
La tabella deve registrare le seguenti informazioni:
\begin{itemize}
    \item \textbf{Versione del file}.
    \item \textbf{Data di rilascio}.
    \item \textbf{Autore}.
    \item \textbf{Verificatore}.
    \item \textbf{Descrizione}: un riassunto delle modifiche apportate.
\end{itemize}
La convezione per il versionamento è presente al paragrafo : ~\ref{sec:versionamento} .
\paragraph{Verbali: struttura generale}\label{sec:Verbali}
Questo documento assume l'importante ruolo di costituire un registro ufficiale, atto a riportare in maniera accurata gli argomenti trattati, le decisioni adottate, le azioni da intraprendere e le figure coinvolte.\\
In particolare, è possibile distinguere tra verbali interni, destinati all'uso interno dell'organizzazione, e verbali esterni, che trovano applicazione quando vi sono terze parti coinvolte nelle discussioni o nelle decisioni documentate.\\
Nella prima pagina oltre alle informazioni comuni a ogni documento vengono specificate:
\begin{itemize}
    \item Data della riunione e tipologia (Interna,Esterna) nel nome del documento;
    \item Luogo: canale di comunicazione adottato;
    \item Ora di inizio e fine dell'incontro;
    \item Amministratore;
    \item Partecipanti della riunione (interni ed esterni);
    \item Il Responsabile (in basso a destra);
\end{itemize}
%Il registro delle modifiche non è incluso nei verbali in quanto non è necessario; una volta redatti, i verbali non subiscono ulteriori modifiche.
Corpo del documento:
\begin{itemize}
    \item  \textbf{Revisione del periodo precedente:} Si analizza lo stato delle attività e, relativamente all'approccio lavorativo, si valutano gli aspetti positivi e le difficoltà incontrate in modo tale da identificare azioni di miglioramento per ottimizzare i processi;
    \item  \textbf{Ordine del giorno:} Elenco delle tematiche discusse durante la riunione, accompagnate dai relativi esiti.\\
    Nel caso del verbale esterno, se sono presenti richieste di chiarimenti effettuate alle terze parti coinvolte, saranno incluse nella sottosezione:
    \begin{itemize}
        \item \textbf{Richieste e chiarimenti}.
    \end{itemize}
    \item  \textbf{Attività da svolgere:} Tabella dove viene specificato:
          \begin{itemize}
              \item Nome della task da svolgere;
              \item Id issue del ITS;
              \item Verificatore dell' attività.
          \end{itemize}
\end{itemize}
Ultima pagina:
\begin{itemize}
    \item Nel caso di verbale esterno, in ultima pagina, deve essere presente la firma delle terze parti coinvolte, il luogo e la data.
\end{itemize}

Il template dei verbali è disponibile al link:\\ \href{https://github.com/ByteOps-swe/Sorgente-documenti/tree/main/Documents/Verbali/Templates}{https://github.com/ByteOps-swe/Sorgente-documenti/tree/main/Documents/Verbali/Templates}

\subsubsection{Regole tipografiche}
\paragraph*{Documenti del progetto e nome dei File}\label{sec:NomeFile}
Il nome dei documenti generati deve essere omogeneo, con la prima lettera maiuscola ed il resto minuscolo e, ad eccezione dei verbali, deve contenere un riferimento alla versione del documento (notazione di versionamento:~\ref{sec:RegistroModifiche}).\\
Nello specifico devono seguire la seguente convenzione:
\begin{itemize}
    \item \textbf{Verbali:} Verbale\_AAAA-MM-DD ;
    \item \textbf{Norme di Progetto:} Norme\_di\_progetto\_vX.Y.Z ;
    \item \textbf{Analisi dei requisiti:} Analisi\_dei\_requisiti\_vX.Y.Z ;
    \item \textbf{Piano di progetto:} Piano\_di\_progetto\_vX.Y.Z ;
    \item \textbf{Glossario:} Glossario\_vX.Y.Z .
\end{itemize}

\paragraph*{Regole sintattiche:}
\begin{itemize}
    \item I titoli delle sezioni iniziano con la lettera maiuscola;
    \item Viene inserito ';' alla fine delle voci dell'elenco tranne l'ultima che termina con '.' . Ogni voce dell'elenco inizia con una lettera maiuscola;
    \item  Le date vengono scritte nel formato GG/MM/AAAA.
    \item  I numeri razionali si scrivono utilizzando la virgola come separatore tra parte intera e parte decimale.
\end{itemize}
\paragraph*{Stile del testo:}
\begin{itemize}
    \item \textbf{Grassetto}: Titoli delle sezioni, parole o frasi ritenute di rilievo.
    \item \textbf{Italico}: Riferimento a paragrafi o sezioni, nomi delle aziende e nomi propri dei membri del team.
\end{itemize}
\subsubsection{Abbreviazioni}
Nei documenti vi sono molte ripetizioni di termini per la quale si possono usare le seguenti abbreviazioni:\\
\vspace{0.2cm}
\begin{tabular}{|c|c|}
    \hline
    \textbf{Abbreviazione} & \textbf{Scrittura Estesa}            \\
    \hline
    RTB                    & Requirements and Technology Baseline \\
    PB                     & Product Baseline                     \\
    CA                     & Customer Acceptance                  \\
    ITS                    & Issue Tracking System                \\
    CI                     & Configuration Item                   \\
    SAL                    & Stato Avanzamento Lavori             \\   
    \hline
\end{tabular}

\subsubsection{Strumenti}
Gli strumenti utilizzati dalle attività di redazione dei documenti vogliono soddisfare il principio adottato di "Documentation as Code".
\begin{itemize}
    \item \textbf{GitHub}: versionamento, collaborazione, integrazione continua, automazione e distribuzione;
    \item \textbf{Latex}: scrittura in formato testuale, linguaggio per la stesura di documenti compilati;
    \item \textbf{Overleaf}: per la redazione dei documenti in \LaTeX\textsubscript{\textit{G}} collaborativa;
    \item \textbf{VSCode} : per la redazione con l'utilizzo del plugin \LaTeX\ Workshop.
\end{itemize}
