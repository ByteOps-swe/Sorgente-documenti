\subsection{Finalità del documento}
L'obiettivo fondamentale del seguente documento è quello di stabilire le linee guida e le \textit{best practice} che ciascun membro del gruppo deve seguire per garantire un approccio efficiente ed efficace nel processo di realizzazione del progetto didattico.\\
I \textit{processi}\textsubscript{\textit{G}} e le relative \textit{attività}\textsubscript{\textit{G}} contenute nel seguente documento sono state definite a partire dallo Standard ISO/IEC 12207:1995. \\
Il documento è strutturato secondo i \textit{processi}\textsubscript{\textit{G}} del ciclo di vita del \textit{software}\textsubscript{\textit{G}} e presenta una gerarchia in cui ogni processo si configura come una serie di \textit{attività}\textsubscript{\textit{G}}. Ciascuna \textit{attività}\textsubscript{\textit{G}} è composta da procedure dotate di obiettivi, scopi e strumenti ben definiti.\\
In aggiunta, il documento dettaglia le convenzioni relative all'utilizzo dei diversi strumenti adottati durante lo sviluppo del prodotto. \\
È importante sottolineare che questo documento è in continua evoluzione poiché le norme definite al suo interno vengono regolarmente riesaminate, aggiornate e ottimizzate seguendo un approccio incrementale.\\
