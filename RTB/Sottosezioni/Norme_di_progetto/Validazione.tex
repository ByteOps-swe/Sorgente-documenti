\subsection{Validazione}

\subsubsection{Introduzione}
La validazione in ingegneria del \textit{software}\textsubscript{\textit{G}} è l'\textit{attività}\textsubscript{\textit{G}} di conferma che il prodotto finale soddisfi effettivamente le aspettative del \textit{committente}\textsubscript{\textit{G}} e del \textit{proponente}\textsubscript{\textit{G}}. 

Questa fase è essenziale per assicurarsi che il \textit{software}\textsubscript{\textit{G}} sviluppato risponda alle esigenze e agli obiettivi iniziali del progetto. 
Un elemento chiave della validazione è l'incontro diretto con il \textit{committente}\textsubscript{\textit{G}} e il \textit{proponente}\textsubscript{\textit{G}} per ottenere un feedback diretto e garantire un chiaro allineamento tra ciò che è stato sviluppato e le aspettative degli utenti finali. 

Il prodotto finale deve quindi: 
\begin{itemize}
    \item 
        soddisfare tutti i requisiti concordati con il \textit{proponente}\textsubscript{\textit{G}}; 
    \item 
        poter essere esguito correttamente nel suo ambiente di utilizzo finale; 
\end{itemize}

L'obiettivo è giungere a un prodotto finale pronto per il rilascio, segnando così la conclusione del ciclo di vita del progetto didattico. 

\subsubsection{Procedura di validazione}
Il processo di validazione riceverà in ingresso i \textit{test}\textsubscript{\textit{G}} programmati per l'\textit{attività}\textsubscript{\textit{G}} di verifica, come definiti nelle corrispondenti sezioni delle Norme di Progetto. Successivamente, sarà eseguito il \textit{test}\textsubscript{\textit{G}} di accettazione, considerato il nucleo essenziale del processo, finalizzato a garantire la validazione del prodotto. 

I \textit{test}\textsubscript{\textit{G}} considerati dovranno valutare: 
\begin{itemize}
    \item  
        soddisfacimento dei casi d’uso;
    \item 
        soddisfacimento dei requisiti obbligatori;
    \item 
        soddisfacimento di altri requisiti concordati con il \textit{committente}\textsubscript{\textit{G}}; 
\end{itemize}

\subsubsection{Strumenti}
\begin{itemize}
    \item 
        strumenti utilizzati per la verifica; 
\end{itemize}
