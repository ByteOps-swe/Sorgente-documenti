\subsection{Validazione}

\subsubsection{Introduzione}
Nello standard \href{https://www.math.unipd.it/~tullio/IS-1/2009/Approfondimenti/ISO_12207-1995.pdf}{ISO/IEC 12207:1995}, il processo di validazione è definito come il processo di conferma attraverso dimostrazione oggettiva che i requisiti specificati per un'operazione o un sistema siano stati soddisfatti. In altre parole, la validazione è il processo finalizzato a garantire che il prodotto software risponda alle esigenze e alle aspettative degli utenti finali, dimostrandolo attraverso prove, test o altri metodi oggettivi. \\
Questo processo è fondamentale per assicurarsi che il software sviluppato corrisponda agli obiettivi iniziali del progetto. \\
Un aspetto cruciale della validazione è l'interazione diretta con il committente e il proponente al fine di ottenere un feedback immediato e garantire un chiaro allineamento tra ciò che è stato sviluppato e le aspettative degli utenti finali.

Il prodotto finale deve quindi: 
\begin{itemize}
    \item 
        soddisfare tutti i requisiti concordati con il \textit{proponente}\textsubscript{\textit{G}}; 
    \item 
        poter essere eseguito correttamente nel suo ambiente di utilizzo finale.
\end{itemize}

L'obiettivo è giungere a un prodotto finale pronto per il rilascio, segnando così la conclusione del ciclo di vita del progetto didattico. 

\subsubsection{Procedura di validazione}
Il processo di validazione riceverà in ingresso i test elencati nella sezione \ref{subsubsec:Testing} (\textit{Testing}). \\
In particolare, il test di accettazione è considerato il nucleo essenziale di questo processo, finalizzato a garantire la validazione del prodotto.

I \textit{test}\textsubscript{\textit{G}} considerati dovranno valutare: 
\begin{itemize}
    \item 
        Soddisfacimento dei casi d’uso;
    \item 
        Soddisfacimento dei requisiti obbligatori;
    \item 
        Soddisfacimento di altri requisiti concordati con il committente.
\end{itemize}

\subsubsection{Strumenti}
\begin{itemize}
    \item 
        Strumenti utilizzati per la \hyperlink{subsubsec:strumentiVerifica}{\textit{Verifica}}.
\end{itemize}
