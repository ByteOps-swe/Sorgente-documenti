\subsection{Validazione}

\subsubsection{Introduzione}
La validazione in ingegneria del software è l'attività di conferma che il prodotto finale soddisfi effettivamente le aspettative del committente e del proponente.

Questa fase è essenziale per assicurarsi che il software sviluppato risponda alle esigenze e agli obiettivi iniziali del progetto.
Un elemento chiave della validazione è l'incontro diretto con il committente e il proponente per ottenere un feedback diretto e garantire un chiaro allineamento tra ciò che è stato sviluppato e le aspettative degli utenti finali.

Il prodotto finale deve quindi:
\begin{itemize}
    \item 
        soddisfare tutti i requisiti concordati con il proponente;
    \item 
        poter essere esguito correttamente nel suo ambiente di utilizzo finale;
\end{itemize}

L'obiettivo è giungere a un prodotto finale pronto per il rilascio, segnando così la conclusione del ciclo di vita del progetto didattico.

\subsubsection{Procedura di validazione}
Il processo di validazione riceverà in ingresso i test programmati per l'attività di verifica, come definiti nelle corrispondenti sezioni delle Norme di Progetto. Successivamente, sarà eseguito il test di accettazione, considerato il nucleo essenziale del processo, finalizzato a garantire la validazione del prodotto.

I test considerati dovranno valutare:
\begin{itemize}
    \item 
        soddisfacimento dei casi d’uso;
    \item 
        soddisfacimento dei requisiti obbligatori;
    \item 
        soddisfacimento di altri requisiti concordati con il committente;
\end{itemize}

\subsubsection{Strumenti}
\begin{itemize}
    \item 
        strumenti utilizzati per la verifica;
\end{itemize}
