\subsection{Gestione dei Processi}
La Gestione dei Processi si occupa di stabilire, implementare e migliorare i \textit{processi}\textsubscript{\textit{G}} che guidano la realizzazione del \textit{software}\textsubscript{\textit{G}}, al fine di raggiungere gli obiettivi prefissati e soddisfare le esigenze degli \textit{stakeholder}\textsubscript{\textit{G}}.
\\
Le \textit{attività}\textsubscript{\textit{G}} di gestione di processo sono:
\begin{enumerate}
    \item \textbf{Definizione dei Processi:}
      \begin{itemize}
        \item Identificare e documentare i \textit{processi}\textsubscript{\textit{G}} chiave coinvolti nello sviluppo \textit{software}\textsubscript{\textit{G}};
        \item Stabilire linee guida e procedure per l'esecuzione di ciascun processo.
      \end{itemize}
  
    \item \textbf{Pianificazione e Monitoraggio:}
      \begin{itemize}
        \item Elaborare piani dettagliati per l'esecuzione dei \textit{processi}\textsubscript{\textit{G}};
        \item Monitorare costantemente l'avanzamento, l'efficacia e la conformità ai requisiti pianificati;
        \item Stimare i tempi, le risorse ed i costi
      \end{itemize}
  
    \item \textbf{Valutazione e Miglioramento Continuo:}
      \begin{itemize}
        \item Condurre valutazioni periodiche dei \textit{processi}\textsubscript{\textit{G}} per identificare aree di miglioramento;
        \item Implementare azioni correttive e preventive per ottimizzare i \textit{processi}\textsubscript{\textit{G}}.
      \end{itemize}
  
    \item \textbf{Formazione e Competenze:}
      \begin{itemize}
        \item Assicurare che il personale coinvolto nei \textit{processi}\textsubscript{\textit{G}} sia adeguatamente formato;
        \item Mantenere e sviluppare le competenze necessarie per l'efficace gestione dei \textit{processi}\textsubscript{\textit{G}}.
      \end{itemize}
  
    \item \textbf{Gestione dei Rischi:}
      \begin{itemize}
        \item Identificare e valutare i rischi associati ai \textit{processi}\textsubscript{\textit{G}};
        \item Definire strategie per mitigare o gestire i rischi identificati.
      \end{itemize}
  \end{enumerate}
  
\subsubsection{Pianificazione}

L'\textit{attività}\textsubscript{\textit{G}} di pianificazione nello sviluppo \textit{software}\textsubscript{\textit{G}}, nell'ambito della gestione dei \textit{processi}\textsubscript{\textit{G}}, è un processo fondamentale finalizzato a definire un piano organizzato e coerente per il corretto svolgimento delle \textit{attività}\textsubscript{\textit{G}} di sviluppo del \textit{software}\textsubscript{\textit{G}}. Tale processo è disciplinato e guidato dal responsabile, il quale ha il compito di predisporre le \textit{attività}\textsubscript{\textit{G}} relative alla pianificazione.

\vspace{0,1cm}

In dettaglio, il responsabile verifica la fattibilità del piano organizzato, garantendo che sia eseguibile in maniera corretta e efficiente da parte dei membri del team. I piani associati all'esecuzione del processo devono includere descrizioni dettagliate delle \textit{attività}\textsubscript{\textit{G}} e delle risorse necessarie, comprese le tempistiche, le tecnologie impiegate, le infrastrutture coinvolte e il personale assegnato.

\vspace{0,1cm}

L'obiettivo primario della pianificazione è assicurare che ciascun membro del team assuma ogni ruolo almeno una volta durante lo svolgimento del progetto, promuovendo così una distribuzione equa delle responsabilità e un arricchimento delle competenze all'interno del team.

\vspace{0,1cm}

La pianificazione, stilata dal responsabile, è integrata nel documento del \textbf{Piano di Progetto}. Questo documento fornisce una descrizione completa delle \textit{attività}\textsubscript{\textit{G}} e dei compiti necessari per raggiungere gli obiettivi prefissati in ogni periodo del progetto.
\paragraph{Assegnazione dei ruoli}

Durante l'intero periodo del progetto, i membri del gruppo assumeranno sei ruoli distinti, ovvero assumeranno le responsabilità e svolgeranno le mansioni tipiche dei professionisti nel campo dello sviluppo \textit{software}\textsubscript{\textit{G}}.
\\I ruoli a disposizione sono:
\paragraph{Responsabile}\label{responsabile} Figura fondamentale che coordina il gruppo, fungendo da punto di riferimento per il \textit{committente}\textsubscript{\textit{G}} e il \textit{fornitore}\textsubscript{\textit{G}} e svolgendo il ruolo di mediatore tra le due parti.\\
In particolare si occupa di:
\begin{itemize}
    \item Gestire le relazioni con l'esterno;
    \item Pianificare le \textit{attività}\textsubscript{\textit{G}}: quali svolgere, data di inizio e fine, assegnazione delle priorità\dots
    \item Valutare i rischi delle scelte da effettuare;
    \item Controllare i progressi del progetto;
    \item Gestire le risorse umane;
    \item Approvazione della documentazione;
\end{itemize}
\paragraph{Amministratore}\label{amministratore}Questa figura professionale è incaricata del controllo e dell'amministrazione dell'ambiente di lavoro utilizzato dal gruppo ed è anche il punto di riferimento per quanto concerne le norme di progetto. Le sue mansioni principali sono:
\begin{itemize}
    \item Affrontare e risolvere le problematiche associate alla gestione dei \textit{processi}\textsubscript{\textit{G}};
    \item Gestire versionamento della documentazione;
    \item Gestire la configurazione del prodotto;
    \item Redigere ed attuare le norme e le procedure per la gestione della qualità;
    \item Amministrare le infrastrutture e i servizi per i \textit{processi}\textsubscript{\textit{G}} di supporto;
\end{itemize}
\paragraph{Analista}\label{analista}Figura professionale con competenze avanzate riguardo l'\textit{attività}\textsubscript{\textit{G}} di \textit{analisi dei requisiti}\textsubscript{\textit{G}} ed il dominio applicativo del problema. Il suo ruolo cruciale è quello di identificare, documentare e comprendere a fondo le esigenze e le specifiche del progetto, traducendole in requisiti chiari e dettagliati. Si occupa di:
\begin{itemize}
    \item Analizzare il contesto di riferimento, definire il problema in esame e stabilire gli obiettivi da raggiungere;
    \item Comprendere il problema e definire la complessità e i requisiti;
    \item Redigere il documento \textit{Analisi dei Requisiti};
    \item Studiare i bisogni espliciti ed impliciti;
\end{itemize}
\paragraph{Progettista}\label{progettista}Il \textit{Progettista} è la figura di riferimento per quanto riguarda le scelte progettuali partendo dal lavoro dell'analista. Spetta al progettista assumere decisioni di natura tecnica e tecnologica, oltre a supervisionare il processo di sviluppo. Tuttavia, non è responsabile della manutenzione del prodotto. In particolare si occupa di:
\begin{itemize}
    \item Progettare l'\textit{architettura}\textsubscript{\textit{G}} del prodotto secondo specifiche tecniche dettagliate;
    \item Prendere decisioni per sviluppare soluzioni che soddisfino i criteri di affidabilità, efficienza, sostenibilità e conformità ai Requisiti;
    \item Redige la \textit{Specifica Architetturale} e la parte pragmatica del \textit{Piano di Qualifica};
\end{itemize}
\paragraph{Programmatore}\label{programmatore}Il programmatore è la figura professionale responsabile della codifica del \textit{software}\textsubscript{\textit{G}}. Il suo ruolo principale consiste nell'implementare codice informatico basato sulle specifiche fornite dall'analista e sull'\textit{architettura}\textsubscript{\textit{G}} fornita dal progettista. In particolare, il \textit{Programmatore}:
\begin{itemize}
    \item Scrivere codice informatico mantenibile in conformità con le \textit{Specifiche di Progetto};
    \item Codificare le varie parti dell'\textit{architettura}\textsubscript{\textit{G}} elaborata seguendo l'\textit{architettura}\textsubscript{\textit{G}} ideata dal Progettista;
    \item Realizza gli strumenti per verificare e validare il codice;
    \item Redigere il \textit{Manuale Utente};
\end{itemize}
\paragraph{Verificatore}\label{verificatore} La principale responsabilità del \textit{Verificatore} consiste nell'ispezionare il lavoro svolto da altri membri del team per assicurare la qualità e la conformità alle attese prefissate. Stabilisce se il lavoro è stato svolto correttamente sulla base delle proprie competenze tecniche, esperienza e conoscenza delle norme. Si occupa di:
\begin{itemize}
    \item Verificare che i prodotti siano conformi alle \textit{Norme di Progetto};
    \item Verificare la conformità dei prodotti ai requisiti funzionali e di qualità;
    \item Ricercare ed in caso segnalare eventuali errori;
    \item Redigere la sezione retrospettiva del \textit{Piano di Qualifica}, descrivendo le verifiche e le prove effettuate durante il processo di sviluppo del prodotto;
\end{itemize}
\paragraph{Ticketing} ~\ref{sec:ticketing}
GitHub è adottato come \textit{sistema}\textsubscript{\textit{G}} di tracciamento degli \textit{issue}\textsubscript{\textit{G}} (ITS), garantendo così una gestione agevole e trasparente dei compiti da svolgere. L'amministratore ha la facoltà di creare e assegnare specifici compiti identificati dal responsabile, assicurando chiarezza sulle responsabilità di ciascun individuo e stabilendo tempi definiti. Inoltre, ogni membro del gruppo può monitorare i progressi compiuti nel periodo corrente, consultando lo stato di avanzamento dei vari task attraverso le Dashboard:
\begin{itemize}
    \item \href{https://github.com/orgs/ByteOps-swe/projects/1}{DashBoard}: Per una chiara visione dello stato dei task;
    \item \href{https://github.com/orgs/ByteOps-swe/projects/3}{RoadMap}: Per visione delle scadenze dei task.
\end{itemize}
La procedura da seguire in caso di necessità di svolgere un compito è la seguente:
\begin{enumerate}
    \item L' amministratore crea e assegna il task su GitHub, seguendo la convenzione descritta in seguito;
    \item L'incaricato apre una \textit{branch}\textsubscript{\textit{G}} su GitHub seguendo la denominazione suggerita;
    \item  All'inizio del lavoro di produzione il task viene marcato in corso sulla DashBoard GitHub;
    \item Finito il lavoro di produzione viene aperta la \textit{pull request}\textsubscript{\textit{G}} su GitHub assegnando il verificatore.
    \item Il task viene marcato nella colonna "Da revisionare" sulla DashBoard GitHub.
    \item Il verificatore si accerta e agisce secondo quanto esposto al punto ~\ref{sec:verifica} nel caso di appartenenza;
    \item Il task viene marcato nella colonna "Done" della DashBoard GitHub.
\end{enumerate}
I task sono creati dall'amministratore e hanno i seguenti attributi:
\begin{itemize}
    \item \textbf{Titolo}: un titolo coinciso e descrittivo;
    \item \textbf{Descrizione}:
    \begin{itemize}
        \item Descrizione testuale di "to-do";
        \item Come come ultima riga il Verificatore della task. (Verificatore: Mario Rossi.)
    \end{itemize} 
    \item \textbf{Assegnatario}: incaricato svolgimento della task;
    \item \textbf{Scadenza}: Data entro la quale la task deve essere completata;
    \item \textbf{labels}: Tag per identificare la categoria della task. (ex. Verbale, Documents, Develop, Bug).\\ Inoltre per associare ad ogni Issue un CI vengono utilizzati i seguenti \textit{label}\textsubscript{\textit{G}}:
    \begin{itemize}
        \item \textbf{NdP:} Norme di progetto;
        \item \textbf{PdQ:} Piano di Qualifica;
        \item \textbf{PdP:} Piano di Progetto;
        \item \textbf{AdR:} Analisi dei Requisiti;
        \item \textbf{Poc:} Proof of concept;
        \item \textbf{Gls:} Glossario.
    \end{itemize}
    \item \textbf{Milestone}: Milestone associata alla task.
\end{itemize}

\subsubsection{Coordinamento}
Il coordinamento rappresenta l'\textit{attività}\textsubscript{\textit{G}} che sovraintende la gestione della comunicazione e la pianificazione degli incontri tra le diverse parti coinvolte in un progetto di ingegneria del \textit{software}\textsubscript{\textit{G}}. Questo comprende sia la gestione della comunicazione interna tra i membri del team del progetto, sia la comunicazione esterna con il \textit{proponente}\textsubscript{\textit{G}} e i committenti. Il coordinamento risulta essere cruciale per assicurare che il progetto proceda in modo efficiente e che tutte le parti coinvolte siano informate e partecipino attivamente in ogni fase del progetto.
\paragraph{Comunicazione}
Il gruppo \textit{ByteOps} mantiene comunicazioni attive, sia interne che esterne al team, le quali possono essere sincrone o asincrone a seconda delle necessità.
\paragraph{Comunicazioni sincrone}
\paragraph*{Comunicazioni sincrone interne}
Per le comunicazioni sincrone interne, il gruppo ByteOps, ha scelto di adottare \textit{Discord}\textsubscript{\textit{G}} in quanto permette di comunicare tramite chiamate vocali, videochiamate, messaggi di testo, media e file in chat private o come membri di un \textit{"server Discord"}.
\paragraph*{Comunicazioni sincrone esterne}
Per le comunicazioni sincrone esterne,in accordo con l'azienda \textit{proponente}\textsubscript{\textit{G}} si è deciso di utilizzare \textit{google meet}\textsubscript{\textit{G}}.
\paragraph{Comunicazioni asincrone}
\paragraph*{Comunicazioni asincrone interne}
Le comunicazioni asincrone interne al nostro team avvengono tramite l'applicazione \textit{Telegram}\textsubscript{\textit{G}} all'interno di un Gruppo dedicato, il quale consente una comunicazione rapida tra tutti i membri del gruppo. Inoltre, tramite la stessa \textit{piattaforma}\textsubscript{\textit{G}}, è possibile avere conversazioni dirette e private (\textit{chat}) tra due membri.
\paragraph*{Comunicazioni asincrone esterne}
Per le comunicazioni asincrone esterne sono stati adottati due canali differenti:
\begin{itemize}
    \item \textbf{E-mail} La posta elettronica è stata utilizzata per comunicare con il \textit{committente}\textsubscript{\textit{G}} e, nelle prime fasi del progetto, per coordinare gli incontri con l'azienda \textit{proponente}\textsubscript{\textit{G}}. Questa forma di comunicazione offre la flessibilità necessaria per coordinare incontri sincroni e revisioni in modo efficace, consentendo a entrambe le parti di pianificare le interazioni in base alla propria disponibilità.
    \item \textbf{Element} E' un client di messaggistica istantanea libero ed open source che supporta conversazioni strutturate e crittografate. La sua flessibilità nell'adattarsi a varie esigenze di comunicazione, inclusa la possibilità di condividere file, immagini e altri documenti, ha reso la \textit{piattaforma}\textsubscript{\textit{G}} un'opzione versatile e completa per soddisfare le esigenze specifiche del nostro contesto lavorativo.
\end{itemize}

\paragraph*{Riunioni}
In ogni riunione, qualunque ne sia la tipologia, verrà designato un segretario con l'incarico di prendere appunti durante il meeting e successivamente redigere un verbale completo, documentando gli argomenti trattati e i risultati emersi durante le discussioni

\paragraph{Riunioni interne}
Il nostro team ha adottato un approccio settimanale attraverso incontri simili a "stand-up meetings" al fine di facilitare una comunicazione costante e coordinare il progresso delle \textit{attività}\textsubscript{\textit{G}} interne.\\

Le riunioni sono fissate ogni venerdi allore ore:
\begin{itemize}
    \item \textbf{10:00}, nel caso in cui non sia prevista un \textit{SAL}\textsubscript{\textit{G}} con l'azienda \textit{proponente}\textsubscript{\textit{G}} nello stesso giorno;
    \item \textbf{11:30}, nel caso in cui sia prevista un \textit{SAL}\textsubscript{\textit{G}} con l'azienda \textit{proponente}\textsubscript{\textit{G}} nello stesso giorno, veranno quindi anche verbalizzate la nuove \textit{attività}\textsubscript{\textit{G}} sorte dall'incontro con la \textit{proponente}\textsubscript{\textit{G}}.
\end{itemize} 
In caso risultasse necessario, ogni membro del gruppo può decidere di richiedere una riunione supplementare. In questo caso la data e l'orario verranno stabilite
tramite il canale Telegram dedicato creando un sondaggio.

Questi incontri rivestono un ruolo cruciale nel monitorare il progresso delle mansioni assegnate, valutare i risultati conseguiti e affrontare le sfide che possono sorgere. Durante questi momenti, i membri del team condividono gli aggiornamenti sulle proprie \textit{attività}\textsubscript{\textit{G}}, identificano le problematiche riscontrate e discutono di opportunità di miglioramento nei \textit{processi}\textsubscript{\textit{G}} di lavoro. Questo ambiente aperto e collaborativo favorisce l'innovazione e la condivisione di nuove prospettive.\\

Per agevolare la comunicazione sincrona, il canale utilizzato per questi incontri è \textit{Discord}\textsubscript{\textit{G}}, ritenuto particolarmente efficace per tali scopi.

\vspace{0,1cm}

Sarà compito del responsabile:
\begin{itemize}
    \item Stabilire preventivamente i punti dell'ordine, considerando l'aggiunta di nuovi durante il corso della riunione;
    \item Guidare la discussione e raccogliere i pareri dei membri in maniera ordinata;
    \item Nominare un segretario per la riunione;
    \item Pianificare e proporre le nuove attiva da svolgere.
\end{itemize}
\paragraph*{Verbali interni}
Lo svolgimento di una riunione ha come obiettivo la risoluzione dei punti stilati nell'ordine del giorno, la pianificazione delle nuove \textit{attività}\textsubscript{\textit{G}} e la retrospettiva del periodo precedente.
Al termine di ogni riunione viene creato un \textit{issue}\textsubscript{\textit{G}} sull'ITS di GitHub che prevede la stesura del verbale interni. Sarà cura del segretario redigere il verbale includendo
tutte le informazioni di rilievo sorte durante la riunione.
Le linee guida per la redazione dei verbali interni sono reperibili alla sezione  ~\ref{sec:Verbali}

\paragraph{Riunioni esterne}
Durante il corso del progetto, si renderà necessaria l'organizzazione di vari incontri con i \textit{Committenti} e/o il \textit{Proponente} allo scopo di valutare lo stato di avanzamento del prodotto.\\
La convocazione di tali incontri è di competenza del \textit{Responsabile}, il quale è incaricato di pianificarli e di agevolarne lo svolgimento in maniera efficiente. Sarà compito del Responsabile anche l'esposizione dei punti di discussione al \textit{proponente}\textsubscript{\textit{G}}/\textit{committente}\textsubscript{\textit{G}}, lasciando la parola ai membri del gruppo interessati quando necessario.
\\
Questo approccio assicura una comunicazione efficace tra il nostro team e i rappresentanti aziendali, garantendo una gestione ottimale del tempo e una registrazione accurata delle informazioni rilevanti emerse durante gli incontri.\\
I membri del gruppo si impegnano a garantire la propria presenza in modo costante alle riunioni, facendo il possibile per riorganizzare eventuali altri impegni al fine di partecipare. Nel caso in cui gli obblighi inderogabili di un membro del gruppo rendessero impossibile la partecipazione, il responsabile assicurerà di informare tempestivamente il \textit{proponente}\textsubscript{\textit{G}} o i committenti, richiedendo la possibilità di rinviare la riunione ad una data successiva.

\paragraph*{Riunioni con la proponente}
In accordo con l'azienda propronente si è deciso di attuare con cadenza bisettimanale gli incontri di \textit{stato avanzamento lavori}\textsubscript{\textit{G}} (\textit{SAL}\textsubscript{\textit{G}}) tramite \textit{google meet}\textsubscript{\textit{G}}.\\
Durante tali incontri, si affrontano diversi aspetti, tra cui:
\begin{itemize}
    \item Discussione delle \textit{attività}\textsubscript{\textit{G}} svolte nel periodo precedente, valutando l'aderenza alle concordanze stabilite e identificando eventuali problematiche riscontrate.
    \item Pianificazione delle \textit{attività}\textsubscript{\textit{G}} per il prossimo periodo, definendo gli obiettivi e le azioni necessarie per il loro raggiungimento;
    \item Chiarezza e risoluzione di eventuali dubbi emersi nel corso delle \textit{attività}\textsubscript{\textit{G}} svolte.
\end{itemize}

\paragraph*{Verbali esterni}
Come per il caso delle riunioni interne verrà redatto un Verbale con le stesse
modalità descritte in precedenza.
Le linee guida per la redazione dei verbali esterni sono reperibili alla sezione  ~\ref{sec:Verbali}

\vspace{0.1cm}

La firma di approvazione dell'esterno è necessaria per ogni verbale esterno in ultima pagina.

\subsubsection{Metriche}
\begin{table}[H]
  \centering
  \begin{tabular}{|C{3cm}|C{3cm}|C{4cm}|}
  \hline
  Metrica & Nome & Riferimento \\
  \hline \hline
  M4BV & Budeget Variance (BV) &  \hyperlink{item:M4BV}{\textcolor{linkcolor}{M4BV}} \\
  M6SV & Schedule Variance (SV) &  \hyperlink{item:M6SV}{\textcolor{linkcolor}{M6SV}} \\
  M12VR & Variazione dei Requisiti (VR) &  \hyperlink{item:M12VR}{\textcolor{linkcolor}{M12VR}} \\
  M21IF & Implementazione delle Funzionalità (IF) & \hyperlink{item:M21IF}{\textcolor{linkcolor}{M21IF}} \\ 
  \hline
  \end{tabular}
  \caption{Metriche relative alla gestione dei processi}
\end{table}