\subsection{Fornitura}
\subsubsection{Scopo}
Conformemente allo standard ISO/IEC 12207:1995, il processo di fornitura definisce un insieme strutturato di attività, metodi, pratiche e procedure mirate a garantire la fornitura del prodotto software richiesto dal committente. 
In dettaglio, il suo scopo consiste nel monitorare e coordinare le attività eseguite dal gruppo nel corso dell'intero processo di realizzazione del progetto.\\
Tale processo sarà attivato una volta completata la redazione integrale del documento \textit{Valutazione Capitolati}, cioè dopo aver correttamente identificato le specifiche richieste dalla proponente.

In seguito, il fornitore dovrà instaurare un contratto con l'azienda proponente, nel quale si concorderanno e accetteranno i requisiti specificati e i tempi di consegna del prodotto finale. Una volta concluso l'accordo con la proponente, sarà possibile iniziare il processo di redazione del documento
\textit{Piano di Progetto}, il quale delineerà le attività, le risorse e i costi indispensabili per la realizzazione del prodotto.

\subsubsection{Fasi della fornitura}
Il processo di fornitura, come descritto dallo standard ISO/IEC 12207:1995, è suddiviso nelle seguenti fasi:
\begin{enumerate}
    \item \textbf{Avvio}
    \item \textbf{Preparazione della risposta alle richieste}
    \item \textbf{Contrattazione}
    \item \textbf{Pianificazione}
    \item \textbf{Esecuzione e controllo}
    \item \textbf{Revisione e valutazione}
    \item \textbf{Consegna e completamento}
\end{enumerate}

\subsubsection{Aspettative}
Il gruppo \textit{Byte Ops} instaurerà e si impegnerà a mantere una comunicazione costante con la proponente \textit{Sync Lab} in modo da ottenere un riscontro sul lavoro svolto fino a quel momento e per verificare che i requisiti individuati siano conformi a quanto stabilito nel capitolato e nei colloqui con la proponente stessa.\\

\subsubsection{Comunicazioni con la proponente}
La proponente \textit{Sync Lab} mette a disposizione un indirizzo email per contattare il gruppo \textit{Byte Ops} per le comunicazioni e le richieste formali, ed un canale \textit{Element} per comunicazioni informali, quali richiedere eventuali chiarimenti o richieste di aiuto in determinate sezioni dello svolgimento.\\
Gli incontri SAL(\textit{Stato Avanzamento Lavori}) con la proponente sono fissati a cadenza bisettimanale, in modo da poter discutere con essa lo stato di avanzamento e le difficoltà riscontrate per quanto concerne gli obiettivi proposti nel periodo che intercorre tra un SAL e il successivo, nonchè le attività future da svolgere.
Per ogni incontro effettuato con la proponente, verrà redatto un \textbf{Verbale esterno} che riporterà data e ora dell'incontro, i partecipanti, gli argomenti trattati e le attività concordate per il prossimo incontro.
I verbali esterni sono archiviati al percorso \textit{Nome\_periodo/Verbali/Esterni} nella \href{https://github.com/ByteOps-swe/Documents}{repository} del gruppo \textit{Byte Ops}.

\subsubsection {Documentazione fornita}
Viene elencata di seguito la documentazione che il gruppo \textit{Byte Ops} consegnerà ai commitenti \textit{Prof.} Tullio Vardanega e \textit{Prof.} Riccardo Cardin e all'azienda proponente \textit{Sync Lab}.

\paragraph{Valutazione dei capitolati}
Il documento \textit{Valutazione Capitolati} contiene l'analisi dei capitolati proposti, con l'obiettivo di individuare il capitolato più adatto alle esigenze del gruppo \textit{Byte Ops}.

Il documento contiene le seguenti sezioni:
\begin{itemize}
    \item \textbf{Informazioni generali}: contiene informazioni generali sul capitolato, come il nome del progetto e la proponente.
    \item \textbf{Obiettivo}: contiene una sintesi del prodotto da sviluppare seguendo quanto richiesto dal capitolato.
    \item \textbf{Tecnologie suggerite}: contiene le tecnologie suggerite dal proponente per lo sviluppo del prodotto.
    \item \textbf{Considerazioni}: contiene le considerazioni del gruppo \textit{Byte Ops} riguardo il capitolato, come i pro e i contro, le criticità e le motivazioni che hanno portato alla scelta o meno del capitolato.
\end{itemize}

\paragraph{Analisi dei requisiti}
L'analisi dei requisiti è un documento che contiene la descrizione in dettaglio dei requisiti, dei casi d'uso e definisce in modo esaustivo le funzionalità che il prodotto offirà.\\
Tale documento ha come scopo quello di eliminare ogni ambiguità che potrebbe sorgere durante la lettura del capitolato e di fornire una base di partenza per la progettazione del prodotto.

Il documento contiene le seguenti sezioni:
%---------------------REVISIONATO FINO A QUA --------------------
\begin{itemize}
    \item \textbf{Introduzione}: contiene una breve descrizione del prodotto e delle sue funzionalità.
    \item \textbf{Casi d'uso}: identifica tutti i possibili scenari di utilizzo da parte dell'utente del prodotto. Ogni caso d'uso è accompagnato da una descrizione che ne descrive il funzionamento.
    \item \textbf{Requisiti}: insieme di tutte le richieste e vincoli definiti dalla proponente, o estratti da discussioni con il gruppo, per realizzare il prodotto software commissionato. Ogni requisito è accompagnato da una descrizione e da un relativo caso d'uso.
\end{itemize}

\paragraph{Piano di progetto}
Il \textit{Piano di Progetto} è un documento stilato ed aggiornato continuamente che tratta i seguenti temi:
\begin{itemize}
    \item \textbf{Analisi dei rischi}: contiene l'analisi dei rischi che il gruppo preventiva di incontrare durante lo svolgimento del progetto. Ad ogni rischio è associata una descrizione, una procedura di identificazione, la probabilità di occorrenza, l'indice di gravità un relativo piano di mitigazione del rischio. Tale documento permetterà di attuare le strategie di mitigazione dei rischi in modo tempestivo.
    \item \textbf{Pianificazione e metodo utilizzato}: contiene la pianificazione delle attività, la suddivisione dei ruoli e l'identificazione delle attività da svolgere per ogni ruolo. In aggiunta, fornisce una dettagliata esposizione del modello di sviluppo adottato, corredato dalle motivazioni sottostanti che hanno orientato la decisione verso tale approccio.
    \item \textbf{Preventivo e consuntivo di periodo}: contiene il preventivo  delle ore e dei costi associati a ciascuna attività, suddivisi per ogni periodo definito. Al termine di ciascun periodo, viene redatto il consuntivo, il quale include un resoconto delle ore e dei costi effettivamente sostenuti per ogni singola attività.
\end{itemize}

\paragraph{Piano di qualifica}
Il \textit{Piano di Qualifica} è un documento essenziale che illustra le strategie di verifica e validazione per garantire che il prodotto soddisfi le aspettative del committente e della proponente. Esso identifica le metodologie di verifica, i criteri di validazione e le risorse coinvolte, assicurando un processo di controllo qualità efficiente.
Il documento contiene le seguenti sezioni:
\begin{itemize}
    \item \textbf{Qualità di processo}: contiene le strategie di verifica e validazione per garantire che i processi di sviluppo del prodotto soddisfi gli obiettivi di qualità.
    \item \textbf{Qualità di prodotto}: contiene le strategie di verifica e validazione per garantire che il prodotto soddisfi gli obiettivi di qualità.
    \item \textbf{Specifica dei test}: contiene la descrizione dei test per garantire che il prodotto soddisfi i requisiti. 
\end{itemize}

\paragraph{Glossario}
Il \textit{Glossario} è un documento essenziale finalizzato alla chiarificazione dei termini tecnici e degli acronimi impiegati all'interno dei documenti formali. La sua funzione principale è fornire agli utenti e ai lettori un riferimento chiaro e uniforme, garantendo così una interpretazione coerente dei concetti specifici presenti nei diversi documenti. Questo strumento si propone inoltre di mitigare eventuali ambiguità e prevenire fraintendimenti, promuovendo una comprensione univoca dei termini tecnici utilizzati nell'ambito della documentazione formale.

\paragraph{Lettera di presentazione}
La \textit{Lettera di presentazione} è un documento che accompagna la presentazione ufficiale della documentazione formale e del prodotto software durante le fasi di revisione di progetto. Questo documento dettaglia l'elenco della documentazione redatta, la quale sarà consegnata ai committenti, il \textit{Prof.} Tullio Vardanega e il \textit{Prof.} Riccardo Cardin, nonché all'azienda proponente, \textit{Sync Lab}.

\subsubsection{Strumenti}
Gli strumenti utilizzati dal gruppo per la gestione del processo di fornitura sono:
\paragraph{Google Meet}
Servizio di videoconferenza utilizzato per gli incontri con la proponente.
\paragraph{Google Calendar}
Servizio di calendario utilizzato per la pianificazione degli incontri con la proponente.
\paragraph{Element}
Servizio di messaggistica istantanea utilizzato per le comunicazioni con la proponente in caso di necessità.
\paragraph{Google Slides}
Servizio cloud utilizzato per la creazione delle presentazioni da mostrare durante i diari di bordo.
\paragraph{Excel}
Servizio cloud utilizzato per la creazione dei grafici per la creazione dei grafici di preventivo e consuntivo di periodo presenti nel documento \textit{Piano di Progetto}.