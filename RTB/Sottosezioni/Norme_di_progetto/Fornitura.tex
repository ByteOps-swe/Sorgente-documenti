\subsection{Fornitura}
\subsubsection{Introduzione}
Conformemente allo \textit{standard}\textsubscript{\textit{G}} \href{https://www.math.unipd.it/~tullio/IS-1/2009/Approfondimenti/ISO_12207-1995.pdf}{ISO/IEC 12207:1995}, il processo di fornitura definisce un insieme strutturato di \textit{attività}\textsubscript{\textit{G}}, metodi, pratiche e procedure mirate a garantire la fornitura del prodotto \textit{software}\textsubscript{\textit{G}} richiesto dal \textit{committente}\textsubscript{\textit{G}}. 
In particolare, il processo di fornitura si concentra sul monitoraggio e sul coordinamento delle \textit{attività}\textsubscript{\textit{G}} eseguite dal team durante la realizzazione del progetto, dalla concezione alla consegna, assicurando che il prodotto finale soddisfi i requisiti specificati dal \textit{committente}\textsubscript{\textit{G}} e venga consegnato nei tempi e nei costi previsti.

\vspace{0.2cm}

Tale processo viene attuato una volta completata la redazione integrale del documento \textit{Valutazione Capitolati}, cioè dopo aver correttamente identificato le specifiche e i vincoli richiesti dal \textit{proponente}\textsubscript{\textit{G}}.

\vspace{0.2cm}

In seguito, il \textit{fornitore}\textsubscript{\textit{G}} dovrà instaurare un contratto con l'azienda \textit{proponente}\textsubscript{\textit{G}}, nel quale si concorderanno i requisiti, i vincoli e i tempi di consegna del prodotto finale. Una volta concluso l'accordo con il \textit{proponente}\textsubscript{\textit{G}}, sarà possibile iniziare il processo di redazione del documento
\textit{Piano di Progetto}, il quale delineerà le \textit{attività}\textsubscript{\textit{G}}, le risorse e i costi indispensabili per la realizzazione del prodotto.

\subsubsection{Attività}
Il processo di fornitura, come descritto dallo \textit{standard}\textsubscript{\textit{G}} \href{https://www.math.unipd.it/~tullio/IS-1/2009/Approfondimenti/ISO_12207-1995.pdf}{ISO/IEC 12207:1995}, è composto dalle seguenti \textit{attività}\textsubscript{\textit{G}}:
\begin{enumerate}
    \item \textbf{Avvio}: si individuano le necessità e i requisiti del cliente e si avvia l'iter per rispondere alle sue esigenze;
    \item \textbf{Preparazione della risposta alle richieste}: si elaborano le proposte per rispondere alle richieste del cliente, che comprendono la definizione dei requisiti e delle condizioni contrattuali;
    \item \textbf{Contrattazione}: questa \textit{attività}\textsubscript{\textit{G}} coinvolge la negoziazione dei termini contrattuali tra il \textit{fornitore}\textsubscript{\textit{G}} e il cliente, incluso il raggiungimento di un accordo sui requisiti del progetto e sulle condizioni di consegna;
    \item \textbf{Pianificazione}: si pianificano le \textit{attività}\textsubscript{\textit{G}} necessarie per soddisfare i requisiti del cliente, inclusi i tempi, le risorse e i costi;
    \item \textbf{Esecuzione e controllo}: vengono eseguite le \textit{attività}\textsubscript{\textit{G}} pianificate e viene monitorato il progresso del progetto per garantire il rispetto dei tempi, dei costi e dei requisiti;
    \item \textbf{Revisione e valutazione}:  si effettuano revisioni periodiche per valutare lo stato del progetto rispetto agli obiettivi pianificati e per identificare eventuali problemi o rischi;
    \item \textbf{Consegna e completamento}: il prodotto \textit{software}\textsubscript{\textit{G}} viene consegnato al cliente includendo la documentazione finale e attuando le ultime procedure e compiti necessari per concludere il progetto in modo completo e soddisfacente.
\end{enumerate}

\vspace{0.1cm}

\subsubsection{Comunicazioni con l'azienda proponente}
Il gruppo \textit{ByteOps} instaurerà e si impegnerà a mantere una comunicazione costante con l'azienda \textit{proponente}\textsubscript{\textit{G}} \textit{Sync Lab} in modo da ottenere un riscontro sul lavoro svolto fino a quel momento e per verificare che i requisiti individuati siano conformi a quanto stabilito nel capitolato e nei colloqui con la \textit{proponente}\textsubscript{\textit{G}} stessa.

\vspace{0.2cm}

L'azienda \textit{proponente}\textsubscript{\textit{G}} \textit{Sync Lab} mette a disposizione un indirizzo mail per contattare il team \textit{ByteOps} relativamente alle comunicazioni e le richieste formali, ed un canale \textit{Element} per comunicazioni informali, quali richiedere eventuali chiarimenti o richieste di aiuto in determinate sezioni dello svolgimento.

\vspace{0.2cm}

Gli incontri \textit{SAL}\textsubscript{\textit{G}} (\textit{Stato Avanzamento Lavori}) con la \textit{proponente}\textsubscript{\textit{G}} sono fissati a cadenza bisettimanale, in modo da poter discutere con essa lo stato di avanzamento e le difficoltà riscontrate per quanto concerne gli obiettivi proposti nel periodo che intercorre tra un \textit{SAL}\textsubscript{\textit{G}} e il successivo, nonchè le \textit{attività}\textsubscript{\textit{G}} future da svolgere.
Per ogni incontro effettuato con la \textit{proponente}\textsubscript{\textit{G}}, verrà redatto un \textit{\textbf{Verbale esterno}} che riporterà data e ora dell'incontro, i partecipanti, gli argomenti trattati e le \textit{attività}\textsubscript{\textit{G}} concordate per il prossimo incontro.
I verbali esterni sono archiviati al percorso \textit{Nome\_periodo/Verbali/Esterni} nella \href{https://github.com/ByteOps-swe/Documents}{repository} del gruppo \textit{ByteOps}.

\subsubsection {Documentazione fornita}
Viene elencata di seguito la documentazione che il gruppo \textit{ByteOps} consegnerà ai commitenti \textit{Prof.} Tullio Vardanega e \textit{Prof.} Riccardo Cardin e all'azienda \textit{proponente}\textsubscript{\textit{G}} \textit{Sync Lab}.

\paragraph{Valutazione dei capitolati}
Il documento \textit{Valutazione Capitolati} contiene l'analisi dei capitolati proposti, con l'obiettivo di individuare il capitolato più adatto alle esigenze del gruppo \textit{ByteOps}. 

Il documento contiene le seguenti sezioni:
\begin{itemize}
    \item \textbf{Informazioni generali}: contiene informazioni generali sul capitolato, come il nome del progetto e la \textit{proponente}\textsubscript{\textit{G}}.
    \item \textbf{Obiettivo}: contiene una sintesi del prodotto da sviluppare seguendo quanto richiesto dal capitolato.
    \item \textbf{Tecnologie suggerite}: contiene le tecnologie suggerite dal \textit{proponente}\textsubscript{\textit{G}} per lo sviluppo del prodotto.
    \item \textbf{Considerazioni}: contiene le considerazioni del gruppo \textit{ByteOps} riguardo il capitolato, come i pro e i contro, le criticità e le motivazioni che hanno portato alla scelta o meno del capitolato.
\end{itemize}

\paragraph{Analisi dei requisiti}
L'\textit{Analisi dei Requisiti} è un documento che contiene la descrizione in dettaglio dei requisiti, dei casi d'uso e definisce in modo esaustivo le funzionalità che il prodotto offirà.\\
Tale documento ha come scopo quello di eliminare ogni ambiguità che potrebbe sorgere durante la lettura del capitolato e di fornire una base di partenza per la progettazione del prodotto.

Il documento contiene le seguenti sezioni:

\begin{itemize}
    \item \textbf{Introduzione}: contiene una breve descrizione del prodotto e delle sue funzionalità.
    \item \textbf{Casi d'uso}: identifica tutti i possibili scenari di utilizzo da parte dell'utente del prodotto. Ogni caso d'uso è accompagnato da una descrizione che ne descrive il funzionamento.
    \item \textbf{Requisiti}: insieme di tutte le richieste e vincoli definiti dalla \textit{proponente}\textsubscript{\textit{G}}, o estratti da discussioni con il gruppo, per realizzare il prodotto \textit{software}\textsubscript{\textit{G}} commissionato. Ogni requisito è accompagnato da una descrizione e da un relativo caso d'uso. 
\end{itemize}

\paragraph{Piano di progetto}
Il \textit{Piano di Progetto} è un documento stilato ed aggiornato continuamente che tratta i seguenti temi: 
\begin{itemize}
    \item \textbf{Analisi dei rischi}: contiene l'analisi dei rischi che il gruppo prevede di incontrare durante lo svolgimento del progetto. Ad ogni rischio è associata una descrizione, una procedura di identificazione, la probabilità di occorrenza, l'indice di gravità e un relativo piano di mitigazione del rischio. Tale documento permetterà di attuare le strategie di mitigazione dei rischi in modo tempestivo.
    \item \textbf{Pianificazione e metodo utilizzato}: contiene la pianificazione delle \textit{attività}\textsubscript{\textit{G}}, la suddivisione dei ruoli e l'identificazione delle \textit{attività}\textsubscript{\textit{G}} da svolgere per ogni ruolo. In aggiunta, fornisce una dettagliata esposizione del modello di sviluppo adottato, corredato dalle motivazioni sottostanti che hanno orientato la decisione verso tale approccio.
    \item \textbf{Preventivo e consuntivo di periodo}: contiene il preventivo  delle ore e dei costi associati a ciascuna \textit{attività}\textsubscript{\textit{G}}, suddivisi per ogni periodo definito. Al termine di ciascun periodo, viene redatto il consuntivo, il quale include un resoconto delle ore e dei costi effettivamente sostenuti per ogni singola \textit{attività}\textsubscript{\textit{G}}.
\end{itemize}

\paragraph{Piano di qualifica}
Il \textit{Piano di Qualifica} è un documento essenziale che illustra le strategie di verifica e validazione per garantire che il prodotto soddisfi le aspettative del \textit{committente}\textsubscript{\textit{G}} e della \textit{proponente}\textsubscript{\textit{G}}. Esso identifica le metodologie di verifica, i criteri di validazione e le risorse coinvolte, assicurando un processo di controllo qualità efficiente.  
Il documento contiene le seguenti sezioni:  
\begin{itemize}
    \item \textbf{Qualità di processo}: contiene le strategie di verifica e validazione per garantire che i \textit{processi}\textsubscript{\textit{G}} di sviluppo del prodotto soddisfino gli obiettivi di qualità.
    \item \textbf{Qualità di prodotto}: contiene le strategie di verifica e validazione per garantire che il prodotto soddisfi gli obiettivi di qualità.
    \item \textbf{Specifica dei test}: contiene la descrizione approfondita dei \textit{test}\textsubscript{\textit{G}}. 
\end{itemize}

\paragraph{Glossario}
Il \textit{Glossario} è un documento essenziale finalizzato alla chiarificazione dei termini tecnici e degli acronimi impiegati all'interno dei documenti. La sua funzione principale è fornire agli utenti e ai lettori un riferimento chiaro e uniforme, garantendo così una interpretazione coerente dei concetti specifici presenti nei diversi documenti. Questo strumento si propone inoltre di mitigare eventuali ambiguità e prevenire fraintendimenti, promuovendo una comprensione univoca dei termini tecnici utilizzati nell'ambito della documentazione.

\paragraph{Lettera di presentazione}
La \textit{Lettera di presentazione} è un documento che accompagna la presentazione della documentazione e del prodotto \textit{software}\textsubscript{\textit{G}} durante le fasi di revisione di progetto. Questo documento dettaglia l'elenco della documentazione redatta, la quale sarà consegnata ai committenti, il \textit{Prof.} Tullio Vardanega e il \textit{Prof.} Riccardo Cardin, nonché all'azienda \textit{proponente}\textsubscript{\textit{G}}, \textit{Sync Lab}.

\subsubsection{Strumenti}
Gli strumenti utilizzati dal gruppo per la gestione del processo di fornitura sono: 
\begin{itemize}
    \item \textbf{Google Meet:}
    \textit{servizio}\textsubscript{\textit{G}} di videoconferenza utilizzato per gli incontri con la \textit{proponente}\textsubscript{\textit{G}};
    \item \textbf{Google Calendar:}
    \textit{servizio}\textsubscript{\textit{G}} di \textit{calendario}\textsubscript{\textit{G}} utilizzato per la pianificazione degli incontri con la \textit{proponente}\textsubscript{\textit{G}};
    \item \textbf{Element:}
    \textit{servizio}\textsubscript{\textit{G}} di messaggistica istantanea utilizzato per le comunicazioni con la \textit{proponente}\textsubscript{\textit{G}} in caso di necessità;
    \item \textbf{Google Slides:}
    \textit{servizio}\textsubscript{\textit{G}} cloud utilizzato per la creazione delle presentazioni da mostrare durante i diari di bordo;
    \item \textbf{Google Sheets:} \textit{servizio}\textsubscript{\textit{G}} cloud utilizzato per la creazione dei grafici di preventivo e consuntivo di periodo presenti nel documento \textit{Piano di Progetto}.
\end{itemize}

