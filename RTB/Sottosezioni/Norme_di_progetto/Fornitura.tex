\subsection{Fornitura}
\subsubsection{Scopo}
Conformemente allo standard ISO/IEC 12207:1995, il processo di fornitura definisce un insieme strutturato di attività, metodi, pratiche e procedure mirate a garantire la fornitura del prodotto software richiesto dal committente. 
In dettaglio, il suo scopo consiste nel monitorare e coordinare le attività eseguite dal gruppo \textit{Byte-Ops} nel corso dell'intero processo di realizzazione del progetto.\\
Tale processo sarà attivato una volta completata la redazione integrale del documento \textit{Valutazione\_Capitolati}, cioè dopo aver correttamente identificato le specifiche richieste dalla proponente.
In seguito, il fornitore dovrà instaurare un contratto con la proponente, nel quale si concorderanno e accetteranno i requisiti specificati e i tempi di consegna del prodotto finale. Una volta concluso l'accordo con la proponente, sarà possibile iniziare il processo di redazione del documento
\textit{Piano\_di\_Progetto}, il quale delineerà le attività, le risorse e i costi indispensabili per la realizzazione del prodotto.

\subsubsection{Fasi della fornitura}
Il processo di fornitura, come descritto dallo standard ISO/IEC 12207:1995, è suddiviso nelle seguenti fasi:
\begin{enumerate}
    \item \textbf{Avvio}
    \item \textbf{Preparazione della risposta alle richieste}
    \item \textbf{Contrattazione}
    \item \textbf{Pianificazione}
    \item \textbf{Esecuzione e controllo}
    \item \textbf{Revisione e valutazione}
    \item \textbf{Consegna e completamento}
\end{enumerate}

\subsubsection{Descrizione}
In questa sezione verranno descritte le norme che il gruppo \textit{Byte-Ops} si impegna a rispettare durante lo svolgimento del progetto didattico al fine di figurare come fornitore nei confronti della proponente \textit{Sync Lab} e dei committenti \textit{Prof. Tullio Vardanega} e \textit{Prof. Riccardo Cardin}.

\subsubsection{Aspettative}
Il gruppo \textit{Byte-Ops} instaurerà e si impegnerà di mantere una comunicazione costante con la proponente \textit{Sync Lab} di modo da ottenere un riscontro sul lavoro svolto fino a quel momento e per verificare che i requisiti individuati siano conformi a quanto stabilito nel capitolato e nei colloqui con la proponente stessa.\\

\subsubsection{Comunicazioni con la proponente}
La proponente \textit{Sync Lab} mette a disposizione un indirizzo email per contattare il gruppo \textit{Byte-Ops} in caso di necessità, ed un canale \textit{Element} per richiedere eventuali chiarimenti o richieste di aiuto in determinate sezioni dello svolgimento.\\
Gli incontri (\textit{SAL}) con la proponente sono organizzati a cadenza bisettimanale, in modo da poter discutere con essa gli obiettivi proposti dalla scorsa riunione e le attività future da svolgere per il successivo SAL.
Per ogni incontro effettuato con la proponente, verrà redatto un \textbf{Verbale esterno} che riporterà data e ora dell'incontro, i partecipanti, gli argomenti trattati e le attività concordate per il prossimo incontro.
I verbali esterni sono archiviati al percorso \textit{Nome\_periodo/Esterni/Verbali} nella \href{https://github.com/ByteOps-swe/Documents}{repository} del gruppo \textit{Byte-Ops}.

\subsubsection {Documentazione fornita}
Viene elencata di seguito la documentazione che il gruppo \textit{Byte-Ops} consegnerà ai commitenti \textit{Prof. Tullio Vardanega} e \textit{Prof. Riccardo Cardin} e alla proponente \textit{Sync Lab}.

\paragraph{Valutazione dei capitolati}
Il documento \textit{Valutazione Capitolati} contiene l'analisi dei capitolati proposti, con l'obiettivo di individuare il capitolato più adatto alle esigenze del gruppo \textit{Byte-Ops}.\\
Il documento contiene le seguenti sezioni:
\begin{itemize}
    \item \textbf{Informazioni generali}: contiene informazioni generali sul capitolato, come il nome del progetto e la proponente.
    \item \textbf{Obiettivo}: contiene una sintesi del prodotto da sviluppare seguendo quanto richiesto dal capitolato.
    \item \textbf{Tecnologie suggerite}: contiene le tecnologie suggerite dal proponente per lo sviluppo del prodotto.
    \item \textbf{Considerazioni}: contiene le considerazioni del gruppo \textit{Byte-Ops} riguardo il capitolato, come i pro e i contro, le criticità e le motivazioni che hanno portato alla scelta o meno del capitolato.
\end{itemize}

\paragraph{Analisi dei requisiti}
L'analisi dei requisiti è un documento che contiene la descrizione in dettaglio dei requisiti, dei casi d'uso e definisce in modo esaustivo le funzionalità che il prodotto offirà.\\
Tale documento ha come scopo quello di eliminare ogni ambiguità che possono sorgere durante la lettura del capitolato e di fornire una base di partenza per la progettazione del prodotto.\\
Il documento contiene le seguenti sezioni:
\begin{itemize}
    \item \textbf{Introduzione}: contiene una breve descrizione del prodotto e delle sue funzionalità.
    \item \textbf{Casi d'uso}: identifica tutti i possibili scenari di utilizzo da parte dell'utente del prodotto. Ogni caso d'uso è accompagnato da una descrizione che ne descrive il funzionamento.
    \item \textbf{Requisiti}: insieme di tutte le richieste e vincoli definiti dalla proponente, o estratti da discussioni con il gruppo, per realizzare il prodotto software commissionato. Ogni requisito è accompagnato da una descrizione e da un relativo caso d'uso.
\end{itemize}

\paragraph{Piano di progetto}
Il documento \textit{Piano di Progetto} è un documento stilato ed aggiornato continuamente che tratta i seguenti temi:
\begin{itemize}
    \item \textbf{Analisi dei rischi}: contiene l'analisi dei rischi che il gruppo preventiva di incontrare durante lo svolgimento del progetto. Ad ogni rischio è associata una descrizione, una procedura di identificazione, la probabilità di occorrenza, l'indice di gravità un relativo piano di mitigazione del rischio. Tale documento permetterà di attuare le strategie di mitigazione dei rischi in modo tempestivo.
\end{itemize}