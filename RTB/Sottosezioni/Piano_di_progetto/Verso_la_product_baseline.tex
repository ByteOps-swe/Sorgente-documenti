\subsection{Verso la Product Baseline}

\subsubsection{Pianificazione}
Periodo previsto: 16/01/2024-12/03/2024\\ 
\vspace{0.2cm} 
Periodo effettivo: (??)-(??)\\ 
\vspace{0.2cm} 
In questa sezione, pur avendo un'idea di ciò che ci attende, al momento abbiamo scelto di strutturare i nostri periodi di lavoro in fasi più lunghe anziché bisettimanali. Questa decisione deriva dalla nostra attuale valutazione delle competenze, poiché riteniamo prematuro definire l'intervallo che precede la seconda revisione con periodi di tempo molto definiti e stretti.

Le fasi attuali, più lunghe e meno specifiche, saranno progressivamente convertite in periodi bisettimanali quando vi sarà una maggiore consapevolezza delle \textit{attività}\textsubscript{\textit{G}} future.

\vspace{0.2cm}

\textbf{Obiettivo}: Nella fase successiva, il focus sarà sullo sviluppo dei Diagrammi delle Classi e di eventuali nuovi documenti. L'obiettivo primario sarà la realizzazione del prodotto effettivo partendo dal \textit{POC}\textsubscript{\textit{G}}, integrando le funzionalità non ancora implementate e migliorandolo nei punti più deboli della sua struttura.

\todo{da rivedere questa parte introduttiva}

\paragraph{Prima fase}
Intervallo temporale previsto: 16/01/2024-13/02/2024\\ 
\vspace{0.2cm} 
Durante la prima fase, l'attenzione sarà rivolta all'inizializzazione di possibili nuovi documenti per la \textit{PB}\textsubscript{\textit{G}}, in parallelo sarà affrontato lo studio dell'\textit{architettura}\textsubscript{\textit{G}} di \textit{sistema}\textsubscript{\textit{G}} e dei design \textit{pattern}\textsubscript{\textit{G}} più appropriati.

\vspace{0.2cm}

I lavori continueranno sul \textit{Piano di Progetto}, sulla correzione dei documenti \textit{Analisi dei Requisiti}, \textit{Glossario} e \textit{Piano di Qualifica}. Si avvierà la realizzazione dei diagrammi di \textit{attività}\textsubscript{\textit{G}} e sequenze, dando anche inizio allo sviluppo della prima versione del prodotto basata sul \textit{PoC}\textsubscript{\textit{G}}.

\paragraph{Seconda fase}
Intervallo temporale previsto: 13/02/2024-12/03/2024
\\ 
\vspace{0.2cm} 
Durante la seconda fase del progetto, l'attenzione sarà rivolta all’avanzamento di possibili nuovi documenti per la \textit{PB}\textsubscript{\textit{G}}, insieme alla continuazione dei documenti inizialmente avviati. In parallelo, saranno eseguite ottimizzazioni del \textit{sistema}\textsubscript{\textit{G}}, attraverso \textit{test}\textsubscript{\textit{G}} specifici per valutare la sua scalabilità e l'implementazione di allarmi per individuare eventuali anomalie o superamento di soglie critiche.

Si proseguirà con il perfezionamento del codice stesso, garantendo un costante miglioramento delle funzionalità e delle prestazioni del prodotto in fase di sviluppo.

\paragraph{Terza fase}
Intervallo temporale previsto: 12/03/2024-25/03/2024\\ 
\vspace{0.2cm} 
Durante la terza fase del nostro progetto, l'attenzione sarà rivolta all'ultimazione delle nuove versioni del \textit{Piano di Qualifica}, delle \textit{Norme di Progetto}, del \textit{Glossario} e del \textit{Piano di Qualifica}, completando contemporaneamente i documenti specifici della revisione \textit{PB}\textsubscript{\textit{G}}.

Sarà inoltre intensivamente testato il prodotto attraverso i \textit{test}\textsubscript{\textit{G}} e le metriche descritte all'interno del documento \textit{Piano di Qualifica} ed infine verrà creata la presentazione per la \textit{PB}\textsubscript{\textit{G}}.


%---------------------OTTAVO PERIODO---------------------------------------

\subsubsection{Ottavo periodo  16/02/2024 - 23/12/2024}
\paragraph{Considerazioni} \todo{scrivila}
    Nel corso del primo periodo, il nostro team ha dedicato risorse significative all'elaborazione e alla standardizzazione dei \textit{processi}\textsubscript{\textit{G}}, formalizzando tali linee guida nel documento \textit{Norme di Progetto}. In quest'ultimo, sono state dettagliatamente redatte le sezioni specificate nella tabella sottostante.

    Durante il primo incontro con l'azienda, abbiamo definito obiettivi chiave da conseguire entro il prossimo \textit{SAL}\textsubscript{\textit{G}} fissato per il 24 novembre 2023, coincidente con l'avvio del prossimo periodo. Questo approccio ricalca la struttura dello sprint backlog del \textit{framework}\textsubscript{\textit{G}} Scrum. \\
    Tra i molteplici obiettivi delineati, si evidenziano la realizzazione di almeno un simulatore di un \textit{sensore}\textsubscript{\textit{G}} in linguaggio \textit{Phyton}\textsubscript{\textit{G}}, il quale interagisca con un server \textit{Apache Kafka}\textsubscript{\textit{G}} mediante \textit{Docker}\textsubscript{\textit{G}}. Opzionalmente, si è prevista l'\textit{integrazione}\textsubscript{\textit{G}} con il \textit{database}\textsubscript{\textit{G}} \textit{ClickHouse}\textsubscript{\textit{G}}e per immagazzinare i dati dei simulatori. In parallelo, ci si è dedicati alla creazione di user story e casi d'uso correlati al capitolato.

    È soddisfacente constatare che tutte le richieste avanzate dal \textit{proponente}\textsubscript{\textit{G}} sono state risolte entro i tempi concordati, includendo le richieste opzionali.

    Parallelamente, durante questa fase, gli amministratori hanno investito risorse per automatizzare il processo di compilazione dei sorgenti \LaTeX, una volta caricati nella \textit{repository}\textsubscript{\textit{G}} condivisa. Inoltre, è stata implementata una procedura automatica di rinomina dei file PDF generati, inclusiva dell'indicazione della versione del documento.

\paragraph{Gestione dei rischi} 

\begin{itemize}
    \item \textbf{Rischi attesi e verificati:}
\begin{itemize}
    \item \textbf{RT-1A-1} - Inesperienza nell'attività di progettazione \todo{non è una tecnologia}(\textit{(\ref{sec:rischioTec})})
    \begin{itemize}
        \item \textbf{Esito mitigazione:} 
            l'autoapprendimento e la conoscenze dei singoli non si sono dimostrate adeguate per acquisire una conoscenza approfondita dell'ambiente \textit{Docker}\textsubscript{\textit{G}} nel breve periodo iniziale, portando all'utilizzo del \textit{sistema}\textsubscript{\textit{G}} senza una comprensione approfondita di ciascuna delle sue componenti e configurazioni. Di conseguenza, è stata formulata una richiesta al \textit{proponente}\textsubscript{\textit{G}} per la realizzazione di un corso di formazione specifico su \textit{Docker}\textsubscript{\textit{G}} seguendo le norme di mitigazione definite nella \textit{sezione~\ref{sec:rischioTec}};
        
        \item \textbf{Impatto:}
            nessuna conseguenza significativa è stata riscontrata, poiché le avvertenze segnalate dalla \textit{proponente}\textsubscript{\textit{G}} riguardavano criticità di lieve entità relative alle best practices di \textit{Docker}\textsubscript{\textit{G}}. Le misure di mitigazione necessarie sono state tempestivamente implementate, e un incontro formativo è stato programmato per approfondire ulteriormente la questione.
            Inoltre, al fine di conformarsi alle best practices dell'ambiente, è stata presa la decisione di regolamentare, nel documento \textit{Norme di Progetto}, la configurazione dell'ambiente \textit{Docker}\textsubscript{\textit{G}}.
    \end{itemize}
\end{itemize}
\item \textbf{Rischi attesi ma non verificati:}
 \begin{itemize}
    \item \textbf{RO-2M-2} - Ritardo nel completamento delle \textit{attività}\textsubscript{\textit{G}} rispetto ai tempi previsti~(\ref{sec:ritAttivita});
    \item \textbf{RP-2B-1} - Contrasti interni al gruppo~(\textit{\ref{subsubsec:contrastiInterni}}).
\end{itemize}
\item \textbf{Rischi non attesi ma verificati:}
\begin{itemize}
    \item Nessuno.
\end{itemize}
\end{itemize}

\paragraph{Definizione ruoli}
Per le \textit{attività}\textsubscript{\textit{G}} registrate nei costi, sono stati assegnati i seguenti ruoli: 

\vspace{0.4cm}

\begin{table}[H]
    \centering
    \begin{tabular}{|L{4cm}|L{2cm}|}
        \hline
        \textbf{Ruolo} & \textbf{Persona} \\
        \hline
        \hline
        Responsabile (Re)   & E. Hysa \\
        \hline
        Amministratore (Am) & R. Smanio \\
        \hline
        Analisti (An)       & D. Diotto \\
        \hline
        Verificatore (Ve)   & F. Pozza \\
                            & R. Smanio \\   
        \hline
        Programmatori (Pr)  & L. Skenderi \\
                            & A. Barutta \\
        \hline
        Progettista (Pt)    & N. Preto \\
                            & E. Hysa \\
        \hline
    \end{tabular}
    \caption{Tabella dei ruoli assegnati - Ottavo periodo}
    \label{tab:Ruoli_persone_1}
    \end{table}

\newpage
\paragraph{Pianificazione attività divise per ruoli con consuntivo e preventivo orario e dei costi}

\vspace{0.4cm}

\begin{figure}[H]
    \centering
    \includegraphics[height=1.1\textwidth]{../Images/periodo8.PNG}
    \caption{Ottavo periodo}
    \label{fig:Ottavo_periodo}
\end{figure}

Al termine dell'ottavo periodo, l'ammontare totale del costo del progetto è \textbf{ 1317,50\euro\ }\todo{da modificare} e sono state completate il \textbf{100\%} delle \textit{attività}\textsubscript{\textit{G}} attese.
Il preventivo a finire rimane invariato a \textbf{12565,00€} e non risulta necessaria una ripianificazione delle \textit{attività}\textsubscript{\textit{G}} future.
\href{https://github.com/orgs/ByteOps-swe/projects/3/views/1?sortedBy%5Bdirection%5D=asc&sortedBy%5BcolumnId%5D=64182560}{Vai al Diagramma di Gantt.}

\begin{figure}[H]
    \centering
    \begin{minipage}[b]{0.70\textwidth}
        \centering
        \includegraphics[width=0.7\textwidth]{../Images/avanzamento8Periodo.png}
        \caption{Avanzamento dei lavori RTB - Ottavo periodo}
        \label{fig:Avanzamento_RTB_8}
    \end{minipage}
\end{figure}

\paragraph{Preventivo orario}

\begin{figure}[H] 
    \centering
    \includegraphics[width=0.9\textwidth]{../Images/preventivoOrario8Periodo.png}
    \caption{Preventivo orario per membro - Ottavo periodo}
    \label{fig:Preventivo_orario_8}
\end{figure}

\vspace{0.6cm}

\begin{figure}[H]
    \centering
    \includegraphics[width=0.6\textwidth]{../Images/preventivoDivisioneRuoli8Periodo.png}
    \caption{Istogramma preventivo della ripartizione oraria dei ruoli - Ottavo periodo}
    \label{fig:Preventivo_ripartizione_oraria_8}
\end{figure}

\paragraph{Consuntivo orario}

\begin{figure}[H]
    \centering
    \includegraphics[width=0.9\textwidth]{../Images/consuntivoOrario8Periodo.png}
    \caption{Consuntivo orario per membro - Ottavo periodo}
    \label{fig:Constuntivo_orario_8}
\end{figure}

\vspace{0.6cm}

\begin{figure}[H]
    \centering
    \includegraphics[width=0.6\textwidth]{../Images/consuntivoDivisioneRuoli1Periodo.png}
    \caption{Istogramma consuntivo della ripartizione oraria dei ruoli - Ottavo periodo}
    \label{fig:Consuntivo_ripartizione_oraria_8}
\end{figure}