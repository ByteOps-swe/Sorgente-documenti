\subsection{Verso la Product Baseline}
Periodo previsto: 16/01/2024-12/03/2024\\ 
\vspace{0.2cm} 
Periodo effettivo: (??)-(??)\\ 
\vspace{0.2cm} 
In questa sezione, pur avendo un'idea di ciò che ci attende, al momento abbiamo scelto di strutturare i nostri periodi di lavoro in fasi più lunghe anziché bisettimanali. Questa decisione deriva dalla nostra attuale valutazione delle competenze, poiché riteniamo prematuro definire l'intervallo che precede la seconda revisione con periodi di tempo molto definiti e stretti.

Le fasi attuali, più lunghe e meno specifiche, saranno progressivamente convertite in periodi bisettimanali quando vi sarà una maggiore consapevolezza delle \textit{attività}\textsubscript{\textit{G}} future.

\vspace{0.2cm}

\textbf{Obiettivo}: Nella fase successiva, il focus sarà sullo sviluppo dei Diagrammi delle Classi e di eventuali nuovi documenti. L'obiettivo primario sarà la realizzazione del prodotto effettivo partendo dal \textit{POC}\textsubscript{\textit{G}}, integrando le funzionalità non ancora implementate e migliorandolo nei punti più deboli della sua struttura.

\subsubsection{Prima fase}
Intervallo temporale previsto: 16/01/2024-13/02/2024\\ 
\vspace{0.2cm} 
Durante la prima fase, l'attenzione sarà rivolta all'inizializzazione di possibili nuovi documenti per la \textit{PB}\textsubscript{\textit{G}}, in parallelo sarà affrontato lo studio dell'\textit{architettura}\textsubscript{\textit{G}} di \textit{sistema}\textsubscript{\textit{G}} e dei design \textit{pattern}\textsubscript{\textit{G}} più appropriati.

\vspace{0.2cm}

I lavori continueranno sul \textit{Piano di Progetto}, sulla correzione dell'Analisi dei Requisiti, del \textit{Glossario} e del \textit{Piano di Qualifica}. Si avvierà la realizzazione dei diagrammi di \textit{attività}\textsubscript{\textit{G}} e sequenze, dando anche inizio allo sviluppo della prima versione del prodotto basata sul \textit{PoC}\textsubscript{\textit{G}}.

\subsubsection{Seconda fase}
Intervallo temporale previsto: 13/02/2024-12/03/2024
\\ 
\vspace{0.2cm} 
Durante la seconda fase del progetto, l'attenzione sarà rivolta all’avanzamento di possibili nuovi documenti per la \textit{PB}\textsubscript{\textit{G}}, insieme alla continuazione dei documenti inizialmente avviati. In parallelo, saranno eseguite ottimizzazioni del \textit{sistema}\textsubscript{\textit{G}}, attraverso \textit{test}\textsubscript{\textit{G}} specifici per valutare la sua scalabilità e l'implementazione di allarmi per individuare eventuali anomalie o superamento di soglie critiche.

Si proseguirà con il perfezionamento del codice stesso, garantendo un costante miglioramento delle funzionalità e delle prestazioni del prodotto in fase di sviluppo.

\subsubsection{Terza fase}
Intervallo temporale previsto: 12/03/2024-25/03/2024\\ 
\vspace{0.2cm} 
Durante la terza fase del nostro progetto, l'attenzione sarà rivolta all'ultimazione delle nuove versioni del \textit{Piano di Qualifica}, delle \textit{Norme di Progetto}, del \textit{Glossario} e del \textit{Piano di Qualifica}, completando contemporaneamente i documenti specifici della revisione \textit{PB}\textsubscript{\textit{G}}.

Sarà inoltre intensivamente testato il prodotto attraverso i \textit{test}\textsubscript{\textit{G}} e le metriche descritte all'interno del documento \textit{Piano di Qualifica} ed infine verrà creata la presentazione per la \textit{PB}\textsubscript{\textit{G}}.

\vspace{0.4cm}