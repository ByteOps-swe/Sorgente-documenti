\subsection{Gestione e monitoraggio dell'avanzamento del Progetto}
In collaborazione con la proponente, si è concordato di organizzare l'avanzamento del progetto in periodi di due settimane, seguendo un approccio simile agli sprint di Scrum. Durante ciascun periodo, verranno selezionate, in collaborazione con l'azienda e i membri del team, le attività da svolgere.

La scelta delle task si baserà sulla loro importanza strategica e sulla fattibilità di completarle entro la durata del periodo. Nel caso in cui alcune attività non vengano portate a termine entro il periodo, verranno riportate nel consuntivo di periodo e proseguiranno nel periodo successivo.
Ogni periodo sarà documentato attraverso una tabella esaustiva in cui saranno identificate le task relative a ciascun ruolo. Per ogni attività, verrà indicato lo stato di completamento, i tempi previsti ed effettivi, e i costi associati.

Le attività assegnate al ruolo "Team" non vengono considerate nel computo dei costi, in quanto sono associate a iniziative di carattere interno e possono essere eseguite da ruoli vari.

Le ore impiegate per tali attività sono regolarmente registrate nella sezione "Preventivo orario per membro" presente in ciascun periodo.

Al termine di ciascun periodo, sarà calcolato il costo totale fino a quel momento del progetto, fornendo una chiara visione del progresso complessivo.

Inoltre ogni periodo conterrà una discussione sui rischi occorsi e sull'esito della loro mitigazione seguendo quanto definito in (\ref{sec:AnalisiRischi})

I dati riportati in ogni periodo sono il resconto di quanto inserito in autonomia dai membri nel foglio google condiviso disponibile
\href{https://docs.google.com/spreadsheets/d/1gbGCTKO6tLKN7lI9kTZHLpoItniJIi9X/edit?usp=sharing&ouid=104272518979154193028&rtpof=true&sd=true}{qui}.

