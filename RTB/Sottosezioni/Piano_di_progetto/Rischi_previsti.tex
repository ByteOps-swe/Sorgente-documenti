\subsection{Rischi previsti}
Di seguito sono riportate le tabelle relative ai rischi previsti che potrebbero presentarsi durante lo sviluppo del progetto.  

La convenzione utilizzata per la codifica dei rischi è presente in \textit{Norme di Progetto} nella sezione: "Gestione dei rischi" in "Risoluzione dei problemi".  


\subsubsection{Impegni personali e accademici}\label{sec:ImpPersonali}
\begin{table}[ht]
    \centering
    \begin{tabularx}{\textwidth}{l>{\RaggedRight}X>{\RaggedRight}X>{\RaggedRight}X>{\RaggedRight}X}
    \toprule
    \rowcolor{gray!50}
    \textbf{Codice} & \textbf{Descrizione del rischio} & \textbf{Identificazione} & \textbf{Mitigazione} \\
    \midrule
    \addlinespace 
    RO-1A-1 & 
    Rischio di rallentamento del progetto dovuto all'armonizzazione delle \textit{attività}\textsubscript{\textit{G}} personali e progettuali, con particolare intensificazione durante la sessione invernale 2023-2024 a causa degli esami. & 
    I membri del gruppo comunicheranno al responsabile i loro impegni durante le riunioni di organizzazione o al momento immediato della conoscenza dell'impedimento. & 
    Il responsabile, considerando gli impegni dei membri del gruppo, avrà la facoltà di riassegnare le varie \textit{attività}\textsubscript{\textit{G}} ad altri membri o estendere il tempo previsto per l'esecuzione dell'\textit{attività}\textsubscript{\textit{G}} assegnata.\\  
    \bottomrule
    \addlinespace 
    \end{tabularx}
\end{table}

\subsubsection{Variazione dei requisiti del progetto}
\begin{table}[ht]
    \centering
    \begin{tabularx}{\textwidth}{l>{\RaggedRight}X>{\RaggedRight}X>{\RaggedRight}X>{\RaggedRight}X}
    \toprule
    \rowcolor{gray!50}
    \textbf{Codice} & \textbf{Descrizione del rischio} & \textbf{Identificazione} & \textbf{Mitigazione} \\
    \midrule
    \addlinespace 
    RO-3A-2 & 
    Potrebbero verificarsi modifiche in corso d'opera dei requisiti del progetto, che potrebbero determinare un cambiamento di direzione delle \textit{attività}\textsubscript{\textit{G}}. &
    Attraverso le riunioni periodiche con la \textit{proponente}\textsubscript{\textit{G}}, vengono comunicate in modo esplicito al gruppo le modifiche di alcuni requisiti. &
    Redigere un'analisi dettagliata dei requisiti all'inizio al fine di identificare e soddisfare completamente le esigenze della \textit{proponente}\textsubscript{\textit{G}}. Presentare tali requisiti e attuare tempestivamente eventuali misure correttive necessarie.\\
    \bottomrule
    \addlinespace 
    \end{tabularx}
\end{table}

\newpage
\subsubsection{Ritardo nel completamento delle attività rispetto ai tempi previsti}
\begin{table}[ht]
    \centering
    \begin{tabularx}{\textwidth}{l>{\RaggedRight}X>{\RaggedRight}X>{\RaggedRight}X>{\RaggedRight}X}
    \toprule
    \rowcolor{gray!50}
    \textbf{Codice} & \textbf{Descrizione del rischio} & \textbf{Identificazione} & \textbf{Mitigazione}\\
    \midrule
    \addlinespace 
    RO-2M-2 & 
    L'inesperienza del gruppo in un progetto \textit{software}\textsubscript{\textit{G}} professionale potrebbe portare a superare i tempi preventivati, specialmente a causa della nuova tecnologia e della necessità di migliorare la gestione delle risorse.& 
    I membri del gruppo devono segnalare al responsabile eventuali difficoltà nel rispettare le scadenze previste per le \textit{attività}\textsubscript{\textit{G}}.&
    Il responsabile, considerando le motivazioni del ritardo, avrà la facoltà di riassegnare le varie \textit{attività}\textsubscript{\textit{G}} ad altri membri o estendere il tempo previsto per l'esecuzione dell'\textit{attività}\textsubscript{\textit{G}} assegnata.\\
    \bottomrule
    \addlinespace 
    \end{tabularx}
\end{table}

\subsubsection{Apprendimento ed utilizzo delle nuove tecnologie} \label{sec:rischioTec}
\begin{table}[ht]
    \centering
    \begin{tabularx}{\textwidth}{l>{\RaggedRight}X>{\RaggedRight}X>{\RaggedRight}X>{\RaggedRight}X}
    \toprule
    \rowcolor{gray!50}
    \textbf{Codice} & \textbf{Descrizione del rischio} & \textbf{Identificazione} & \textbf{Mitigazione} \\
    \midrule
    \addlinespace 
    RT-1A-1 & 
    L’apprendimento e l'implementazione delle tecnologie proposte possono rappresentare un rischio considerevole per lo  sviluppo di un progetto, in quanto esiste la possibilità che lo studio accurato di queste tecnologie richieda più tempo del previsto.& 
    I membri del gruppo sono tenuti a notificare tempestivamente al responsabile qualsiasi difficoltà riscontrata durante il processo di studio delle tecnologie proposte.&
    Ogni membro deve studiare le nuove tecnologie, e in caso di difficoltà, organizzare workshop interni e sfruttare le opportunità di formazione dell'azienda \textit{proponente}\textsubscript{\textit{G}}.\\
    \bottomrule
    \addlinespace 
    \end{tabularx}
\end{table}

\newpage
\subsubsection{Perdita di file}
\begin{table}[ht]
    \centering
    \begin{tabularx}{\textwidth}{l>{\RaggedRight}X>{\RaggedRight}X>{\RaggedRight}X>{\RaggedRight}X}
    \toprule
    \rowcolor{gray!50}
    \textbf{Codice} & \textbf{Descrizione del rischio} & \textbf{Identificazione} & \textbf{Mitigazione} \\
    \midrule
    \addlinespace 
    RT-3M-2 & 
    È presente il rischio che alcuni file vengano persi a causa di malfunzionamenti hardware o errori umani.&
    Il danneggiamento o l'eliminazione accidentale di file su cui i membri hanno lavorato che compromette il lavoro svolto su quei documenti.&
    Adottare un \textit{sistema}\textsubscript{\textit{G}} di versionamento dei file fornisce ai membri del gruppo la capacità di tracciare e recuperare agevolmente versioni precedenti dei documenti, garantendo una robusta protezione contro modifiche indesiderate, danneggiamenti o eliminazioni accidentali.\\
    \bottomrule
    \addlinespace 
    \end{tabularx}
\end{table}

\subsubsection{Contrasti interni al gruppo}
\begin{table}[ht]
    \centering
    \begin{tabularx}{\textwidth}{l>{\RaggedRight}X>{\RaggedRight}X>{\RaggedRight}X>{\RaggedRight}X}
    \toprule
    \rowcolor{gray!50}
    \textbf{Codice} & \textbf{Descrizione del rischio} & \textbf{Identificazione} & \textbf{Mitigazione} \\
    \midrule
    \addlinespace 
    RP-2B-1 & 
    La comunicazione inefficace tra i membri del gruppo potrebbe causare ritardi significativi nello sviluppo del progetto, specialmente data la natura collaborativa del lavoro di gruppo, che richiede il rispetto di norme concordate collettivamente.& 
    Clima di disaccordo tra i membri del gruppo evidente, con segnali di divergenze di opinioni e contrasti nelle dinamiche di collaborazione. Si manifesta attraverso la mancanza di convergenza di idee, complicando il processo decisionale. &
    Il responsabile è tenuto a mitigare il clima di disaccordo e a perseguire una soluzione che soddisfi la maggioranza dei membri del gruppo.\\
    \bottomrule
    \addlinespace 
    \end{tabularx}
\end{table}

\newpage
\subsubsection{Contatti con la proponente}
\begin{table}[ht]
    \centering
    \begin{tabularx}{\textwidth}{l>{\RaggedRight}X>{\RaggedRight}X>{\RaggedRight}X>{\RaggedRight}X}
    \toprule
    \rowcolor{gray!50}
    \textbf{Codice} & \textbf{Descrizione del rischio} & \textbf{Identificazione} & \textbf{Mitigazione} \\
    \midrule
    \addlinespace 
    RP-3M-2 & 
    La comunicazione con l'azienda \textit{proponente}\textsubscript{\textit{G}} potrebbe non essere più efficace e potrebbe non essere sempre possibile, il che potrebbe portare alla comparsa di dubbi e richieste. & 
    Le risposte assenti o incomplete non contribuiscono alla risoluzione dei dubbi o delle domande proposte. Frequenza degli incontri che diminuisce. &
    Il responsabile è tenuto a comunicare la situazione alla parte \textit{proponente}\textsubscript{\textit{G}}, cercando di trovare una soluzione. Se non si riesce a risolvere il problema con la parte \textit{proponente}\textsubscript{\textit{G}}, si richiederà l’intervento del \textit{committente}\textsubscript{\textit{G}}. \\
    \bottomrule
    \addlinespace 
    \end{tabularx}
\end{table}
