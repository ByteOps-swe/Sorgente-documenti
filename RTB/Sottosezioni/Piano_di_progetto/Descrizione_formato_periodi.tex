\subsection{Presentazione della struttura espositiva dei periodi}
Ogni periodo di avanzamento verra esposto in seguito nel seguente formato:
\begin{enumerate}
    \item \textbf{Considerazioni:} Considerazioni retrospettive sul periodo effettuate una volta terminato;
    \item \textbf{Gestione dei rischi:} Un elenco di 
            \begin{itemize}
                \item \textbf{Rischi attesi e verificati};
                \item \textbf{Rischi attesi ma non verificati};
                \item \textbf{Rischi non attesi ma verificati}.
            \end{itemize}
        Nel caso in cui i rischi si verifichino essi conterranno considerazioni su:
        \begin{itemize}
            \item \textbf{Esito mitigazione:} Considerazioni sulla validità della mitigazione pianificata;
            \item \textbf{Impatto:} Impatto avuto nelle \textit{attività}\textsubscript{\textit{G}} pianificate.
        \end{itemize}
    \item \textbf{Definizione ruoli:} Esposizione dei ruoli occupati dai membri del team nel periodo;
    \item \textbf{Pianificazione attività divise per ruoli con consuntivo e preventivo orario e dei costi:} Tabella descritta in~\ref{sec:DescrTabella};
    \item \textbf{Grafico a torta dello stato avanzamento dei lavori};
    \item \textbf{Preventivo orario:} Espone le informazioni quali le ore preventivate svolte dai membri nei ruoli che la tabella descritta in~\ref{sec:DescrTabella} non contiene;
    \item \textbf{Consuntivo orario:} Espone le informazioni quali le ore consuntivate svolte dai membri nei ruoli che la tabella descritta in~\ref{sec:DescrTabella} non contiene;


\end{enumerate}
