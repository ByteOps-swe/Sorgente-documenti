\subsection{Presentazione della struttura espositiva dei periodi}
Ogni periodo di avanzamento verra esposto in seguito nel seguente formato:
\begin{enumerate}
    \item \textbf{Considerazioni:} Considerazioni retrospettive e consuntive sul periodo effettuate una volta terminato;
    \item \textbf{Gestione dei rischi:} Un elenco di 
            \begin{itemize}
                \item \textbf{Rischi attesi e verificati};
                \item \textbf{Rischi attesi ma non verificati};
                \item \textbf{Rischi non attesi ma verificati}.
            \end{itemize}
        Nel caso in cui i rischi si verifichino essi conterranno considerazioni su:
        \begin{itemize}
            \item \textbf{Esito mitigazione:} Considerazioni sulla validità della mitigazione pianificata;
            \item \textbf{Impatto:} Impatto avuto nelle \textit{attività}\textsubscript{\textit{G}} pianificate.
        \end{itemize}
    \item \textbf{Definizione ruoli:} Esposizione dei ruoli occupati dai membri del team nel periodo;
    \item \textbf{Pianificazione attività divise per ruoli con consuntivo e preventivo orario e dei costi:} Tabella descritta in~\ref{sec:DescrTabella}.
    Questa, come precedentemente indicato, svolge simultaneamente il ruolo di pianificazione e stima delle risorse durante la compilazione iniziale del responsabile, nonché quello di rendicontazione delle risorse e di monitoraggio dell'avanzamento effettivo. L'obiettivo è fornire una visione complessiva che rappresenti efficacemente l'esito del periodo in esame.
    \vspace*{0.2cm}
    Al di sotto della tabella, considerando i dati presentati, saranno incluse le osservazioni del responsabile riguardanti il totale speso fino al periodo in questione, la percentuale di \textit{attività}\textsubscript{\textit{G}} svolte rispetto a quelle pianificate per il periodo, nonché il nuovo preventivo a finire rivalutato al termine delle \textit{attività}\textsubscript{\textit{G}}. Inoltre, sarà valutata la necessità di rivalutare le \textit{attività}\textsubscript{\textit{G}} successive al termine di questo periodo.
    \item \textbf{Grafico a torta dello stato avanzamento dei lavori};
    \item \textbf{Preventivo orario:} Espone le informazioni quali le ore preventivate svolte dai membri nei ruoli che la tabella descritta in~\ref{sec:DescrTabella} non contiene;
    \item \textbf{Consuntivo orario:} Espone le informazioni quali le ore consuntivate svolte dai membri nei ruoli che la tabella descritta in~\ref{sec:DescrTabella} non contiene;


\end{enumerate}
