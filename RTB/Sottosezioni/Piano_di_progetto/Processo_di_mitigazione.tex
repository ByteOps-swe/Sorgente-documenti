\subsection{Processo di mitigazione}
\subsubsection{Identificazione}
    Individuare le possibili problematiche che potrebbero verificarsi durante lo sviluppo del progetto. 
    Le fonti dalle quali potrebbero derivare i rischi sono: 
    \begin{itemize}
        \item \textbf{Gruppo:} collaborazione, comunicazione, competenze tecniche, organizzazione.
        \item \textbf{Prodotto del capitolato:} requisiti, tecnologie, strumenti.
    \end{itemize}

\subsubsection{Processo di analisi}
Per ogni rischio identificato assegnare un indice identificativo e stabilire secondo i seguenti parametri:
\begin{itemize}
    \item \textbf{Probabilità di occorrenza:} quanto è probabile che il rischio si verifichi.
    \item \textbf{Grado di pericolosità:} quali effetti negativi potrebbe causare nello sviluppo del progetto.
\end{itemize}

\subsubsection{Pianificazione}
Per ogni rischio identificato, definire un piano di contingenza che preveda:
\begin{itemize}
    \item \textbf{Strategia preventiva:} definire le azioni da intraprendere per prevenire l’insorgenza del rischio.
    \item \textbf{Riduzione dell'impatto:} stabilire le misure da adottare per ridurre al minimo l'impatto del rischio, nel caso non si riesca ad evitarlo.
\end{itemize}

\subsubsection{Processo di controllo e aggiornamento}
Effettuare un monitoraggio periodico delle attività in corso e degli artefatti prodotti, al fine di identificare potenziali nuovi rischi o modificare quelli preesistenti, aggiornando di conseguenza le relative strategie di mitigazione.

\begin{comment}
Qui dentro viene mostrato il codice dei rischi che andrà normato nella sezione Gestione dei Rischi nel file Norme di Progetto.
Il rischio ha il seguente codice:   R[Tipo]-[Probabilità][Priorità]-[Indice]

R: rischio
Tipo: natura del rischio (tecnologico (T), organizzativo (O), personale (P), requisiti (R))
Probabilità: probabilità di occorrenza (Alta(1), Media(2), Bassa(3))
Priorità: grado di pericolosità  (Alta(A), Media(M), Bassa(B))
Indice: numero progressivo che identifica il rischio (1, 2, 3, ecc..)

\end{comment}