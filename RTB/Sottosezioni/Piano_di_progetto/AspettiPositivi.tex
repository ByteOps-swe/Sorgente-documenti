\subsection{Aspetti Positivi}
\begin{itemize}
    \item \textbf{Collaborazione:} in situazioni in cui si sono verificati ritardi nello svolgimento di alcune \textit{attività}\textsubscript{\textit{G}}, i membri del gruppo hanno manifestato una notevole disponibilità nel compensare eventuali lacune temporali e di conoscenza. Questa prontezza nell'affrontare le sfide ha contribuito a mantenere l'efficienza complessiva del team, evidenziando un elevato senso di responsabilità e collaborazione tra i membri;
    \item \textbf{Norme di progetto:} la maggior parte delle direttive del progetto sono state sviluppate, almeno in una fase iniziale, all'inizio delle \textit{attività}\textsubscript{\textit{G}}. Questo ha fornito una base iniziale su cui costruire convenzioni interne, evitare situazioni di caos non controllato e applicare il ciclo PDCA;
    \item \textbf{Rimozione del vincolo di ruolo:} durante l'analisi dell'andamento del progetto, il team ha notato che l'imposizione del vincolo di assegnare ai membri un unico ruolo durante i periodi ha causato situazioni di inattività per alcuni membri, mentre allo stesso tempo ha generato un carico di lavoro eccessivo per altri.
    Al fine di affrontare questa problematica, è stata presa la decisione di assegnare più ruoli ai membri il cui carico di \textit{attività}\textsubscript{\textit{G}} per il periodo associato fosse inferiore, al fine di fornire supporto ai ruoli con una maggiore richiesta oraria;
    \item \textbf{Riunioni frequenti:} al fine di garantire un costante monitoraggio dello stato del progetto, sono state programmate riunioni di aggiornamento settimanali. Tale frequenza si è dimostrata adeguata alle necessità del progetto, consentendo di gestire efficacemente eventuali inadempienze e ritardi;
    \item \textbf{Automazione:} il gruppo si ritiene soddisfatto del grado di automazione raggiunto per quanto riguarda \textit{attività}\textsubscript{\textit{G}} ripetitive e per le quali l'intervento umano potrebbe provocare errori, quali:
    \begin{itemize}
        \item Build e pubblicazione dei documenti;
        \item Rinomina documenti con versione in registro delle modifiche;
        \item Notifiche in relazione a eventi sui \textit{repository}\textsubscript{\textit{G}} di progetto;
        \item Inserimento "G" a pedice per i termini da \textit{Glossario}.
    \end{itemize}
\end{itemize}