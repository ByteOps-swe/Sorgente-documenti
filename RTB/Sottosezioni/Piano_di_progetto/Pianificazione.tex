\subsection{Pianificazione}
    In conformità con la filosofia di sviluppo moderna e dinamica, abbiamo scelto di adottare il modello agile, con un focus specifico sul framework Scrum.
    Lo Scrum, con le sue pratiche iterative e collaborative, offre una risposta efficace alle sfide e alle mutevoli esigenze del mondo contemporaneo dello sviluppo software.\\
    Attraverso l'implementazione di Scrum, il nostro team mira a ottenere numerosi benefici positivi che influenzeranno in modo significativo il successo del progetto.
    
\vspace{0.3cm}

\subsubsection*{Vantaggi del Modello Agile e Scrum}
    L'adozione del modello Agile, e in particolare di Scrum, introduce una serie di lati positivi che contribuiranno al raggiungimento dei nostri obiettivi di progetto.
    Alcuni dei principali vantaggi che ci aspettiamo di acquisire includono:

\begin{itemize}
    \item \textbf{Flessibilità e Adattabilità:}
        Lo Scrum consente una rapida risposta ai cambiamenti nei requisiti del cliente, garantendo una maggiore flessibilità durante tutto il ciclo di sviluppo;
    \item \textbf{Collaborazione e Comunicazione:}
        La struttura collaborativa di Scrum promuove una comunicazione aperta e continua tra i membri del team e le parti interessate, migliorando la comprensione reciproca e la condivisione di conoscenze;
        \begin{itemize}
            \item In particolare con l'azienda proponente sono fissati SAL \textit{(Stato Avanzamento Lavori)} ogni due settimane.
        \end{itemize}
    \item \textbf{Consegna Incrementale:}
        Attraverso la pratica di rilasci incrementali, Scrum consente la distribuzione graduale delle funzionalità, fornendo valore al cliente fin dalle prime fasi del progetto;
    \item \textbf{Miglioramento Continuo:}
        Le retrospettive regolari incoraggiano il miglioramento continuo del processo, permettendo al team di identificare e risolvere eventuali problematiche in modo tempestivo.
\end{itemize}

La scelta di adottare il modello Agile con Scrum riflette la nostra dedizione a fornire un prodotto di qualità, rispondendo in modo efficiente ai cambiamenti e alle esigenze del cliente.
