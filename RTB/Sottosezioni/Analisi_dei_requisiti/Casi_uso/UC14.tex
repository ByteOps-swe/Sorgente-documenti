\subsubsection{UC14 -  Selezione dell'intervallo temporale per l'aggregazione delle misurazioni storiche}
\begin{itemize}
    \item \textbf{Attore principale:} Autorità locale;
    \item \textbf{Descrizione:} L'autorità locale ha la capacità di selezionare l'intervallo temporale desiderato per l'aggregazione delle misurazioni nel contesto di un'unica visualizzazione dati. Tale intervallo può essere specificato in termini di secondo, minuto, ora, giorno, mese o anno, al fine di personalizzare la scala temporale dell'analisi.
    \item \textbf{Scenario principale:}
          \begin{enumerate}
             \item L'utente sceglie tra le opzioni di aggregazione un intervallo temporale disponibili tra secondo, minuto, ora, giorno, mese o anno.
             \item Il sistema adatta dinamicamente la rappresentazione dei dati, inclusi i grafici, secondo l'intervallo temporale di aggregazione selezionato.
          \end{enumerate}
    \item \textbf{Precondizioni:}
          \begin{itemize}
              \item  L'utente si trova in un interfaccia per la visualizzazione di uno storico dati.  (UC3)
          \end{itemize}
    \item \textbf{Postcondizioni:}
          \begin{itemize}
              \item  L'utente ottiene una visualizzazione chiara e personalizzata delle misurazioni aggregate sull'intervallo temporale d'appartenenza specificato.
          \end{itemize}
    \item \textbf{User story associata:}
          \begin{itemize}
              \item Come autorità locale, desidero essere in grado di personalizzare l'intervallo temporale di aggregazione delle misurazioni, scegliendo tra le opzioni di secondo, minuto, ora, giorno, mese o anno. Questa funzionalità mi consentirà di adattare la visualizzazione dei dati alle mie specifiche esigenze temporali, agevolando l'analisi dettagliata dei trend e delle variazioni nel corso del tempo.
          \end{itemize}
\end{itemize}