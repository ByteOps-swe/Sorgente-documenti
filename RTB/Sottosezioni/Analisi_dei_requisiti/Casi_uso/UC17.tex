\subsubsection{UC17 - APPLICAZIONE FILTRI MULTIPLI AI WIDGET}
\begin{itemize}
    \item \textbf{Attore principale:} Autorità locale;
   % \item \textbf{Descrizione:} L’autorità locale seleziona i/il sensore/i della quale vuole visionare lo storico dei dati e filtra la visualizzazione ai soli dati compresi tra due valori.
    \item \textbf{Scenario principale:}
          \begin{enumerate}
              \item L'utente seleziona più funzionalità di filtraggio;
              \item Il sistema reimposta la visualizzazione esclusivamente delle misurazioni che rispettano i vincoli.
          \end{enumerate}
    \item \textbf{Precondizioni:}
          \begin{itemize}
              \item  L'utente si trova in un interfaccia per la visualizzazione di un widget (UC3).
          \end{itemize}
    \item \textbf{Postcondizioni:}
          \begin{itemize}
              \item  L'autorità locale visualizza solamente i dati relativi ai filtri applicati.
          \end{itemize}
    \item \textbf{User story associata:}
          \begin{itemize}
            \item Come autorità locale, desidero avere la possibilità di filtrare la visualizzazione delle misurazioni in base a più filtri simultanei. In questo modo sarà possibile effettuare analisi mirate.
          \end{itemize}
\end{itemize}
