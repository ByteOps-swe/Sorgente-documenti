\subsubsection{UC2 - FILTRO VISUALIZZAZIONE DASHBOARD CELLA}
\begin{itemize}
    \item \textbf{Attore principale:} Autorità locale.
    \item \textbf{Precondizioni:}
        \begin{itemize}
            \item Il sistema ha caricato con successo la visualizzazione generale della dashboard (UC1). 
            \item Almeno una cella urbana è presente nella città. 
        \end{itemize}
    \item \textbf{Postcondizioni:}
        \begin{itemize}
            \item L’autorità locale ha una visione aggiornata dello stato di salute della cella urbana tramite widget e grafici interattivi, basati esclusivamente sui dati correlati alla cella selezionata;
            \item Viene visualizzata una mappa dettagliata dei sensori presenti nella cella selezionata;
            \item L'autorità locale ha accesso a un punteggio di salute relativo alla specifica cella urbana. 
        \end{itemize}
    \item \textbf{Scenario principale:}
        \begin{enumerate}
            \item L’autorità locale seleziona la cella urbana di interesse per visualizzare una dashboard dedicata contenente esclusivamente i dati correlati a essa;
            \item Il sistema rielabora dinamicamente le informazioni presenti nella dashboard, considerando solo quelle provenienti dalla cella selezionata.
        \end{enumerate}
    \item \textbf{User story associata:} \\ 
        Come autorità locale, desidero poter selezionare una specifica cella urbana per visualizzare i dati provenienti dai vari sensori presenti in quell’area. Questo mi permetterà di valutare rapidamente lo stato complessivo della cella e prendere decisioni informate per migliorare la qualità e l'efficienza dei servizi di una specifica area della città.
\end{itemize}

