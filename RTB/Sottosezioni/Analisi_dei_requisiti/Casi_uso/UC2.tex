\subsubsection{UC2 - Visualizzazione stato cella}
\begin{itemize}
    \item \textbf{Attore principale:} Autorità locale.
    \item \textbf{Descrizione:} L'autorità locale effettua la selezione della cella, ossia la specifica zona urbana, al fine di visualizzare in tempo reale i dati provenienti da varie tipologie di sensori ubicati nella suddetta area. Ciò permette una valutazione reapida dello stato complessivo della cella.
    \item \textbf{Scenario principale:}
          \begin{enumerate}
              \item L'utente seleziona la cella per la quale desidera visualizzare la dashboard contenente esclusivamente i dati correlati a essa.
          \end{enumerate}
    \item \textbf{Precondizioni:}
          \begin{itemize}
              \item  Almeno un sensore presente nella cella è attivo e ha trasmesso dati;
              \item L'utente di trova  nella piattaforma per la visualizzazione della dashboard sullo stato della città (UC1);
          \end{itemize}
    \item \textbf{Postcondizioni:}
          \begin{itemize}
              \item  L'utente ha una visione aggiornata dello stato di salute della cella tramite widget e grafici interattivi aggiornati in tempo reale sulla base di dati correlati esclusivamente alla cella
                    Inoltre visualizza una mappa dei sensori presenti nella cella e un punteggio di salute relativo alla cella.
          \end{itemize}
    \item \textbf{User story associata:}
          \begin{itemize}
              \item Come autorità locale, desidero poter selezionare una specifica cella urbana sulla piattaforma al fine di visualizzare immediatamente i dati provenienti da vari sensori presenti nell'area. Questo mi permetterà di valutare rapidamente lo stato complessivo della cella e prendere decisioni informate.
          \end{itemize}
\end{itemize}