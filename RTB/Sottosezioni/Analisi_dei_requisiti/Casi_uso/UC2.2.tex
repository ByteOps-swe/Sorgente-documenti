\subsubsection{UC2.2 - Visualizzazione mappa sensori cella}
\begin{itemize}
    \item \textbf{Attore principale:} Autorità locale.
    \item \textbf{Descrizione:} L'autorità locale accede alla dashboard di una cella e visualizza una mappa con una visione dei sensori posizionati nella cella.
    La mappa consentirà di raggiungere la visualizzazione dello storico dei dati trasmessi da ciascun sensore presente nella cella.
    \item \textbf{Scenario principale:}
    \begin{enumerate}
      \item L'utente accede alla piattaforma per la visualizzazione della dashboard di una cella. (UC2)
    \end{enumerate}
\item \textbf{Precondizioni:}
    \begin{itemize}
        \item  Almeno un sensore nella cella ha trasmesso dati durante la giornata;
        \item L'utente si trova nella dashboard relativa alla città (UC1).
    \end{itemize}
    \item \textbf{Postcondizioni:}
          \begin{itemize}
              \item      L'utente ha una visione grafica aggiornata della mappa dei sensori nella cella e della loro tipologia.
          \end{itemize}
    \item \textbf{User story associata:}
          \begin{itemize}
              \item Come autorità locale, voglio essere in grado di visualizzare una mappa contenente solo i sensori attivi e operativi all'interno di una specifica cella. La mappa deve mostrare chiaramente la posizione di ciascun sensore e deve essere etichettata per consentire un riconoscimento immediato della tipologia di ogni sensore.
              Attraverso la mappa interattiva della cella è possibile accedere alla visualizzazione dello storico dei dati trasmessi da uno specifico sensore presente in essas.
          \end{itemize}
\end{itemize}