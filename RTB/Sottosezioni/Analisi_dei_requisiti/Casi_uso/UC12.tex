\subsubsection{UC12 - FILTRO VISUALIZZAZIONE MISURAZIONI IN UN INTERVALLO TEMPORALE}
\begin{itemize}
    \item \textbf{Attore principale:} Autorità locale;
    %\item \textbf{Descrizione:} L’autorità locale seleziona i/il sensore/i della quale vuole visionare lo storico dei dati in formato testuale in ordine decrescente rispetto alle misurazioni del/i sensore/i.
    \item \textbf{Scenario principale:}
          \begin{enumerate}
            \item L'attore seleziona la funzionalità relativa al filtro dei dati per intervallo temporale;
              \item L'utente imposta un intervallo temporale valido.
              \item Il sistema reimposta la visualizzazione con le sole misurazioni effettuate nell'intervallo temporale.
          \end{enumerate}
          \item \textbf{Estensioni:}
          \begin{enumerate}
              \item VISUALIZZAZIONE ERRORE (UC13)
          \end{enumerate}
    \item \textbf{Precondizioni:}
          \begin{itemize}
              \item  L'utente visualizza un widget delle misurazioni(UC3);
          \end{itemize}
    \item \textbf{Postcondizioni:}
          \begin{itemize}
              \item  L'utente ha una visione dello storico dei dati trasmessi nell'intervallo temporale selezionato.
          \end{itemize}
    \item \textbf{User story associata:}
          \begin{itemize}
            \item Come autorità locale, desidero essere in grado di visualizzare i soli dati storici provenienti dai sensori in un intervallo temporale.
          \end{itemize}
\end{itemize}