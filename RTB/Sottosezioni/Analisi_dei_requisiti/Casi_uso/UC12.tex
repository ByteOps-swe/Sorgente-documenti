\subsubsection{UC12 - FILTRO VISUALIZZAZIONE MISURAZIONI IN UN INTERVALLO TEMPORALE}
\begin{itemize}
    \item \textbf{Attore principale:} Autorità locale;
    \item \textbf{Precondizioni:}
        \begin{itemize}
            \item L'autorità locale si trova nell'interfaccia di visualizzazione di un widget associato ad una specifica tipologia di sensori (UC3); 
        \end{itemize}
    \item \textbf{Postcondizioni:}
        \begin{itemize}
            \item L'autorità locale visualizza le sole misurazioni trasmesse da una specifica tipolgia di sensori nell'intervallo temporale selezionato.
        \end{itemize}
    \item \textbf{Scenario principale:}
        \begin{enumerate}
            \item L'autorità locale seleziona la funzionalità relativa al filtro dei dati per intervallo temporale;
            \item L'autorità locale imposta un intervallo temporale;
            \item Il sistema verifica la validità dell'intervallo temporale inserito;
            \item Il sistema aggiorna la visualizzazione mostrando solo le misurazioni effettuate durante l'intervallo temporale. selezionato.
        \end{enumerate}
    \item \textbf{Estensioni:}
    \begin{enumerate}
        \item VISUALIZZAZIONE ERRORE INTERVALLO TEMPORALE NON VALIDO (UC30)
    \end{enumerate}
    \item \textbf{User story associata:} \\
        Come autorità locale, voglio avere la capacità di definire un intervallo temporale personalizzato per poter filtrare le misurazioni trasmesse da una specifica tipologia di sensori. Ciò mi permetterà di analizzare dettagliatamente le misurazioni raccolte in un periodo di interesse specifico.
\end{itemize}