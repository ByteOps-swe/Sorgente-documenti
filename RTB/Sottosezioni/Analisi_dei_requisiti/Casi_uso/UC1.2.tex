\subsubsection{UC1.2 - Visualizzazione mappa sensori}
\begin{itemize}
    \item \textbf{Attore principale:} Autorità locale.
    \item \textbf{Descrizione:} L'autorità locale accede alla dashboard della città e visualizza una mappa con una visione dei sensori posizionati nella città. 
    La mappa consentirà di raggiungere la visualizzazione dello storico dei dati trasmessi da ciascun sensore presente nella città.
    \item \textbf{Scenario principale:}
          \begin{enumerate}
            \item L'utente accede alla piattaforma per la visualizzazione della dashboard della città. (UC1)
        \end{enumerate}
    \item \textbf{Precondizioni:}
          \begin{itemize}
              \item  Almeno un sensore è attivo e ha trasmesso dati durante la giornata;
              %\item L'utente si trova nella dashboard della città. (UC1)
          \end{itemize}
    \item \textbf{Postcondizioni:}
          \begin{itemize}
              \item      L'utente ha una visione grafica aggiornata della mappa dei sensori nella città e della loro tipologia.
          \end{itemize}
    \item \textbf{User story associata:}
          \begin{itemize}
              \item Come autorità locale, voglio essere in grado di visualizzare una mappa interattiva contenente i sensori attivi e operativi all'interno della città. La mappa deve mostrare chiaramente la posizione di ciascun sensore e devono essere etichettati per consentire un riconoscimento immediato della tipologia di ogni sensore.
              Attraverso la mappa interattiva della città è possibile accedere alla visualizzazione dello storico dei dati trasmessi da uno specifico sensore.

          \end{itemize}
\end{itemize}