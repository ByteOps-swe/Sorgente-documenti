\subsection{Funzionalità del prodotto}

Il software di monitoraggio della Smart City è progettato per offrire una serie di funzionalità cruciali per gestire e migliorare le condizioni della città. \\
Le principali funzionalità includono:

\begin{enumerate}
    \item \textbf{Monitoraggio in tempo reale:} Il sistema raccoglie dati in tempo reale dai sensori simulati, fornendo uno stato sempre aggiornato della città.

    \item \textbf{Memorizzazione dei dati:} I dati trasmessi dai sensori vengono memorizzati in un database per garantire la disponibilità a lungo termine e consentire analisi storiche.

    \item \textbf{Visualizzazione attraverso Dashboard:} Gli utenti possono accedere ad una dashboard che offre una visione d’insieme delle condizioni della città in tempo reale. La dashboard è composta da widget e grafici che facilitano la comprensione e l'analisi dei dati.
    
    \item \textbf{Visualizzazione mappa dei sensori:} La dashboard offre una mappa interattiva della città che mostra con precisione la posizione dei sensori, ciascuno dei quali è contraddistinto da un'etichetta che ne indica la tipologia.

    \item \textbf{Visualizzazione punteggio di salute:} Le informazioni ottenute dai simulatori consentono al sistema di calcolare un indice di benessere, valutato su una scala da zero a cento in base all'ultima rilevazione di ciascun sensore. Un punteggio più alto corrisponde a condizioni di vita migliori.

    \item \textbf{Supporto alle decisioni:} L'applicativo fornisce alle autorità locali strumenti per prendere decisioni informate e tempestive sulla gestione delle risorse e sull'implementazione di servizi.
    
    \item \textbf{Analisi dettagliata delle misurazioni:} Il sistema offre strumenti di filtraggio avanzati per esaminare le misurazioni con precisione. Gli utenti possono selezionare intervalli temporali, esplorare gli assi spaziali per visualizzare l'intera rete di sensori cittadini o concentrarsi su aree specifiche. Inoltre, è possibile filtrare le misurazioni in base a intervalli o soglie specifiche di rilevamento.
    
    \item \textbf{Sistema di Notifica per Superamento delle Soglie:} Quando un sensore rileva una misurazione che supera i valori preimpostati come soglia critica, il sistema attiva immediatamente un meccanismo di notifica. Questo avviso viene inviato istantaneamente alle autorità competenti, consentendo loro di essere prontamente informate sull'evento. L'obiettivo principale di questo sistema è garantire una risposta tempestiva ed efficace di fronte a situazioni che richiedono un'azione immediata.
\end{enumerate}