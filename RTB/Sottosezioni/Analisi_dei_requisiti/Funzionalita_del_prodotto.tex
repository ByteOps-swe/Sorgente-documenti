\subsection{Funzionalità del prodotto}

Il software di monitoraggio della Smart City è progettato per offrire una serie di funzionalità cruciali per gestire e migliorare le condizioni della città. \\
Le principali funzionalità includono:

\begin{enumerate}
    \item \textbf{Simulazione di sensori:} Il software consente la simulazione di sensori posizionati in diverse aree della città per raccogliere informazioni su parametri come temperatura, umidità, quantità di polveri sottili nell’aria, traffico, livelli di acqua, stato di riempimento delle isole ecologiche, guasti elettrici, e altro ancora.

    \item \textbf{Monitoraggio in tempo reale:} Il sistema raccoglie dati in tempo reale dai sensori simulati, fornendo uno stato sempre aggiornato della città.

    \item \textbf{Memorizzazione dei dati:} I dati trasmessi dai sensori vengono memorizzati in un database per garantire la disponibilità a lungo termine e consentire analisi storiche.

    \item \textbf{Visualizzazione attraverso Dashboard:} Gli utenti possono accedere ad una dashboard che offre una visione d’insieme delle condizioni della città in tempo reale. La dashboard è composta da widget e grafici che facilitano la comprensione e l'analisi dei dati.

    \item \textbf{Visualizzazione punteggio di salute:} Le informazioni ottenute dai simulatori consentono al sistema di calcolare un indice di benessere, valutato su una scala da zero a cento in base all'ultima rilevazione di ciascun sensore. Un punteggio più alto corrisponde a condizioni di vita migliori.

    \item \textbf{Supporto alle decisioni:} L'applicativo fornisce alle autorità locali strumenti per prendere decisioni informate e tempestive sulla gestione delle risorse e sull'implementazione di servizi.

    \item \textbf{Coinvolgimento dei Cittadini:} Il software può essere utilizzato come strumento per coinvolgere i cittadini nella gestione e nel miglioramento della città, fornendo loro accesso alle informazioni e coinvolgendoli attivamente nelle decisioni.

\end{enumerate}