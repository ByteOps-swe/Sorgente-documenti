\subsection{Copertura dei test}
Rappresenta la misura in cui i \textit{test}\textsubscript{\textit{G}} eseguiti coprono l'ampiezza delle funzionalità del \textit{software}\textsubscript{\textit{G}}, valutando quanto il prodotto sia stato esaminato e valutato rispetto ai \textit{test}\textsubscript{\textit{G}} definiti. Una copertura completa implica che la maggior parte, se non tutti, gli aspetti del \textit{software}\textsubscript{\textit{G}} siano stati testati, offrendo una maggiore fiducia nella sua affidabilità e nella conformità alle aspettative.  
\begin{table}[H]
    \centering
    \begin{tabular}{|C{1.5cm}|L{3cm}|L{4cm}|L{2.5cm}|L{2.5cm}|}
        \hline
        \textbf{Metrica} & \centering{\textbf{Nome}} & \textbf{Descrizione} & \textbf{Valore di accettazione} & \textbf{Valore preferibile} \\
        \hline
        \stepcounter{metriccounter}\textbf{M\arabic{metriccounter}TS} & Test Superati & Percentuale di \textit{test}\textsubscript{\textit{G}} Superati. & $ 100\%$ & $100\%$ \\
        \hline
    \end{tabular}
    \caption{Copertura dei test - Metriche e indici di qualità}
    \label{tab:copertura_qualita_prodotto}
\end{table}