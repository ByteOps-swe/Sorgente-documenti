\subsection{Test di sistema}
Questa sezione illustra i \textit{test}\textsubscript{\textit{G}} di \textit{sistema}\textsubscript{\textit{G}}, i quali mirano a dimostrare la copertura completa dei requisiti identificati nel documento di Analisi dei Requisiti. Di seguito è fornito l'elenco di questi \textit{test}\textsubscript{\textit{G}}:
\\
\begin{longtable}{|c|p{5cm}|>{\raggedright}p{2cm}|c|}
        \hline
        Codice Test & Descrizione & Requisito & Stato Test \\
        \hline
        TS01 & Verificare che l'accesso al \textit{sistema}\textsubscript{\textit{G}} non richieda alcuna procedura di login e che sia immediatamente accessibile all'utente. & RF1 & NI \\
        \hline
        TS02 & Verificare che il prodotto non abbia alcuna sezione o funzionalità di amministrazione o gestione riservata. & RF2 & NI \\
        \hline
        TS03 & Verificare che i simulatori integrati producano dati di misurazione coerenti con l'ambito del \textit{sensore}\textsubscript{\textit{G}} simulato. & RF3 & NI \\
        \hline
        TS04 & Verificare che ogni misurazione inviata dal simulatore contenga l'id del \textit{sensore}\textsubscript{\textit{G}}, un timestamp e la misurazione stessa. & RF4 & NI \\
        \hline
        TS05 & Verificare che il \textit{sistema}\textsubscript{\textit{G}} sia in grado di simulare almeno un \textit{sensore}\textsubscript{\textit{G}} che rilevi la temperatura, espressa in gradi Celsius. & RF5 & NI \\
        \hline
        TS06 & Verificare che il \textit{sistema}\textsubscript{\textit{G}} sia in grado di simulare almeno un \textit{sensore}\textsubscript{\textit{G}} che misuri l'umidità, espressa in percentuale di umidità nell'aria.
        & RF6 & NI \\
        \hline
        TS07 & Verificare che il \textit{sistema}\textsubscript{\textit{G}} sia in grado di simulare almeno un \textit{sensore}\textsubscript{\textit{G}} per rilevare le particelle di polveri sottili nell'aria, esprimendole in microgrammi per metro cubo. & RF7 & NI \\
        \hline
        TS08 & Verificare che il \textit{sistema}\textsubscript{\textit{G}} includa almeno un \textit{sensore}\textsubscript{\textit{G}} per individuare guasti elettrici, segnalando le interruzioni nella fornitura di energia tramite un \textit{bit}\textsubscript{\textit{G}} binario. & RF8 & NI \\
        \hline
        TS09 & Verificare che il \textit{sistema}\textsubscript{\textit{G}} sia in grado di simulare almeno un \textit{sensore}\textsubscript{\textit{G}} per monitorare lo stato di riempimento dei contenitori nelle isole ecologiche, segnalando con un \textit{bit}\textsubscript{\textit{G}} binario se il contenitore è pieno. & RF9 & NI \\
        \hline
        TS10 & Verificare che il \textit{sistema}\textsubscript{\textit{G}} includa almeno un \textit{sensore}\textsubscript{\textit{G}} per le colonnine di ricarica, indicando tramite un \textit{bit}\textsubscript{\textit{G}} binario se la colonnina è occupata o libera. & RF10 & NI \\
        \hline
        TS59 & Verificare che il \textit{sistema}\textsubscript{\textit{G}} contenga almeno un \textit{sensore}\textsubscript{\textit{G}} per il livello dell'acqua, indicando con un \textit{bit}\textsubscript{\textit{G}} binario se il \textit{sensore}\textsubscript{\textit{G}} rileva liquidi o meno. & RF59 & NI \\
        \hline
        TS11 & Verificare che ogni dato generato dai simulatori dei sensori sia strettamente correlato al dato successivo, garantendo una transizione realistica tra le misurazioni. & RF11 & NI \\
        \hline
        TS12 & Verificare che il \textit{sistema}\textsubscript{\textit{G}} memorizzi in modo sicuro e efficiente i dati generati dai sensori, registrando accuratamente ogni misurazione per assicurare l'integrità e la coerenza dei dati. & RF12 & NI \\
        \hline
        TS13 & Verificare che la \textit{piattaforma}\textsubscript{\textit{G}} supporti la visualizzazione di dati provenienti da diversi tipi di sensori, permettendo una rappresentazione corretta e coerente. & RF13 & NI \\
        \hline
        TS14 & Verificare che l'utente possa visualizzare una \textit{dashboard}\textsubscript{\textit{G}} completa dello stato della città tramite l'uso di \textit{widget}\textsubscript{\textit{G}} rappresentanti le misurazioni dei sensori. & RF14 & NI \\
        \hline
        TS15 & Verificare che l'utente possa vedere le misurazioni all'interno dei \textit{widget}\textsubscript{\textit{G}} dedicati alla rappresentazione delle rilevazioni dei sensori in un formato grafico, facilitando la comprensione dei dati. & RF15 & NI \\
        \hline
        TS16 & Verificare che l'utente possa vedere le misurazioni all'interno dei \textit{widget}\textsubscript{\textit{G}} dedicati alla rappresentazione delle rilevazioni dei sensori in un formato testuale. & RF16 & NI \\
        \hline
        TS17 & Verificare che la visualizzazione delle misurazioni in formato testuale segua il formato richiesto: IDSensore, TIMESTAMP, Dato. & RF17 & NI \\
        \hline
        TS18 & Verificare che la \textit{dashboard}\textsubscript{\textit{G}} si aggiorni quasi istantaneamente per riflettere i dati provenienti dai sensori entro un massimo di 10 secondi. & RF18 & NI \\
        \hline
        TS19 & Verificare che la \textit{dashboard}\textsubscript{\textit{G}} mostri \textit{widget}\textsubscript{\textit{G}} distinti per ciascun tipo di \textit{sensore}\textsubscript{\textit{G}} attivo che trasmette dati al \textit{sistema}\textsubscript{\textit{G}}, contenenti le misurazioni in formato grafico. & RF19 & NI \\
        \hline
        TS21 & Verificare che ogni \textit{widget}\textsubscript{\textit{G}} che visualizza le misurazioni includa informazioni sull'identificativo dei sensori che hanno contribuito a quelle misurazioni. & RF21 & NI \\
        \hline
        TS62 & Verificare che il \textit{widget}\textsubscript{\textit{G}} destinato alla rappresentazione temporale delle misurazioni effettuate dai sensori di temperatura offra all'utente la possibilità di visualizzare tali dati in un formato grafico a linee, con una linea corrispondente a ciascun \textit{sensore}\textsubscript{\textit{G}} coinvolto. & RF62 & NI \\
        \hline
        TS23 & Verificare che la \textit{dashboard}\textsubscript{\textit{G}} includa un \textit{widget}\textsubscript{\textit{G}} dedicato alle misurazioni dei sensori di umidità. & RF23 & NI \\
        \hline
        TS63 & Verificare che il \textit{widget}\textsubscript{\textit{G}} destinato alla rappresentazione temporale delle misurazioni effettuate dai sensori di umidità offra all'utente la possibilità di visualizzare tali dati in un formato grafico a linee, con una linea corrispondente a ciascun \textit{sensore}\textsubscript{\textit{G}} coinvolto. & RF63 & NI \\
        \hline
        TS64 & Verificare che il \textit{widget}\textsubscript{\textit{G}} dedicato alla rappresentazione temporale delle misurazioni dei sensori di polveri sottili offra all'utente la possibilità di visualizzare tali dati in un formato grafico a linee, con una linea corrispondente a ciascun \textit{sensore}\textsubscript{\textit{G}} coinvolto. & RF64 & NI \\
        \hline
        TS25 & Verificare che la \textit{dashboard}\textsubscript{\textit{G}} includa un \textit{widget}\textsubscript{\textit{G}} dedicato alle misurazioni dei sensori dei guasti elettrici. & RF25 & NI \\
        \hline
        TS65 & Verificare che il \textit{widget}\textsubscript{\textit{G}} dedicato alla rappresentazione temporale delle misurazioni dei sensori dei guasti elettrici offra all'utente la possibilità di visualizzare tali dati con un grafico a linee per ciascun \textit{sensore}\textsubscript{\textit{G}} coinvolto. & RF65 & NI \\
        \hline
        TS26 & Verificare che la \textit{dashboard}\textsubscript{\textit{G}} includa un \textit{widget}\textsubscript{\textit{G}} dedicato alle misurazioni dei sensori di soglia delle isole ecologiche. & RF26 & NI \\
        \hline
        TS66 & Verificare che il \textit{widget}\textsubscript{\textit{G}} destinato alla rappresentazione temporale delle misurazioni dei sensori di soglia delle isole ecologiche offra all'utente la possibilità di visualizzare tali dati con un grafico a linee per ciascun \textit{sensore}\textsubscript{\textit{G}} coinvolto. & RF66 & NI \\
        \hline
        TS27 & Verificare che la \textit{dashboard}\textsubscript{\textit{G}} includa un \textit{widget}\textsubscript{\textit{G}} dedicato alle misurazioni dei sensori delle colonnine di ricarica. & RF27 & NI \\
        \hline
        TS67 & Verificare che il \textit{widget}\textsubscript{\textit{G}} destinato alla rappresentazione temporale delle misurazioni dei sensori delle colonnine di ricarica offra all'utente la possibilità di visualizzare tali dati con un grafico a linee per ciascun \textit{sensore}\textsubscript{\textit{G}} coinvolto. & RF67 & NI \\
        \hline
        TS60 & Verificare che la \textit{dashboard}\textsubscript{\textit{G}} includa un \textit{widget}\textsubscript{\textit{G}} dedicato alle misurazioni dei sensori del livello dell'acqua. & RF60 & NI \\
        \hline
        TS68 & Verificare che il \textit{widget}\textsubscript{\textit{G}} destinato alla rappresentazione temporale delle misurazioni dei sensori del livello dell'acqua offra all'utente la possibilità di visualizzare tali dati con un grafico a linee per ciascun \textit{sensore}\textsubscript{\textit{G}} coinvolto. & RF68 & NI \\
        \hline
        TS28 & Verificare che la \textit{dashboard}\textsubscript{\textit{G}} della città includa una mappa interattiva che mostri la posizione dei diversi sensori nella città. & RF28 & NI \\
        \hline
        TS29 & Verificare che i sensori sulla mappa siano etichettati in modo chiaro e distinguibile, permettendo il riconoscimento della loro tipologia. & RF29 & NI \\
        \hline
        TS61 & Verificare che i sensori posizionati sulla mappa mostrino l'ultimo valore registrato quando il puntatore del mouse è posizionato sopra di essi. & RF61 & NI \\
        \hline
        TS30 & Verificare che la \textit{dashboard}\textsubscript{\textit{G}} fornisca un \textit{widget}\textsubscript{\textit{G}} con il punteggio di salute relativo alla città basato sui dati aggregati provenienti dai sensori. & RF30 & NI \\
        \hline
        TS31 & Verificare che l'utente possa selezionare una cella specifica della città e visualizzare una \textit{dashboard}\textsubscript{\textit{G}} dedicata contenente esclusivamente sensori, misurazioni e punteggio di salute correlati a essa. & RF31 & NI \\
        \hline
        TS32 & Verificare che l'utente possa filtrare la visualizzazione delle misurazioni di una specifica tipologia di sensori inserendo uno specifico intervallo temporale. & RF32 & NI \\
        \hline
        TS33 & Verificare che il \textit{sistema}\textsubscript{\textit{G}} verifichi la validità dell'intervallo temporale inserito dall'utente. & RF33 & NI \\
        \hline
        TS34 & Verificare che, in caso di inserimento di un intervallo temporale non valido, il \textit{sistema}\textsubscript{\textit{G}} generi una notifica di errore. & RF34 & NI \\
        \hline
        TS35 & Verificare che la notifica di errore relativa all'inserimento di un intervallo temporale non valido richieda all'utente di reinserire date valide. & RF35 & NI \\
        \hline
        TS36 & Verificare che la notifica di errore relativa all'inserimento di un intervallo temporale non valido sia chiara e informativa, indicando il motivo specifico dell'invalidità dell'intervallo temporale. & RF36 & NI \\
        \hline
        TS37 & Verificare che l'utente possa selezionare l'intervallo temporale desiderato (secondo, minuto, ora, giorno, mese, anno) per aggregare le misurazioni in base al periodo di registrazione corrispondente. & RF37 & NI \\
        \hline
        TS38 & Verificare che il \textit{sistema}\textsubscript{\textit{G}} adatti dinamicamente la rappresentazione delle misurazioni secondo l'intervallo temporale di aggregazione selezionato dall'utente. & RF38 & NI \\
        \hline
        TS40 & Verificare che il \textit{sistema}\textsubscript{\textit{G}} verifichi la validità dell'intervallo di rilevamento inserito dall'utente. & RF40 & NI \\
        \hline
        TS41 & Verificare che, in caso di inserimento di un intervallo di rilevamento non valido, il \textit{sistema}\textsubscript{\textit{G}} generi una notifica di errore. & RF41 & NI \\
        \hline
        TS42 & Verificare che la notifica di errore relativa all'inserimento di un intervallo di rilevamento non valido richieda all'utente di reinserire valori validi. & RF42 & NI \\
        \hline
        TS43 & Verificare che la notifica generata in caso di inserimento di un intervallo di rilevamento non valido sia chiara e informativa, indicando specificamente il motivo dell'invalidità (ad esempio, data fine precedente a data inizio, arco temporale precedente o antecedente all’inizio della trasmissione dati). & RF43 & NI \\
        \hline
        TS44 & Verificare che l'utente possa filtrare le misurazioni selezionando uno o più sensori di una specifica categoria e visualizzare solo le misurazioni corrispondenti. & RF44 & NI \\
        \hline
        TS45 & Verificare che l'utente possa filtrare la visualizzazione delle misurazioni di una tipologia di sensori selezionando una o più specifiche celle come criterio di filtro. & RF45 & NI \\
        \hline
        TS46 & Verificare che l'utente possa applicare più filtri simultaneamente per la visualizzazione delle misurazioni di una specifica tipologia di sensori. & RF46 & NI \\
        \hline
        TS47 & Verificare che l'utente possa rimuovere i filtri applicati e ripristinare la visualizzazione senza tali filtri. & RF47 & NI \\
        \hline
        TS48 & Verificare che l'utente possa salvare una misurazione trasmessa da un \textit{sensore}\textsubscript{\textit{G}} in una lista di misurazioni rilevanti. & RF48 Opzionale & NI \\
        \hline
        TS49 & Verificare che il \textit{sistema}\textsubscript{\textit{G}} effettui una verifica per assicurarsi che la misurazione non sia già presente nella lista delle misurazioni rilevanti prima di salvarla. & RF49 Opzionale & NI \\
        \hline
        TS50 & Verificare che l'utente possa visualizzare la lista delle misurazioni rilevanti. & RF50 Opzionale & NI \\
        \hline
        TS69 & Verificare che ogni misurazione nella lista dei rilevanti fornisca correttamente l'identificativo del \textit{sensore}\textsubscript{\textit{G}}. & RF69 Opzionale & NI \\
        \hline
        TS70 & Verificare che ogni misurazione nella lista dei rilevanti fornisca correttamente la tipologia del \textit{sensore}\textsubscript{\textit{G}}. & RF70 Opzionale & NI \\
        \hline
        TS71 & Verificare che ogni misurazione nella lista dei rilevanti fornisca correttamente l'orario e la data di misurazione. & RF71 Opzionale & NI \\
        \hline
        TS72 & Verificare che ogni misurazione nella lista dei rilevanti fornisca correttamente il valore misurato e l'unità di misura corrispondente. & RF72 Opzionale & NI \\
        \hline
        TS51 & Verificare che l'utente possa rimuovere una misurazione specifica dalla lista delle misurazioni rilevanti. & RF51 Opzionale & NI \\
        \hline
        TS52 & Verificare che l'utente riceva notifiche quando i sensori superano determinate soglie di sicurezza. & RF52 & NI \\
        \hline
        TS53 & Verificare che l'utente possa visualizzare correttamente le informazioni richieste per i sensori. & RF53 & NI \\
        \hline
        TS54 & Verificare che l'utente possa visualizzare correttamente l'\textit{ID}\textsubscript{\textit{G}} del \textit{sensore}\textsubscript{\textit{G}}. & RF54 & NI \\
        \hline
        TS55 & Verificare che l'utente possa visualizzare correttamente il tipo di \textit{sensore}\textsubscript{\textit{G}}. & RF55 & NI \\
        \hline
        TS56 & Verificare che l'utente possa visualizzare correttamente la posizione in coordinate dei sensori. & RF56 & NI \\
        \hline
        TS57 & Verificare che l'utente possa visualizzare correttamente la cella di installazione del \textit{sensore}\textsubscript{\textit{G}}. & RF57 & NI \\
        \hline
        TS58 & Verificare che l'utente possa visualizzare correttamente la data di installazione del \textit{sensore}\textsubscript{\textit{G}}. & RF58 & NI \\
        \hline
        TS59 & Verificare che l'utente possa visualizzare correttamente l'unità di misura associata al \textit{sensore}\textsubscript{\textit{G}}. & RF59 & NI \\
        \hline
    \caption{Tabella test di sistema}
    \\textit{label}\textsubscript{\textit{G}}{tab:testsSistema}
\end{longtable}