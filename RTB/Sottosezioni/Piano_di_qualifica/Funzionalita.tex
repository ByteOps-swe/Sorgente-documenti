\subsection{Funzionalità}
La funzionalità di un prodotto si riferisce alla sua capacità di eseguire specifiche azioni o fornire servizi richiesti, come definito nei requisiti. La valutazione della funzionalità misura quanto il prodotto soddisfi i requisiti identificati nell'Analisi dei Requisiti, indicando se il prodotto adempie pienamente alle necessità previste.  
\begin{table}[H]
    \centering
    \begin{tabular}{|C{1.5cm}|L{3cm}|L{4cm}|L{2.5cm}|L{2.5cm}|}
        \hline
        \textbf{Metrica} & \centering{\textbf{Nome}} & \textbf{Descrizione} & \textbf{Valore di accettazione} & \textbf{Valore preferibile} \\
        \hline
        \stepcounter{metriccounter}\textbf{M\arabic{metriccounter}ROS} & Requisiti Obbligatori Soddisfatti & Misura la percentuale di requisiti obbligatori completamente soddisfatti rispetto al totale dei requisiti obbligatori. & $100\%$ & $100\%$ \\
        \hline
        \stepcounter{metriccounter}\textbf{M\arabic{metriccounter}RDS} & Requisiti Desiderabili Soddisfatti & Misura la percentuale di requisiti desiderabili soddisfatti rispetto al totale dei requisiti desiderabili. & $\geq 0\%$ & $ 100\%$ \\
        \hline
        \stepcounter{metriccounter}\textbf{\makecell{\footnotesize M\arabic{metriccounter}ROPZS}} & Requisiti Opzionali Soddisfatti & Misura la percentuale di requisiti opzionali soddisfatti rispetto al totale dei requisiti opzionali. & $\geq 0\%$ & $ 100\%$ \\
        \hline
    \end{tabular}
    \caption{Funzionalità - Metriche e indici di qualità}
    \label{tab:funzionalità_qualita_prodotto}
\end{table}
