\subsection{Qualità per obiettivo}
Le metriche menzionate in precedenza vengono ora categorizzate secondo la struttura delineata nello standard ISO/IEC 12207:1995, che le suddivide nei processi primari, di supporto e organizzativi. Questo adattamento semplificato è stato realizzato per allineare le metriche alle specifiche esigenze del progetto:

\subsubsection{Processi primari}
\paragraph{Analisi dei requisiti}
\hspace{1pt}
    \begin{longtable}{|C{1.5cm}|L{3cm}|L{4cm}|L{2.5cm}|L{2.5cm}|}
        \hline
        \textbf{Metrica} & \textbf{Nome} & \textbf{\makecell{Descrizione}} & \textbf{\makecell{Valore di \\ accettazione}} & \textbf{\makecell{Valore \\ preferibile}} \\
        \hline\textbf{M16PROS} & Percentuale di Requisiti Obbligatori Soddisfatti &  Metrica che valuta quanto del lavoro svolto durante lo sviluppo corrisponda ai requisiti essenziali o obbligatori definiti in fase di analisi dei requisiti.  & $ 100\%$  & $ 100\%$ \\
        \hline
        \textbf{M17PRDS} & Percentuale di Requisiti Desiderabili Soddisfatti & Metrica usata per valutare quanti di quei requisiti, che se integrati arricchirebbero l'esperienza dell'utente o fornirebbero vantaggi aggiuntivi non strettamente necessari, sono stati implementati o soddisfatti nel prodotto. & $\geq 0\%$ & $100\%$ \\
        \hline
        \textbf{M18PRPS} & Percentuale di Requisiti oPzionali Soddisfatti & Metrica per valutare quanti dei requisiti aggiuntivi, non essenziali o di bassa priorità, sono stati implementati o soddisfatti nel prodotto & $\geq 0\%$ & $100\%$ \\
        \hline
    \caption{Analisi dei requisiti - Metriche e indici di qualità.}
    \label{tab:analisi_requisiti_progetto}
\end{longtable}

\paragraph{Progettazione}
La progettazione è un processo in cui vengono definite le specifiche tecniche e architetturali del software che si intende sviluppare. Questo processo traduce i requisiti raccolti durante la fase di acquisizione in un piano strutturato e dettagliato per la creazione del software.

\hspace{1pt}
    \begin{longtable}{|C{1.5cm}|L{3cm}|L{4cm}|L{2.5cm}|L{2.5cm}|}
        \hline
        \textbf{Metrica} & \textbf{Nome} & \textbf{\makecell{Descrizione}} & \textbf{\makecell{Valore di \\ accettazione}} & \textbf{\makecell{Valore \\ preferibile}} \\
        \hline
        \textbf{M24ATC} & Accoppiamento Tra Classi &   Misura della dipendenza e dell'interconnessione tra le classi all'interno di un sistema software.   & $\leq 4$  & $\leq 2$ \\
        \hline
        \textbf{M25PG} & Profondità delle Gerarchie & Metrica che misura il numero di livelli tra una classe base (superclasse) e le sue sottoclassi (classi derivate). & $\leq 5$  & $\leq 3$ \\
        \hline
        \textbf{M27FU} & Facilità di Utilizzo & Metrica che misura l'usabilità di un sistema software & $\leq 7$ \textit{click}  & $\leq 5$ \textit{click} \\
        \hline
    \caption{Progettazione - Metriche e indici di qualità.}
    \label{tab:progettazione_progetto}
\end{longtable}

\paragraph{Fornitura}
Processo che consiste nel decidere procedure e risorse
adatte allo sviluppo del progetto.

\hspace{1pt}
    \begin{longtable}{|C{1.5cm}|L{3cm}|L{4cm}|L{2.5cm}|L{2.5cm}|}
        \hline
        \textbf{Metrica} & \textbf{Nome} & \textbf{\makecell{Descrizione}} & \textbf{\makecell{Valore di \\ accettazione}} & \textbf{\makecell{Valore \\ preferibile}} \\
        \hline
        \textbf{M2EAC} & Estimated at Completion &  Misura il costo realizzativo stimato per terminare il progetto.  & $\pm 5\%$ rispetto al preventivo & Pari al preventivo \\
        \hline
        \textbf{M3CPI} & Cost Performance Index & Misura il rapporto tra il valore del lavoro effettivamente svolto ed il 
        costo reale del lavoro fino al periodo di riferimento. & $\pm 10\%$ & $0\%$ \\
        \hline
        \textbf{M4BV} & Budget Variance & misura la differenza percentuale di budget tra quanto previsto nella 
        pianificazione di un periodo e l’effettiva realizzazione. & $\geq -10\%$ & $0\%$ \\
        \hline
        \textbf{M5AC} & Actual Cost & Misura i costi effettivamente sostenuti dall’inizio del progetto fino 
        all’attualità.
         & $\geq 0 $ & $ \leq$ EAC  \\
        \hline
        \textbf{M6SV} & Schedule Variance & Indica in percentuale quanto si è in anticipo o in ritardo con le attività
        pianificate. & $\geq -10\%$ & $0\%$ \\
        \hline
        \textbf{M7EV} & Earned Value & Valore del lavoro effettivamente svolto fino a quel periodo.
        & $\geq 0 $ & $\leq$ EAC  \\
        \hline
        \textbf{M8PV} & Planned Value & Stima la somma dei costi realizzativi delle attività imminenti periodo 
        per periodo. & $\geq 0  $ & $ \leq$ BAC  \\
        \hline
        \textbf{M9ETC} & Estimate to Complete &  Stima i costi realizzativi fino alla fine del progetto. & $\geq 0  $ & $ \leq$ EAC  \\
        \hline
    \caption{Fornitura - Metriche e indici di qualità.}
    \label{tab:controllo_progetto}
\end{longtable}

\paragraph{Codifica}
La fase di codifica è essenziale in quanto trasforma il progetto e le specifiche del software in istruzioni comprensibili dalla macchina, permettendo al prodotto software di prendere vita e funzionare effettivamente.

\hspace{1pt}
    \begin{longtable}{|C{1.5cm}|L{3cm}|L{4cm}|L{2.5cm}|L{2.5cm}|}
        \hline
        \textbf{Metrica} & \textbf{Nome} & \textbf{Descrizione} & \textbf{Valore di accettazione} & \textbf{Valore preferibile} \\
        \hline
        \textbf{M28CCM} & Complessità Ciclomatica per Metodo & Rappresenta la complessità di un metodo in base ai percorsi possibili & $\leq 5$ & $\leq 3$ \\
        \hline
        \textbf{M29PM} & Parametri per Metodo & Numero massimo di parametri per metodo & $\leq 6$ & $\leq 5$ \\
        \hline
        \textbf{M30APC} & Attributi Per Classe & Misura il numero massimo di attributi per classe. & $\leq 6$ & $\leq 4$ \\
        \hline
        \textbf{M1LCM} & Linee di Codice per Metodo & Limite massimo di linee di codice per metodo & $\leq 30$ & $\leq 20$ \\
        \hline
        \textbf{M26TMR} & Tempo Medio di Risposta & Metrica che misura quanto è efficiente e reattivo un sistema software & $\leq 10$ \textit{secondi}  & $\leq 4$ \textit{secondi} \\
        \hline
    \caption{Codifica - Metriche e indici di qualità.}
    \label{tab:metriche}
\end{longtable}




\subsubsection{Processi di supporto}

\paragraph{Documentazione}
\hspace{1pt}
    \begin{longtable}{|C{1.5cm}|L{3cm}|L{4cm}|L{2.5cm}|L{2.5cm}|}
        \hline
        \textbf{Metrica} & \textbf{Nome} & \textbf{Descrizione} & \textbf{Valore di accettazione} & \textbf{Valore preferibile} \\
        \hline
        \textbf{M19CO} & Correttezza Ortografica & Misura la presenza di errori ortografici nei documenti & $0$ & $0$ \\
        \hline
        \textbf{M20IG} & Indice Gulpease & Misura la leggibilità di un testo in base alla lunghezza delle parole e delle frasi & $\geq 40$ & $\geq 60$ \\
        \hline
    \caption{Documentazione - Metriche e indici di qualità.}
    \label{tab:metriche_testo}
\end{longtable}

\paragraph{Verifica}
\hspace{1pt}
\begin{longtable}{|C{1.5cm}|L{3cm}|L{4cm}|L{2.5cm}|L{2.5cm}|}
    \hline
    \textbf{Metrica} & \textbf{Nome} & \textbf{Descrizione} & \textbf{Valore di accettazione} & \textbf{Valore preferibile} \\
    \hline
    \textbf{M15CC} & Code Coverage & Percentuale del codice sorgente coperto dai test & $\geq 80\%$ & $100\%$ \\
    \hline
    \textbf{M16PCTS} & Percentuale di Casi di Test Superati & Percentuale di casi di test superati & $\geq 80\%$ & $100\%$ \\
    \hline
    \textbf{M17PCTF} & Percentuale di Casi di Test Falliti & Percentuale di casi di test falliti & $\leq 20\%$ & $0\%$ \\
    \hline
\caption{Verifica - Metriche e indici di qualità.}
\label{tab:metriche_testo}
\end{longtable}


\paragraph{Gestione dei rischi}
\hspace{1pt}
    \begin{longtable}{|C{1.5cm}|L{3cm}|L{4cm}|L{2.5cm}|L{2.5cm}|}
        \hline
      \textbf{Metrica} & \textbf{Nome} & \textbf{Descrizione} & \textbf{Valore di accettazione} & \textbf{Valore preferibile} \\
      \hline
      \textbf{M11RNP}    & Rischi non previsti   & Misura il numero di rischi non previsti nel corso del progetto. & $\leq 5$ &   $0$ \\
      \hline
    \caption{Gestione dei rischi - Metriche e indici di qualità.}
    \label{tab:tabella2}
\end{longtable}


\paragraph{Gestione della qualità}
\hspace{1pt}
    \begin{longtable}{|C{1.5cm}|L{3cm}|L{4cm}|L{2.5cm}|L{2.5cm}|}
        \hline
        \textbf{Metrica} & \textbf{Nome} & \textbf{Descrizione} & \textbf{Valore di accettazione} & \textbf{Valore preferibile} \\
        \hline
        \textbf{M1PMS} & Percentuale di Metriche Soddisfatte & Misura che valuta quante metriche che sono state definite sono state effettivamente adottate o soddisfatte & $\geq 80\%$ & $100\%$ \\
        \hline
    \caption{Gestione della qualità - Metriche e indici di qualità.}
    \label{tab:gestione_metriche_testo}
\end{longtable}



\subsubsection{Processi organizzativi}

\paragraph{Pianificazione}
La pianificazione organizza obiettivi, risorse e tempistiche per guidare il successo di un progetto.

\hspace{1pt}
    \begin{longtable}{|C{1.5cm}|L{3cm}|L{4cm}|L{2.5cm}|L{2.5cm}|}
        \hline
        \textbf{Metrica} & \textbf{Nome} & \textbf{Descrizione} & \textbf{Valore di accettazione} & \textbf{Valore preferibile} \\
        \hline
        \textbf{M6SV} & Schedule Variance & Indica in percentuale quanto si è in anticipo o in ritardo con le attività
        pianificate. & $\geq -10\%$ & $0\%$ \\
        \hline
        \textbf{M4BV} & Budget Variance & misura la differenza percentuale di budget tra quanto previsto nella 
        pianificazione di un periodo e l’effettiva realizzazione. & $\geq -10\%$ & $0\%$ \\
        \hline
        \textbf{M12VR} & Variazione dei Requisiti & Misura la variazione nei requisiti dal momento della pianificazione & $\leq 3$ & $0$ \\
        \hline
        \textbf{M22IF} & Implementazione delle Funzionalità & Misura qual è la quantità di funzionalità pianificate che sono state implementate & $ 100\%$ & $ 100\%$ \\
        \hline
    \caption{Pianificazione - Metriche e indici di qualità.}
    \label{tab:metriche_pianificazione}
\end{longtable}

\paragraph{Miglioramento}
\hspace{1pt}
    \begin{longtable}{|C{1.5cm}|L{3cm}|L{4cm}|L{2.5cm}|L{2.5cm}|}
        \hline
        \textbf{Metrica} & \textbf{Nome} & \textbf{Descrizione} & \textbf{Valore di accettazione} & \textbf{Valore preferibile} \\
        \hline
        \textbf{M21DE} & Densità Errori & Misura la leggibilità di un testo in base alla lunghezza delle parole e delle frasi & $\leq 10\%$ & $ 0\%$ \\
        \hline
    \caption{Miglioramento - Metriche e indici di qualità.}
    \label{tab:metriche_miglioramento}
\end{longtable}