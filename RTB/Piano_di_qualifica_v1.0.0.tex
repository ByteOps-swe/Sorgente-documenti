\documentclass{article}
\usepackage[utf8]{inputenc}
\usepackage[default]{raleway}
\usepackage{titlesec, comment, tabularx, makecell, listings, array, setspace, geometry, graphicx, xcolor, xparse, fancyvrb, relsize, fancyhdr, booktabs, ragged2e, hyperref, float}
\usepackage[hypcap=false]{caption}
\usepackage{todonotes}
\usepackage{longtable}

\geometry{a4paper, margin=1in}
%\geometry{a4paper, left=2cm, right=2cm, top=2cm, bottom=2.5cm}
\renewcommand{\headrulewidth}{0pt}

% Definisci uno stile per i comandi git
\definecolor{light-gray}{gray}{0.92}
\definecolor{darkgreen}{rgb}{0.0, 0.5, 0.0}
\definecolor{navyblue}{rgb}{0.0, 0.0, 0.8}

\lstdefinestyle{code}{
    frame=single,
    framesep=1mm,
    rulecolor=\color{light-gray},
    backgroundcolor=\color{light-gray},
    basicstyle=\ttfamily,
}
% ----------------------------- Definizione tabella ---------------------------

\newcolumntype{C}[1]{>{\centering\arraybackslash}m{#1}}
\newcolumntype{L}[1]{>{\RaggedRight\arraybackslash}m{#1}}

% ------------------------------Metadati indice --------------------------------
\title{\textbf{\fontsize{28}{6}\selectfont Indice}}
\author{\fontsize{14}{6}\selectfont ByteOps}
\date{}

% -----------------------------Creazione footer --------------------------------

\pagestyle{fancy}
\fancyhf{}
\renewcommand{\footrulewidth}{0.4pt} 
\lfoot{
    \parbox[c]{2cm}{\includegraphics[width=2cm]{../Images/logo.png}}
    \textcolor[RGB]{120, 120, 120}{$\cdot$ Piano di qualifica}
}
\rfoot{\thepage}

% --------------------------Modifica formato hyperlinks ------------------------

\hypersetup{
    colorlinks=true,
    linkcolor=darkgreen,
    urlcolor=navyblue,
    filecolor=black,
    pdftitle={Piano di qualifica},
    pdfpagemode=FullScreen,
}

% ------------------------------- Valore sotto-paragrafi indice --------------------------------------

\setcounter{secnumdepth}{4}
\setcounter{tocdepth}{4}

\titleformat{\section}
{\normalfont\huge\bfseries}{\thesection}{0.2cm}{}
\titlespacing*{\paragraph}{0pt}{0.5cm}{0.1cm}

\titleformat{\subsection}
{\normalfont\Large\bfseries}{\thesubsection}{0.2cm}{}
\titlespacing*{\paragraph}{0pt}{0.5cm}{0.1cm}

\titleformat{\subsubsection}
{\normalfont\large\bfseries}{\thesubsubsection}{0.2cm}{}
\titlespacing*{\paragraph}{0pt}{0.5cm}{0.1cm}

\titleformat{\paragraph}
{\normalfont\normalsize\bfseries}{\theparagraph}{0.2cm}{}
\titlespacing*{\paragraph}{0pt}{0.5cm}{0.1cm}

% ------------------------------- Front Page ---------------------------------------

\begin{document}
\pagestyle{fancy}
\begin{center}
    \includegraphics[width = 0.7\textwidth]{../Images/logo.png} \\
    \vspace{0.2cm}
    \textcolor[RGB]{60, 60, 60}{\textit{ByteOps.swe@gmail.com}} \\
    \vspace{2cm}
    \fontsize{16}{6}\selectfont Piano di qualifica \\
    \vspace{0.5cm}
\end{center}

\section*{Informazioni documento}
\def\arraystretch{1.2}
\begin{tabular}{>{\raggedleft\arraybackslash}p{0.2\textwidth}|>{\raggedright\arraybackslash}p{0.6\textwidth}c}
    \hline
    \addlinespace 
    \textbf{Redattori}    & A. Barutta\\ & R.Smanio\\ & E.Hysa\\ & L. Skenderi\\ & F.Pozza \vspace{10pt} \\
    \textbf{Verificatori} & E. Hysa\\ & A.Barutta\\ & N.Preto\\ & D.Diotto\\ & L.Skenderi \vspace{10pt} \\
    \textbf{Destinatari}  & ByteOps\\ & T. Vardanega   \\ & R. Cardin \vspace{10pt} \\
\end{tabular}
\pagebreak 

% ------------------------- Changelog ----------------------------
\section*{Registro delle modifiche}
\begin{longtable}{|C{1.5cm}|C{2.1cm}|C{2cm}|C{2cm}|C{5cm}|}
    \hline
    \textbf{Versione} & \textbf{Data}   & \textbf{Autore}                         & \textbf{Verificatore} & \textbf{Dettaglio} \\
    \hline \hline
    \endhead % Questo comando indica la fine dell'header ripetuto

    \label{Git_Action_Version}
    1.0.0
    & 13/02/2024      & \makecell{E. Hysa}      & F. Pozza & \makecell{Completata sezione\\ Cruscotto.} \\
    \hline 
    0.3.0
    & 05/01/2024      & \makecell{N. Preto}      & L. Skenderi & \makecell{Completamento sezione\\Specifica dei test.} \\
    \hline
    0.2.3
    & 02/01/2024      & \makecell{D. Diotto}      & E. Hysa & \makecell{Correzioni su Test\\di sistema e\\Test di accettazione.} \\
    \hline
    0.2.2
    & 30/12/2023      & \makecell{E. Hysa}      & L. Skenderi & \makecell{Iniziale stesura\\Test di accettazione.} \\
    \hline
    0.2.1
    & 29/12/2023      & \makecell{N. Preto}      & L. Skenderi & \makecell{Iniziale stesura\\Test di sistema.} \\
    \hline
    0.2.0
    & 27/12/2023      & \makecell{N. Preto}      & E. Hysa & \makecell{Correzione e\\riadattamenti della\\sezione degli obiettivi\\metrici.} \\
    \hline
    0.1.0
    & 12/12/2023      & \makecell{D. Diotto}      & A. Barutta & \makecell{Completati contenuti\\sezione Obiettivi metrici\\di qualità.} \\
    \hline
    0.0.4
    & 09/12/2023      & \makecell{R. Smanio}      & D. Diotto & \makecell{Aggiunto contenuto\\per la sottosezione\\Qualità di prodotto.} \\
    \hline
    0.0.3
    & 05/12/2023      & \makecell{D. Diotto}      & R. Smanio & \makecell{Prima stesura sottosezione\\Qualità di processo,\\Qualità di Prodotto\\e Qualità per obiettivo.} \\
    \hline
    0.0.2
    & 17/11/2023      & \makecell{A. Barutta}      & E. Hysa & \makecell{Finalizzata scrittura\\sottosez. Finalità documento\\ e Glossario.} \\
    \hline
    0.0.1
    & 15/11/2023      & \makecell{A. Barutta}      & E. Hysa & \makecell{Iniziale scrittura sezione\\Introduzione.} \\
    \hline
\end{longtable}

\pagebreak

% ------------------------- Generazione automatica indice ----------------------
\setstretch{1.5}
\maketitle
\thispagestyle{fancy}
{
    \hypersetup{linkcolor=black}
    \tableofcontents
}
\setstretch{1.2}
\pagebreak


% ---------------------------- Inizio valutazione -------------------------------
\flushleft
\section{Introduzione}
\subsection{Scopo del Manuale}
Il presente manuale è concepito per fornire un supporto completo agli utenti Autorità Locale nell'utilizzo efficace del software, consentendo loro di sfruttare appieno tutte le sue funzionalità al fine di garantire un'esperienza ottimale. \\
Poiché l'installazione del software è gestita da personale tecnico specializzato, questo manuale non include istruzioni dettagliate per l'installazione, ma si concentra piuttosto sui passaggi necessari per utilizzare il software una volta installato correttamente.
\subsection{Scopo del Prodotto}
L'obiettivo del progetto è quello di creare un'applicazione web per il monitoraggio di una "Smart City", consentendo un controllo completo sul suo stato di salute. Ciò permetterà di prendere decisioni rapide ed efficaci, oltre ad analizzare gli effetti delle azioni intraprese.\\
La piattaforma è in grado di fornire informazioni chiare e in tempo reale sullo stato della città tramite una dashboard Grafana, che mette a disposizione tutti gli strumenti necessari per l'analisi delle misurazioni provenienti dai sensori. \\
Come detto in precedenza, questa piattaforma è destinata alle autorità cittadine desiderose di ottenere una visione globale della situazione urbana, fornendo informazioni chiare e in tempo reale sullo stato della città.

\subsection{Accesso alla piattaforma}
La piattaforma è presentata come una web-application accessibile esclusivamente agli utenti autorizzati. L'accesso al servizio avviene tramite un browser web, senza richiedere l'installazione di alcun software aggiuntivo sul dispositivo dell'utente. Al fine di garantire la massima sicurezza e riservatezza dei dati, l'accesso è limitato esclusivamente agli utenti in possesso del link e delle credenziali di accesso, le quali vengono fornite dal team amministrativo o da personale autorizzato. Una volta ottenuto il link e le credenziali, gli utenti possono accedere alla web-application da qualsiasi dispositivo connesso a Internet, garantendo un'esperienza di utilizzo flessibile e accessibile ovunque si trovi
\subsection{Glossario}
Per evitare possibili ambiguità che potrebbero sorgere durante la lettura dei documenti,
alcuni termini utilizzati sono stati inseriti nel documento \textit{Glossario v 2.0.0 }. \\
Sarà possibile individuare il riferimento al Glossario per mezzo
di una G a pedice del termine considerato ambiguo.
\subsection{Riferimenti}
\subsubsection{Riferimenti informativi}
    \begin{itemize}
        \item \href {https://www.math.unipd.it/~tullio/IS-1/2023/Progetto/C6.pdf} {Capitolato d'appalto C6 - InnovaCity}
        \item \href{https://www.math.unipd.it/~tullio/IS-1/2023/Dispense/T4.pdf} {Slide del corso di Ingegneria del Software - Gestione di progetto}
        \item \href{https://www.math.unipd.it/~tullio/IS-1/2023/Dispense/T2.pdf} {Slide del corso di Ingegneria del Software - Ciclo di vita del software}
    \end{itemize}

\subsubsection{Riferimenti normativi}
    \begin{itemize}
    \item Norme di progetto
    \item \href {https://www.math.unipd.it/~tullio/IS-1/2023/Dispense/PD2.pdf} {Regolamento del progetto didattico}
    \end{itemize}


\newcounter{metriccounter}
\setcounter{metriccounter}{1} 

\section{Obiettivi metrici di qualità}
\subsection{Introduzione}
Durante l'implementazione del progetto, i processi adottano criteri di qualità al fine di perseguire un miglioramento continuo finalizzato a soddisfare appieno tali criteri.
Nel contesto di questo progetto, si è scelto di utilizzare il metodo PDCA (Plan-Do-Check-Act) e lo standard ISO/IEC 15504 noto anche come SPICE (Software Process Improvement and Capability Determination).

Ciò consente di garantire un'implementazione dei processi che, basandosi sull'esperienza accumulata, si orienta al miglioramento costante e assicura al cliente il raggiungimento di un prodotto di elevata qualità. Nella presente sezione, vengono delineati i livelli di qualità accettabili e ottimali in base alle metriche definite nel documento "Norme di progetto".


\subsection{Qualità di processo}

\hspace{1pt}
        \begin{longtable}{|C{1.5cm}|L{3cm}|L{4cm}|L{2.5cm}|L{2.5cm}|}
        \hline
        \textbf{Metrica} & \textbf{Nome} & \textbf{Descrizione} & \textbf{Valore di accettazione} & \textbf{Valore preferibile} \\
        \hline
        \textbf{M\arabic{metriccounter}PMS} & Percentuale di Metriche Soddisfatte & Misura che valuta quante metriche che sono state definite sono state effettivamente adottate o soddisfatte & $\geq 80\%$ & $100\%$ \\
        \hline
        \stepcounter{metriccounter}
        \textbf{M\arabic{metriccounter}EAC} & Estimated at Completion &  Misura il costo realizzativo stimato per terminare il progetto.  & $\pm 5\%$ rispetto al preventivo & Pari al preventivo \\
        \hline
        \stepcounter{metriccounter}\textbf{M\arabic{metriccounter}CPI} & Cost Performance Index & Misura il rapporto tra il valore del lavoro effettivamente svolto ed il 
        costo reale del lavoro fino al periodo di riferimento. & $\pm 10\%$ & $0\%$ \\
        \hline
        \stepcounter{metriccounter}\textbf{M\arabic{metriccounter}BV} & Budget Variance & Misura la differenza percentuale di budget tra quanto previsto nella 
        pianificazione di un periodo e l’effettiva realizzazione. & $\geq -10\%$ & $0\%$ \\
        \hline
        \stepcounter{metriccounter}\textbf{M\arabic{metriccounter}AC} & Actual Cost & Misura i costi effettivamente sostenuti dall’inizio del progetto fino 
        all’attualità.
         & $\geq 0 $ & $ \leq$ EAC  \\
        \hline
        \stepcounter{metriccounter}\textbf{M\arabic{metriccounter}SV} & Schedule Variance & Indica in percentuale quanto si è in anticipo o in ritardo con le attività
        pianificate. & $\geq -10\%$ & $0\%$ \\
        \hline
        \stepcounter{metriccounter}\textbf{M\arabic{metriccounter}EV} & Earned Value & Valore del lavoro effettivamente svolto fino a quel periodo.
        & $\geq 0 $ & $\leq$ EAC  \\
        \hline
        \stepcounter{metriccounter}\textbf{M\arabic{metriccounter}PV} & Planned Value & Stima la somma dei costi realizzativi delle attività imminenti periodo 
        per periodo. & $\geq 0  $ & $ \leq$ BAC  \\
        \hline
        \stepcounter{metriccounter}\textbf{M\arabic{metriccounter}ETC} & Estimate to Complete &  Stima i costi realizzativi fino alla fine del progetto. & $\geq 0  $ & $ \leq$ EAC  \\
        \hline
        \stepcounter{metriccounter}\stepcounter{metriccounter}\textbf{M\arabic{metriccounter}RNP}    & Rischi non previsti   & Misura il numero di rischi non previsti nel corso del progetto. & $\leq 5$ &   $0$ \\
        \hline
        \stepcounter{metriccounter}\textbf{M\arabic{metriccounter}VR} & Variazione dei Requisiti & Misura la variazione nei requisiti dal momento della pianificazione & $\leq 3$ & $0$ \\
        \hline
        \stepcounter{metriccounter}\textbf{M\arabic{metriccounter}CC} & Code Coverage & Percentuale del codice sorgente coperto dai test & $\geq 80\%$ & $100\%$ \\
        \hline
        \stepcounter{metriccounter}\textbf{M\arabic{metriccounter}PCTS} & Percentuale di Casi di Test Superati & Percentuale di casi di test superati & $\geq 80\%$ & $100\%$ \\
        \hline
        \stepcounter{metriccounter}\textbf{M\arabic{metriccounter}PCTF} & Percentuale di Casi di Test Falliti & Percentuale di casi di test falliti & $\leq 20\%$ & $0\%$ \\
        \hline
        \caption{Metriche per la qualità dei processi}
        \label{tab:qualità_processo_progetto}
        \end{longtable}

\subsection{Qualità di prodotto}

\subsubsection{Caratteristica di qualità: Funzionalità}
\hspace{1pt}
    \begin{longtable}{|C{1.5cm}|L{3cm}|L{4cm}|L{2.5cm}|L{2.5cm}|}
        \hline
        \textbf{Metrica} & \textbf{Nome} & \textbf{Descrizione} & \textbf{Valore di accettazione} & \textbf{Valore preferibile} \\
        \hline
        \stepcounter{metriccounter}\textbf{M\arabic{metriccounter}PROS} & Percentuale di Requisiti Obbligatori Soddisfatti &  Metrica che valuta quanto del lavoro svolto durante lo sviluppo corrisponda ai requisiti essenziali o obbligatori definiti in fase di analisi dei requisiti.  & $ 100\%$  & $ 100\%$ \\
        \hline
        \stepcounter{metriccounter}\textbf{M\arabic{metriccounter}PRDS} & Percentuale di Requisiti Desiderabili Soddisfatti & Metrica usata per valutare quanti di quei requisiti, che se integrati arricchirebbero l'esperienza dell'utente o fornirebbero vantaggi aggiuntivi non strettamente necessari, sono stati implementati o soddisfatti nel prodotto. & $\geq 0\%$ & $100\%$ \\
        \hline
        \stepcounter{metriccounter}\textbf{M\arabic{metriccounter}PRPS} & Percentuale di Requisiti oPzionali Soddisfatti & Metrica per valutare quanti dei requisiti aggiuntivi, non essenziali o di bassa priorità, sono stati implementati o soddisfatti nel prodotto. & $\geq 0\%$ & $100\%$ \\
        \hline
        \stepcounter{metriccounter}\textbf{M\arabic{metriccounter}IF} & Implementazione delle Funzionalità & Misura qual è la quantità di funzionalità pianificate che sono state implementate. & $ 100\%$ & $ 100\%$ \\
        \hline
        \caption{Funzionalità - Metriche e indici di qualità.}
        \label{tab:metriche_funzionalità_testo}
    \end{longtable}

\subsubsection{Caratteristica di qualità: Affidabilità}
\hspace{1pt}
    \begin{longtable}{|C{1.5cm}|L{3cm}|L{4cm}|L{2.5cm}|L{2.5cm}|}
            \hline
            \textbf{Metrica} & \textbf{Nome} & \textbf{Descrizione} & \textbf{Valore di accettazione} & \textbf{Valore preferibile} \\
            \hline
            \stepcounter{metriccounter} \textbf{M\arabic{metriccounter}CO} & Correttezza Ortografica & Misura la presenza di errori ortografici nei documenti. & $0$ & $0$ \\
            \hline
            \stepcounter{metriccounter}\textbf{M\arabic{metriccounter}IG} & Indice Gulpease & Misura la leggibilità di un testo in base alla lunghezza delle parole e delle frasi. & $\geq 40$ & $\geq 80$ \\
            \hline
            \stepcounter{metriccounter}\textbf{M\arabic{metriccounter}DE} & Densità Errori & Percentuale rappresentante la resistenza a malfunzionamenti del prodotto.  & $\leq 10\%$ & $ 0\%$ \\
            \hline
            \caption{Affidabilità - Metriche e indici di qualità.}
        \label{tab:metriche_affidabilità_testo}
    \end{longtable}

\subsubsection{Caratteristica di qualità: Manutenibilità}
\hspace{1pt}
    \begin{longtable}{|C{1.8cm}|L{3cm}|L{4cm}|L{2.5cm}|L{2.5cm}|}
                \hline
                \textbf{Metrica} & \textbf{Nome} & \textbf{Descrizione} & \textbf{Valore di accettazione} & \textbf{Valore preferibile} \\
                \hline
                \stepcounter{metriccounter}\textbf{M\arabic{metriccounter}ATC} & Accoppiamento Tra Classi &   Misura della dipendenza e dell'interconnessione tra le classi all'interno di un sistema software.   & $\leq 4$  & $\leq 2$ \\
                \hline
                \stepcounter{metriccounter}\textbf{M\arabic{metriccounter}MCCM} & Complessità Ciclomatica per Metodo & Rappresenta la complessità di un metodo in base ai percorsi possibili. & $\leq 5$ & $\leq 3$ \\
                \hline
                \stepcounter{metriccounter}\textbf{M\arabic{metriccounter}PM} & Parametri per Metodo & Numero massimo di parametri per metodo. & $\leq 6$ & $\leq 5$ \\
                \hline
                \stepcounter{metriccounter}\textbf{M\arabic{metriccounter}APC} & Attributi Per Classe & Misura il numero massimo di attributi per classe. & $\leq 6$ & $\leq 4$ \\
                \hline
                \stepcounter{metriccounter} \textbf{M\arabic{metriccounter}LCM} & Linee di Codice per Metodo & Limite massimo di linee di codice per metodo. & $\leq 30$ & $\leq 20$ \\
                \hline
                \stepcounter{metriccounter}\textbf{M\arabic{metriccounter}PG} & Profondità delle Gerarchie & Metrica che misura il numero di livelli tra una classe base (superclasse) e le sue sottoclassi (classi derivate). & $\leq 5$  & $\leq 3$ \\
                \hline
                \caption{Manutenibilità - Metriche e indici di qualità.}
        \label{tab:metriche_manutenibilità_testo}
    \end{longtable}


\subsubsection{Caratteristica di qualità: Efficienza}
\hspace{1pt}
        \begin{longtable}{|C{1.5cm}|L{3cm}|L{4cm}|L{2.5cm}|L{2.5cm}|}
                    \hline
                    \textbf{Metrica} & \textbf{Nome} & \textbf{Descrizione} & \textbf{Valore di accettazione} & \textbf{Valore preferibile} \\
                    \hline
                    \stepcounter{metriccounter}\textbf{M\arabic{metriccounter}TMR} & Tempo Medio di Risposta & Metrica che misura quanto è efficiente e reattivo un sistema software. & $\leq 10$ \textit{secondi}  & $\leq 4$ \textit{secondi} \\
                    \hline
                    \caption{Efficienza - Metriche e indici di qualità.}
        \label{tab:metriche_efficienza_testo}
    \end{longtable}

\subsubsection{Caratteristica di qualità: Usabilità}
\hspace{1pt}
            \begin{longtable}{|C{1.5cm}|L{3cm}|L{4cm}|L{2.5cm}|L{2.5cm}|}
                        \hline
                        \textbf{Metrica} & \textbf{Nome} & \textbf{Descrizione} & \textbf{Valore di accettazione} & \textbf{Valore preferibile} \\
                        \hline
                        \stepcounter{metriccounter}\textbf{M\arabic{metriccounter}FU} & Facilità di Utilizzo & Metrica che misura l'usabilità di un sistema software. & $\leq 7$ \textit{click}  & $\leq 5$ \textit{click} \\
                        \hline
                        \stepcounter{metriccounter}\textbf{M\arabic{metriccounter}TA} & Tempo di Apprendimento & Misura il tempo massimo richiesto per apprendere l'utilizzo del prodotto. & $\leq 15$ \textit{minuti}  & $\leq 10$ \textit{minuti} \\
                        \hline
                        \caption{Usabilità - Metriche e indici di qualità.}
            \label{tab:metriche_usabilità_testo}
        \end{longtable}

\subsubsection{Caratteristica di qualità: Portabilità}
\hspace{1pt}
            \begin{longtable}{|C{1.5cm}|L{3cm}|L{4cm}|L{2.5cm}|L{2.5cm}|}
                    \hline
                    \textbf{Metrica} & \textbf{Nome} & \textbf{Descrizione} & \textbf{Valore di accettazione} & \textbf{Valore preferibile} \\
                    \hline
                    \stepcounter{metriccounter}\textbf{M\arabic{metriccounter}VBS} & Versioni dei Browser Supportate (VBS) & Metrica che misura la percentuale delle versioni di browser supportate rispetto al totale delle versioni disponibili. & $\geq 80\%$ & $100\%$ \\
                    \hline
                    \caption{Portabilità - Metriche e indici di qualità.}
                \label{tab:metriche_portabilità_testo}
            \end{longtable}

\subsection{Qualità per obiettivo}
Le metriche menzionate in precedenza vengono ora categorizzate secondo la struttura delineata nello \textit{standard}\textsubscript{\textit{G}} ISO/IEC 12207:1995, che le suddivide nei \textit{processi}\textsubscript{\textit{G}} primari, di supporto e organizzativi. Questo adattamento semplificato è stato realizzato per allineare le metriche alle specifiche esigenze del progetto:

\subsubsection{Processi primari}
\paragraph{Analisi dei requisiti}
L'Analisi dei Requisiti coinvolge la raccolta, l'analisi e la definizione dei requisiti del \textit{sistema}\textsubscript{\textit{G}} che si intende sviluppare. Coinvolge l'interazione con gli \textit{stakeholder}\textsubscript{\textit{G}} per comprendere le loro esigenze e tradurle in requisiti dettagliati e comprensibili per il team di sviluppo. Un'\textit{analisi dei requisiti}\textsubscript{\textit{G}} efficace è cruciale per garantire che il \textit{software}\textsubscript{\textit{G}} soddisfi le aspettative degli utenti finali.
\hspace{1pt}
    \begin{longtable}{|C{1.5cm}|L{3cm}|L{2.5cm}|L{2.5cm}|}
        \hline
        \textbf{Metrica} & \textbf{Nome} & \textbf{\makecell{Valore di \\ accettazione}} & \textbf{\makecell{Valore \\ preferibile}} \\
        \hline\textbf{M18PROS} & Percentuale di Requisiti Obbligatori Soddisfatti & $ 100\%$  & $ 100\%$ \\
        \hline
        \textbf{M19PRDS} & Percentuale di Requisiti Desiderabili Soddisfatti & $\geq 0\%$ & $100\%$ \\
        \hline
        \textbf{M20PRPS} & Percentuale di Requisiti oPzionali Soddisfatti & $\geq 0\%$ & $100\%$ \\
        \hline
    \caption{Analisi dei requisiti - Metriche e indici di qualità.}
    \label{tab:analisi_requisiti_progetto}
\end{longtable}

\paragraph{Progettazione}
La Progettazione è un processo in cui vengono definite le specifiche tecniche e architetturali del \textit{software}\textsubscript{\textit{G}} che si intende sviluppare. Questo processo traduce i requisiti raccolti durante la fase di acquisizione in un piano strutturato e dettagliato per la creazione del \textit{software}\textsubscript{\textit{G}}.

\hspace{1pt}
    \begin{longtable}{|C{1.5cm}|L{3cm}|L{2.5cm}|L{2.5cm}|}
        \hline
        \textbf{Metrica} & \textbf{Nome} & \textbf{\makecell{Valore di \\ accettazione}} & \textbf{\makecell{Valore \\ preferibile}} \\
        \hline
        \textbf{M25ATC} & Accoppiamento Tra Classi & $\leq 4$  & $\leq 2$ \\
        \hline
        \textbf{M30PG} & Profondità delle Gerarchie & $\leq 5$  & $\leq 3$ \\
        \hline
        \textbf{M32FU} & Facilità di Utilizzo & $\leq 7$ \textit{click}  & $\leq 5$ \textit{click} \\
        \hline
        \textbf{M33TA} & Tempo di Apprendimento & $\leq 15$ \textit{minuti}  & $\leq 10$ \textit{minuti} \\
        \hline
    \caption{Progettazione - Metriche e indici di qualità.}
    \label{tab:progettazione_progetto}
\end{longtable}

\paragraph{Fornitura}
La Fornitura è un processo che consiste nel decidere procedure e risorse
adatte allo sviluppo del progetto.

\hspace{1pt}
    \begin{longtable}{|C{1.5cm}|L{3cm}|L{2.5cm}|L{2.5cm}|}
        \hline
        \textbf{Metrica} & \textbf{Nome} & \textbf{\makecell{Valore di \\ accettazione}} & \textbf{\makecell{Valore \\ preferibile}} \\
        \hline
        \textbf{M2EAC} & Estimated at Completion & $\pm 5\%$ rispetto al preventivo & Pari al preventivo \\
        \hline
        \textbf{M3CPI} & Cost Performance Index & $\pm 10\%$ & $0\%$ \\
        \hline
        \textbf{M5AC} & Actual Cost & $\geq 0 $ & $ \leq$ \textit{EAC}\textsubscript{\textit{G}}  \\
        \hline
        \textbf{M7EV} & Earned Value & $\geq 0 $ & $\leq$ \textit{EAC}\textsubscript{\textit{G}}  \\
        \hline
        \textbf{M8PV} & Planned Value & $\geq 0  $ & $ \leq$ BAC  \\
        \hline
        \textbf{M9ETC} & Estimate to Complete & $\geq 0  $ & $ \leq$ \textit{EAC}\textsubscript{\textit{G}}  \\
        \hline
    \caption{Fornitura - Metriche e indici di qualità.}
    \label{tab:controllo_progetto}
\end{longtable}

\paragraph{Codifica}
La fase di codifica è essenziale in quanto trasforma il progetto e le specifiche del \textit{software}\textsubscript{\textit{G}} in istruzioni comprensibili dalla macchina, permettendo al prodotto \textit{software}\textsubscript{\textit{G}} di prendere vita e funzionare effettivamente.

\hspace{1pt}
    \begin{longtable}{|C{1.8cm}|L{3cm}|L{2.5cm}|L{2.5cm}|}
        \hline
        \textbf{Metrica} & \textbf{Nome} & \textbf{Valore di accettazione} & \textbf{Valore preferibile} \\
        \hline
        \textbf{M26MCCM} & Complessità Ciclomatica per Metodo & $\leq 5$ & $\leq 3$ \\
        \hline
        \textbf{M27PM} & Parametri per Metodo & $\leq 6$ & $\leq 5$ \\
        \hline
        \textbf{M28APC} & Attributi Per Classe & $\leq 6$ & $\leq 4$ \\
        \hline
        \textbf{M29LCM} & Linee di Codice per Metodo & $\leq 30$ & $\leq 20$ \\
        \hline
        \textbf{M31TMR} & Tempo Medio di Risposta & $\leq 10$ \textit{secondi}  & $\leq 4$ \textit{secondi} \\
        \hline
        \textbf{M34VBS} & Versioni dei Browser Supportate & $\geq 80\%$ & $100\%$ \\
        \hline
    \caption{Codifica - Metriche e indici di qualità.}
    \label{tab:metriche}
\end{longtable}




\subsubsection{Processi di supporto}

\paragraph{Documentazione}
La Documentazione è un processo essenziale che coinvolge la creazione e la gestione di documenti correlati allo sviluppo del \textit{software}\textsubscript{\textit{G}}. Una documentazione accurata e completa è fondamentale per comprendere, mantenere e supportare il \textit{software}\textsubscript{\textit{G}} nel tempo.
\hspace{1pt}
    \begin{longtable}{|C{1.5cm}|L{3cm}|L{2.5cm}|L{2.5cm}|}
        \hline
        \textbf{Metrica} & \textbf{Nome} & \textbf{Valore di accettazione} & \textbf{Valore preferibile} \\
        \hline
        \textbf{M22CO} & Correttezza Ortografica & $0$ & $0$ \\
        \hline
        \textbf{M23IG} & Indice Gulpease & $\geq 40$ & $\geq 60$ \\
        \hline
    \caption{Documentazione - Metriche e indici di qualità.}
    \label{tab:metriche_testo}
\end{longtable}

\paragraph{Verifica}
La Verifica è un processo che assicura che i prodotti del \textit{software}\textsubscript{\textit{G}} siano conformi ai requisiti specificati e agli \textit{standard}\textsubscript{\textit{G}} stabiliti. Coinvolge l'analisi, l'esecuzione di \textit{test}\textsubscript{\textit{G}} e l'ispezione dei prodotti \textit{software}\textsubscript{\textit{G}} per identificare e correggere eventuali difetti o discrepanze.
\hspace{1pt}
\begin{longtable}{|C{1.5cm}|L{3cm}|L{2.5cm}|L{2.5cm}|}
    \hline
    \textbf{Metrica} & \textbf{Nome} & \textbf{Valore di accettazione} & \textbf{Valore preferibile} \\
    \hline
    \textbf{M15SC} & Statement Coverage & $\geq 80\%$ & $100\%$ \\
    \hline
    \textbf{M16BC} & Branch Coverage & $\geq 80\%$ & $100\%$ \\
    \hline
    \textbf{M17CNC} & CoNdition Coverage & $\geq 80\%$ & $100\%$ \\
    \hline
    \textbf{M13PCTS} & Percentuale di Casi di Test Superati & $\geq 80\%$ & $100\%$ \\
    \hline
    \textbf{M14PCTF} & Percentuale di Casi di Test Falliti & $\leq 20\%$ & $0\%$ \\
    \hline
\caption{Verifica - Metriche e indici di qualità.}
\label{tab:metriche_verifica}
\end{longtable}


\paragraph{Gestione dei rischi}
Questo processo implica l'identificazione, l'analisi, la valutazione e il controllo dei rischi associati allo sviluppo del \textit{software}\textsubscript{\textit{G}}. 
\hspace{1pt}
    \begin{longtable}{|C{1.5cm}|L{3cm}|L{2.5cm}|L{2.5cm}|}
        \hline
      \textbf{Metrica} & \textbf{Nome} & \textbf{Valore di accettazione} & \textbf{Valore preferibile} \\
      \hline
      \textbf{M11RNP}    & Rischi non previsti  & $\leq 5$ &   $0$ \\
      \hline
    \caption{Gestione dei rischi - Metriche e indici di qualità.}
    \label{tab:tabella2}
\end{longtable}


\paragraph{Gestione della qualità}
Questo processo riguarda l'implementazione di \textit{standard}\textsubscript{\textit{G}}, procedure e metodologie atte a garantire che il \textit{software}\textsubscript{\textit{G}} soddisfi i requisiti di qualità stabiliti.
\hspace{1pt}
    \begin{longtable}{|C{1.5cm}|L{3cm}|L{2.5cm}|L{2.5cm}|}
        \hline
        \textbf{Metrica} & \textbf{Nome} & \textbf{Valore di accettazione} & \textbf{Valore preferibile} \\
        \hline
        \textbf{M1PMS} & Percentuale di Metriche Soddisfatte & $\geq 80\%$ & $100\%$ \\
        \hline
    \caption{Gestione della qualità - Metriche e indici di qualità.}
    \label{tab:gestione_metriche_testo}
\end{longtable}



\subsubsection{Processi organizzativi}

\paragraph{Pianificazione}
La Pianificazione organizza obiettivi, risorse e tempistiche per guidare il successo di un progetto.

\hspace{1pt}
    \begin{longtable}{|C{1.5cm}|L{3cm}|L{2.5cm}|L{2.5cm}|}
        \hline
        \textbf{Metrica} & \textbf{Nome} & \textbf{Valore di accettazione} & \textbf{Valore preferibile} \\
        \hline
        \textbf{M6SV} & Schedule Variance & $\geq -10\%$ & $0\%$ \\
        \hline
        \textbf{M4BV} & Budget Variance & $\geq -10\%$ & $0\%$ \\
        \hline
        \textbf{M12VR} & Variazione dei Requisiti & $\leq 3$ & $0$ \\
        \hline
        \textbf{M21IF} & Implementazione delle Funzionalità & $ 100\%$ & $ 100\%$ \\
        \hline
    \caption{Pianificazione - Metriche e indici di qualità.}
    \label{tab:metriche_pianificazione}
\end{longtable}

\paragraph{Miglioramento}
Il processo di miglioramento mira a identificare le aree che possono essere ottimizzate o migliorate.
\hspace{1pt}
    \begin{longtable}{|C{1.5cm}|L{3cm}|L{2.5cm}|L{2.5cm}|}
        \hline
        \textbf{Metrica} & \textbf{Nome} & \textbf{Valore di accettazione} & \textbf{Valore preferibile} \\
        \hline
        \textbf{M24DE} & Densità Errori & $\leq 10\%$ & $ 0\%$ \\
        \hline
    \caption{Miglioramento - Metriche e indici di qualità.}
    \label{tab:metriche_miglioramento}
\end{longtable}


\section{Specifica dei test}
L'esecuzione dei test è un passaggio imprescindibile per confermare che il prodotto, nel suo insieme, rispecchi fedelmente e adempia pienamente a tutti i requisiti espressi e definiti all'interno del documento di Analisi dei Requisiti. I test utili all'interno di un progetto sono:
\begin{itemize}
    \item \textbf{Test di unità}.
    \item \textbf{Test di integrazione}
    \item \textbf{Test di sistema}
    \item \textbf{Test di regressione}
    \item \textbf{Test di accettazione}
\end{itemize}
\subsection{Test di unità}
Sono test mirati a verificare singole parti (unità) del codice, come funzioni, classi o metodi. L'obiettivo è assicurarsi che ciascuna unità funzioni correttamente in isolamento.
\\
\begin{table}[htbp]
    \centering
    \begin{tabular}{|c|p{3cm}|p{5cm}|c|}
        \hline
        Codice Test & Descrizione & Stato Test \\
        \hline
        TU01 & Esempio. & N-I \\
        \hline
        TU02 & Esempio. & N-I \\
        \hline
    \end{tabular}
    \caption{Tabella test di unità}
    \label{tab:testsUnità}
\end{table}

\subsection{Test di integrazione}
Sono test finalizzati a verificare la corretta interazione tra le varie unità di codice o moduli. Si eseguono dopo i test di unità per assicurare che le unità, una volta combinate, lavorino insieme correttamente.
\\
\begin{table}[htbp]
    \centering
    \begin{tabular}{|c|p{3cm}|p{5cm}|c|}
        \hline
        Codice Test & Descrizione & Stato Test \\
        \hline
        TU01 & Esempio. & N-I \\
        \hline
        TU02 & Esempio. & N-I \\
        \hline
    \end{tabular}
    \caption{Tabella test di integrazione}
    \label{tab:testsIntegrazione}
\end{table}

\subsection{Test di sistema}Questa sezione illustra i test di sistema, i quali mirano a dimostrare la copertura completa dei requisiti identificati nel documento di Analisi dei Requisiti. Di seguito è fornito l'elenco di questi test.
\\
\begin{table}[htbp]
    \centering
    \begin{tabular}{|c|p{3cm}|p{5cm}|c|}
        \hline
        Codice Test & Descrizione & Stato Test \\
        \hline
        TF01 & Esempio. & N-I \\
        \hline
        TF02 & Esempio. & N-I \\
        \hline
    \end{tabular}
    \caption{Tabella test di sistema}
    \label{tab:testsSistema}
\end{table}

\subsection{Test di regressione}
Sono test che vengono eseguiti per assicurarsi che le modifiche apportate al codice non abbiano introdotto nuovi errori nelle funzionalità già testate in precedenza. Si eseguono dopo ogni modifica al codice per garantire che le funzionalità esistenti continuino a funzionare come previsto.
\\
\begin{table}[htbp]
    \centering
    \begin{tabular}{|c|p{3cm}|p{5cm}|c|}
        \hline
        Codice Test & Descrizione & Stato Test \\
        \hline
        TU01 & Esempio. & N-I \\
        \hline
        TU02 & Esempio. & N-I \\
        \hline
    \end{tabular}
    \caption{Tabella test di regressione}
    \label{tab:testsRegressione}
\end{table}

\subsection{Test di accettazione}
Nella sezione in questione, sono illustrati i test di accettazione del prodotto, condotti sia dai membri del team che dal proponente con il supporto del team di sviluppo. L'obiettivo finale di tali test è concludere il processo di validazione del prodotto.
\\
\begin{table}[htbp]
    \centering
    \begin{tabular}{|c|p{3cm}|p{5cm}|c|}
        \hline
        Codice Test & Descrizione & Stato Test \\
        \hline
        TF01 & Esempio. & N-I \\
        \hline
        TF02 & Esempio. & N-I \\
        \hline
    \end{tabular}
    \caption{Tabella test funzionalità}
    \label{tab:testsAccettazione}
\end{table}





\section{Cruscotto}

\subsection{Qualità di processo - Fornitura}

\vspace{0.3cm}

\subsubsection{M1PMS - Percentuale di Metriche Soddisfatte}
\begin{figure}[H]
    \centering
    \includegraphics[width=1\textwidth]{../Images/PianoDiQualifica/M1PMS.png}
    \caption{Proiezione della percentuale di metriche soddisfatte nei vari periodi di progetto.}
    \label{fig:1}
\end{figure}

\vspace{0.2cm}

\textbf{RTB}\\
Nel corso dei primi periodi, è evidente l'adozione di tutte le metriche di qualità; tuttavia, è solamente nell'ultimo periodo che si osserva il superamento dei valori di accettazione per due metriche, M2EAC e M4BV, fenomeno attribuibile al periodo di esami universitari.

\vspace{0.3cm}

\textbf{PB} \\
Il grafico evidenzia un’adozione efficace delle metriche di qualità. Nonostante ci siano stati periodi con percentuali inferiori, queste rimangono sempre all’interno dei range di accettazione. \\
La diminuzione osservata è attribuibile alle metriche dedicate ai test dato il mancato raggiungimento dei valori di accettazione per le relative metriche, a causa della limitata esperienza del gruppo in questo specifico ambito. \\
Durante il decimo periodo, si è verificato un superamento dei valori di accettazione per la metrica M24DE, causato dall’introduzione di uno strumento per la valutazione della qualità del codice. Questi problemi sono stati prontamente risolti e, nel periodo immediatamente successivo, la percentuale di metriche soddisfatte è ritornata ai valori ottimali.

\subsubsection{M2EAC - Estimed at Completion}

\vspace{0.3cm}

\begin{figure}[H]
    \centering
    \includegraphics[width=1\textwidth]{../Images/PianoDiQualifica/M2EAC.png}
    \caption{Proiezione della stima del costo totale nei vari periodi di progetto.}
    \label{fig:2}
\end{figure}

\vspace{0.2cm}

\textbf{RTB} \\
Si nota come nei primi periodi la stima del costo totale sia in linea con il budget inzialmente preventivato. \\
Tuttavia al quinto periodo, periodo di sessione degli esami, il costo totale è di molto inferiore al budget preventivato. \\
Questo è dovuto al fatto che in quel periodo c'è stato un calo di \textit{attività}\textsubscript{\textit{G}}, in quanto i membri del gruppo erano impegnati con gli esami universitari. Le \textit{attività}\textsubscript{\textit{G}} però rimanenti sono state completate con un costo inferiore a quello preventivato e questo ha portato ad una riduzione del costo totale.

\vspace{0.3cm}

\textbf{PB} \\
Si nota come, nonostante una diminuzione registrata nel quinto e sesto periodo, nei restanti dei periodi la stima del costo totale è rientrata nei limiti di accettazione. Inoltre, tale stima è rimasta in stretta aderenza al budget preventivato. Questo fenomeno indica un’efficace gestione dei costi.

\subsubsection{M7EV- Earned Value + M8PV - Planned Value} 

\vspace{0.3cm}

\begin{figure}[H]
    \centering
    \includegraphics[width=1\textwidth]{../Images/PianoDiQualifica/EV_PV.png}
    \caption{Proiezione dell’EV e del PV nei vari periodi di progetto.}
    \label{fig:3}
\end{figure}

\vspace{0.2cm}

\textbf{RTB} \\
Dall'analisi del grafico, è chiaro che le curve del valore guadagnato (Earned Value) e del valore pianificato (Planned Value) si sovrappongono, suggerendo che il lavoro effettivamente completato corrisponde alla pianificazione. 
Questa coincidenza implica un progresso positivo rispetto alla pianificazione del progetto.

\vspace{0.3cm}

\textbf{PB} \\ 
Dall'analisi del grafico, si evince una congruenza pressoché totale tra le traiettorie del valore guadagnato (EV) e del valore pianficato (PV), a conferma dell'allineamento del completamento del lavoro con la pianificazione originaria.
La convergenza delle due curve con la stima dei costi totali (EAC) nell'ultimo periodo convalida il rispetto del budget preventivato e il completamento dei lavori in linea con le previsioni iniziali.

\subsubsection{M5AC - Actual Cost + M9ETC - Estimate to Complete}

\vspace{0.3cm}

\begin{figure}[H]
    \centering
    \includegraphics[width=1\textwidth]{../Images/PianoDiQualifica/AC_ETC.png}
    \caption{Proiezione dell’AC e dell’ETC nei vari periodi di progetto.}
    \label{fig:4}
\end{figure}

\vspace{0.2cm}

\textbf{RTB} \\
Il grafico illustra l'Actual Cost (AC), che rappresenta i costi effettivamente sostenuti fino al periodo corrente per il lavoro eseguito, e l'Estimate to Complete (\textit{ETC}\textsubscript{\textit{G}}), che denota la stima dei costi rimanenti per completare il progetto durante i vari periodi. \\
Si osserva che l'\textit{ETC}\textsubscript{\textit{G}} tende a diminuire, come atteso, mentre l'AC mostra un incremento proporzionale alla riduzione dell'\textit{ETC}\textsubscript{\textit{G}}.

\vspace{0.3cm}

\textbf{PB} \\
Il grafico illustra come l'Actual Cost (AC), ossia i costi effettivi sostenuti per il lavoro completato fino al periodo corrente, converge con l'Estimated at Completion (EAC) nell'ultimo periodo. \\
Ciò significa che i costi sostenuti sono stati in linea con la stima del costo totale del progetto, a dimostrazione di un efficace controllo e gestione delle risorse finanziarie. Inoltre, l'andamento dell'Estimate to Complete (ETC), ovvero la stima dei costi residui necessari per completare il progetto, evidenzia una graduale diminuzione nel corso del tempo, fino ad azzerarsi nel periodo finale. \\
Tale risultato conferma il completamento del progetto in linea con le previsioni di budget e la corretta pianificazione delle attività.

\subsubsection{M4BV - Budget Variance + M6SV - Schedule Variance}

\vspace{0.3cm}

\begin{figure}[H]
    \centering
    \includegraphics[width=1\textwidth]{../Images/PianoDiQualifica/BV_SV.png}
    \caption{Proiezione della BV e della SV nei vari periodi di progetto.}
    \label{fig:5}
\end{figure}

\vspace{0.2cm}

\textbf{RTB} \\
Il grafico mostra l'andamento della Budget Variance \textit{(BV)} rappresentante la differenza tra il valore guadagnato \textit{(EV)} e i costi sostenuti \textit{(AC)} e la Schedule Variance \textit{(SV)} che indica la differenza tra il valore guadagnato \textit{(EV)} e il valore pianificato (\textit{PV}\textsubscript{\textit{G}}). \\
Si nota come la Budget Variance risulti altalenante, suggerendo che ad ogni periodo, tranne il primo e il terzo dove abbiamo un valore molto vicino a zero, ci sia una discrepanza dal costo preventivato a quello effettivo fino al periodo di riferimento. \\
Nell'ultimo periodo si nota un grande aumento della Budget Variance, questo è dovuto al fatto che le \textit{attività}\textsubscript{\textit{G}} rimanenti sono state completate con un costo inferiore a quello preventivato. \\
La Schedule Variance risulta altalenante in alcuni periodi, suggerendo che durante questi intervalli di tempo ci sono stati ritardi o anticipi rispetto alla pianificazione prevista. \\
Nell'ultimo periodo si è raggiunto l'ottimo per la Schedule Variance. 

\vspace{0.3cm}

\textbf{PB} \\
Dall'analisi del grafico si nota che la Budget Variance (BV), che rappresenta la differenza tra il valore guadagnato (EV) e i costi sostenuti (AC), si attesta su valori prossimi all'ottimale dopo i periodi relativi all'RTB, indicando un'efficace gestione dei costi e un'accurata stima delle risorse necessarie per il completamento dei lavori.
Analogamente, la Schedule Variance (SV), che indica la differenza tra il valore guadagnato (EV) e il valore pianificato (PV), raggiunge valori ottimali. Questo conferma una pianificazione efficace e un puntuale rispetto dei tempi previsti per il completamento dei lavori.

\subsubsection{M3CPI - Cost Performance Index}

\vspace{0.3cm}

\begin{figure}[H]
    \centering
    \includegraphics[width=1\textwidth]{../Images/PianoDiQualifica/M3CPI.png}
    \caption{Proiezione del CPI nei vari periodi di progetto.}
    \label{fig:6}
\end{figure}

\vspace{0.2cm}

\textbf{RTB} \\
Il grafico evidenzia la costante prossimità del nostro Cost Performance Index (\textit{CPI}\textsubscript{\textit{G}}) a 1, suggerendo che il progetto stia mantenendo i costi in linea con la pianificazione.
In particolare, nell'ultimo periodo, si osserva un incremento del \textit{CPI}\textsubscript{\textit{G}}, indicando che le \textit{attività}\textsubscript{\textit{G}} rimanenti sono state completate con un costo inferiore rispetto a quanto inizialmente previsto.

\vspace{0.3cm}

\textbf{PB} \\
Il grafico mostra un Cost Performance Index (CPI) costantemente prossimo a 1, a conferma di un'efficace gestione dei costi e di un'accurata stima delle risorse necessarie per il completamento dei lavori.

\subsubsection{M11RNP - Rischi non previsti}

\vspace{0.3cm}

\begin{figure}[H]
    \centering
    \includegraphics[width=1\textwidth]{../Images/PianoDiQualifica/M11RNP.png}
    \caption{Proiezione rischi non previsti nei vari periodi di progetto.}
    \label{fig:7}
\end{figure}

\vspace{0.2cm}

\textbf{RTB} \\
Il grafico mostra come i rischi non previsti siano rimasti costanti durante tutto il progetto. Questo è un buon segno, in quanto indica che il gruppo è stato in grado di gestire i rischi in modo efficace e che non sono emersi nuovi rischi inaspettati.

\vspace{0.3cm}

\textbf{PB} \\
Il grafico mostra che, nei periodi successivi, i rischi imprevisti sono rimasti stabili, ad eccezione dell’ultimo periodo. Durante quest’ultimo, si è verificato un rischio non anticipato, che ha causato un lieve incremento nei tempi di preparazione per i colloqui previsti per la revisione PB.


\subsection{Qualità di processo - Documentazione}

\subsubsection{Errori Ortografici}
\begin{figure}[H]
    \centering
    \includegraphics[width=0.8\textwidth]{../Images/PianoDiQualifica/errori_ortografici.png}
    \caption{Resoconto errori ortografici}
    \label{fig:Errori ortografici}
\end{figure}

\textbf{RTB}: Il grafico mostra l'andamento degli errori ortografici rilevati nei documenti. Si nota come il numero di errori ortografici sia inizialmente alto, ma tenda a diminuire con l'avanzare del progetto. Questo è dovuto al fatto che il gruppo ha iniziato a prestare maggiore attenzione alla scrittura dei documenti raggiugendo l'ottimo nell'ultimo periodo.

\subsubsection{Indice di Gulpease}
\begin{figure}[H]
    \centering
    \includegraphics[width=0.8\textwidth]{../Images/PianoDiQualifica/Gulpease.png}
    \caption{Resoconto indice di Gulpease}
    \label{fig:Indice di Gulpease}
\end{figure}

\textbf{RTB}: Dalla valutazione del grafico si nota un tendenza generale di crescita e/o mantenimento dell'indice per ogni documento durante i vari periodi considerati. Si osserva che il glossario presenta un indice di Gulpease molto basso, il che è attribuibile alla sua natura tecnica e alla conseguente impossibilità di aumentare tale indice. Gli altri documenti, invece, mostrano un indice di Gulpease elevato, in parte dovuto al loro contenuto meno tecnico e più accessibile.

\subsection{Qualità di processo - Analisi dei requisiti}

\vspace{0.3cm}

\subsubsection{M18PROS - Percentuale di requisiti obbligatori soddisfatti}

\vspace{0.3cm}

\begin{figure}[H]
    \centering
    \includegraphics[width=0.9\textwidth]{../Images/PianoDiQualifica/PROS.png}
    \caption{Proiezione della copertura dei requisiti obbligatori soddisfatti nei vari periodi di progetto.}
    \label{fig:10}
\end{figure}

\vspace{0.2cm}

\textbf{PB}: La rappresentazione grafica evidenzia l’adempimento completo, pari al 100\%, dei requisiti obbligatori nel periodo più recente. L’attuazione di tali requisiti ha avuto inizio nel periodo immediatamente successivo alla fase di progettazione del sistema, poiché era imprescindibile disporre di una comprensione accurata dell’architettura da implementare.

\subsubsection{M19PRDS - Percentuale di requisiti desiderabili soddisfatti}

\vspace{0.3cm}

\begin{figure}[H]
    \centering
    \includegraphics[width=0.9\textwidth]{../Images/PianoDiQualifica/PRDS.png}
    \caption{Proiezione della copertura dei requisiti desiderabili soddisfatti nei vari periodi di progetto.}
    \label{fig:11}
\end{figure}

\vspace{0.2cm}

\textbf{PB}: La rappresentazione grafica evidenzia anche in questo caso, l'adempimento completo, pari al 100\%, dei requisiti desiderabili nel periodo più recente. La soddisfazione di tali requisiti è stata possibile grazie alla loro implementazione in un momento successivo rispetto ai requisiti obbligatori e all’incoraggiamento alla realizzazione da parte della proponente, in quanto considerata una funzionalità ritenuta utile per l’utente finale.

\subsubsection{M20PRPS - Percentuale di requisiti opzionali soddisfatti}

\vspace{0.3cm}
\begin{figure}[H]
    \centering
    \includegraphics[width=0.9\textwidth]{../Images/PianoDiQualifica/PRPS.png}
    \caption{Proiezione della copertura dei requisiti desiderabili soddisfatti nei vari periodi di progetto.}
    \label{fig:12}
\end{figure}

\vspace{0.2cm}

\textbf{PB}: La rappresentazione grafica illustra, con dispiacere, che solo una parte dei requisiti opzionali è stata soddisfatta. L’implementazione di tali requisiti è stata parzialmente realizzata, in quanto la loro esecuzione è stata ritenuta di importanza secondaria rispetto ai requisiti obbligatori e desiderabili. Inoltre, le risorse disponibili non sono state adeguate per adempiere completamente a tali requisiti.

\end{document}
