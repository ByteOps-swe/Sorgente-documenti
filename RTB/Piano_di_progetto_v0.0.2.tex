\documentclass{article}
\usepackage[utf8]{inputenc}
\usepackage[default]{raleway}
\usepackage[table]{xcolor}
\usepackage{booktabs} 
\usepackage{ragged2e}
\usepackage{titlesec, comment, tabularx, makecell, listings, array, setspace, geometry, graphicx, xcolor, xparse, fancyvrb, relsize, fancyhdr, booktabs, hyperref,verbatim, float}
%\geometry{a4paper, left=2cm, right=2cm, top=2cm, bottom=2.5cm}

\renewcommand{\headrulewidth}{0pt}

% ----------------------------- Definizione tabella ---------------------------

\newcolumntype{C}[1]{>{\centering\arraybackslash}m{#1}}

% ------------------------------Metadati indice --------------------------------
\title{\textbf{\fontsize{30}{6}\selectfont Indice}}
\author{\fontsize{14}{6}\selectfont ByteOps}
\date{\today}

% -----------------------------Creazione footer --------------------------------
\pagestyle{fancy} 
\fancyhf{}
\renewcommand{\footrulewidth}{0.4pt} 
\lfoot{ 
    \parbox[c]{2cm}{\includegraphics[width=2cm]{../Images/logo.png}}
    \textcolor[RGB]{120, 120, 120}{$\cdot$ Piano di progetto}
}
\rfoot{\thepage}

% --------------------------Modifica formato hyperlinks ------------------------
\hypersetup{
    colorlinks=true,
    linkcolor=black,
    filecolor=black,      
    pdftitle={Piano di progetto},
    pdfpagemode=FullScreen,
}




\begin{document}
\pagestyle{fancy}
\begin{center}
\includegraphics[width = 0.7\textwidth]{../Images/logo.png} \\
\vspace{0.2cm}
\textcolor[RGB]{60, 60, 60}{\textit{ByteOps.swe@gmail.com}} \\
\vspace{1cm}
\fontsize{16}{6}\selectfont Piano di progetto \\
\vspace{0.5cm}
\end{center}

\section*{Informazioni documento}
\def\arraystretch{1.2}
\begin{tabular}{>{\raggedleft\arraybackslash}p{0.2\textwidth}|>{\raggedright\arraybackslash}p{0.6\textwidth}c}
\hline
\addlinespace
    \textbf{Redattori} & A. Barutta \\ & R. Smanio \\ & N. Preto \vspace{10pt} \\
    \textbf{Verificatori} & E. Hysa \\ & L. Skenderi \\ & D. Diotto \vspace{10pt} \\
    \textbf{Amministratore} & F. Pozza \vspace{10pt} \\
    \textbf{Destinatari} & T. Vardanega \\ & R. Cardin \vspace{10pt} \\
    \textbf{Partecipanti} & A. Barutta \\ & E. Hysa \\ & R. Smanio \\ & D. Diotto \\ & F. Pozza \\ & L. Skenderi \\ & N. Preto \vspace{10pt} \\
\end{tabular}
\pagebreak 

% ------------------------- Changelog ----------------------------

\begin{tabular}{|C{2.5cm}|C{2.5cm}|C{2.5cm}|C{2.5cm}|C{2.5cm}|}
    \hline
    \textbf{Versione} & \textbf{Data} & \textbf{Autore} & \textbf{Verificatore} & \textbf{Dettaglio} \\
    \hline
    \label{Git_Action_Version} 0.0.2 & 05/11/2023 & Nome autore & Nome verificatore & Aggiunta sezione rischi \\
    \hline 
    0.0.1 & 03/11/2023 & Nome autore & Nome verificatore & Prima impostazione documento\\
    \hline 
 
    
\end{tabular}

\pagebreak

% ------------------------- Generazione automatica indice ----------------------
\setstretch{1.5}
\maketitle
\thispagestyle{fancy}
\tableofcontents
\setstretch{1.2}
\pagebreak

% ---------------------------- Inizio documento -------------------------------

\section{Introduzione}
\subsection{Scopo del documento}
Il presente documento ha lo scopo di identificare la metodologia di pianificazione e illustrare le modalità con cui il gruppo \textit{ByteOps} sta sviluppando il progetto assegnato, al fine di garantire efficacia ed efficienza nel processo di sviluppo.\\
I contenuti che vengono trattati sono:
\begin{itemize}
    \item Analisi dei rischi
    \item Assegnazione ruoli dei membri del gruppo
    \item Descrizione del modello adottato con relative motivazioni della scelta
\end{itemize}



\subsection{Scopo del capitolato}
Il Capitolato C6 affidato al gruppo, si prefigge come obiettivo la realizzazione di una piattaforma di monitoraggio di una \textit{"Smart City"} che consenta di avere sotto controllo lo stato di salute della città in modo tale da prendere decisioni veloci, efficaci ed analizzare poi gli effetti conseguenti.
A tale scopo il proponente richiede di simulare dei sensori posti in diverse aree per reperire informazioni relative alle condizioni della città. 
I dati trasmessi in tempo reale dai sensori devono poter essere memorizzati in modo tale da renderli disponibili per la visualizzazione tramite una dashboard, composta anche da widget e grafici, per una visione d'insieme delle condizioni della città in tempo reale. 
L'applicativo potrà consentire alle autorità locali di prendere decisioni informate e tempestive sulla gestione delle risorse e sull'implementazione di servizi e, inoltre, si potrebbe rivelare uno strumento essenziale per coinvolgere i cittadini nella gestione e nel miglioramento della città. 

\subsection{Glossario}
Per evitare possibili incomprensioni con la terminologia utilizzata, verrà utilizzato il seguente simbolo:
\begin{itemize}
    \item \textit{G}: indica un termine presente nel documento \textit{Glossario}.  
\end{itemize}
%Glossario da definire


\subsection{Riferimenti}
\subsubsection{Riferimenti informativi}
\begin{itemize}
    \item \href {https://www.math.unipd.it/~tullio/IS-1/2023/Progetto/C6.pdf} {Capitolato d'appalto C6 - InnovaCity}
    \item \href{https://www.math.unipd.it/~tullio/IS-1/2023/Dispense/T4.pdf} {Slide del corso di Ingegneria del Software - Gestione di progetto}
    \item \href{https://www.math.unipd.it/~tullio/IS-1/2023/Dispense/T2.pdf} {Slide del corso di Ingegneria del Software - Ciclo di vita del software}

\end{itemize}

\subsubsection{Riferimenti normativi}
\begin{itemize}
\item Norme di progetto
\item \href {https://www.math.unipd.it/~tullio/IS-1/2023/Dispense/PD2.pdf} {Regolamento del progetto didattico}
\end{itemize}

%--Il seguente documento è stato redatto con l'obiettivo di valutare e disaminare i potenziali rischi che il gruppo potrebbe affrontare nel corso delle attività di sviluppo del capitolato. In questo documento veranno esposti i vari rischi identificati, accompagnati da una relativa previsione di impatto che ciascuno di essi potrebbe avere e una possibile soluzione.

\section{Analisi dei rischi}
%------------------------------------------------------------------------
\subsection{Descrizione}
Durante lo sviluppo di un progetto è probabile incorrere in problematiche e imprevisti vari. Questi possono provocare effetti indesiderati, quali:
\begin{itemize}
    \item Aumento dei costi previsti per un dato periodo.
    \item Sforamento dei tempi preventivati per la realizzazione dei vari compiti.
    \item Rendimento complessivo condizionato negativamente.
    \item Deterioramento della qualità del prodotto.
\end{itemize}
È necessario quindi attuare un processo utile ad indentificare i rischi ed avere un piano di contingenza per mitigarli o eliminarli.
%------------------------------------------------------------------------

%------------------------------------------------------------------------
\subsection{Processo di mitigazione}
\subsubsection{Identificazione}
Individuare le possibili problematiche che potrebbero verificarsi durante lo sviluppo del progetto. 
Le fonti dalle quali potrebbero derivare i rischi sono: 
\begin{itemize}
    \item Gruppo: collaborazione, comunicazione, competenze tecniche, organizzazione.
    \item Prodotto del capitolato: requisiti, tecnologie, strumenti.
\end{itemize}
%------------------------------------------------------------------------

%------------------------------------------------------------------------
\subsubsection{Processo di analisi}
Per ogni rischio identificato assegnare un indice identificativo e stabilire secondo i seguenti parametri:
\begin{itemize}
    \item Probabilità di occorrenza: quanto è probabile che il rischio si verifichi.
    \item Grado di pericolosità: quali effetti negativi potrebbe causare nello sviluppo del progetto.
\end{itemize}
%------------------------------------------------------------------------

%------------------------------------------------------------------------
\subsubsection{Pianificazione}
Per ogni rischio identificato, definire un piano di contingenza che preveda:
\begin{itemize}
    \item Strategia preventiva: definire le azioni da intraprendere per prevenire l’insorgenza del rischio.
    \item Riduzione dell'impatto: stabilire le misure da adottare per ridurre al minimo l'impatto del rischio, nel caso non si riesca ad evitarlo.
\end{itemize}
%------------------------------------------------------------------------

%------------------------------------------------------------------------
\subsubsection{Processo di controllo e aggiornamento}
Effettuare un monitoraggio periodico delle attività in corso e degli artefatti prodotti, al fine di identificare potenziali nuovi rischi o modificare quelli preesistenti, aggiornando di conseguenza le relative strategie di mitigazione.
%------------------------------------------------------------------------


\begin{comment}
Qui dentro viene mostrato il codice dei rischi che andrà normato nella sezione Gestione dei Rischi nel file Norme di Progetto.
Il rischio ha il seguente codice:   R[tipo]-[probabilità][priorità]-[id_identificativo]

R: rischio
tipo: natura del rischio (tecnologico (T), organizzativo (O), personale (P), requisiti (R))
probabilità: probabilità di occorrenza (Alta(1), Media(2), Bassa(3))
priorità: grado di pericolosità  (Alta(A), Media(M), Bassa(B))
id_identificativo: numero progressivo che identifica il rischio (1, 2, 3, ecc..)

\end{comment}



\subsection{Rischi previsti}
Di seguito sono riportate le tabelle relative ai rischi previsti che potrebbero presentarsi durante lo sviluppo del progetto. Per il codice identificativo dei rischi si rimanda al documento \textit{Norme di progetto}.


%------------------------------------------------------------------------
\subsubsection{Impegni personali e accademici}
\begin{table}[h]
    \centering
    \begin{tabularx}{\textwidth}{l>{\RaggedRight}X>{\RaggedRight}X>{\RaggedRight}X>{\RaggedRight}X}
    \toprule
    \rowcolor{gray!50}
    \textbf{Codice} & \textbf{Descrizione del rischio} & \textbf{Identificazione} & \textbf{Mitigazione} \\
    \midrule
    \addlinespace 
    RO-1A-1 & 
    Rischio di rallentamento del progetto dovuto all'armonizzazione delle attività personali e progettuali, con particolare intensificazione durante la sessione invernale 2023-2024 a causa degli esami. & 
    I membri del gruppo comunicheranno al responsabile i loro impegni durante le riunioni di organizzazione o al momento immediato della conoscenza dell'impedimento.& 
    Il responsabile, considerando gli impegni dei membri del gruppo, avrà la facoltà di riassegnare le varie attività ad altri membri o estendere il tempo previsto per l'esecuzione dell'attività assegnata. \\  
    \bottomrule
    \addlinespace 
    \end{tabularx}
\end{table}
%------------------------------------------------------------------------
\newpage
\subsubsection{Ritardo nel completamento delle attività rispetto ai tempi previsti.}
\begin{table}[h]
    \centering
    \begin{tabularx}{\textwidth}{l>{\RaggedRight}X>{\RaggedRight}X>{\RaggedRight}X>{\RaggedRight}X}
    \toprule
    \rowcolor{gray!50}
    \textbf{Codice} & \textbf{Descrizione del rischio} & \textbf{Identificazione} & \textbf{Mitigazione} \\
    \midrule
    \addlinespace 
    RO-2M-2 & 
    L'inesperienza del gruppo in un progetto software professionale potrebbe portare a superare i tempi preventivati, specialmente a causa della nuova tecnologia e della necessità di migliorare la gestione delle risorse.& 
    I membri del gruppo devono segnalare al responsabile eventuali difficoltà nel rispettare le scadenze previste per le attività. &
    Il responsabile, considerando le motivazioni del ritardo, avrà la facoltà di riassegnare le varie attività ad altri membri o estendere il tempo previsto per l'esecuzione dell'attività assegnata. \\
    \bottomrule
    \addlinespace 
    \end{tabularx}
\end{table}

%------------------------------------------------------------------------


\begin{comment}
    \subsection{Apprendimento ed utilizzo delle nuove tecnologie}
L’apprendimento e l'implementazione delle tecnologie proposte possono rappresentare un rischio considerevole per lo 
sviluppo di un progetto, in quanto esista la possibilità che lo studio accurato di queste tecnologie richieda più tempo del previsto. \\
Per quanto riguarda le possibili soluzioni per attenuare il problema, si propone:
\begin{itemize}
    \item Pianificazione accurata del tempo necessario per l’assimilazione e l’implementazione \\delle nuove tecnologie.
    \item Sfruttare le competenze individuali di ciascun membro del gruppo, al fine di accelerare \\ il processo di apprendimento.
    \item Approfittare, ove possibile, delle opportunità di formazione offerte dall’azienda proponente, in modo da risolvere eventuali dubbi e consolidare i concetti appresi.
\end{itemize}
\subsection{Inefficacia della comunicazione} 
L’inefficacia della comunicazione tra i componenti del gruppo potrebbe condurre a ritardi significativi nello sviluppo del progetto. Questo rischio assume una rilevanza particolare considerando la natura collaborativa del lavoro di gruppo, che richiede l’adesione a norme concordate collettivamente.
\\Per quanto riguarda le possibili soluzioni per attenuare il problema, si propone:
   \begin{itemize} 
    \item Impiego di strumenti di comunicazione efficaci e l’istituzione di regole chiare per le riunioni e le discussioni tra i componenti del gruppo.
    \item Incentivare un ambiente di lavoro che favorisca la collaborazione e sia aperto all'espres-sione di idee o feedback da parte dei membri del gruppo.
\end{itemize}
\end{comment}









\end{document} 