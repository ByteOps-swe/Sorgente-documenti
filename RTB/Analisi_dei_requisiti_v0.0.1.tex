\documentclass{article}
\usepackage[utf8]{inputenc}
\usepackage[absolute]{textpos}
\usepackage[default]{raleway}
\usepackage{titlesec, multirow, comment, tabularx, longtable, makecell, listings, array, setspace, geometry, graphicx, xcolor, xparse, fancyvrb, relsize, fancyhdr, booktabs, hyperref, float}
\usepackage{colortbl}
\usepackage{lipsum}
%\geometry{a4paper, left=2cm, right=2cm, top=2cm, bottom=2.5cm}
\renewcommand{\headrulewidth}{0pt}

% Definisci uno stile per i comandi git
\definecolor{light-gray}{gray}{0.92}

\lstdefinestyle{code}{
    frame=single,
    framesep=1mm,
    rulecolor=\color{light-gray},
    backgroundcolor=\color{light-gray},
    basicstyle=\ttfamily,
}

% ----------------------------- Definizione tabella ---------------------------

\newcolumntype{C}[1]{>{\centering\arraybackslash}m{#1}}

%\setcellgapes{2ex} % Imposta l'altezza dell'header (2ex)


% ------------------------------Metadati indice --------------------------------
\title{\textbf{\fontsize{28}{6}\selectfont Indice}}
\author{\fontsize{14}{6}\selectfont ByteOps}
\date{}


% -----------------------------Creazione footer --------------------------------

\pagestyle{fancy}
\fancyhf{}
\renewcommand{\footrulewidth}{0.4pt}
\lfoot{
    \parbox[c]{2cm}{\includegraphics[width=2cm]{../Images/logo.png}}
    \textcolor[RGB]{120, 120, 120}{$\cdot$ Analisi dei requisiti}
}
\rfoot{\thepage}

% --------------------------Modifica formato hyperlinks ------------------------

\hypersetup{
    colorlinks=true,
    linkcolor=black,
    filecolor=black,      
    pdftitle={Analisi dei Requisiti},
    pdfpagemode=FullScreen,
}

% ------------------------------- Valore sotto-paragrafi indice --------------------------------------

\setcounter{secnumdepth}{4}
\setcounter{tocdepth}{4}

\titleformat{\section}
{\normalfont\huge\bfseries}{\thesection}{0.2cm}{}
\titlespacing*{\paragraph}{0pt}{0.5cm}{0.1cm}

\titleformat{\subsection}
{\normalfont\Large\bfseries}{\thesubsection}{0.2cm}{}
\titlespacing*{\paragraph}{0pt}{0.5cm}{0.1cm}

\titleformat{\subsubsection}
{\normalfont\large\bfseries}{\thesubsubsection}{0.2cm}{}
\titlespacing*{\paragraph}{0pt}{0.5cm}{0.1cm}

\titleformat{\paragraph}
{\normalfont\normalsize\bfseries}{\theparagraph}{0.2cm}{}
\titlespacing*{\paragraph}{0pt}{0.5cm}{0.1cm}

% ------------------------------- Front Page ---------------------------------------

\begin{document}

% --------------------------Aggiunta firma finale ------------------------
%\begin{textblock*}{\textwidth}(0.85\textwidth, 1.16\textheight)
%   Il responsabile: Nome Cognome
%\end{textblock*}
% ------------------------------------------------------------------------

\pagestyle{fancy}
\begin{center}
    \includegraphics[width = 0.7\textwidth]{../Images/logo.png} \\
    \vspace{0.2cm}
    \textcolor[RGB]{60, 60, 60}{\textit{ByteOps.swe@gmail.com}} \\
    \vspace{1cm}
    \fontsize{16}{6}\selectfont Analisi dei requisiti\\
    \vspace{0.5cm}
\end{center}

\section*{Informazioni documento}
\def\arraystretch{1.2}
\begin{tabular}{>{\raggedleft\arraybackslash}p{0.2\textwidth}|>{\raggedright\arraybackslash}p{0.6\textwidth}c}
    \hline
    \addlinespace
    \textbf{Redattori}    & A. Barutta\\ & R.Smanio \vspace{10pt} \\
    \textbf{Verificatori} & E. Hysa \vspace{10pt} \\
    \textbf{Destinatari}  & ByteOps\\ & T. Vardanega   \\ & R. Cardin \vspace{10pt} \\
\end{tabular}
\pagebreak

% ------------------------- Changelog ----------------------------

\section*{Registro delle modifiche}

\begin{tabular}{|C{1.5cm}|C{2cm}|C{2cm}|C{2cm}|C{5cm}|}
    \hline
    \textbf{Versione} & \textbf{Data} & \textbf{Autore} & \textbf{Verificatore} & \textbf{Dettaglio} \\
    \hline \hline
    \label{Git_Action_Version} 0.4.0    
    & 27/11/2023    & \makecell{A. Barutta}    & F. Pozza          & Sez. R. Qualitativi        \\
    \hline
     0.3.0     & 27/11/2023    & \makecell{A. Barutta}    & F. Pozza          & Sez. R. Vincolo        \\
    \hline
    0.2.0             & 22/11/2023    & \makecell{A. Barutta\\ R. Smanio}    & E. Hysa          & Sez. R. Funzionali        \\
    \hline
    0.1.9             & 21/11/2023    & \makecell{A. Barutta\\ R. Smanio}    & E. Hysa          &  \makecell{Sottocasi di UC6, UC7}       \\
    \hline
    0.1.8             & 21/11/2023    & \makecell{A. Barutta\\ R. Smanio}    & E. Hysa          &  \makecell{Casi d'uso UC6, UC7, \\ UC9, UC10}       \\
    \hline
    0.1.7             & 20/11/2023    & \makecell{A. Barutta\\ R. Smanio}    & E. Hysa          &  \makecell{Sottocasi di UC1, UC2, UC3}       \\
    \hline
    0.1.6             & 18/11/2023    & \makecell{A. Barutta\\ R. Smanio}    & E. Hysa          &  \makecell{Casi d'uso UC1, UC2, UC3}       \\
    \hline
    0.1.5             & 17/11/2023    & \makecell{A. Barutta\\ R. Smanio}    & E. Hysa          &  \makecell{Obiettivi del prodotto}       \\
    \hline
    0.1.4             & 16/11/2023    & \makecell{A. Barutta\\ R. Smanio}    & N. Preto         &  \makecell{Aggiunto glossario}       \\
    \hline
    0.1.3             & 15/11/2023    & \makecell{A. Barutta\\ R. Smanio}    & N. Preto         &  \makecell{Integrazione di casi d'uso}       \\
    \hline
    0.1.2             & 13/11/2023    & \makecell{A. Barutta\\ R. Smanio}    & N. Preto         &  \makecell{Casi d'uso}       \\
    \hline
    0.1.1             & 12/11/2023    & \makecell{A. Barutta\\ R. Smanio}    & N. Preto         &  \makecell{Descrizione del Prodotto}       \\
    \hline
    0.1.0             & 12/11/2023    & \makecell{A. Barutta\\ R. Smanio}    & N. Preto         &  \makecell{Introduzione}       \\
    \hline
\end{tabular}
\pagebreak

% ------------------------- Generazione automatica indice ----------------------
\setstretch{1.5}
\maketitle
\thispagestyle{fancy}
\tableofcontents
\listoffigures % Indice delle figure
\setstretch{1.2}
\pagebreak

% ------------------------ INIZIO DOCUMENTO ----------------------
\flushleft

\section{Introduzione}
\subsection{Scopo del Manuale}
Il presente manuale è concepito per fornire un supporto completo agli utenti Autorità Locale nell'utilizzo efficace del software, consentendo loro di sfruttare appieno tutte le sue funzionalità al fine di garantire un'esperienza ottimale. \\
Poiché l'installazione del software è gestita da personale tecnico specializzato, questo manuale non include istruzioni dettagliate per l'installazione, ma si concentra piuttosto sui passaggi necessari per utilizzare il software una volta installato correttamente.
\subsection{Scopo del Prodotto}
L'obiettivo del progetto è quello di creare un'applicazione web per il monitoraggio di una "Smart City", consentendo un controllo completo sul suo stato di salute. Ciò permetterà di prendere decisioni rapide ed efficaci, oltre ad analizzare gli effetti delle azioni intraprese.\\
La piattaforma è in grado di fornire informazioni chiare e in tempo reale sullo stato della città tramite una dashboard Grafana, che mette a disposizione tutti gli strumenti necessari per l'analisi delle misurazioni provenienti dai sensori. \\
Come detto in precedenza, questa piattaforma è destinata alle autorità cittadine desiderose di ottenere una visione globale della situazione urbana, fornendo informazioni chiare e in tempo reale sullo stato della città.

\subsection{Accesso alla piattaforma}
La piattaforma è presentata come una web-application accessibile esclusivamente agli utenti autorizzati. L'accesso al servizio avviene tramite un browser web, senza richiedere l'installazione di alcun software aggiuntivo sul dispositivo dell'utente. Al fine di garantire la massima sicurezza e riservatezza dei dati, l'accesso è limitato esclusivamente agli utenti in possesso del link e delle credenziali di accesso, le quali vengono fornite dal team amministrativo o da personale autorizzato. Una volta ottenuto il link e le credenziali, gli utenti possono accedere alla web-application da qualsiasi dispositivo connesso a Internet, garantendo un'esperienza di utilizzo flessibile e accessibile ovunque si trovi
\subsection{Glossario}
Per evitare possibili ambiguità che potrebbero sorgere durante la lettura dei documenti,
alcuni termini utilizzati sono stati inseriti nel documento \textit{Glossario v 2.0.0 }. \\
Sarà possibile individuare il riferimento al Glossario per mezzo
di una G a pedice del termine considerato ambiguo.
\subsection{Riferimenti}
\subsubsection{Riferimenti informativi}
    \begin{itemize}
        \item \href {https://www.math.unipd.it/~tullio/IS-1/2023/Progetto/C6.pdf} {Capitolato d'appalto C6 - InnovaCity}
        \item \href{https://www.math.unipd.it/~tullio/IS-1/2023/Dispense/T4.pdf} {Slide del corso di Ingegneria del Software - Gestione di progetto}
        \item \href{https://www.math.unipd.it/~tullio/IS-1/2023/Dispense/T2.pdf} {Slide del corso di Ingegneria del Software - Ciclo di vita del software}
    \end{itemize}

\subsubsection{Riferimenti normativi}
    \begin{itemize}
    \item Norme di progetto
    \item \href {https://www.math.unipd.it/~tullio/IS-1/2023/Dispense/PD2.pdf} {Regolamento del progetto didattico}
    \end{itemize}


\section{Descrizione del prodotto}

\subsection{Obiettivi del prodotto}
Sviluppare una \textit{piattaforma}\textsubscript{\textit{G}} di monitoraggio di una "\textit{Smart City}\textsubscript{\textit{G}}" che consenta di avere sotto
controllo lo stato di salute della città in modo tale da prendere decisioni veloci, efficaci
ed analizzare poi gli effetti conseguenti.  
A tale scopo il \textit{proponente}\textsubscript{\textit{G}} richiede di simulare dei sensori posti in diverse aree per reperire
informazioni relative alle condizioni della città come, ad esempio, temperatura, umidità,
quantità di polveri sottili nell’aria, traffico, livelli di acqua, stato di riempimento delle isole ecologiche,
guasti elettrici e molto altro.  

I dati trasmessi in tempo reale dai sensori devono poter essere memorizzati in un \textit{database}\textsubscript{\textit{G}}
in modo tale da renderli disponibili per la visualizzazione tramite una \textit{dashboard}\textsubscript{\textit{G}}, composta da \textit{widget}\textsubscript{\textit{G}} e grafici, per una visione d’insieme delle condizioni della città in
tempo reale.  
L’applicativo potrà consentire alle autorità locali di prendere decisioni informate e tempestive sulla gestione delle risorse e sull’implementazione di servizi e, inoltre, si potrebbe
rivelare uno strumento essenziale per coinvolgere i cittadini nella gestione e nel miglioramento della città.
\vspace{0.3cm}  

L’implementazione di una città monitorata da sensori rappresenta un approccio promettente nell’ottica di ottimizzare l’efficienza e la qualità della vita urbana. Tale \textit{sistema}\textsubscript{\textit{G}} consente
una raccolta continua di dati e informazioni cruciali, fornendo una base solida per l’ottimizzazione dei servizi pubblici, la gestione del traffico, la sicurezza e la sostenibilità ambientale.  



\subsection{Funzionalità del prodotto}

Il software di monitoraggio della Smart City è progettato per offrire una serie di funzionalità cruciali per gestire e migliorare le condizioni della città. \\
Le principali funzionalità includono:

\begin{enumerate}
    \item \textbf{Simulazione di sensori:} Il software consente la simulazione di sensori posizionati in diverse aree della città per raccogliere informazioni su parametri come temperatura, umidità, quantità di polveri sottili nell’aria, traffico, livelli di acqua, stato di riempimento delle isole ecologiche, guasti elettrici, e altro ancora.

    \item \textbf{Monitoraggio in tempo reale:} Il sistema raccoglie dati in tempo reale dai sensori simulati, fornendo uno stato sempre aggiornato della città.

    \item \textbf{Memorizzazione dei dati:} I dati trasmessi dai sensori vengono memorizzati in un database per garantire la disponibilità a lungo termine e consentire analisi storiche.

    \item \textbf{Visualizzazione attraverso Dashboard:} Gli utenti possono accedere ad una dashboard che offre una visione d’insieme delle condizioni della città in tempo reale. La dashboard è composta da widget e grafici che facilitano la comprensione e l'analisi dei dati.

    \item \textbf{Visualizzazione punteggio di salute:} Le informazioni ottenute dai simulatori consentono al sistema di calcolare un indice di benessere, valutato su una scala da zero a cento in base all'ultima rilevazione di ciascun sensore. Un punteggio più alto corrisponde a condizioni di vita migliori.

    \item \textbf{Supporto alle decisioni:} L'applicativo fornisce alle autorità locali strumenti per prendere decisioni informate e tempestive sulla gestione delle risorse e sull'implementazione di servizi.

    \item \textbf{Coinvolgimento dei Cittadini:} Il software può essere utilizzato come strumento per coinvolgere i cittadini nella gestione e nel miglioramento della città, fornendo loro accesso alle informazioni e coinvolgendoli attivamente nelle decisioni.

\end{enumerate}

\subsection{Caratteristiche utente}
\textbf{Autorità locali:} Gli utenti principali sono le autorità locali responsabili della gestione e del monitoraggio della Smart City. Questi utenti devono essere in grado di prendere decisioni consapevoli sulla base delle informazioni raccolte e analizzate dal \textit{sistema}\textsubscript{\textit{G}}.
\begin{itemize}
    \item L'utente dovrà utilizzare un dispositivo (Desktop o Mobile) connesso alla \textit{rete}\textsubscript{\textit{G}} per poter accedere alla \textit{piattaforma}\textsubscript{\textit{G}}.  
\end{itemize}


\section{Tecnologie}
In questa sezione sono definiti gli strumenti e le tecnologie impiegati per lo sviluppo e l'implementazione del software relativo al progetto InnovaCity.

Si procederà quindi con la descrizione delle tecnologie e dei linguaggi di programmazione utilizzati, delle librerie e dei framework necessari, nonché delle infrastrutture richieste.

L'obiettivo principale è garantire che il software sia sviluppato utilizzando le tecnologie più appropriate in termini di efficienza, sicurezza e affidabilità.

\subsection{Docker}
Per lo sviluppo, il testing e il rilascio del prodotto sono stati utilizzati container Docker in modo tale da garantire ambienti consistenti e riproducibili.

\subsubsection{Ambienti}
\begin{itemize}
  \item \textbf{Ambiente di sviluppo:}
    \begin{itemize}
      \item È l'ambiente dove i software developer scrivono, testano e modificano il codice sorgente;
      \item Può includere strumenti di debug e monitoraggio per facilitare lo sviluppo e la correzione di errori;
      \item Non è accessibile agli utenti finali.
    \end{itemize}
    \item \textbf{Ambiente di test:}
    \begin{itemize}
      \item Simula l'ambiente di produzione;
      \item Viene utilizzato per testare il software in modo completo e realistico prima del rilascio in produzione;
      \item I test vengono eseguiti automaticamente tramite Github workflow oppure manualmente in locale tramite profilo di testing docker e includono test di unità, integrazione, sistema (sicurezza, carico e prestazioni).
    \end{itemize}
    \item \textbf{Ambiente di produzione:}
    \begin{itemize}
      \item È l'ambiente dove il software viene rilasciato per poter essere utilizzato dagli utenti finali;
      \item Deve essere stabile, sicuro e performante per garantire un'esperienza utente ottimale;
      \item Le modifiche al software in produzione sono controllate rigorosamente per minimizzare i rischi di errori o downtime.
    \end{itemize}
\end{itemize}

\subsubsection{Docker images}

Di seguito sono elencate le immagini Docker utilizzate:

\begin{itemize}

  \item \textbf{Simulators - Python} 
    \begin{itemize}
      \item \textbf{Image:} Python: 3.9;
      \item \textbf{Riferimento:} \url{https://hub.docker.com/layers/library/python/3.9/images/sha256-1023bd4c5e0e6b7f4f612b034627826d91ec78ae0439313450ec30c0ad60c908?context=explore}~(consultato: 19/03/2024);
      \item \textbf{Ambiente:}
        \begin{itemize}
          \item Develop;
          \item Production.
        \end{itemize}
    \end{itemize}

  \item \textbf{Broker - Apache Kafka} 
    \begin{itemize}
      \item \textbf{Image:} confluentinc/cp-kafka: 7.6.0;
      \item \textbf{Riferimento:} \url{https://hub.docker.com/layers/confluentinc/cp-kafka/7.6.0/images/sha256-8fc15a671986983b83beecae14e013a91adcd3999f687de8b6b8153fd47e8f67?context=explore}~(consultato: 19/03/2024);
      \item \textbf{Ambiente:}
        \begin{itemize}
          \item Develop;
          \item Production;
          \item Testing;
        \end{itemize}
    \end{itemize}

  \item \textbf{Zookeeper} 
    \begin{itemize}
      \item \textbf{Image:} confluentinc/cp-zookeeper: 7.6.0;
      \item \textbf{Riferimento:} \url{https://hub.docker.com/layers/confluentinc/cp-zookeeper/7.6.0/images/sha256-6a0822643ceb4725db4f24bf2d228eee39bb5ade88f586449d87263cbc81bc97?context=explore}~(consultato: 19/03/2024);
      \item \textbf{Ambiente:}
        \begin{itemize}
          \item Develop;
          \item Production;
          \item Testing;
        \end{itemize}
    \end{itemize}

  \item \textbf{Apache Kafka UI} 
    \begin{itemize}
      \item \textbf{Image:} provectuslabs/kafka-ui: 53a6553765a806eda9905c43bfcfe09da6812035;
      \item \textbf{Riferimento:} \url{https://hub.docker.com/layers/provectuslabs/kafka-ui/53a6553765a806eda9905c43bfcfe09da6812035/images/sha256-633606ca07677d1c4b9405c5df1b6f0087aa75b36528a17eed142d06f65d0881?context=explore}~(consultato: 19/03/2024);
      \item \textbf{Ambiente:}
        \begin{itemize}
          \item Develop.
        \end{itemize}
    \end{itemize}

  \item \textbf{Schema registry} 
    \begin{itemize}
      \item \textbf{Image:} confluentinc/cp-schema-registry: 7.6.0;
      \item \textbf{Riferimento:} \url{https://hub.docker.com/layers/confluentinc/cp-schema-registry/7.6.0/images/sha256-7cea5369377b52823d3101dd22073a235a501256f6f140c66d2111224803af0b?context=explore}~(consultato: 19/03/2024);
      \item \textbf{Ambiente:}
        \begin{itemize}
          \item Develop;
          \item Production;
          \item Testing;
        \end{itemize}
    \end{itemize}

  \item \textbf{Schema registry UI} 
    \begin{itemize}
      \item \textbf{Image:} landoop/schema-registry-ui: latest;
      \item \textbf{Riferimento:} \url{https://hub.docker.com/layers/landoop/schema-registry-ui/latest/images/sha256-c8b7baf7c53224eaa066937410adae388384e3f7c6f26296ba6a98cc5880f866?context=explore}~(consultato: 19/03/2024);
      \item \textbf{Ambiente:}
        \begin{itemize}
          \item Develop.
        \end{itemize}
    \end{itemize}

  \item \textbf{Faust processing - Python} 
    \begin{itemize}
      \item \textbf{Image:} Python: 3.9;
      \item \textbf{Riferimento:} \url{https://hub.docker.com/layers/library/python/3.9/images/sha256-1023bd4c5e0e6b7f4f612b034627826d91ec78ae0439313450ec30c0ad60c908?context=explore}~(consultato: 19/03/2024);
      \item \textbf{Ambiente:}
        \begin{itemize}
          \item Develop;
          \item Production;
          \item Testing;
        \end{itemize}
    \end{itemize}

  \item \textbf{ClickHouse} 
    \begin{itemize}
      \item \textbf{Image:} clickhouse/clickhouse-server: 24.2.1.2248;
      \item \textbf{Riferimento:} \url{https://hub.docker.com/layers/clickhouse/clickhouse-server/24.2.1.2248/images/sha256-a5921f08bc3ab230e20db5970698b300279b29a353620e62729325fa8d1dc601?context=explore}~(consultato: 19/03/2024);
      \item \textbf{Ambiente:}
        \begin{itemize}
          \item Develop;
          \item Production;
          \item Testing;
        \end{itemize}
    \end{itemize}

  \item \textbf{Grafana} 
    \begin{itemize}
      \item \textbf{Image:} grafana/grafana-oss: 10.4.0;
      \item \textbf{Riferimento:} \url{https://hub.docker.com/layers/grafana/grafana-oss/10.4.0/images/sha256-c7ae30e06ee76656f4faf37df1f0d0dfb6941a706b66800a7b289a304d31d771?context=explore}~(consultato: 19/03/2024);
      \item \textbf{Ambiente:}
        \begin{itemize}
          \item Develop;
          \item Production.
        \end{itemize}
    \end{itemize}
\end{itemize}

\subsection{Linguaggi e formato dati}
\subsection{Python}

\subsubsection{Librerie o framework}
\paragraph{Confluent kafka}

\subsubsection{Utilizzo nel progetto}

\subsubsection{SQL (Structured Query Language)}
Linguaggio standard per la gestione e la manipolazione dei
database che lo supportano

\paragraph{Utilizzo nel progetto}
Gestione e interrogazione database Clickhouse.

\subsubsection{JSON (JavaScript Object Notation)}
JSON è un formato di scrittura leggibile dalle persone e facilmente interpretabile dai computer. È utilizzato principalmente per lo scambio di dati strutturati attraverso le reti, come Internet. \todo{non mi piace questa parte}

Il formato JSON si basa su due strutture di dati principali:

\begin{itemize}
  \item \textbf{Oggetti}: Rappresentati da coppie chiave-valore racchiuse tra parentesi graffe \{ \}, dove la chiave è una stringa e il valore può essere un altro oggetto, un array, una stringa, un numero, un booleano o \texttt{null}.
  \item \textbf{Array}: Una raccolta ordinata di valori, racchiusi tra parentesi quadre [ ], in cui ogni elemento può essere un oggetto, un array, una stringa, un numero, un booleano o \texttt{null}.
\end{itemize}

JSON è ampiamente impiegato in diversi contesti, tra cui lo sviluppo web, le API di servizi web e lo scambio di dati tra applicazioni, grazie alla sua sintassi semplice e chiara per la rappresentazione dei dati. La sua struttura basata su testo e la facilità di lettura lo rendono ideale per facilitare l'interazione tra sistemi eterogenei.

\paragraph{Utilizzo nel progetto}
\begin{itemize}
  \item Formato dei messaggi trasmessi dai simulatori dei sensori al broker Kafka;
  \item Configurazione dashboard Grafana.
\end{itemize}

\subsubsection{YAML (YAML Ain't Markup Language)}
Formato di serializzazione leggibile dall'uomo utilizzato per rappresentare dati strutturati in modo chiaro e semplice.

\paragraph{Utilizzo nel progetto}
\begin{itemize}
    \item Configurazione Docker Compose;
    \item Configurazione pipeline Github workflow per Countinuous Integration;
    \item Configurazione provisioning Grafana e politiche di notifica allerte.
\end{itemize}

\subsection{Database e servizi}
\subsubsection{Apache Kafka}
Apache Kafka è una piattaforma open-source di streaming distribuito sviluppata dall'Apache Software Foundation. Progettata per gestire flussi di dati in tempo reale in modo scalabile e affidabile, è ampiamente utilizzata nel data streaming e nell'integrazione dei dati nelle moderne applicazioni.

\paragraph{Versione}
La versione utilizzata è: 3.7.0
\paragraph{Documentazione}
\href{https://kafka.apache.org/20/documentation.html}{https://kafka.apache.org/20/documentation.html}

\paragraph{Funzionalità e vantaggi di Apache Kafka}
Le principali funzionalità e vantaggi di Apache Kafka includono:

\begin{itemize}
  \item \textbf{Pub-Sub Messaging:} Kafka utilizza un modello di messaggistica publish-subscribe, dove i produttori di dati inviano messaggi ad uno o più topic e i consumatori possono sottoscriversi a tali topic per ricevere i messaggi;
  
  \item \textbf{Disaccoppiamento Produttore - Consumatore:} questo principio si realizza grazie al fatto che i Produttori e i Consumatori non necessitano di essere consapevoli l'uno dell'altro o di interagire direttamente. Invece, essi comunicano attraverso il broker Kafka, che svolge il ruolo di intermediario per la trasmissione dei messaggi. Ciò consente una maggiore scalabilità e flessibilità nell'architettura del sistema, facilitando la gestione e il mantenimento delle applicazioni;
  
  \item \textbf{Architettura Distribuita:} Kafka è progettato per essere distribuito su un cluster di nodi, consentendo una scalabilità orizzontale per gestire grandi volumi di dati e carichi di lavoro. Questo approccio distribuito offre resilienza e alta disponibilità, garantendo che il sistema possa crescere in modo flessibile con l'aumentare delle richieste;
  
  \item \textbf{Persistenza e Affidabilità:} Kafka offre la possibilità di definire politiche specifiche per la conservazione dei dati, garantendo la durabilità dei messaggi. Questo non solo assicura la disponibilità dei dati anche in caso di eventuali interruzioni del servizio, ma consente anche ai consumatori di recuperare i messaggi dopo tali anomalie, garantendo un alto livello di affidabilità nel sistema.
  
  \item \textbf{Alta Disponibilità:} Kafka assicura un'elevata disponibilità e tolleranza ai guasti grazie alla sua architettura distribuita e al meccanismo di replica dei dati. Anche in caso di malfunzionamenti dei nodi o delle componenti, i cluster di Kafka mantengono la loro operatività, garantendo la continuità del servizio.
  
  \item \textbf{Elaborazione degli Stream:} Kafka supporta anche l'elaborazione degli stream di dati in tempo reale tramite API come Kafka Streams e Kafka Connect, consentendo agli sviluppatori di scrivere applicazioni per l'analisi e l'elaborazione dei dati in tempo reale.
\end{itemize}

\paragraph{Casi d'uso di Apache Kafka}

Apache Kafka è utilizzato in una vasta gamma di casi d'uso, tra cui:

\begin{itemize}
  \item \textbf{Data Integration:} Kafka viene utilizzato per integrare dati provenienti da diverse fonti e sistemi, consentendo lo scambio di dati in tempo reale tra applicazioni e sistemi eterogenei.
  
  \item \textbf{Streaming di Eventi:} Molte applicazioni moderne, come le applicazioni IoT (Internet of Things) e le applicazioni di monitoraggio in tempo reale, utilizzano Kafka per lo streaming di eventi in tempo reale e l'analisi dei dati.
  
  \item \textbf{Analisi dei Log:} Kafka è spesso utilizzato per l'analisi dei log di sistema e applicativi in tempo reale, consentendo il monitoraggio delle prestazioni, la rilevazione degli errori e l'analisi dei pattern di utilizzo.
  
  \item \textbf{Elaborazione di Big Data:} Kafka è integrato con tecnologie di big data come Apache Hadoop e Apache Spark, consentendo l'elaborazione di grandi volumi di dati in tempo reale.
  
  \item \textbf{Messaggistica Real-time:} Kafka è ampiamente utilizzato per la messaggistica real-time in applicazioni di social media, e-commerce e finanziarie, dove la velocità e l'affidabilità della messaggistica sono cruciali.
\end{itemize}

\paragraph{Utilizzo nel progetto}
\textit{Kafka} funge da intermediario dei messaggi, ricevendo i dati dai produttori e rendendoli disponibili ai consumatori. Nel contesto del progetto, i dati provenienti dalle simulazioni di sensori vengono inviati a \textit{Kafka} come messaggi in formato \textit{JSON}.

\paragraph*{Consumatori di dati:}
\begin{itemize}
  \item \textbf{\textit{ClickHouse:}} \textit{Kafka} invia \todo{è Kafka che li invia o i consumatori che se li prendono da Kafka?} i dati ai consumatori, inclusi i database come \textit{ClickHouse}, dove i dati vengono salvati per l'analisi e l'archiviazione a lungo termine.
  \item \textbf{\textit{Faust:}} per soddisfare il requisito opzionale del calcolo del punteggio di salute, \textit{Kafka} rende disponibili i dati in tempo reale a un'applicazione di Faust\todo{è corretto applicazione di Faust?}. Quest'ultima elabora i dati utilizzando una funzione di aggregazione per calcolare il punteggio e quindi mette a disposizione il risultato in una coda dedicata di Kafka per i servizi interessati.
\end{itemize}

In breve, \textit{Kafka} funge da ponte tra i produttori di dati (simulazioni di sensori) e i consumatori di dati (\textit{ClickHouse} o altri servizi futuri). Gestisce il flusso dei dati in tempo reale e garantisce che i dati siano disponibili per l'elaborazione e la visualizzazione in modo efficiente e scalabile.
\subsubsection{Schema Registry}
Schema Registry è un componente importante nell'ecosistema di Apache Kafka, progettato per la gestione e la convalida degli schemi dei dati utilizzati all'interno di un sistema di messaggistica distribuita.
\paragraph{Versione}
Versione utilizzata: 7.6.0
\paragraph{Documentazione}
\url{https://docs.confluent.io/platform/current/schema-registry/index.html}

\paragraph{Funzionalità e Vantaggi di Schema Registry}
Le funzionalità principali di Schema Registry includono:
\begin{itemize}
    \item \textbf{Gestione centralizzata degli schemi}: Fornisce un repository centralizzato per la gestione degli schemi dei dati.
    Contribuisce alla governance dei dati garantendo la qualità, la conformità agli standard e la tracciabilità dei dati;
    \item \textbf{Convalida degli schemi}: Assicura la validità e la compatibilità degli schemi dei dati;
    \item \textbf{Serializzazione e deserializzazione}: Supporta la serializzazione e la deserializzazione dei dati basati sugli schemi su reti distribuite.
\end{itemize}

\paragraph{Utilizzo nel progetto}
Nell'ambito del progetto didattico schema registry permette di validare i messaggi nell'ambito del topic kakfa di appartenenza definendo un contratto che i produttori, ovvero i sensori, dovranno rispettare nell'invio delle misurazioni.
\input{Sottosezioni/Specifica_tecnica/Zookeper.tex}
\subsection{ClickHouse} \label{sec:clickHouse}
ClickHouse è un sistema di gestione di database (DBMS) di tipo column-oriented, progettato principalmente per l'analisi di grandi volumi di dati in tempo reale. È un progetto open-source sviluppato da Yandex, un motore di ricerca russo, ed è stato creato per rispondere alle esigenze di elaborazione analitica ad alte prestazioni.
\subsubsection{Versione}
La versione utilizzata è: 24.1.5.6
\subsubsection{Link download}
\href{https://clickhouse.com/}{https://clickhouse.com/}

\subsubsection*{Funzionalità e Vantaggi di ClickHouse}
\begin{itemize}
    \item \textbf{ Modello di dati column-oriented:} a differenza dei tradizionali DBMS che memorizzano i dati in modo row-oriented, dove le righe complete sono memorizzate in sequenza, ClickHouse memorizza i dati in modo column-oriented. Questo significa che i dati di ogni colonna sono memorizzati insieme, permettendo una maggiore compressione e velocità di query per le analisi che coinvolgono molte colonne;
    \item \textbf{Architettura Distribuita e scalabilità:} ClickHouse è progettato per funzionare in un ambiente distribuito, consentendo la scalabilità orizzontale per gestire grandi carichi di lavoro;
    \item \textbf{Compressione dei Dati:} utilizza algoritmi efficienti per ridurre lo spazio di archiviazione richiesto per i dati, riducendo i costi di archiviazione;
    \item \textbf{Alte Prestazioni:} ottimizzato per eseguire query analitiche su grandi volumi di dati in tempo reale, garantendo tempi di risposta bassi anche con carichi di lavoro elevati.
    \item \textbf{Supporto per SQL:} supporta un sottoinsieme del linguaggio SQL, consentendo agli sviluppatori di scrivere query complesse per l'analisi dei dati;
    \item \textbf{Integrazione con Strumenti di Business Intelligence (BI):} può essere integrato con strumenti di BI popolari come Tableau, Power BI, Qlik, Grafana per la visualizzazione e l'analisi dei dati.
\end{itemize}


\subsubsection*{Casi d'Uso di ClickHouse}
ClickHouse è adatto per una vasta gamma di casi d'uso, tra cui:
\begin{itemize}
    \item \textbf{Analisi dei Log:} clickHouse può essere utilizzato per analizzare i log di grandi dimensioni generati da server, applicazioni web e dispositivi IoT;
    \item \textbf{Analisi dei Dati in Tempo Reale:} ClickHouse è ideale per l'analisi dei dati in tempo reale, consentendo agli utenti di eseguire query complesse su flussi di dati in continua evoluzione;
    \item \textbf{Reporting e Dashboard:} ClickHouse può essere utilizzato per generare report e dashboard interattivi per monitorare le prestazioni del business e identificare tendenze.
\end{itemize}



\subsection{Grafana}
Grafana è una piattaforma open-source per la visualizzazione e l'analisi dei dati, utilizzata per creare dashboard interattive e grafici da fonti di dati eterogenee. 
\subsubsection{Versione}
La versione utilizzata è: x.x.x
\subsubsection{Link download}
\href{https://clickhouse.com/}{https://clickhouse.com/}

\subsubsection{Funzionalità e Vantaggi di Grafana}
\begin{itemize}
    \item \textbf{Dashboard interattive}: Creazione di dashboard personalizzate e interattive per visualizzare dati provenienti da diverse fonti in un'unica interfaccia.
    
    \item \textbf{Connessione a sorgenti di dati eterogenee}: Supporto per una vasta gamma di sorgenti di dati, inclusi database, servizi cloud, sistemi di monitoraggio, API e altro ancora.
    
    \item \textbf{Ampia varietà di visualizzazioni}: Selezione di pannelli e visualizzazioni, tra cui grafici a linea, a barre, a torta, termometri, mappe geografiche e altro ancora, per adattarsi alle esigenze specifiche di visualizzazione dei dati.
    
    \item \textbf{Query e aggregazioni flessibili}: Esecuzione di query flessibili e aggregazione dei dati in modi personalizzati per ottenere insight approfonditi dai dati.
    
    \item \textbf{Notifiche e allarmi}: Impostazione di avvisi in base a criteri predefiniti, come soglie di performance, e ricezione di notifiche tramite diversi canali, tra cui email, Slack e molti altri.
    
    \item \textbf{Gestione degli accessi e dei permessi}: Controllo degli accessi e dei permessi degli utenti in modo granulare, gestendo chi può visualizzare, modificare o creare dashboard e pannelli.
    
    \item \textbf{Integrazione con altre applicazioni e strumenti}: Integrazione con una vasta gamma di applicazioni e strumenti, tra cui sistemi di log management, strumenti di monitoraggio delle prestazioni, sistemi di allerta e altro ancora.
    
   \end{itemize}
\subsubsection{Casi d'Uso di Grafana}
\begin{itemize}
    \item \textbf{Monitoraggio delle prestazioni}: Monitoraggio in tempo reale delle metriche di sistema come CPU, memoria e rete per identificare e risolvere rapidamente problemi di prestazioni.
    
    \item \textbf{Analisi dei log}: Analisi e visualizzazione dei log delle applicazioni e dell'infrastruttura per individuare pattern e risolvere problemi operativi.
    
    \item \textbf{Monitoraggio dell'infrastruttura}: Monitoraggio dello stato e delle prestazioni di server, servizi cloud, database e altri componenti IT per garantire un funzionamento ottimale dell'infrastruttura.
    
    \item \textbf{DevOps e CI/CD}: Monitoraggio dei processi di sviluppo, test e distribuzione del software per migliorare la collaborazione e l'efficienza del team.
    
    \item \textbf{Monitoraggio di dispositivi IoT}: Monitoraggio dei dispositivi IoT per raccogliere e visualizzare dati di sensori e dispositivi connessi, consentendo una gestione efficiente degli ambienti IoT.
\end{itemize}





\section{Casi d'uso}
\section{Introduzione}
\subsection{Scopo del Manuale}
Il presente manuale è concepito per fornire un supporto completo agli utenti Autorità Locale nell'utilizzo efficace del software, consentendo loro di sfruttare appieno tutte le sue funzionalità al fine di garantire un'esperienza ottimale. \\
Poiché l'installazione del software è gestita da personale tecnico specializzato, questo manuale non include istruzioni dettagliate per l'installazione, ma si concentra piuttosto sui passaggi necessari per utilizzare il software una volta installato correttamente.
\subsection{Scopo del Prodotto}
L'obiettivo del progetto è quello di creare un'applicazione web per il monitoraggio di una "Smart City", consentendo un controllo completo sul suo stato di salute. Ciò permetterà di prendere decisioni rapide ed efficaci, oltre ad analizzare gli effetti delle azioni intraprese.\\
La piattaforma è in grado di fornire informazioni chiare e in tempo reale sullo stato della città tramite una dashboard Grafana, che mette a disposizione tutti gli strumenti necessari per l'analisi delle misurazioni provenienti dai sensori. \\
Come detto in precedenza, questa piattaforma è destinata alle autorità cittadine desiderose di ottenere una visione globale della situazione urbana, fornendo informazioni chiare e in tempo reale sullo stato della città.

\subsection{Accesso alla piattaforma}
La piattaforma è presentata come una web-application accessibile esclusivamente agli utenti autorizzati. L'accesso al servizio avviene tramite un browser web, senza richiedere l'installazione di alcun software aggiuntivo sul dispositivo dell'utente. Al fine di garantire la massima sicurezza e riservatezza dei dati, l'accesso è limitato esclusivamente agli utenti in possesso del link e delle credenziali di accesso, le quali vengono fornite dal team amministrativo o da personale autorizzato. Una volta ottenuto il link e le credenziali, gli utenti possono accedere alla web-application da qualsiasi dispositivo connesso a Internet, garantendo un'esperienza di utilizzo flessibile e accessibile ovunque si trovi
\subsection{Glossario}
Per evitare possibili ambiguità che potrebbero sorgere durante la lettura dei documenti,
alcuni termini utilizzati sono stati inseriti nel documento \textit{Glossario v 2.0.0 }. \\
Sarà possibile individuare il riferimento al Glossario per mezzo
di una G a pedice del termine considerato ambiguo.
\subsection{Riferimenti}
\subsubsection{Riferimenti informativi}
    \begin{itemize}
        \item \href {https://www.math.unipd.it/~tullio/IS-1/2023/Progetto/C6.pdf} {Capitolato d'appalto C6 - InnovaCity}
        \item \href{https://www.math.unipd.it/~tullio/IS-1/2023/Dispense/T4.pdf} {Slide del corso di Ingegneria del Software - Gestione di progetto}
        \item \href{https://www.math.unipd.it/~tullio/IS-1/2023/Dispense/T2.pdf} {Slide del corso di Ingegneria del Software - Ciclo di vita del software}
    \end{itemize}

\subsubsection{Riferimenti normativi}
    \begin{itemize}
    \item Norme di progetto
    \item \href {https://www.math.unipd.it/~tullio/IS-1/2023/Dispense/PD2.pdf} {Regolamento del progetto didattico}
    \end{itemize}


\subsection{Attori}
Il \textit{sistema}\textsubscript{\textit{G}} si interfaccerà con due attori distinti:

\begin{itemize}
    \item \textbf{Autorità locale:} avrà accesso esclusivo alla visualizzazione della \textit{dashboard}\textsubscript{\textit{G}} relativa allo stato della città; l'applicazione non richiede autenticazione.  
    \item \textbf{Sensore:} un dispositivo di misurazione in grado di acquisire dati dal suo dominio di interesse e di inserirli nel \textit{sistema}\textsubscript{\textit{G}} per consentirne l'archiviazione permanente.  
\end{itemize}

\subsection{Elenco dei casi d'uso}

%---------------------------- UC1 ---------------------------------
\begin{figure}[H]
    \centering
    \includegraphics[width=0.9\textwidth]{../Images/uc1.png}
    \caption{UC1 - Visualizzazione stato città}
    \label{fig:UC1}
\end{figure}
\subsubsection{UC1 - VISUALIZZAZIONE DASHBOARD}
\begin{itemize}
    \item \textbf{Attore principale:} Autorità locale.
    \item \textbf{Precondizioni:}
        \begin{itemize}
            \item Il \textit{sistema}\textsubscript{\textit{G}} è operativo e accessibile.
        \end{itemize}
    \vspace{0,5cm}
    \item \textbf{Postcondizioni:}
    \begin{itemize}
        \item  L'autorità locale ha una visione aggiornata dello stato di salute della città tramite \textit{widget}\textsubscript{\textit{G}} e grafici interattivi aggiornati in tempo reale, una mappa dei sensori presenti nella città e un punteggio di salute relativo alla città.
    \end{itemize}
    \item \textbf{Scenario principale:}
        \begin{enumerate}
            \item L'autorità locale accede alla \textit{piattaforma}\textsubscript{\textit{G}} per la visualizzazione della \textit{dashboard}\textsubscript{\textit{G}};
            \item Il \textit{sistema}\textsubscript{\textit{G}} elabora le informazioni ricevute dai sensori;
            \item Il \textit{sistema}\textsubscript{\textit{G}} imposta la visualizzazione dei \textit{widget}\textsubscript{\textit{G}} sulla \textit{dashboard}\textsubscript{\textit{G}}.
        \end{enumerate}
    \item \textbf{User story associata:} \\
        Come autorità locale, voglio accedere alla \textit{dashboard}\textsubscript{\textit{G}} per visualizzare in tempo reale i dati provenienti dai diversi tipi di sensori presenti nella città. Questo mi consentirà di valutare rapidamente lo stato generale della città e prendere decisioni informate e tempestive sulla gestione delle risorse e sull'implementazione di servizi.
\end{itemize}

%---------------------------- SUB_UC1 ---------------------------------
\begin{figure}[H]
    \centering
    \includegraphics[width=0.9\textwidth]{../Images/uc1_Subcase.PNG} 
    \caption{Sottocasi UC1 - Visualizzazione stato città}
    \label{fig:UC1_sub}
\end{figure}

%---------------------------- UC1.1 ---------------------------------
\subsubsection{UC1.1 - VISUALIZZAZIONE COMPLESSIVA DEI WIDGET DI MISURAZIONE PER OGNI TIPOLOGIA DI SENSORE}
\begin{itemize}
    \item \textbf{Attore principale:} Autorità locale.
    \item \textbf{Precondizioni:}
        \begin{itemize}
            \item Il sistema è operativo e accessibile.
        \end{itemize}
    \item \textbf{Postcondizioni:}
        \begin{itemize}
            \item L'autorità locale visualizza tutti i widget che mostrano le misurazioni aggiornate in tempo reale per ogni tipo di sensore. Questi widget presentano i dati registrati nell'ultima ora, se presenti.
        \end{itemize}
    \item \textbf{Scenario principale:}
        \begin{enumerate}
            \item L'autorità locale accede alla piattaforma per la visualizzazione della dashboard della città. (UC1)
            \item Il sistema elabora le informazioni ricevute dai sensori e imposta la visualizzazione di un widget con le misurazioni dell'ultima ora per ogni tipolgia di sensore;
        \end{enumerate}
    \item \textbf{User story associata:} \\
        Come autorità locale, nella dashboard desidero visualizzare un widget dedicato per ciascun tipo di sensore, contenente le misurazioni in tempo reale relative all'ultima ora, al fine di ottenere una panoramica completa sulle ultime misurazioni.
\end{itemize}


%---------------------------- UC1.2 ---------------------------------
\subsubsection{UC1.2 - VISUALIZZAZIONE WIDGET MAPPA INTERATTIVA DEI SENSORI}
\begin{itemize}
    \item \textbf{Attore principale:} Autorità locale.
    \item \textbf{Precondizioni:}
        \begin{itemize}
            \item  Il \textit{sistema}\textsubscript{\textit{G}} è operativo e accessibile.
        \end{itemize}
    \item \textbf{Postcondizioni:}
        \begin{itemize}
            \item L’autorità locale ha una visione grafica aggiornata della mappa dei sensori nella città, se presenti, con indicazione chiara della loro posizione e tipologia.
        \end{itemize}
    \item \textbf{Scenario principale:}
        \begin{enumerate}
            \item L'autorità locale accede alla \textit{piattaforma}\textsubscript{\textit{G}} per la visualizzazione della \textit{dashboard}\textsubscript{\textit{G}}. (UC1)
            \item Il \textit{sistema}\textsubscript{\textit{G}} elabora i dati e imposta i sensori nella posizione corretta all'interno della mappa.
        \end{enumerate}
    \item \textbf{User story associata:} \\
        Come autorità locale, desidero essere in grado di visualizzare una mappa interattiva contenente i sensori attivi e operativi all’interno della città. La mappa deve mostrare chiaramente la posizione di ciascun \textit{sensore}\textsubscript{\textit{G}} e deve essere etichettata per consentire un riconoscimento immediato della tipologia di ogni \textit{sensore}\textsubscript{\textit{G}}. Questa visualizzazione intuitiva e dettagliata mi permetterà di valutare rapidamente la distribuzione dei sensori nella città e di prendere decisioni informate per ottimizzare la copertura e l'efficacia del monitoraggio ambientale.
\end{itemize}

%---------------------------- UC1.3 ---------------------------------
\subsubsection{UC1.3 - Visualizzazione punteggio di salute città}
\begin{itemize}
    \item \textbf{Attore principale:} Autorità locale.
    \item \textbf{Descrizione:} L'autorità locale accede alla dashboard della città e visualizza un widget contenente un punteggio che rappresenta lo stato di salute della città.
    \item \textbf{Scenario principale:}
          \begin{enumerate}
              \item L'utente visualizza un widget contente un punteggio che rappresenta lo stato di salute della città.
          \end{enumerate}
    \item \textbf{Precondizioni:}
          \begin{itemize}
              \item Almeno un sensore è attivo e ha trasmesso dati;
              \item L'utente si trova nella dashboard della città. (UC1)
          \end{itemize}
    \item \textbf{Postcondizioni:}
          \begin{itemize}
              \item      L'utente ha una visione aggiornata di un punteggio, un numero intero, rappresentante lo stato di salute della città.
          \end{itemize}
    \item \textbf{User story associata:}
          \begin{itemize}
              \item Come autorità locale, desidero avere la capacità di visualizzare un punteggio ottenuto tramite una funzione di aggregazione, il quale fornisca una visione immediata di eventuali dati anomali rilevati dai sensori disseminati nella città.
          \end{itemize}
\end{itemize}

%---------------------------- UC2 ---------------------------------
\begin{figure}[H]
    \centering
    \includegraphics[width=0.9\textwidth]{../Images/uc2.png}
    \caption{UC2 - Visualizzazione stato cella}
    \label{fig:UC2}
\end{figure}
\subsubsection{UC2 - FILTRO VISUALIZZAZIONE DASHBOARD CELLA}
\begin{itemize}
    \item \textbf{Attore principale:} Autorità locale.
    %\item \textbf{Descrizione:} L'autorità locale effettua la selezione della cella, ossia la specifica zona urbana, al fine di visualizzare in tempo reale i dati provenienti da varie tipologie di sensori ubicati nella suddetta area. Ciò permette una valutazione reapida dello stato complessivo della cella.
    \item \textbf{Scenario principale:}
        \begin{enumerate}
            \item L'utente seleziona la cella per la quale desidera visualizzare la dashboard contenente esclusivamente i dati correlati a essa;
            \item Il sistema rielabora le informazioni presenti nella dashboard considerando solo quelle provenienti dalla cella selezionata.
        \end{enumerate}
    \item \textbf{Precondizioni:}
        \begin{itemize}
            \item  Almeno una cella è presente nella città.
            \item Il sistema ha caricato la visualizzazione della dashboard (UC1);
        \end{itemize}
    \item \textbf{Postcondizioni:}
        \begin{itemize}
            \item  L'utente ha una visione aggiornata dello stato di salute della cella tramite widget e grafici interattivi aggiornati in tempo reale sulla base di dati correlati esclusivamente alla cella, inoltre visualizza una mappa dei sensori presenti nella cella e un punteggio di salute relativo alla cella.
          \end{itemize}
    \item \textbf{User story associata:}
        \begin{itemize}
            \item Come autorità locale, desidero poter selezionare una specifica cella urbana sulla piattaforma al fine di visualizzare immediatamente i dati provenienti da vari sensori presenti nell'area. Questo mi permetterà di valutare rapidamente lo stato complessivo della cella e prendere decisioni informate.
        \end{itemize}
\end{itemize}

%---------------------------- UC3 ---------------------------------
\begin{figure}[H]
    \centering
    \includegraphics[width=0.9\textwidth]{../Images/uc3.png}
    \caption{UC3 - Visualizzazione storico dati }
    \label{fig:UC3}
\end{figure}
\subsubsection{UC3 - VISUALIZZAZIONE WIDGET MISURAZIONI DI UNA TIPOLOGIA SPECIFICA DI SENSORI}

\begin{itemize}
    \item \textbf{Attore principale:} Autorità locale;
    %\item \textbf{Descrizione:} L'autorità locale procede all'accesso alla piattaforma e visualizza dati storici dei sensori.
    \item \textbf{Scenario principale:}
        \begin{enumerate}
            \item Il sistema carica i dati e imposta la visualizzazione di un widget di misurazioni di una specifica tipoligia di sensori.
        \end{enumerate}
    \item \textbf{Precondizioni:}
        \begin{itemize}
            \item L'autorità locale ha effettuato l'accesso alla piattaforma.
            \item Il sistema ha caricato la visualizzazione della dashboard (UC1);
        \end{itemize}
    \item \textbf{Postcondizioni:}
        \begin{itemize}
            \item   L'autorità locale ha una visione di un widget di misurazioni per una specifica tipoligia di sensori.
        \end{itemize}
    \item \textbf{User story associata:}
        \begin{itemize}
            \item Come autorità locale, voglio accedere alla piattaforma e visualizzare un widget che rappresenti le misurazioni dei sensori di una tipolgia specifica per poter analizzare le informazione che li riguardano.
        \end{itemize}
\end{itemize}

%---------------------------- SUB_UC3 ---------------------------------
\begin{figure}[H]
    \centering
    \includegraphics[width=0.9\textwidth]{../Images/uc3_Subcase.png}
    \caption{Sottocasi UC3 - Visualizzazione storico dati }
    \label{fig:UC3_sub}
\end{figure}

%---------------------------- UC3.1 ---------------------------------
\subsubsection{UC3.1 - Visualizzazione storico dati città per una specifica tipologia di sensore}
\begin{itemize}
    \item \textbf{Attore principale:} Autorità locale;
    \item \textbf{Descrizione:} L'autorità locale procede all'accesso alla piattaforma e, mediante la dashboard relativa allo stato di salute della città, effettua la selezione di una specifica categoria di sensori al fine di visualizzare esclusivamente i dati storici ad essi pertinenti.
    \item \textbf{Scenario principale:}
        \begin{enumerate}
            \item L'utente seleziona la tipologia di sensori di cui vuole visionare lo storico dei dati trasmessi.
        \end{enumerate}
    \item \textbf{Precondizioni:}
        \begin{itemize}
            \item  L'utente di trova  nella piattaforma per la visualizzazione della dashboard sullo stato della città (UC1);
            \item  Almeno un sensore della categoria desiderata per la visualizzazione dello storico ha trasmesso dati.
        \end{itemize}
    \item \textbf{Postcondizioni:}
        \begin{itemize}
            \item  L'utente ha una visione esclusiva aggregata dello storico dei dati trasmessi da tutti sensori della tipologia selezionata.
        \end{itemize}
    \item \textbf{User story associata:}
        \begin{itemize}
            \item Come Autorità Locale, desidero accedere alla piattaforma e visualizzare esclusivamente lo storico dei dati trasmessi da una specifica tipologia di sensori nella città, attraverso la dashboard relativa allo stato di salute della città.
        \end{itemize}
\end{itemize}

%---------------------------- UC3.2 ---------------------------------
\subsubsection{UC3.2 - Visualizzazione Storico Dati cella per una Specifica Tipologia di Sensore}
\begin{itemize}
      \item \textbf{Attore principale:} Autorità locale;
      \item \textbf{Descrizione:} L’autorità locale procede all’accesso alla piattaforma e, mediante la dashboard relativa allo stato di salute della cella, effettua la selezione di una specifica categoria di sensori al fine di visualizzare esclusivamente i dati storici ad essi pertinenti.
      \item \textbf{Scenario principale:}
      \begin{enumerate}
            \item L'utente seleziona la tipologia di sensori di cui vuole visionare lo storico dei dati trasmessi nella cella.
      \end{enumerate}
      \item \textbf{Precondizioni:}
      \begin{itemize}
            \item L'utente di trova nella piattaforma per la visualizzazione della dashboard sullo stato della cella (UC2);
            \item  Almeno un sensore della categoria desiderata per la visualizzazione dello storico è presente nella cella ha trasmesso dati.
      \end{itemize}
      \item \textbf{Postcondizioni:}
      \begin{itemize}
            \item  L'utente ha una visione dello storico dei dati trasmessi dai sensori della tipologia selezionata presenti nella cella.
      \end{itemize}
      \item \textbf{User story associata:}
      \begin{itemize}
            \item Come autorità locale, desidero accedere alla piattaforma e visualizzare lo storico dei dati trasmessi da una specifica tipologia di sensori in una cella, in modo che possa analizzare e comprendere l'andamento storico dei dati relativi a quella categoria di sensori per prendere decisioni informate sullo stato della cella.
      \end{itemize}
\end{itemize}

\begin{figure}[H]
    \centering
    \includegraphics[width=0.9\textwidth]{../Images/uc3_1Gen.PNG}
    \caption{UC3.2 - Visualizzazione storico dati }
    \label{fig:UC3_gen}
\end{figure}

%---------------------------- UC3.1 ---------------------------------
\subsubsection{UC4 - VISUALIZZAZIONE MISURAZIONI IN FORMATO TESTUALE TIME SERIES}
\begin{itemize}
      \item \textbf{Attore principale:} Autorità locale;
      \item \textbf{Precondizioni:}
            \begin{itemize}
                  \item Il \textit{sistema}\textsubscript{\textit{G}} è operativo e accessibile;
                  \item Il \textit{sistema}\textsubscript{\textit{G}} carica e configura la visualizzazione di un \textit{widget}\textsubscript{\textit{G}} di una specifica tipologia di sensori (UC1.1.1);
                  \item Il \textit{sistema}\textsubscript{\textit{G}} carica e configura la visualizzazione all'interno del \textit{widget}\textsubscript{\textit{G}} delle misurazioni dei sensori coinvolti.
            \end{itemize}
      \item \textbf{Postcondizioni:}
            \begin{itemize}
                  \item L'utente visualizza le misurazioni associate al \textit{widget}\textsubscript{\textit{G}} nel formato: \texttt{ID\_sensore, TIMESTAMP, dato}.
            \end{itemize}
      \item \textbf{Scenario principale:}
            \begin{enumerate}
                  \item L'autorità locale seleziona la visualizzazione delle misurazioni associato al \textit{widget}\textsubscript{\textit{G}} in formato testuale.
            \end{enumerate}
      \item \textbf{User story associata:} \\
            Come autorità locale, desidero visualizzare le misurazioni all'interno di uno specifico \textit{widget}\textsubscript{\textit{G}} nel formato testuale: \texttt{ID\_sensore, TIMESTAMP, Dato}. Questo consente di ottenere una visione dettagliata di ogni misurazione trasmessa dai sensori.
\end{itemize}

%---------------------------- UC3.2 ---------------------------------
\subsubsection{UC5 - VISUALIZZAZIONE MISURAZIONI IN FORMATO GRAFICO}
\begin{itemize}
    \item \textbf{Attore principale:} Autorità locale;
   % \item \textbf{Descrizione:} L’autorità locale seleziona i/il sensore/i della quale vuole visionare lo storico dei dati e imposta la visulizzazione in formato grafico.
    \item \textbf{Scenario principale:}
          \begin{enumerate}
              \item L'utente imposta la visualizzazione in formato grafico.
          \end{enumerate}
    \item \textbf{Precondizioni:}
          \begin{itemize}
            \item Il sistema ha caricato e impostato la visualizzazione di un widget di una specifica tipolgia di sensori (UC3);
          \end{itemize}
    \item \textbf{Postcondizioni:}
          \begin{itemize}
              \item  L'utente ha una visione delle misurazioni trasmesse nel formato grafico.
          \end{itemize}
    \item \textbf{User story associata:}
          \begin{itemize}
              \item Come Autorità locale, voglio visualizzare le misurazioni dei sensori in formato grafico per analizzarne le tendenze nel tempo.
          \end{itemize}
\end{itemize}


%---------------------------- UC6 ---------------------------------
\begin{figure}[H]
    \centering
    \includegraphics[width=0.9\textwidth]{../Images/uc6.png}
    \caption{UC6 - VISUALIZZAZIONE WIDGET SENSORI TEMPERATURA}
    \label{fig:UC6}
\end{figure}
\subsubsection{UC6 - Visualizzazione informazioni sensore}
\begin{itemize}
    \item \textbf{Attore principale:} Autorità locale;
    \item \textbf{Descrizione:} L’autorità locale dalla pagina adibita alla visione dello storico dati di un sensore seleziona la visualizzazione delle informazioni sul sensore.
    \item \textbf{Scenario principale:}
          \begin{enumerate}
              \item L'utente seleziona la visualizzazione dell'informazioni del sensore.
          \end{enumerate}
    \item \textbf{Precondizioni:}
          \begin{itemize}
              \item  L'utente di trova nella pagina di visualizzazione dello storico dei dati di un sensore. (3.3)
          \end{itemize}
    \item \textbf{Postcondizioni:}
          \begin{itemize}
              \item  L'utente ha una visione delle informazioni del sensore.
          \end{itemize}
    \item \textbf{User story associata:}
          \begin{itemize}
              \item Come Autorità locale, desidero visualizzare le informazioni dettagliate di un sensore dalla pagina dedicata al relativo storico dei dati, al fine di ottenere una visione completa delle caratteristiche e delle specifiche del sensore selezionato.
          \end{itemize}
\end{itemize}

%---------------------------- uc7 ---------------------------------
\begin{figure}[H]
    \centering
    \includegraphics[width=0.9\textwidth]{../Images/uc7.png}
    \caption{Sottocasi UC6 - Visualizzazione informazioni sensore}
    \label{fig:UC6_sub}
\end{figure}

\subsubsection{UC7 - VISUALIZZAZIONE WIDGET SENSORI UMIDITÀ}
\begin{itemize}
    \item \textbf{Attore principale:} Autorità locale;
    \item \textbf{Precondizioni:}
        \begin{itemize}
            \item Il sistema ha caricato la visualizzazione della dashboard (UC1).
        \end{itemize}
    \item \textbf{Postcondizioni:}
        \begin{itemize}
            \item L'autorità locale visualizza un widget contenente le misurazioni relative ai sensori di umidità.
        \end{itemize}
    \item \textbf{Scenario principale:}
        \begin{enumerate}
            \item Il sistema carica i dati e imposta la visualizzazione del widget contenente le misurazioni relative ai sensori di umidità.
        \end{enumerate}
    \item \textbf{User story associata:} \\
        Come autorità locale, desidero visualizzare un widget per la visualizzazione delle misurazioni trasmesse dai sensori di umidità. Questo mi permetterà di analizzare in modo approfondito i dati relativi a quella tipologia di sensori, aiutandomi a prendere decisioni mirate per migliorare i servizi della città.
\end{itemize}
%---------------------------- uc8 ---------------------------------
\begin{figure}[H]
    \centering
    \includegraphics[width=0.9\textwidth]{../Images/uc8.PNG}
    \caption{Sottocasi UC6 - Visualizzazione informazioni sensore}
    \label{fig:UC6_sub}
\end{figure}

\subsubsection{UC8 - Inserimento date non valide}
\begin{itemize}
    \item \textbf{Attore principale:} Autorità locale;
    \item \textbf{Descrizione:} L’autorità locale seleziona date di arco temporale non valide (Data fine precedente a data inizio, Arco temporale precedente o antecedente all'inizio della trasmissione dati).
    \item \textbf{Scenario principale:}
          \begin{enumerate}
              \item L'utente imposta due date delimitanti entro cui intende esaminare l'andamento storico delle informazioni;
              \item Le date inserite non sono valide.
          \end{enumerate}
    \item \textbf{Precondizioni:}
          \begin{itemize}
              \item  L'utente si trova in un interfaccia per la visualizzazione di uno storico dati  (UC3).
              \item L'utente imposta due date delimitanti entro cui intende esaminare l'andamento storico delle informazioni
                    non valide.
          \end{itemize}
    \item \textbf{Postcondizioni:}
          \begin{itemize}
              \item  L'utente riceve una notifica di errore che richiede il reinserimento di date valide.
          \end{itemize}
    \item \textbf{User story associata:}
          \begin{itemize}
              \item Come autorità locale, voglio essere avvisato quando inserisco date non valide durante la configurazione dello storico dati, in modo da correggere immediatamente e ottenere informazioni accurate.
          \end{itemize}
\end{itemize}
%---------------------------- uc9 ---------------------------------
\begin{figure}[H]
    \centering
    \includegraphics[width=0.9\textwidth]{../Images/uc9.png}
    \caption{Sottocasi UC6 - Visualizzazione informazioni sensore}
    \label{fig:UC6_sub}
\end{figure}

\subsubsection{UC9 - VISUALIZZAZIONE WIDGET SENSORI DI GUASTI ELETTRICI}
\begin{itemize}
    \item \textbf{Attore principale:} Autorità locale;
    %\item \textbf{Descrizione:} L’autorità locale dalla pagina adibita alla visione dello storico dati di un sensore seleziona la visualizzazione delle informazioni sul sensore.
    \item \textbf{Scenario principale:}
          \begin{enumerate}
              \item L'autorità locale seleziona la visualizzazione del widget dei sensori di guasti elettrici.
          \end{enumerate}
    \item \textbf{Precondizioni:}
          \begin{itemize}
              \item  Almeno un sensore di guasti elettrici ha trasmesso dati al sistema.
          \end{itemize}
    \item \textbf{Postcondizioni:}
          \begin{itemize}
              \item  L'autorità locale ha una visione di un widget contenente le informazioni dei sensori di guasti elettrici.
          \end{itemize}
    \item \textbf{User story associata:}
          \begin{itemize}
              \item Come Autorità locale, desidero visualizzare un widget per la visualizzazione delle misurazione dei sensori di guasti elettrici.
          \end{itemize}
\end{itemize}
%---------------------------- uc10 ---------------------------------
\begin{figure}[H]
    \centering
    \includegraphics[width=0.9\textwidth]{../Images/uc10.png}
    \caption{Sottocasi UC6 - Visualizzazione informazioni sensore}
    \label{fig:UC6_sub}
\end{figure}

\subsubsection{UC10 - Rimozione dati preferiti}
\begin{itemize}
    \item \textbf{Attore principale:} Autorità locale;
    \item \textbf{Descrizione:} L’autorità locale, dalla pagina adibita alla visione dei dati salvati tra i preferiti, rimuove un dato dalla lista.
    \item \textbf{Scenario principale:}
          \begin{enumerate}
              \item L'utente accede alla piattaforma per la visualizzazione della dashboard sullo stato della città o di una cella(UC1) (UC1.1);
              \item L'utente sceglie di visualizzare la pagina dedicata alla visualizzazione dei dati preferiti.
              \item L'utente rimuove uno dei dati dalla lista.
          \end{enumerate}
    \item \textbf{Precondizioni:}
          \begin{itemize}
              \item  L'utente di trova nella pagina per la visulizzazione dei dati preferiti.
          \end{itemize}
    \item \textbf{Postcondizioni:}
          \begin{itemize}
              \item  Il dato del sensore viene rimosso dai preferiti.
          \end{itemize}
    \item \textbf{User story associata:}
          \begin{itemize}
              \item Come autorità locale,
                    Desidero poter rimuovere un dato dalla lista dei preferiti sulla piattaforma,
                    Per poter gestire in modo efficiente i dati visualizzati sulla dashboard.
          \end{itemize}
\end{itemize}
%---------------------------- uc11 ---------------------------------
\begin{figure}[H]
    \centering
    \includegraphics[width=0.9\textwidth]{../Images/uc11.PNG}
    \caption{Sottocasi UC6 - Visualizzazione informazioni sensore}
    \label{fig:UC6_sub}
\end{figure}

\subsubsection{UC11 - VISUALIZZAZIONE WIDGET SENSORI COLONNINE DI RICARICA}
\begin{itemize}
    \item \textbf{Attore principale:} Autorità locale;
    \item \textbf{Precondizioni:}
        \begin{itemize}
            \item Almeno un sensore di occupazione delle colonnine di ricarica ha trasmesso dati al sistema;
            \item Il sistema ha caricato la visualizzazione della dashboard (UC1).
        \end{itemize}
    \item \textbf{Postcondizioni:}
        \begin{itemize}
            \item L'autorità locale visualizza un widget contenente le misurazioni relative ai sensori di occupazione delle colonnine di ricarica.
        \end{itemize}
    \item \textbf{Scenario principale:}
        \begin{enumerate}
            \item Il sistema carica i dati e imposta la visualizzazione del widget contenente le misurazioni relative ai sensori di occupazione delle colonnine di ricarica.
        \end{enumerate}
    \item \textbf{User story associata:} \\
        Come autorità locale, desidero visualizzare un widget per la visualizzazione delle misurazioni trasmesse dai sensori di occupazione delle colonnine di ricarica. Questo mi permetterà di analizzare in modo approfondito i dati relativi a quella tipologia di sensori, aiutandomi a prendere decisioni mirate per migliorare i servizi della città.
\end{itemize}




%---------------------------- UC12 ---------------------------------
\begin{figure}[H]
    \centering
    \includegraphics[width=0.9\textwidth]{../Images/uc12.PNG}
    \caption{UC12 - FILTRO VISUALIZZAZIONE MISURAZIONI IN UN INTERVALLO TEMPORALE}
    \label{fig:UC7}
\end{figure}
\subsubsection{UC12 - FILTRO VISUALIZZAZIONE MISURAZIONI IN UN INTERVALLO TEMPORALE}
\begin{itemize}
    \item \textbf{Attore principale:} Autorità locale;
    \item \textbf{Precondizioni:}
        \begin{itemize}
            \item L'autorità locale si trova nell'interfaccia di visualizzazione di un widget associato ad una specifica tipologia di sensori (UC3); 
        \end{itemize}
    \item \textbf{Postcondizioni:}
        \begin{itemize}
            \item L'autorità locale visualizza le sole misurazioni trasmesse da una specifica tipolgia di sensori nell'intervallo temporale selezionato.
        \end{itemize}
    \item \textbf{Scenario principale:}
        \begin{enumerate}
            \item L'autorità locale seleziona la funzionalità relativa al filtro dei dati per intervallo temporale;
            \item L'autorità locale imposta un intervallo temporale.
            \item Il sistema aggiorna la visualizzazione mostrando solo le misurazioni effettuate durante l'intervallo temporale selezionato.
        \end{enumerate}
    \item \textbf{Estensioni:}
    \begin{enumerate}
        \item VISUALIZZAZIONE ERRORE (UC13)
    \end{enumerate}
    \item \textbf{User story associata:} \\
        \begin{itemize}
          \item Come autorità locale, voglio avere la capacità di definire un intervallo temporale personalizzato per poter filtrare le misurazioni trasmesse da una specifica tipologia di sensori. Ciò mi permetterà di analizzare dettagliatamente le misurazioni raccolte in un periodo di interesse specifico.
        \end{itemize}
\end{itemize}
%---------------------------- UC12 ---------------------------------
\begin{figure}[H]
    \centering
    \includegraphics[width=0.9\textwidth]{../Images/uc13.PNG}
    \caption{UC12 - FILTRO VISUALIZZAZIONE MISURAZIONI IN UN INTERVALLO TEMPORALE}
    \label{fig:UC7}
\end{figure}
\subsubsection{UC13 - AGGREGAZIONE MISURAZIONI PER UNITÀ TEMPORALI}
\begin{itemize}
    \item \textbf{Attore principale:} Autorità locale;
    \item \textbf{Precondizioni:}
            \begin{itemize}
                \item L'autorità locale si trova nell'interfaccia di visualizzazione di un widget associato ad una specifica tipologia di sensori (UC3); 
            \end{itemize}
    \item \textbf{Postcondizioni:}
          \begin{itemize}
              \item L'autorità locale visualizza le misurazioni aggregate sulla base dell'intervallo temporale specificato.
          \end{itemize}
    \item \textbf{Scenario principale:}
          \begin{enumerate}
             \item L'autorità locale sceglie tra le opzioni di aggregazione un intervallo temporale disponibile tra secondo, minuto, ora, giorno, mese o anno;
             \item Il sistema aggiorna la visualizzazione secondo l'intervallo temporale di aggregazione selezionato.
          \end{enumerate}
    \item \textbf{User story associata:} \\
        Come autorità locale, desidero essere in grado di personalizzare l'intervallo temporale di aggregazione delle misurazioni, scegliendo tra le opzioni di secondo, minuto, ora, giorno, mese o anno. Questa funzionalità mi permetterà di personalizzare la visualizzazione dei dati in base alle mie esigenze temporali specifiche, agevolando l'analisi dettagliata dei trend e delle variazioni nel corso del tempo.
\end{itemize}
%---------------------------- UC12 ---------------------------------
\begin{figure}[H]
    \centering
    \includegraphics[width=0.9\textwidth]{../Images/uc14.PNG}
    \caption{UC12 - FILTRO VISUALIZZAZIONE MISURAZIONI IN UN INTERVALLO TEMPORALE}
    \label{fig:UC7}
\end{figure}
\subsubsection{UC14 - VISUALIZZAZIONE FILTRATA SULLE MISURAZIONI}
\begin{itemize}
    \item \textbf{Attore principale:} Autorità locale;
   % \item \textbf{Descrizione:} L’autorità locale seleziona i/il sensore/i della quale vuole visionare lo storico dei dati e filtra la visualizzazione ai soli dati compresi tra due valori.
    \item \textbf{Scenario principale:}
          \begin{enumerate}
              \item L'utente configura due valori specifici 
              \item Il sistema reimposta la visualizzazione esclusivamente delle misurazioni che ricadono all'interno di tale intervallo.
          \end{enumerate}
    \item \textbf{Precondizioni:}
          \begin{itemize}
              \item  L'utente si trova in un interfaccia per la visualizzazione di un widget (UC3).
          \end{itemize}
    \item \textbf{Postcondizioni:}
          \begin{itemize}
              \item  L'utente accede a una rappresentazione dello storico dei dati trasmessi, la cui visualizzazione è stata filtrata esclusivamente per includere le misurazioni con il dato compreso tra i due valori specificati.
          \end{itemize}
    \item \textbf{User story associata:}
          \begin{itemize}
            \item Come autorità locale, desidero avere la possibilità di filtrare la visualizzazione dello storico dei dati in base alle misurazioni del sensore comprese tra due valori specifici. Questo mi consentirà di analizzare in modo più mirato e focalizzato i dati che rientrano in un determinato intervallo, facilitando l'identificazione di pattern o anomalie significative.
          \end{itemize}
\end{itemize}



\newcounter{rowcounter}
\setcounter{rowcounter}{1}



\section{Requisiti}
\subsection{Requisiti funzionali}

\begin{longtable}{|C{1cm}|C{2cm}|>{\raggedright}m{5cm}|C{2cm}|C{1.5cm}|}
    \hline
    \textbf{Codice} & \textbf{Importanza} & \textbf{Descrizione} & \textbf{Fonte}  & \textbf{Casi d'uso} \\

    \hline
    RF\arabic{rowcounter} & Obbligatorio & Il prodotto deve essere ad accesso pubblico, ovvero senza login. & Capitolato & \\

    \hline
    \stepcounter{rowcounter} RF\arabic{rowcounter} & Obbligatorio & Il prodotto non deve avere una gestione di amministrazione. & Capitolato & \\

    \hline
    \stepcounter{rowcounter} RF\arabic{rowcounter} & Obbligatorio & Il sistema deve integrare simulatori di diverso tipo al fine di generare dati di misurazioni che siano coerenti con l'ambito del sensore simulato. & Capitolato & \\

    \hline
    \stepcounter{rowcounter} RF\arabic{rowcounter} & Obbligatorio & Ogni misurazione trasmessa dal simulatore del sensore deve essere composta dall'id del sensore, il timestamp e la misurazione. & Capitolato & \\

    \hline
    \stepcounter{rowcounter} RF\arabic{rowcounter} & Obbligatorio &  Il sistema deve essere in grado di simulare almeno un sensore che rilevi la temperatura espressa in gradi Celsius. & Capitolato & \\

    \hline
    \stepcounter{rowcounter} RF\arabic{rowcounter} & Obbligatorio &  Il sistema deve essere in grado di simulare almeno un sensore che misuri l'umidità, espressa in percentuale di umidità nell'aria. & Capitolato & \\

    \hline
    \stepcounter{rowcounter} RF\arabic{rowcounter} & Obbligatorio &  Il sistema deve essere in grado di simulare almeno un sensore per la rilevazione delle particelle di polveri sottili presenti nell'aria, espresse in microgrammi per metro cubo. & Capitolato & \\

    \hline
    \stepcounter{rowcounter} RF\arabic{rowcounter} & Obbligatorio &  Il sistema deve includere la simulazione di almeno un sensore per individuare guasti elettrici. Questi sensori segnalano interruzioni nella fornitura di energia elettrica tramite un bit binario, con il valore 0 che indica l'assenza di energia elettrica. & Capitolato & \\
    
    \hline
    \stepcounter{rowcounter} RF\arabic{rowcounter} & Obbligatorio &  Il sistema deve essere in grado di simulare almeno un sensore per monitorare lo stato di riempimento dei diversi conferitori nelle isole ecologiche. L'indicazione fornita sarà un bit binario, dove il valore 1 segnalerà che il contenitore è pieno. & Capitolato & \\

    \hline
    \stepcounter{rowcounter} RF\arabic{rowcounter} & Obbligatorio &  Il sistema deve includere la simulazione di almeno un sensore per le colonnine di ricarica. Questi sensori indicheranno tramite un bit binario se la colonnina è occupata (bit 1) o libera (bit 0). & Capitolato & \\

    \hline
    \stepcounter{rowcounter} RF\arabic{rowcounter} & Obbligatorio &  Ogni dato generato dai simulatori dei sensori deve essere strettamente correlato al dato successivo, garantendo così una transizione realistica e plausibile tra le misurazioni. & Verbale interno & \\
    
    \hline
    \stepcounter{rowcounter} RF\arabic{rowcounter} & Obbligatorio & Il sistema deve essere in grado di memorizzare in modo sicuro e efficiente i dati generati dai sensori. Ciò include la registrazione accurata di ogni misurazione, assicurando l'integrità e la coerenza dei dati. & Capitolato & \\

    \hline
    \stepcounter{rowcounter} RF\arabic{rowcounter} & Obbligatorio & La piattaforma deve supportare la visualizzazione di dati provenienti da diversi tipi di sensori. & Capitolato & \\
    
    \hline
    \stepcounter{rowcounter} RF\arabic{rowcounter} & Obbligatorio & L'utente deve poter visualizzare una dashboard con una panoramica completa dello stato della città tramite l'utilizzo di widget adibiti alla rappresentazione delle misurazioni e dei sensori. & Capitolato & UC1 \\

    \hline
    \stepcounter{rowcounter} RF\arabic{rowcounter} & Obbligatorio & L'utente deve avere la possibilità di visualizzare le misurazioni all'interno dei widget in formato grafico. & Capitolato & \\

    \hline
    \stepcounter{rowcounter} RF\arabic{rowcounter} & Obbligatorio & L'utente deve avere la possibilità di visualizzare le misurazioni all'interno dei widget in formato testuale. & Capitolato & \\

    \hline
    \stepcounter{rowcounter} RF\arabic{rowcounter} & Obbligatorio & La visualizzazione delle misurazioni in formato testuale deve presentare le informazioni nel formato: \texttt{TIMESTAMP, Dato}. & Capitolato & UC4 \\

    \hline
    \stepcounter{rowcounter} RF\arabic{rowcounter} & Obbligatorio & La dashboard richiede un aggiornamento quasi istantaneo per garantire che i dati provenienti dai sensori siano riflessi nel minor tempo possibile, entro un massimo di 10 secondi. & Capitolato & UC1 \\

    \hline
    \stepcounter{rowcounter} RF\arabic{rowcounter} & Obbligatorio & La dashboard deve mostrare un widget distinto per ciascun tipo di sensore attivo che trasmette dati al sistema, contenente le misurazioni in formato grafico. & Capitolato & UC1.1 \\

    \hline
    \stepcounter{rowcounter} RF\arabic{rowcounter} & Obbligatorio & Ogni widget che visualizza le misurazioni deve includere, insieme ai dati stessi, informazioni sull'identificativo dei sensori che hanno contribuito a quelle misurazioni. & Capitolato & \\

    \hline
    \stepcounter{rowcounter} RF\arabic{rowcounter} & Obbligatorio & La dashboard deve includere un widget dedicato alle misurazioni dei sensori di temperatura. & Capitolato & UC6 \\
    
    \hline
    \stepcounter{rowcounter} RF\arabic{rowcounter} & Obbligatorio & La dashboard deve includere un widget dedicato alle misurazioni dei sensori di umidità. & Capitolato & UC7 \\

    \hline
    \stepcounter{rowcounter} RF\arabic{rowcounter} & Obbligatorio & La dashboard deve includere un widget dedicato alle misurazioni dei sensori delle polveri sottili. & Capitolato & UC8 \\

    \hline
    \stepcounter{rowcounter} RF\arabic{rowcounter} & Obbligatorio & La dashboard deve includere un widget dedicato alle misurazioni dei sensori dei guasti elettrici. & Capitolato & UC9 \\

    \hline
    \stepcounter{rowcounter} RF\arabic{rowcounter} & Obbligatorio & La dashboard deve includere un widget dedicato alle misurazioni dei sensori delle isole ecologiche. & Capitolato & UC10 \\

    \hline
    \stepcounter{rowcounter} RF\arabic{rowcounter} & Obbligatorio & La dashboard deve includere un widget dedicato alle misurazioni dei sensori delle colonnine di ricarica. & Capitolato & UC11 \\

    \hline
    \stepcounter{rowcounter} RF\arabic{rowcounter} & Obbligatorio & La dashboard della città deve includere una mappa interattiva che mostra la posizione dei diversi sensori nella città. & Capitolato & UC1.2 \\

    \hline
    \stepcounter{rowcounter} RF\arabic{rowcounter} & Obbligatorio & I sensori nella mappa devono essere etichettati in modo da consentirne il riconoscimento della tipologia. & Capitolato & UC1.2 \\

    \hline
    \stepcounter{rowcounter} RF\arabic{rowcounter} & Obbligatorio & La dashboard deve fornire un widget con il punteggio di salute relativo alla città basato sui dati aggregati provenienti dai sensori. & Verbale interno & UC1.3 \\
    
    \hline
    \stepcounter{rowcounter} RF\arabic{rowcounter} & Obbligatorio & L'utente deve avere la possibilità di selezionare una cella, ovvero un'area specifica della città, al fine di visualizzare una dashboard dedicata contenente esclusivamente sensori, misurazioni e punteggio di salute correlati a essa. & Capitolato & UC2 \\

    \hline
    \stepcounter{rowcounter} RF\arabic{rowcounter} & Obbligatorio & L'utente deve poter filtrare la visualizzazione delle misurazioni inserendo uno specifico intervallo temporale. & Capitolato & UC12 \\

    \hline
    \stepcounter{rowcounter} RF\arabic{rowcounter} & Obbligatorio & Il sistema deve verificare la validità dell'intervallo temporale inserito dall'utente. & Capitolato & UC3 \\

    \hline
    \stepcounter{rowcounter} RF\arabic{rowcounter} & Obbligatorio & In caso di intervallo temporale non valido, il sistema deve generare una notifica di errore. & Capitolato & UC30 \\

    \hline
    \stepcounter{rowcounter} RF\arabic{rowcounter} & Obbligatorio & La notifica di errore relativa all'inserimento di un intervallo temporale non valido deve richiedere all'utente di reinserire date valide. & Capitolato & UC30 \\

    \hline
    \stepcounter{rowcounter} RF\arabic{rowcounter} & Obbligatorio & La notifica di errore relativa all'inserimento di un intervallo temporale non valido deve essere chiara e informativa, indicando il motivo specifico dell'invalidità dell'intervallo temporale (data fine precedente a data inizio, arco temporale precedente o antecedente all'inizio della trasmissione dati). & Capitolato & UC30 \\

    \hline
    \stepcounter{rowcounter} RF\arabic{rowcounter} & Obbligatorio & L'utente ha la possibilità di selezionare l'intervallo temporale desiderato (secondo, minuto, ora, giorno, mese, anno) per aggregare le misurazioni in base al relativo periodo di registrazione corrispondente. & Capitolato & UC14 \\

    \hline
    \stepcounter{rowcounter} RF\arabic{rowcounter} & Obbligatorio & Il sistema deve adattare dinamicamente la rappresentazione delle misurazioni secondo l'intervallo temporale di aggregazione selezionato dall'utente. & Capitolato & UC14 \\

    \hline
    \stepcounter{rowcounter} RF\arabic{rowcounter} & Obbligatorio &  L'utente deve avere la possibilità di definire due valori (un minimo e un massimo) per filtrare le misurazioni dei sensori, utilizzando questi limiti come criterio per visualizzare solo i dati compresi in quei range. & Capitolato & UC14 \\

    \hline
    \stepcounter{rowcounter} RF\arabic{rowcounter} & Obbligatorio & L'utente deve avere la possibilità di filtrare le misurazioni selezionando uno o più sensori di una specifica categoria in modo tale da visualizzare esclusivamente le misurazioni corrispondenti ai sensori selezionati. & Capitolato & UC16 \\

    \hline
    \stepcounter{rowcounter} RF\arabic{rowcounter} & Obbligatorio & L'utente deve poter filtrare la visualizzazione delle misurazioni selezionando una o più specifiche celle come criterio di filtro. & Capitolato & UC15 \\
    
    \hline
    \stepcounter{rowcounter} RF\arabic{rowcounter} & Obbligatorio & L'utente deve poter rimuovere i filtri applicati e ripristinare la visualizzazione senza tali filtri. & Capitolato & UC31 \\

    \hline
    \stepcounter{rowcounter} RF\arabic{rowcounter} & Obbligatorio & L'utente deve poter salvare in una lista di misurazioni rilevanti una misurazione trasmessa da un sensore. & Verbale interno & UC19 \\
    
    \hline
    \stepcounter{rowcounter} RF\arabic{rowcounter} & Obbligatorio & Il sistema deve effettuare una verifica prima di salvare la misurazione tra le misurazioni rilevanti, assicurandosi che il dato non sia già presente in tale lista. & Verbale interno &  \\

    \hline
    \stepcounter{rowcounter} RF\arabic{rowcounter} & Obbligatorio & L'utente deve poter visualizzare la lista delle misurazioni rilevanti. & Verbale interno & UC20 \\

    \hline
    \stepcounter{rowcounter} RF\arabic{rowcounter} & Obbligatorio & L'utente deve poter rimuovere una misurazione dalla lista delle misurazioni rilevanti. & Verbale interno & UC21 \\

    \hline
    \stepcounter{rowcounter} RF\arabic{rowcounter} & Obbligatorio & L'utente deve essere in grado di visualizzare le informazioni dei sensori. & Verbale interno & UC18 \\

    \hline
    \stepcounter{rowcounter} RF\arabic{rowcounter} & Obbligatorio & L'utente deve essere in grado di visualizzare l'ID dei sensori. & Verbale interno  & UC18.1 \\

    \hline
    \stepcounter{rowcounter} RF\arabic{rowcounter} & Obbligatorio & L'utente deve essere in grado di visualizzare il tipo dei sensori. & Verbale interno & UC18.2 \\
    \hline
    \stepcounter{rowcounter} RF\arabic{rowcounter} & Obbligatorio & L'utente deve essere in grado di visualizzare la posizione dei sensori in coordinate. & Verbale interno & UC18.3 \\
    
    \hline
    \stepcounter{rowcounter} RF\arabic{rowcounter} & Obbligatorio & L'utente deve essere in grado di visualizzare la cella in cui è installato il sensore. & Verbale interno & UC18.4 \\

    \hline
    \stepcounter{rowcounter} RF\arabic{rowcounter} & Obbligatorio & L'utente deve essere in grado di visualizzare la data di installazione dei sensori. & Verbale interno & UC18.5 \\
    
    \hline
    \stepcounter{rowcounter} RF\arabic{rowcounter} & Obbligatorio & L'utente deve essere in grado di visualizzare l'unità di misura associata al sensore. & Verbale interno & UC18.6 \\
    
    \hline
    
\end{longtable}



\subsection{Requisiti qualitativi}
\setcounter{rowcounter}{0}
\begin{longtable}{|C{1cm}|C{2cm}|>{\raggedright}m{5cm}|C{2cm}|}
    \hline
    \textbf{Codice} & \textbf{Importanza} & \textbf{Descrizione} & \textbf{Fonte} \\
    \hline
    RQ\arabic{rowcounter} & Obbligatorio &  Devono essere rispettate tutte le norme definite in Norme di progetto.  & Decisione interna \\
    \hline
    \stepcounter{rowcounter} RQ\arabic{rowcounter} & Obbligatorio &  Devono essere rispettati i vincoli e le metriche definiti in Piano di Qualifica.  & Decisione interna \\
    \hline
    \stepcounter{rowcounter} RQ\arabic{rowcounter} & Obbligatorio &  Devono essere consegnati i diagrammi UML relativi ai casi d'uso.  & Capitolato \\
    \hline
    \stepcounter{rowcounter} RQ\arabic{rowcounter} & Obbligatorio &  Devono essere consegnate le user stories relative ai casi d'uso.  & Verbale esterno 10/11/2023 \\
    \hline
    \stepcounter{rowcounter} RQ\arabic{rowcounter} & Obbligatorio &  Deve essere fornita la documentazione relativa al corretto funzionamento dei simulatori dei sensori.  & Capitolato \\
    \hline
    \stepcounter{rowcounter} RQ\arabic{rowcounter} & Obbligatorio & Deve essere fornita la documentazione sulle scelte implementative e progettuali effettuate con relative motivazioni.  & Capitolato \\
    \hline
    \stepcounter{rowcounter} RQ\arabic{rowcounter} & Obbligatorio &  Deve essere fornita una lista dei problemi aperti e relative soluzioni. & Capitolato \\
    \hline
    \stepcounter{rowcounter} RQ\arabic{rowcounter} & Obbligatorio &  Devono essere consegnati i file Docker e relativa documentazione per l'avvio e la configurazione del sistema. & Verbale esterno 10/11/2023 \\
    \hline
    \stepcounter{rowcounter} RQ\arabic{rowcounter} & Obbligatorio &  Il sistema deve essere testato nella sua interezza tramite test end-to-end  & Capitolato \\
    \hline
    \stepcounter{rowcounter} RQ\arabic{rowcounter} & Obbligatorio &  Deve essere consegnato il Manuale Utente & Decisione interna \\
    \hline
    \stepcounter{rowcounter} RQ\arabic{rowcounter} & Obbligatorio &  È necessario superare test che attestino l'adeguato funzionamento dei servizi impiegati e delle funzionalità implementate. La copertura di tali test deve raggiungere almeno l'80\% e deve essere documentata attraverso un apposito report. & Capitolato \\
    \hline
    \stepcounter{rowcounter} RQ\arabic{rowcounter} & Obbligatorio &  La dashboard deve essere intuitiva e di facile utilizzo, consentendo alle autorità locali una rapida comprensione delle condizioni della città attraverso l'uso di widget e grafici chiari e comprensibili.  & Decisione interna \\
    \hline
    \stepcounter{rowcounter} RQ\arabic{rowcounter} & Obbligatorio & L’interfaccia deve presentare una struttura che sia utilizzabile da qualsiasi tipologia di utente, in modo tale che riesca a raggiungere i propri obiettivi derivanti dall’uso della piattaforma con facilità e immediatezza. Non è richiesto all’utente di configurare ne la rete di sensori ne il database da utilizzare. Tali funzionalità sono intrinseche nell’architettura del progetto complessivo. & Decisione interna \\
    \hline
   
\end{longtable}

\subsection{Requisiti di vincolo}
\setcounter{rowcounter}{0}
\begin{longtable}{|C{1cm}|C{2cm}|>{\raggedright}m{5cm}|C{2cm}|}
    \hline
    \textbf{Codice}                                & \textbf{Importanza} & \textbf{Descrizione}                                                                                                                                                                               & \textbf{Fonte}  \\
    \hline
    RV\arabic{rowcounter}                          & Obbligatorio        & La piattaforma deve essere progettata per gestire un numero crescente di sensori e dati, garantendo la scalabilità per futuri aggiornamenti e ampliamenti.                                         & Verbale interno \\
    \hline
    \stepcounter{rowcounter} RV\arabic{rowcounter} & Obbligatorio        & La piattaforma deve essere accessibile tramite Web perciò dovrà essere compatibile con le ultime versioni dei principali browser. Affinchè
    sia soddisfatta la visualizzazione real-time, il sistema necessita di una
    connessione Internet stabile.                  & Capitolato                                                                                                                                                                                                                                 \\
    \hline
    \stepcounter{rowcounter} RV\arabic{rowcounter} & Obbligatorio        & L’interfaccia deve presentare una struttura che sia utilizzabile da qualsiasi tipologia di utente, in modo tale che riesca a raggiungere i propri
    obiettivi derivanti dall’uso della piattaforma con facilità e immediatezza. Non è richiesto all’utente di configurare ne la rete di sensori ne il database da utilizzare. Tali funzionalità sono intrinseche nell’architettura
    del progetto complessivo.                      & Verbale interno                                                                                                                                                                                                                            \\
    \hline
    \stepcounter{rowcounter} RV\arabic{rowcounter} & Obbligatorio        & La piattaforma deve garantire un elevato livello di disponibilità per consentire un monitoraggio continuo della città senza interruzioni significative.                                            & Verbale interno \\
    \hline
    \stepcounter{rowcounter} RV\arabic{rowcounter} & Obbligatorio        & I dati trasmessi in tempo reale dai sensori devono essere visualizzati sulla dashboard con la minima latenza possibile per consentire decisioni tempestive.                                        & Verbale interno \\
    \hline
    \stepcounter{rowcounter} RV\arabic{rowcounter} & Obbligatorio        & Tutte le fasi di sviluppo e implementazione devono essere documentate in modo esaustivo per consentire una comprensione chiara e agevole del sistema.                                              & Capitolato      \\
    \hline
    \stepcounter{rowcounter} RV\arabic{rowcounter} & Obbligatorio        & La piattaforma deve essere progettata in modo flessibile per consentire l'aggiunta di nuovi tipi di sensori e la futura espansione delle funzionalità senza compromettere l'integrità del sistema. & Verbale interno \\
    \hline
    \stepcounter{rowcounter} RV\arabic{rowcounter} & Obbligatorio        & Il personale incaricato dell'utilizzo e della gestione della piattaforma deve ricevere una formazione adeguata per massimizzarne l'efficacia e garantirne un utilizzo corretto.                    & Verbale interno \\
    \hline
\end{longtable}





\subsection{Tracciamento}

\subsubsection{Requisito - Fonte}

\begin{center}
    \begin{longtable}{|C{5cm}|C{5cm}|}
        \hline
        \textbf{Requisito} & \textbf{Fonte} \\
        \hline
        RF1 & Capitolato \\
        \hline
        RF2 & Capitolato \\
        \hline
        RF3 & Capitolato \\
        \hline
        RF4 & Capitolato \\
        \hline
        RF5 & Capitolato \\
        \hline
        RF6 & Capitolato \\
        \hline
        RF7 & Capitolato \\
        \hline
        RF8 & Capitolato \\
        \hline
        RF8 & Capitolato \\
        \hline
        RF9 & Capitolato \\
        \hline
        RF10 & Capitolato \\
        \hline
        RF59 & Capitolato \\
        \hline
        RF11 & Decisione interna \\
        \hline
        RF12 & Capitolato \\
        \hline
        RF13 & Capitolato \\
        \hline
        RF14 & Capitolato, UC1 \\
        \hline
        RF15 & Capitolato, UC5 \\
        \hline
        RF16 & Capitolato, UC4 \\
        \hline
        RF17 & Decisione interna, UC4 \\
        \hline
        RF18 & Decisione interna, UC1 \\
        \hline
        RF19 & Decisione interna, UC1.1 \\
        \hline
        RF20 & Decisione interna, UC1.1.1.1 \\
        \hline
        RF21 & Capitolato, UC6 \\
        \hline
        RF62 &  Decisione interna, UC6 \\
        \hline
        RF22 & Capitolato, UC7 \\
        \hline
        RF63 &  Decisione interna, UC7 \\
        \hline
        RF23 & Capitolato, UC8 \\
        \hline
        RF64 &  Decisione interna, UC8 \\
        \hline
        RF24 & Capitolato, UC9 \\
        \hline
        RF65 &  Decisione interna, UC9 \\
        \hline
        RF25 & Capitolato, UC10 \\
        \hline
        RF66 &  Decisione interna, UC10 \\
        \hline
        RF26 & Capitolato, UC11 \\
        \hline
        RF67 &  Decisione interna, UC11 \\
        \hline
        RF60 & Capitolato, UC33 \\
        \hline
        RF68 &  Decisione interna, UC33 \\
        \hline
        RF27 & Capitolato, UC1.2 \\
        \hline
        RF28 & Decisione interna, UC1.2 \\
        \hline
        RF29 & Decisione interna, UC1.3 \\
        \hline
        RF30 & Capitolato, UC2 \\
        \hline
        RF31 & Decisione interna, UC12.1 \\
        \hline
        RF32 & Decisione interna, UC12.1 \\
        \hline
        RF33 & Decisione interna, UC30 \\
        \hline
        RF34 & Decisione interna, UC30 \\
        \hline
        RF35 & Decisione interna, UC30 \\
        \hline
        RF36 & Decisione interna, UC13 \\
        \hline
        RF37 & Decisione interna, UC13 \\
        \hline
        RF38 & Decisione interna, UC12.2 \\
        \hline
        RF39 & Decisione interna, UC12.2 \\
        \hline
        RF40 & Decisione interna, UC32 \\
        \hline
        RF41 & Decisione interna, UC32 \\
        \hline
        RF42 & Decisione interna, UC32 \\
        \hline
        RF43 & Decisione interna, UC12.4 \\
        \hline
        RF44 & Decisione interna, UC12.3 \\
        \hline
        RF45 & Decisione interna, UC12 \\
        \hline
        RF46 & Decisione interna, UC31 \\
        \hline
        RF47 & Decisione interna, UC19 \\
        \hline
        RF48 & Decisione interna, UC19 \\
        \hline
        RF49 & Decisione interna, UC20 \\
        \hline
        RF50 & Decisione interna, UC21 \\
        \hline
        RF51 & Capitolato, UC22 \\
        \hline
        RF52 & Decisione interna, UC18 \\
        \hline
        RF53 & Decisione interna, UC18.1 \\
        \hline
        RF54 & Decisione interna, UC18.2 \\
        \hline
        RF55 & Decisione interna, UC18.3 \\
        \hline
        RF56 & Decisione interna, UC18.4 \\
        \hline
        RF57 & Decisione interna, UC18.5 \\
        \hline
        RF58 & Decisione interna, UC18.6 \\
        \hline
        RF61 & Decisione interna, UC1.2 \\
        \hline
        RQ0 & Decisione interna \\
        \hline
        RQ1 & Decisione interna \\
        \hline
        RQ2 & Capitolato \\
        \hline
        RQ3 & Verbale esterno 10/11/2023 \\
        \hline
        RQ4 & Capitolato \\
        \hline
        RQ5 & Capitolato \\
        \hline
        RQ6 & Capitolato \\
        \hline
        RQ7 & Verbale esterno 10/11/2023 \\
        \hline
        RQ8 & Capitolato \\
        \hline
        RQ9 &  Decisione interna \\
        \hline
        RQ11 & Decisione interna \\
        \hline
        RQ12 & Decisione interna \\
        \hline
        RV0 & Decisione interna \\
        \hline
        RV1 & Capitolato \\
        \hline
        RP0 & Capitolato \\
        \hline
        RP1 & Capitolato \\
        \hline
        RP2 & Decisione interna \\
        \hline
        RP3 & Capitolato \\
        \hline
    \end{longtable}
\end{center}


  \subsubsection{Fonte - Requisito}

\begin{center}
  \begin{longtable}{|C{5cm}|C{5cm}|}
    \hline
    \textbf{Fonte} & \textbf{Requisito} \\
    \hline
    Capitolato & RF1, RF2, RF3, RF4, RF5, RF6, RF7, RF8, RF8, RF9, RF10, RF21, RF22, RF23, RF24, RF25, RF26, RF27, RF51, RQ2, RQ4, RQ5, RQ6, RQ8, RV1, RP0, RP1, RP3  \\
    \hline
    Decisione interna & RF11, RF17, RF49, RF50, RF52, RF53, RF54, RF55, RF56, RF57, RF58, RF12, RF48, RF13, RF29, RF51, RF30, RF44, RF45, RF46, RF47, RF48, RF49, RF50, RF51, RF52, RF53, RF54, RF55, RF56, RF57, RF58, RF61, RF62, RF63, RF64, RF65, RF66, RF67, RF68, RQ0 ,RQ1, RQ9, RQ11, RQ12, RV0, RP2 \\
    \hline
    Verbale esterno 10/11/2023 & RQ3, RQ7 \\
    \hline
    UC1 & RF14, RF27, RF51 \\
    \hline
    UC1.1 & RF19  \\
    \hline
    UC1.2 & RF27, RF28, RF61 \\
    \hline
    UC1.3 &  RF29 \\
    \hline
    UC2 & RF30 \\
    \hline
    UC1.1.1.1 & RF20 \\
    \hline
    UC4 & RF16, RF17 \\
    \hline
    UC5 & RF15 \\
    \hline
    UC6 & RF21, RF62\\
    \hline
    UC7 & RF22, RF63 \\
    \hline
    UC8 & RF23, RF64 \\
    \hline
    UC9 & RF24, RF65 \\
    \hline
    UC10 & RF25, RF66 \\
    \hline
    UC11 & RF26, RF67 \\
    \hline
    UC12.1 & RF31, RF32 \\
    \hline
    UC13 & RF36, RF37 \\
    \hline
    UC12.2 &  RF38, RF39 \\
    \hline
    UC12.3 & RF44 \\
    \hline
    UC12.4 & RF43 \\
    \hline
    UC12 & RF45 \\
    \hline
    UC18 & RF52 \\
    \hline
    UC18.1 & RF53 \\
    \hline
    UC18.2 & RF54 \\
    \hline
    UC18.3 & RF55 \\
    \hline
    UC18.4 & RF56 \\
    \hline
    UC18.5 & RF57 \\
    \hline
    UC18.6 & RF58 \\
    \hline
    UC19 & RF47, RF48 \\
    \hline
    UC19.1 & RF48 \\
    \hline
    UC20 & RF49 \\
    \hline
    UC21 & RF50 \\
    \hline
    UC22 & RF51, RF52 \\
    \hline
    UC30 & RF33, RF34, RF35 \\
    \hline
    UC31 & RF46 \\
    \hline
    UC32 & RF40, RF41, RF42 \\
    \hline
    UC33 & RF60, RF68 \\
    \hline
\end{longtable}
\end{center}



\end{document}