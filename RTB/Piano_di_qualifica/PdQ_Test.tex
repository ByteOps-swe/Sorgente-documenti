\section{Test}
I test nel contesto dello sviluppo software sono strumenti essenziali utilizzati dal team per vari scopi. Servono per verificare se il prodotto sviluppato rispetta i requisiti stabiliti nell'Analisi dei Requisiti, individuare problemi e errori nel software, dimostrarli in parte la qualità del prodotto e migliorare la manutenibilità del codice.
Lo stato in cui i test potranno essere sarà:
\begin{itemize}
    \item \textbf{I:} Implementato;
    \item \textbf{NI:} Non implementato;
\end{itemize}
\subsection{Formato codice dei test}
Per identificare i test all’interno delle tabelle presenti nelle prossime pagine, verrà utilizzato un codice identificativo che è stato definito con la seguente struttura:
\centering{\textbf{T[Tipologia][Applicazione]-[Numero del test]}}\\
In cui:
\begin{itemize}
    \item \textbf{T:} acronimo di test;
    \item \textbf{Tipologia:} serve per segnale a quale tipologia di test questo appartiene:
    \subitem \textbf{U:} Unità\textit{G};
    \subitem \textbf{I:} Integrazione\textit{G};
    \subitem \textbf{S:} Sistema\textit{G};
    \subitem \textbf{A:} Accettazione\textit{G};
    \item \textbf{Argomento:} prime due lettere dell’argomento principale del test, il quale verrà specificato come sotto-sotto-sezione prima della tabella contenente i test che riguardano quello specifico argomento (Ex: Temperatura —> TE). 
\end{itemize}
\subsection{Test unità}
Servono a verificare il corretto funzionamento di una singola parte autonoma del progetto.
\begin{tabular}{|c|p{4cm}|c|}
    \hline
    \textbf{ID-test} & \textbf{Descrizione} & \textbf{Stato} \\
    \hline
    TUTC01 & Esempio & I \\
    % Aggiungi altre righe come necessario, separando le colonne con "&" e terminando ogni riga con "\\".
    \hline
\end{tabular}
\subsection{Test di integrazione}
Servono a verificare che le singole componenti interagiscano correttamente tra loro.
\begin{tabular}{|c|p{4cm}|c|}
    \hline
    \textbf{ID-test} & \textbf{Descrizione} & \textbf{Stato} \\
    \hline
    TUTC01 & Esempio & I  \\
    % Aggiungi altre righe come necessario, separando le colonne con "&" e terminando ogni riga con "\\".
    \hline
\end{tabular}
\subsection{Test di sistema}
Servono a verificare che il comportamento del sistema sia conforme a quanto stabilito con il proponente.
\begin{tabular}{|c|p{4cm}|c|}
    \hline
    \textbf{ID-test} & \textbf{Descrizione} & \textbf{Stato} \\
    \hline
    TUTC01 & Esempio & I \\
    % Aggiungi altre righe come necessario, separando le colonne con "&" e terminando ogni riga con "\\".
    \hline
\end{tabular}
\subsection{Test di accettazione}
\begin{tabular}{|c|p{4cm}|c|}
    \hline
    \textbf{ID-test} & \textbf{Descrizione} & \textbf{Stato} \\
    \hline
    TUTC01 & Esempio & I \\
    % Aggiungi altre righe come necessario, separando le colonne con "&" e terminando ogni riga con "\\".
    \hline
\end{tabular}
