\subsubsection{UC12.2 - FILTRO VISUALIZZAZIONE PER MISURAZIONI}
\begin{itemize}
    \item \textbf{Attore principale:} Autorità locale;
    \item \textbf{Precondizioni:}
        \begin{itemize}
            \item L'autorità locale si trova nell'interfaccia di visualizzazione di un widget associato ad una specifica tipologia di sensori (UC1.1.1);
            \item L'autorità locale ha impostato la vista in formato testuale time series o in formato grafico time series.
        \end{itemize}
    \item \textbf{Postcondizioni:}
        \begin{itemize}
            \item L'utente visualizza le misurazioni filtrate includendo soltanto i dati rilevati che si collocano tra i due valori specificati.
        \end{itemize}
    \item \textbf{Scenario principale:}
        \begin{enumerate}
            \item L'autorità locale seleziona la funzionalità relativa al filtro dei dati per intervallo di rilevamento;
            \item L'utente inserisce un valore di minimo ed un valore di massimo per filtrare le misurazioni;
            \item Il sistema verifica la validità dell'intervallo di rilevamento inserito;
            \item Il sistema aggiorna la visualizzazione mostrando esclusivamente le misurazioni con il dato rilevato che ricade all'interno dell'intervallo specificato.
        \end{enumerate}
    \vspace{0,5cm}
    \item \textbf{Estensioni:}
        \begin{enumerate}
            \item VISUALIZZAZIONE ERRORE INTERVALLO DI RILEVAMENTO NON VALIDO (UC32)
        \end{enumerate}
    \item \textbf{User story associata:} \\
        Come autorità locale, desidero avere la possibilità di visualizzare le misurazioni filtrate includendo soltanto i dati rilevati che si collocano tra un valore di minimo e di massimo specifici. Questo mi consentirà di analizzare in modo più mirato e focalizzato le misurazioni che rientrano in un determinato intervallo di rilevamento, facilitando l'identificazione di pattern o anomalie significative.
\end{itemize}
