\subsubsection{UC1.2 - VISUALIZZAZIONE WIDGET MAPPA INTERATTIVA DEI SENSORI}
\begin{itemize}
    \item \textbf{Attore principale:} Autorità locale.
    \item \textbf{Precondizioni:}
        \begin{itemize}
            \item  Il sistema è operativo e accessibile.
        \end{itemize}
    \item \textbf{Postcondizioni:}
        \begin{itemize}
            \item L’autorità locale ha una visione grafica aggiornata della mappa dei sensori nella città, se presenti, con indicazione chiara della loro posizione e tipologia.
        \end{itemize}
    \item \textbf{Scenario principale:}
        \begin{enumerate}
            \item L'autorità locale accede alla piattaforma per la visualizzazione della dashboard. (UC1)
            \item Il sistema elabora i dati e imposta i sensori nella posizione corretta all'interno della mappa.
        \end{enumerate}
    \item \textbf{User story associata:} \\
        Come autorità locale, desidero essere in grado di visualizzare una mappa interattiva contenente i sensori attivi e operativi all’interno della città. La mappa deve mostrare chiaramente la posizione di ciascun sensore e deve essere etichettata per consentire un riconoscimento immediato della tipologia di ogni sensore. Questa visualizzazione intuitiva e dettagliata mi permetterà di valutare rapidamente la distribuzione dei sensori nella città e di prendere decisioni informate per ottimizzare la copertura e l'efficacia del monitoraggio ambientale.
\end{itemize}