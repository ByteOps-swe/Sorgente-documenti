\subsection{Funzionalità del prodotto}

Il \textit{software}\textsubscript{\textit{G}} di monitoraggio della Smart City è progettato per offrire una serie di funzionalità cruciali per gestire e migliorare le condizioni della città. \\
Le principali funzionalità includono:

\begin{enumerate}
    \item \textbf{Monitoraggio in tempo reale:} Il \textit{sistema}\textsubscript{\textit{G}} raccoglie dati in tempo reale dai sensori simulati, fornendo uno stato sempre aggiornato della città.

    \item \textbf{Memorizzazione dei dati:} I dati trasmessi dai sensori vengono memorizzati in un \textit{database}\textsubscript{\textit{G}} per garantire la disponibilità a lungo termine e consentire analisi storiche.

    \item \textbf{Visualizzazione attraverso Dashboard:} Gli utenti possono accedere ad una \textit{dashboard}\textsubscript{\textit{G}} che offre una visione d’insieme delle condizioni della città in tempo reale. La \textit{dashboard}\textsubscript{\textit{G}} è composta da \textit{widget}\textsubscript{\textit{G}} e grafici che facilitano la comprensione e l'analisi dei dati.
    
    \item \textbf{Visualizzazione mappa dei sensori:} La \textit{dashboard}\textsubscript{\textit{G}} offre una mappa interattiva della città che mostra con precisione la posizione dei sensori, ciascuno dei quali è contraddistinto da un'etichetta che ne indica la tipologia.

    \item \textbf{Visualizzazione punteggio di salute:} Le informazioni ottenute dai simulatori consentono al \textit{sistema}\textsubscript{\textit{G}} di calcolare un indice di benessere, valutato su una scala da zero a cento in base all'ultima rilevazione di ciascun \textit{sensore}\textsubscript{\textit{G}}. Un punteggio più alto corrisponde a condizioni di vita migliori.

    \item \textbf{Supporto alle decisioni:} L'applicativo fornisce alle autorità locali strumenti per prendere decisioni informate e tempestive sulla gestione delle risorse e sull'implementazione di servizi.
    
    \item \textbf{Analisi dettagliata delle misurazioni:} Il \textit{sistema}\textsubscript{\textit{G}} offre strumenti di filtraggio per esaminare e confrontare le misurazioni con precisione. Le misurazioni possono essere filtrate in diverse modalità: selezionando intervalli temporali specifici, concentrando l'analisi su precise aree della mappa o sensori specifici, oppure focalizzandosi su soglie di rilevamento specifiche. Questa flessibilità permette di esaminare i dati in modo mirato, sia nel tempo che nello spazio, fornendo un'analisi dettagliata e rilevante per le esigenze specifiche.
    
    \item \textbf{Sistema di notifica:} Quando un \textit{sensore}\textsubscript{\textit{G}} rileva una misurazione che supera i valori preimpostati come soglia critica, il \textit{sistema}\textsubscript{\textit{G}} attiva immediatamente un meccanismo di notifica. Questo avviso viene inviato istantaneamente alle autorità competenti, consentendo loro di essere prontamente informate sull'evento. L'obiettivo principale di questo \textit{sistema}\textsubscript{\textit{G}} è garantire una risposta tempestiva ed efficace di fronte a situazioni che richiedono un'azione immediata.
\end{enumerate}