\documentclass{article}
\usepackage[utf8]{inputenc}
\usepackage[absolute]{textpos}
\usepackage[default]{raleway}
\usepackage{titlesec, comment, tabularx, makecell, listings, array, setspace, geometry, graphicx, xcolor, xparse, fancyvrb, relsize, fancyhdr, booktabs, hyperref}
\usepackage{colortbl}
%\geometry{a4paper, left=2cm, right=2cm, top=2cm, bottom=2.5cm}
\renewcommand{\headrulewidth}{0pt}

% Definisci uno stile per i comandi git
\definecolor{light-gray}{gray}{0.92}

\lstdefinestyle{code}{
    frame=single,
    framesep=1mm,
    rulecolor=\color{light-gray},
    backgroundcolor=\color{light-gray},
    basicstyle=\ttfamily,
}

% ----------------------------- Definizione tabella ---------------------------

\newcolumntype{C}[1]{>{\centering\arraybackslash}m{#1}}

%\setcellgapes{2ex} % Imposta l'altezza dell'header (2ex)


% ------------------------------Metadati indice --------------------------------
\title{\textbf{\fontsize{28}{6}\selectfont Indice}}
\author{\fontsize{14}{6}\selectfont ByteOps}
\date{Novembre 24, 2023}


% -----------------------------Creazione footer --------------------------------

\pagestyle{fancy}
\fancyhf{}
\renewcommand{\footrulewidth}{0.4pt}
\lfoot{
    \parbox[c]{2cm}{\includegraphics[width=2cm]{../../Images/logo.png}}
    \textcolor[RGB]{120, 120, 120}{$\cdot$ Verbale Esterno}
}
\rfoot{\thepage}

% --------------------------Modifica formato hyperlinks ------------------------

\hypersetup{
    colorlinks=true,
    linkcolor=black,
    filecolor=black,      
    pdftitle={Verbale Esterno 24/11/2023},  %inserisci data verbale
    pdfpagemode=FullScreen,
}

% ------------------------------- Valore sotto-paragrafi indice --------------------------------------

\setcounter{secnumdepth}{4}
\setcounter{tocdepth}{4}

\titleformat{\section}
{\normalfont\huge\bfseries}{\thesection}{0.2cm}{}
\titlespacing*{\paragraph}{0pt}{0.5cm}{0.1cm}

\titleformat{\subsection}
{\normalfont\Large\bfseries}{\thesubsection}{0.2cm}{}
\titlespacing*{\paragraph}{0pt}{0.5cm}{0.1cm}

\titleformat{\subsubsection}
{\normalfont\large\bfseries}{\thesubsubsection}{0.2cm}{}
\titlespacing*{\paragraph}{0pt}{0.5cm}{0.1cm}

\titleformat{\paragraph}
{\normalfont\normalsize\bfseries}{\theparagraph}{0.2cm}{}
\titlespacing*{\paragraph}{0pt}{0.5cm}{0.1cm}

% ------------------------------- Front Page ---------------------------------------

\begin{document}

% --------------------------Aggiunta firma finale ------------------------
\begin{textblock*}{\textwidth}(0.85\textwidth, 1.16\textheight)
    Il responsabile: Francesco Pozza
\end{textblock*}
% ------------------------------------------------------------------------

\pagestyle{fancy}
\begin{center}
\includegraphics[width = 0.7\textwidth]{../../Images/logo.png} \\
\vspace{0.2cm}
\textcolor[RGB]{60, 60, 60}{\textit{ByteOps.swe@gmail.com}} \\
\vspace{1cm}
\fontsize{16}{6}\selectfont Verbale Esterno $\cdot$ Data: 24/11/2023 \\
\vspace{0.5cm}
\end{center}

\section*{Informazioni documento}
\def\arraystretch{1.2}
\begin{tabular}{>{\raggedleft\arraybackslash}p{0.2\textwidth}|>{\raggedright\arraybackslash}p{0.6\textwidth}c}
\hline
\addlinespace
\textbf{Luogo} & Google Meet \vspace{10pt} \\
\textbf{Orario} & 11:30 - 12:30 \vspace{10pt} \\
\textbf{Redattore} & R.Smanio \vspace{10pt} \\
\textbf{Verificatore} & E.Hysa \vspace{10pt} \\
\textbf{Amministratore} & L.Skenderi \vspace{10pt} \\
\textbf{Destinatari} & T. Vardanega \\ & R. Cardin \vspace{10pt} \\
\textbf{Partecipanti} & A. Barutta \\ & E. Hysa \\ & R. Smanio \\ & D. Diotto \\ & F. Pozza \\ & L. Skenderi \\ & N. Preto \\ & A. Dorigo \\ & D. Zorzi \\ & F. Pallaro \vspace{10pt} \\
\end{tabular}
\pagebreak 

% ------------------------- Changelog ----------------------------

\section*{Registro delle modifiche}

\begin{tabular}{|C{2.5cm}|C{2.5cm}|C{2.5cm}|C{2.5cm}|C{2.5cm}|}
    \hline
    \textbf{Versione} & \textbf{Data} & \textbf{Autore} & \textbf{Verificatore} & \textbf{Dettaglio} \\
    \hline \hline
    0.0.1 & 24/11/2023 & R. Smanio & E. Hysa & Redazione documento. \\
    \hline
\end{tabular}
\pagebreak

% ------------------------- Generazione automatica indice ----------------------
\setstretch{1.5}
\maketitle
\thispagestyle{fancy}
\tableofcontents
\setstretch{1.2}
\pagebreak

% ------------------------ INIZIO DOCUMENTO ----------------------
\flushleft

\section{Revisione del periodo precedente}
Durante il periodo trascorso dal precedente \textit{SAL}\textsubscript{\textit{G}}, il gruppo si è concentrato sulla realizzazione degli obiettivi fissati dall'azienda \textit{proponente}\textsubscript{\textit{G}}. 
Le \textit{attività}\textsubscript{\textit{G}} sono state condotte con efficienza, e le metodologie impiegate si sono dimostrate efficaci nel compimento di tali \textit{attività}\textsubscript{\textit{G}}. Di conseguenza tutte le \textit{attività}\textsubscript{\textit{G}} pianificate per il \textit{SAL}\textsubscript{\textit{G}} sono state completate e verificate e non si ritiene necessario applicare accorgimenti relativamente al \textit{way of working}\textsubscript{\textit{G}}. 

\section{Ordine del giorno}
\subsection{Confronto sul lavoro svolto}
È stato presentato il lavoro svolto al \textit{proponente}\textsubscript{\textit{G}}, che dopo aver revisionato e approvato quanto fatto, ha fornito alcuni spunti su possibili migliorie da apportare al codice presentato, mettendo in discussione alcune delle nostre scelte. Nello specifico è stata analizzata la configurazione dei sensori, implementata in Python, linguaggio raccomandato dall’azienda che offre numerose \textit{librerie}\textsubscript{\textit{G}} esterne per il processo di generazione dei dati.  

Si è posta poi particolare enfasi sull'importanza dell'indipendenza reciproca tra sensori, nonché sulla generazione dei dati basata su una distribuzione realistica della temperatura precedentemente registrata.  

L'azienda consiglia di sfruttare le \textit{API}\textsubscript{\textit{G}} ufficiali per connettere Kafka con \textit{Clickhouse}\textsubscript{\textit{G}}, evidenziandone la maggiore stabilità e documentazione più approfondita.  

In merito all'ambiente \textit{Docker}\textsubscript{\textit{G}}, si raccomanda l'utilizzo di immagini più aggiornate e mantenute di quelle scelte per garantire coerenza e affidabilità.  

\subsection{Webinar su Docker}
Viene proposta l'opportunità di partecipare a un webinar focalizzato su \textit{Docker}\textsubscript{\textit{G}}, organizzato dall'azienda \textit{proponente}\textsubscript{\textit{G}}. Questo evento mira a approfondire le competenze legate a questo strumento e si terrà in collaborazione con l'altro gruppo coinvolto nello stesso progetto.  
La sessione è pianificata per mercoledì 29 dicembre alle ore 15:30. Durante questo incontro, sarà possibile porre domande per favorire una comprensione approfondita dell'argomento trattato. 

\subsection{Presentazione dei casi d'uso}
Nonostante siano ancora in fase di sviluppo, sono stati esaminati brevemente alcuni dei casi d'uso definiti finora, ottenendo un'approvazione preliminare.
Tuttavia, poiché permane incertezza nella definizione di alcuni di essi, si è concordato di consultare il professor R. Cardin per ottenere indicazioni su come procedere.  
Una volta completati, verrano nuovamente presentati all'azienda \textit{proponente}\textsubscript{\textit{G}} i casi d'uso per una revisione completa.  

\subsection{Obiettivi del prossimo sprint}
Per il prossimo sprint, l'obiettivo è realizzare una \textit{dashboard}\textsubscript{\textit{G}} preliminare in \textit{Grafana}\textsubscript{\textit{G}} con diversi \textit{widget}\textsubscript{\textit{G}} per visualizzare i dati provenienti dai sensori. Successivamente, si dovranno mettere in comunicazione \textit{Grafana}\textsubscript{\textit{G}} e ClickHouse per poter visualizzare i dati memorizzati nel \textit{database}\textsubscript{\textit{G}} direttamente all'interno della \textit{dashboard}\textsubscript{\textit{G}}.  

\section{Richieste e chiarimenti}
\begin{enumerate}
    \item \textbf{Domanda riguardante una possibile estensione di Kafka nella rete}
    
    Durante l'incontro è stata sollevata una domanda riguardante l'inoltro dei dati generati a un server Kafka, al fine di consentire l'invio di dati simulati da diverse fonti. Tuttavia, la \textit{proponente}\textsubscript{\textit{G}} al momento sconsiglia questa implementazione a causa delle complessità di networking ad essa associate.  
    Inoltre, viene dichiarato che questa funzionalità potrà essere presa in considerazione una volta completato il lavoro complessivo. 

    \item \textbf{Approccio ai PoC da parte dell'azienda proponente}
    
    Sotto l'indicazione del Professor Vardanega, è stato chiesto all’azienda \textit{proponente}\textsubscript{\textit{G}} il loro approccio riguardo ai PoC. L'azienda ha chiarito di effettuare dei PoC principalmente per progetti di ampia portata oppure quando è necessario presentare una versione preliminare del prodotto al cliente. Inoltre, è stato precisato che i PoC possono essere di due tipologie: "usa e getta" o s"riutilizzabili". Nella prassi aziendale, prevale l'uso dei PoC del secondo tipo.  

\end{enumerate}
    
    
    
% ------------------------ Firma Azienda ----------------------
\begin{textblock*}{\textwidth}(0.35\textwidth, 1.08\textheight)
    Padova, 24/11/2023
\end{textblock*}

\begin{textblock*}{\textwidth}(0.80\textwidth, 1.08\textheight)
        Firma referente Sync Lab:
\end{textblock*}
% -------------------------------------------------------------
\end{document}
