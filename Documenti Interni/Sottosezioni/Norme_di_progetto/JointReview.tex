\subsection{Joint review} 

\subsubsection{Introduzione}
Il processo di revisione congiunta costituisce un metodo formale per valutare lo stato e i risultati di un'\textit{attività}\textsubscript{\textit{G}} all'interno di un progetto, coinvolgendo sia il livello gestionale che tecnico. Tale procedura viene attuata durante l'intero periodo contrattuale.

Può essere attivata da due entità qualsiasi, di cui una (la parte recensita) esamina criticamente l'altra parte (la parte recensente).

\vspace{0.2cm}

Nel nostro caso i recensori sono gli \textit{stakeholder}\textsubscript{\textit{G}}: \textit{committente}\textsubscript{\textit{G}}, \textit{proponente}\textsubscript{\textit{G}} mentre noi fornitori siamo i recensiti.

\vspace{0.2cm}

Nelle sezioni successive saranno descritte in dettaglio le seguenti \textit{attività}\textsubscript{\textit{G}} che fanno parte del processo di Joint Review:
\begin{itemize}
    \item Implementazione del processo;
    \item Project management reviews;
    \item Revisioni tecniche;
\end{itemize}

\subsubsection{Implementazione del processo}
Questa \textit{attività}\textsubscript{\textit{G}} comprende i seguenti compiti: 

\paragraph{Revisioni periodiche}
Saranno condotte revisioni periodiche in corrispondenza di \textit{milestone}\textsubscript{\textit{G}} prestabilite come specificato nel documento \textit{Piano di Progetto}.

\paragraph{SAL}
Ogni due settimane si tiene una revisione \textit{SAL}\textsubscript{\textit{G}} (Stato Avanzamento Lavori) tra il \textit{fornitore}\textsubscript{\textit{G}} e il \textit{proponente}\textsubscript{\textit{G}}. Lo scopo è valutare il lavoro svolto nelle due settimane successive alla revisione \textit{SAL}\textsubscript{\textit{G}} precedente, per determinare se le aspettative e le scadenze sono state rispettate e se sono emerse difficoltà. Inoltre, durante ogni \textit{SAL}\textsubscript{\textit{G}}, si pianificano le successive task da svolgere.

\paragraph{Revisioni ad hoc}
È prevista la possibilità di programmare revisioni ad hoc nel caso in cui uno qualsiasi degli \textit{stakeholder}\textsubscript{\textit{G}} le ritenga necessarie.
Durante tali revisioni, le parti coinvolte valutano dettagliatamente lo stato avanzamento dei lavori, discutono eventuali problematiche emerse e definiscono le azioni correttive necessarie.

\paragraph{Risorse per le revisioni}
Tutte le risorse necessarie per condurre le revisioni sono concordate tra le parti. Queste risorse includono personale, strutture, hardware, \textit{software}\textsubscript{\textit{G}} e strumenti.

\paragraph{Elementi da concordare}
Le parti concordano i seguenti elementi in ciascuna revisione: 
\begin{itemize}
    \item Agenda della riunione;
    \item Prodotti \textit{software}\textsubscript{\textit{G}} (risultati di un'\textit{attività}\textsubscript{\textit{G}}) e relativi problemi da esaminare;
    \item Ambito e procedure;
    \item Criteri di ingresso e uscita per la revisione.
\end{itemize}

%\paragraph{Registrazione dei problemi}
%I problemi rilevati durante le revisioni devono essere registrati ed inseriti nel Processo di Risoluzione dei Problemi (6.8) come richiesto.

\paragraph{Documentazione e distribuzione dei risultati}
I risultati della revisione devono essere documentati e distribuiti tramite i \textit{Verbali Esterni}.

La parte recensente riconoscerà alla parte recensita l'adeguatezza (ad esempio approvazione, disapprovazione o approvazione condizionale) dei risultati della revisione. 

\subsubsection{Project management reviews}

\paragraph{Introduzione}
Nello \textit{standard}\textsubscript{\textit{G}} \href{https://www.math.unipd.it/~tullio/IS-1/2009/Approfondimenti/ISO_12207-1995.pdf}{ISO/IEC 12207:1995}, l'\textit{attività}\textsubscript{\textit{G}} di "project management reviews" si riferisce a un processo di revisione e valutazione dei progetti \textit{software}\textsubscript{\textit{G}} in corso o completati.

\vspace{0.2cm}

Questa \textit{attività}\textsubscript{\textit{G}} è volta a garantire che il progetto venga eseguito in modo efficiente e conforme agli obiettivi e ai requisiti definiti.
Le project management reviews sono condotte periodicamente durante l'intero ciclo di vita del progetto e coinvolgono tipicamente il team di gestione del progetto, i responsabili delle diverse aree funzionali coinvolte nel progetto e altri \textit{stakeholder}\textsubscript{\textit{G}} pertinenti.

\paragraph{Stato del Progetto}
Lo scopo è quello di verificare se il progetto è in linea con i piani stabiliti in termini di budget, tempistiche e obiettivi di qualità.
L'esito della revisione è discusso tra gli \textit{stakeholder}\textsubscript{\textit{G}} e prevede quanto segue:
\begin{enumerate}
    \item Garantire che le \textit{attività}\textsubscript{\textit{G}} progrediscano secondo i piani, basandosi su una valutazione dello stato dell'\textit{attività}\textsubscript{\textit{G}} e/o del prodotto \textit{software}\textsubscript{\textit{G}};
    \item Mantenere il controllo globale del progetto attraverso l'allocazione adeguata delle risorse;
    \item Modificare la direzione del progetto o determinare la necessità di pianificazioni alternative;
    \item Valutare e gestire le questioni legate al rischio che potrebbero compromettere il successo del progetto.
\end{enumerate}

\subsubsection{Revisioni Tecniche}
Le revisioni tecniche devono essere condotte per valutare i prodotti o servizi \textit{software}\textsubscript{\textit{G}} presi in considerazione e fornire evidenze che:

\begin{enumerate}
    \item Siano completi;
    \item Siano conformi agli \textit{standard}\textsubscript{\textit{G}} e alle specifiche;
    \item Le modifiche ad essi siano correttamente implementate e influiscano solo sulle aree identificate dalla change request;
    \item Siano conformi agli schemi temporali applicabili;
    \item Siano pronti per la successiva \textit{attività}\textsubscript{\textit{G}};
    \item Lo sviluppo, l'operatività o la manutenzione siano condotti secondo i piani, gli schemi temporali, gli \textit{standard}\textsubscript{\textit{G}} e le linee guida del progetto.
\end{enumerate}

\subsubsection{Strumenti}

\begin{itemize}
    \item \textbf{Zoom:} per revisioni tecniche con i committenti;
    \item \textbf{Google Meet:} per \textit{SAL}\textsubscript{\textit{G}} con la \textit{proponente}\textsubscript{\textit{G}}; 
    \item \textbf{Element:} per richieste di revisioni ad hoc alla \textit{proponente}\textsubscript{\textit{G}}.
\end{itemize}