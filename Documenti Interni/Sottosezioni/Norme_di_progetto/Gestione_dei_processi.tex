\subsection{Gestione dei Processi}

\subsubsection{Introduzione}
La Gestione dei Processi si occupa di stabilire, implementare e migliorare i \textit{processi}\textsubscript{\textit{G}} che guidano la realizzazione del \textit{software}\textsubscript{\textit{G}}, al fine di raggiungere gli obiettivi prefissati e soddisfare le esigenze degli \textit{stakeholder}\textsubscript{\textit{G}}.

\vspace{0.2cm}

Le \textit{attività}\textsubscript{\textit{G}} di gestione di processo sono:
\begin{enumerate}
	\item \textbf{Definizione dei Processi:}
	\begin{itemize}
		\item Identificare e documentare i \textit{processi}\textsubscript{\textit{G}} chiave coinvolti nello sviluppo \textit{software}\textsubscript{\textit{G}};
		\item Stabilire linee guida e procedure per l'esecuzione di ciascun processo.
	\end{itemize}

	\item \textbf{Pianificazione e Monitoraggio:}
	\begin{itemize}
		\item Elaborare piani dettagliati per l'esecuzione dei \textit{processi}\textsubscript{\textit{G}};
		\item Monitorare costantemente l'avanzamento, l'efficacia e la conformità ai requisiti pianificati;
		\item Stimare i tempi, le risorse ed i costi.
	\end{itemize}
	\pagebreak

	\item \textbf{Valutazione e Miglioramento Continuo:}
	\begin{itemize}
		\item Condurre valutazioni periodiche dei \textit{processi}\textsubscript{\textit{G}} per identificare aree di miglioramento;
		\item Implementare azioni correttive e preventive per ottimizzare i \textit{processi}\textsubscript{\textit{G}}.
	\end{itemize}

	\item \textbf{Formazione e Competenze:}
	\begin{itemize}
		\item Assicurare che il personale coinvolto nei \textit{processi}\textsubscript{\textit{G}} sia adeguatamente formato;
		\item Mantenere e sviluppare le competenze necessarie per l'efficace gestione dei \textit{processi}\textsubscript{\textit{G}}.
	\end{itemize}

	\item \textbf{Gestione dei Rischi:}
	\begin{itemize}
		\item Identificare e valutare i rischi associati ai \textit{processi}\textsubscript{\textit{G}};
		\item Definire strategie per mitigare o gestire i rischi identificati.
	\end{itemize}
\end{enumerate}
	
\subsubsection{Pianificazione}

\paragraph{Descrizione}
La pianificazione riveste un ruolo centrale nella gestione dei \textit{processi}\textsubscript{\textit{G}}, poiché mira a creare un piano organizzato e coerente per assicurare un'efficace esecuzione delle \textit{attività}\textsubscript{\textit{G}} durante l'intero ciclo di vita del \textit{software}\textsubscript{\textit{G}}.

\vspace{0.2cm}

Il responsabile del progetto assume il compito di coordinare ogni aspetto della pianificazione delle \textit{attività}\textsubscript{\textit{G}}, che include l'allocazione delle risorse, la definizione dei tempi e la redazione di piani dettagliati. Inoltre, il responsabile si assicura che il piano elaborato sia fattibile e possa essere eseguito correttamente ed efficientemente dai membri del team.

\vspace{0.2cm}

I piani associati all'esecuzione del processo devono comprendere descrizioni dettagliate delle \textit{attività}\textsubscript{\textit{G}} e delle risorse necessarie, specificando le tempistiche, le tecnologie impiegate, le infrastrutture coinvolte e il personale assegnato.

\paragraph{Obiettivi}

L'obiettivo primario della pianificazione è assicurare che ciascun membro del team assuma ogni ruolo almeno una volta durante lo svolgimento del progetto, promuovendo così una distribuzione equa delle responsabilità e un arricchimento delle competenze all'interno del team.

\vspace{0.2cm}

La pianificazione, stilata dal responsabile, è integrata nel documento del \textit{Piano di Progetto}. Questo documento fornisce una descrizione completa delle \textit{attività}\textsubscript{\textit{G}} e dei compiti necessari per raggiungere gli obiettivi prefissati in ogni periodo del progetto.
\paragraph{Assegnazione dei ruoli}

Durante l'intero periodo del progetto, i membri del gruppo assumeranno sei ruoli distinti, ovvero assumeranno le responsabilità e svolgeranno le mansioni tipiche dei professionisti nel campo dello sviluppo \textit{software}\textsubscript{\textit{G}}. \\
Nei successivi paragrafi sono descritti in dettaglio i seguenti ruoli:
\begin{itemize}
	\item Responsabile;
	\item Amministratore;
	\item Analista;
	\item Progettista;
	\item Programmatore;
	\item Verificatore.
\end{itemize}

\paragraph{Responsabile}\label{responsabile} Figura fondamentale che coordina il gruppo, fungendo da punto di riferimento per il \textit{committente}\textsubscript{\textit{G}} e il team, svolgendo il ruolo di mediatore tra le due parti.

In particolare si occupa di:
\begin{itemize}
	\item Gestire le relazioni con l'esterno;
	\item Pianificare le \textit{attività}\textsubscript{\textit{G}}: quali svolgere, data di inizio e fine, assegnazione delle priorità;
	\item Valutare i rischi delle scelte da effettuare;
	\item Controllare i progressi del progetto;
	\item Gestire le risorse umane;
	\item Approvazione della documentazione.
\end{itemize}

\paragraph{Amministratore}\label{amministratore}Questa figura professionale è incaricata del controllo e dell'amministrazione dell'ambiente di lavoro utilizzato dal gruppo ed è anche il punto di riferimento per quanto concerne le norme di progetto. Le sue mansioni principali sono:
\begin{itemize}
		\item Affrontare e risolvere le problematiche associate alla gestione dei \textit{processi}\textsubscript{\textit{G}};
		\item Gestire versionamento della documentazione;
		\item Gestire la configurazione del prodotto;
		\item Redigere ed attuare le norme e le procedure per la gestione della qualità;
		\item Amministrare le infrastrutture e i servizi per i \textit{processi}\textsubscript{\textit{G}} di supporto.
\end{itemize}

\paragraph{Analista}\label{analista}Figura professionale con competenze avanzate riguardo l'\textit{attività}\textsubscript{\textit{G}} di \textit{analisi dei requisiti}\textsubscript{\textit{G}} ed il dominio applicativo del problema. Il suo ruolo è quello di identificare, documentare e comprendere a fondo le esigenze e le specifiche del progetto, traducendole in requisiti chiari e dettagliati. Si occupa di:
\begin{itemize}
		\item Analizzare il contesto di riferimento, definire il problema in esame e stabilire gli obiettivi da raggiungere;
		\item Comprendere il problema e definire la complessità e i requisiti;
		\item Redigere il documento \textit{Analisi dei Requisiti};
		\item Studiare i bisogni espliciti ed impliciti.
\end{itemize}

\paragraph{Progettista}\label{progettista}
Il progettista è la figura di riferimento per quanto riguarda le scelte progettuali partendo dal lavoro dell'analista. Spetta al progettista assumere decisioni di natura tecnica e tecnologica, oltre a supervisionare il processo di sviluppo. Tuttavia, non è responsabile della manutenzione del prodotto. \\
In particolare si occupa di:
\begin{itemize}
		\item Progettare l'\textit{architettura}\textsubscript{\textit{G}} del prodotto secondo specifiche tecniche dettagliate;
		\item Prendere decisioni per sviluppare soluzioni che soddisfino i criteri di affidabilità, efficienza, sostenibilità e conformità ai requisiti;
		\item Redige la \textit{Specifica Architetturale} e la parte pragmatica del \textit{Piano di Qualifica};
\end{itemize}

\paragraph{Programmatore}
\label{par:programmatore}
Il programmatore è la figura professionale incaricata della scrittura del codice \textit{software}\textsubscript{\textit{G}}.

Il suo compito primario è implementare il codice conformemente alle specifiche fornite dall'analista e all'\textit{architettura}\textsubscript{\textit{G}} definita dal progettista.

In particolare, il programmatore:
\begin{itemize}
	\item Scrive codice manutenibile in conformità con le \textit{Specifiche Tecniche};
	\item Codifica le varie componenti dell'\textit{architettura}\textsubscript{\textit{G}} seguendo quanto ideato dai progettisti;
	\item Realizza gli strumenti per verificare e validare il codice;
	\item Redige il \textit{Manuale Utente}.
\end{itemize}

\paragraph{Verificatore}\label{verificatore} La principale responsabilità del verificatore consiste nell'ispezionare il lavoro svolto da altri membri del team per assicurare la qualità e la conformità alle attese prefissate.
Stabilisce se il lavoro è stato svolto correttamente sulla base delle proprie competenze tecniche, esperienza e conoscenza delle norme.

In particolare il verificatore si occupa di:
\begin{itemize}
	\item Verificare che i prodotti siano conformi alle \textit{Norme di Progetto};
	\item Verificare la conformità dei prodotti ai requisiti funzionali e di qualità;
	\item Ricercare ed in caso segnalare eventuali errori;
	\item Redigere la sezione retrospettiva del \textit{Piano di Qualifica}, descrivendo le verifiche e le prove effettuate durante il processo di sviluppo del prodotto.
\end{itemize}

\hypertarget{par:ticketing}{\paragraph{Ticketing}}
GitHub è adottato come \textit{sistema}\textsubscript{\textit{G}} di tracciamento delle \textit{issue}\textsubscript{\textit{G}} (ITS), garantendo così una gestione agevole e trasparente delle \textit{attività}\textsubscript{\textit{G}} da svolgere. \\
L'amministratore ha la facoltà di creare e assegnare specifiche \textit{issue}\textsubscript{\textit{G}} sulla base delle \textit{attività}\textsubscript{\textit{G}} identificate dal responsabile, assicurando chiarezza sulle responsabilità di ciascun membro del team e stabilendo tempi definiti entro cui ciascuna \textit{attività}\textsubscript{\textit{G}} deve essere completata. Inoltre, ogni membro del gruppo può monitorare i progressi compiuti nel periodo corrente, consultando lo stato di avanzamento delle varie \textit{issue}\textsubscript{\textit{G}} attraverso le Dashboard:
\begin{itemize}
		\item \href{https://github.com/orgs/ByteOps-swe/projects/1}{DashBoard}: per una panoramica dettagliata sullo stato delle \textit{issue}\textsubscript{\textit{G}};
		\item \href{https://github.com/orgs/ByteOps-swe/projects/3}{RoadMap}: per una panoramica temporale dettagliata delle \textit{issue}\textsubscript{\textit{G}}.
\end{itemize}

\hypertarget{par:proceduraCreazioneIssue}{\paragraph*{Procedura per la creazione delle issue}}
Le \textit{issue}\textsubscript{\textit{G}} vengono create dall'amministratore e devono essere specificati i seguenti attributi:
\begin{itemize}
		\item \textbf{Titolo}: un titolo coinciso e descrittivo;
		\item \textbf{Descrizione}:
		\begin{itemize}
				\item Descrizione testuale oppure "to-do" tramite bullet points;
				\item Nell'ultima riga viene specificato il verificatore della \textit{issue}\textsubscript{\textit{G}} nel formato: "Verificatore: Mario Rossi".
		\end{itemize} 
		\item \textbf{Assegnatario}: incaricato/i allo svolgimento della \textit{issue}\textsubscript{\textit{G}};
		\item \textbf{Scadenza}: data entro la quale la \textit{issue}\textsubscript{\textit{G}} deve essere completata;
		\item \textbf{Labels}: tag per identificare la categoria della \textit{issue}\textsubscript{\textit{G}}. (ex. Verbale, Documents, Develop, Bug, Feature).\\
		Inoltre per associare ad ogni \textit{issue}\textsubscript{\textit{G}} un Configuration Item vengono utilizzati i seguenti \textit{label}\textsubscript{\textit{G}}:
		\begin{itemize}
				\item \textbf{NdP:} Norme di progetto;
				\item \textbf{PdQ:} Piano di Qualifica;
				\item \textbf{PdP:} Piano di Progetto;
				\item \textbf{AdR:} Analisi dei Requisiti;
				\item \textbf{Poc:} Proof of concept;
				\item \textbf{Gls:} Glossario.
		\end{itemize}
		\item \textbf{Milestone}: \textit{milestone}\textsubscript{\textit{G}} associata alla \textit{issue}\textsubscript{\textit{G}};
		\item \textbf{Projects}: progetti a cui la \textit{issue}\textsubscript{\textit{G}} è associata. \\
		Se sono presenti \textit{dashboard}\textsubscript{\textit{G}} associate ad un progetto, le \textit{issue}\textsubscript{\textit{G}} correlate a tale progetto verranno visualizzate nella relativa/e \textit{dashboard}\textsubscript{\textit{G}} di progetto;
		\item \textbf{Development:} \textit{branch}\textsubscript{\textit{G}} e Pull Request associate alla \textit{issue}\textsubscript{\textit{G}}. \\
		Quando una Pull Request viene accettata, la relativa \textit{issue}\textsubscript{\textit{G}} viene automaticamente chiusa ed eventualmente spostata nella sezione "Done" della Dashboard di progetto.
\end{itemize}

\paragraph*{Ciclo di vita di una issue}
Il ciclo di vita di una \textit{issue}\textsubscript{\textit{G}} è il seguente:
\begin{enumerate}
		\item L'amministratore accede alla \textit{repository}\textsubscript{\textit{G}} GitHub, crea la \textit{issue}\textsubscript{\textit{G}} e la assegna ad un assegnatario, seguendo la convenzione descritta nel \hyperlink{par:proceduraCreazioneIssue}{\textit{paragrafo} precedente};
		\item L'amministratore accede alla \textit{dashboard}\textsubscript{\textit{G}} di progetto e sposta la \textit{issue}\textsubscript{\textit{G}} dalla colonna "No Status" alla colonna "To Do";
		\item L'assegnatario apre un \textit{branch}\textsubscript{\textit{G}} su GitHub seguendo la denominazione suggerita in \hyperlink{subsubsec:sincronizzazione&branching}{"\textit{Sincronizzazione e Branching}"};
		\item Quando la \textit{issue}\textsubscript{\textit{G}} viene presa in carico dall'assegnatario, questo accede alla DashBoard e sposta la \textit{issue}\textsubscript{\textit{G}} dalla colonna "To Do" alla colonna "In Progress";
		\item Una volta che la \textit{issue}\textsubscript{\textit{G}} è considerata terminata, l'assegnatario apre una Pull Request su GitHub seguendo la convenzione descitta in dettaglio nella sezione \hyperlink{par:creazionePR}{"\textit{Procedura per la creazione di Pull Request}"}.
		\item All'interno della Dashboard GitHub la \textit{issue}\textsubscript{\textit{G}} deve essere spostata dalla colonna "In Progress" alla colonna "Da revisionare";
		\item Il verificatore o i verificatori designati seguono le procedure esposte nella \textit{sezione~\ref{subsec:verifica}} per verificare le modifiche apportate al progetto;
		\item Se la verifica ha esito positivo, la \textit{issue}\textsubscript{\textit{G}} viene trasferita dalla colonna "Da revisionare" alla colonna "Done" della Dashboard di GitHub. \\
		Nel caso in cui la \textit{issue}\textsubscript{\textit{G}} sia associata ad una Pull Request, una volta che quest'ultima viene accettata dal verificatore, la \textit{issue}\textsubscript{\textit{G}} viene automaticamente chiusa e spostata nella colonna "Done" della Dashboard di progetto. 
\end{enumerate}

\paragraph{Strumenti}
\begin{itemize}
	\item \textbf{GitHub:} \textit{piattaforma}\textsubscript{\textit{G}} utilizzata per il tracciamento e la gestione delle \textit{issue}\textsubscript{\textit{G}}.
\end{itemize}

\subsubsection{Coordinamento}

\paragraph{Descrizione}
Il coordinamento rappresenta l'\textit{attività}\textsubscript{\textit{G}} che sovraintende la gestione della comunicazione e la pianificazione degli incontri tra le diverse parti coinvolte in un progetto di ingegneria del \textit{software}\textsubscript{\textit{G}}.

Questo comprende sia la gestione della comunicazione interna tra i membri del team del progetto, sia la comunicazione esterna con il \textit{proponente}\textsubscript{\textit{G}} e i committenti. Il coordinamento risulta essere cruciale per assicurare che il progetto proceda in modo efficiente e che tutte le parti coinvolte siano informate e partecipino attivamente in ogni fase del progetto.

\paragraph{Obiettivi}
Il coordinamento in un progetto è fondamentale per gestire la comunicazione e pianificare gli incontri tra gli \textit{stakeholder}\textsubscript{\textit{G}}.

L'obiettivo principale è garantire efficienza, evitando ritardi e confusioni, assicurando che tutte le parti in causa siano informate e coinvolte in ogni fase del progetto.

Inoltre, promuove la collaborazione e la coesione nel team, facilitando lo scambio di idee e la risoluzione dei problemi in modo collaborativo, creando un ambiente lavorativo positivo e produttivo.

\paragraph*{Comunicazione}
Il gruppo \textit{ByteOps} mantiene comunicazioni attive, sia interne che esterne al team, le quali possono essere sincrone o asincrone, a seconda delle necessità.

\paragraph{Comunicazioni sincrone}
\begin{itemize}
	\item \textbf{Comunicazioni sincrone interne} \\
	Per le comunicazioni sincrone interne, il gruppo ByteOps, ha scelto di adottare \textit{Discord}\textsubscript{\textit{G}} in quanto permette di comunicare tramite chiamate vocali, videochiamate, messaggi di testo, media e file in chat private o come membri di un \textit{"server Discord"};

	\item \textbf{Comunicazioni sincrone esterne} \\
	Per le comunicazioni sincrone esterne,in accordo con l'azienda \textit{proponente}\textsubscript{\textit{G}} si è deciso di utilizzare Google Meet.
\end{itemize}

\paragraph{Comunicazioni asincrone}
\begin{itemize}
	\item \textbf{Comunicazioni asincrone interne} \\
	Le comunicazioni asincrone interne avvengono tramite l'applicazione \textit{Telegram}\textsubscript{\textit{G}} all'interno di un gruppo dedicato, il quale consente una comunicazione rapida tra tutti i membri del gruppo. Inoltre, tramite la stessa \textit{piattaforma}\textsubscript{\textit{G}}, è possibile avere conversazioni dirette e private (chat) tra due membri;
	
	\item \textbf{Comunicazioni asincrone esterne} \\
	Per le comunicazioni asincrone esterne sono stati adottati due canali differenti:
	\begin{itemize}
		\item \textbf{E-mail:} utilizzata per comunicare con il \textit{committente}\textsubscript{\textit{G}} e per coordinare gli incontri con l'azienda \textit{proponente}\textsubscript{\textit{G}};
		\item \textbf{Element:} client di messaggistica istantanea gratuito ed open source che supporta conversazioni strutturate e crittografate. La sua flessibilità nell'adattarsi a varie esigenze di comunicazione, inclusa la possibilità di condividere file, immagini e altri documenti, ha reso la \textit{piattaforma}\textsubscript{\textit{G}} un'opzione versatile e completa per soddisfare le esigenze specifiche del nostro contesto lavorativo. Si fa uso della \textit{piattaforma}\textsubscript{\textit{G}} Element come canale di comunicazione con l'azienda \textit{proponente}\textsubscript{\textit{G}} per richiedere eventuali chiarimenti e informazioni di natura urgente, evitando l'attesa del meeting dedicato allo stato di avanzamento lavori (\textit{SAL}\textsubscript{\textit{G}}), il quale potrebbe tenersi a diversi giorni di distanza.
	\end{itemize}
\end{itemize}
\pagebreak

\paragraph*{Riunioni}
In ogni riunione, qualunque ne sia la tipologia, verrà designato un segretario con l'incarico di prendere appunti durante il meeting e successivamente redigere un verbale completo, documentando gli argomenti trattati e i risultati emersi durante le discussioni.

\paragraph{Riunioni interne}
Si è scelto di svolgere i meeting interni a cadenza settimanale, al fine di facilitare una comunicazione costante e coordinare il progresso delle \textit{attività}\textsubscript{\textit{G}}. \\

Generalmente le riunioni sono programmate per ogni venerdi alle ore:
\begin{itemize}
		\item \textbf{10:00} nel caso in cui non sia previsto un \textit{SAL}\textsubscript{\textit{G}} con l'azienda \textit{proponente}\textsubscript{\textit{G}} nello stesso giorno;
		\item \textbf{11:30} nel caso in cui sia previsto un \textit{SAL}\textsubscript{\textit{G}} con l'azienda \textit{proponente}\textsubscript{\textit{G}} nello stesso giorno.
\end{itemize}
Se qualche membro del gruppo non può partecipare alla riunione nella data e nell'orario stabiliti, si procede programmando un nuovo incontro, concordando data e ora tramite un sondaggio sul canale Telegram dedicato.

\vspace{0.2cm}

Ogni membro del gruppo ha la facoltà di richiedere una riunione supplementare se necessario. In questo caso, la data e l'orario saranno concordati sempre attraverso il canale Telegram dedicato, mediante la creazione di un sondaggio.

\vspace{0.2cm}

Le riunioni interne rivestono un ruolo cruciale nel monitorare il progresso delle mansioni assegnate, valutare i risultati conseguiti e affrontare i dubbi e le difficoltà che possono sorgere. Durante i meeting interni, i membri del team condividono gli aggiornamenti sulle proprie \textit{attività}\textsubscript{\textit{G}}, identificano le problematiche riscontrate e discutono di opportunità di miglioramento nei \textit{processi}\textsubscript{\textit{G}} di lavoro. Questo ambiente aperto e collaborativo favorisce l'interazione, l'innovazione e la condivisione di nuove prospettive.

\vspace{0.2cm}

Per agevolare la comunicazione sincrona, il canale utilizzato per i meeting interni è \textit{Discord}\textsubscript{\textit{G}}, ritenuto particolarmente efficace per tali scopi.

\vspace{0,1cm}

Relativamente ai meeting interni, sarà compito del responsabile:
\begin{itemize}
		\item Stabilire preventivamente i principali temi da trattare durante la riunione, considerando la possibilità di aggiungerne di nuovi nel corso della riunione stessa;
		\item Guidare la discussione e raccogliere i pareri dei membri in maniera ordinata;
		\item Nominare un segretario per la riunione;
		\item Pianificare e proporre le nuove \textit{attività}\textsubscript{\textit{G}} da svolgere.
\end{itemize}

\hypertarget{par:verbaliInterni}{\paragraph*{Verbali interni}}
Lo svolgimento di una riunione interna ha come obiettivo la retrospettiva del periodo precedente, la discussione dei punti stilati nell'ordine del giorno e la pianificazione delle nuove \textit{attività}\textsubscript{\textit{G}}.

Alla conclusione di ciascuna riunione, l'amministratore \hyperlink{par:ticketing}{apre un'issue nell'ITS di GitHub} e assegna l'incarico di redigere il verbale interno al segretario della riunione. È compito quindi del segretario redigere il verbale, includendo tutte le informazioni rilevanti emerse durante la riunione. Le indicazioni dettagliate per la compilazione dei verbali interni sono disponibili nella \textit{sezione~\ref{sec:Verbali}}.

\paragraph{Riunioni esterne}
Durante lo svolgimento del progetto, è essenziale organizzare vari incontri con i \textit{Committenti} e/o il \textit{Proponente} al fine di valutare lo stato di avanzamento del prodotto e chiarire eventuali dubbi o questioni.

\vspace{0.2cm}

La responsabilità di convocare tali incontri ricade sul responsabile, il quale è incaricato di pianificarli e agevolarne lo svolgimento in modo efficiente ed efficace. \\
Sarà compito del responsabile anche l'esposizione dei punti di discussione al \textit{proponente}\textsubscript{\textit{G}}/\textit{committente}\textsubscript{\textit{G}}, lasciando la parola ai membri del gruppo interessati quando necessario. Questo approccio assicura una comunicazione efficace tra le varie parti in causa, garantendo una gestione ottimale del tempo e una registrazione accurata delle informazioni rilevanti emerse durante gli incontri.

\vspace{0.2cm}

I membri del gruppo si impegnano a garantire la propria presenza in modo costante alle riunioni, facendo il possibile per riorganizzare eventuali altri impegni al fine di partecipare. \\
Nel caso in cui gli obblighi inderogabili di un membro del gruppo rendessero impossibile la partecipazione, il responsabile assicurerà di informare tempestivamente il \textit{proponente}\textsubscript{\textit{G}} o i committenti, richiedendo la possibilità di rinviare la riunione ad una data successiva.

\paragraph*{Riunioni con l'azienda proponente}
In accordo con l'azienda \textit{proponente}\textsubscript{\textit{G}}, si è stabilito di tenere incontri di \textit{stato avanzamento lavori}\textsubscript{\textit{G}} (\textit{SAL}\textsubscript{\textit{G}}) con cadenza bisettimanale tramite Google Meet. \\
Durante tali incontri, si affrontano diversi aspetti, tra cui:
\begin{itemize}
		\item Discussione delle \textit{attività}\textsubscript{\textit{G}} svolte nel periodo precedente, valutando l'aderenza a quanto concordato e identificando eventuali problematiche riscontrate;
		\item Pianificazione delle \textit{attività}\textsubscript{\textit{G}} per il prossimo periodo, definendo gli obiettivi e le azioni necessarie per il loro raggiungimento;
		\item Chiarezza e risoluzione di eventuali dubbi emersi nel corso delle \textit{attività}\textsubscript{\textit{G}} svolte.
\end{itemize}

\paragraph*{Verbali esterni}
Come nel caso delle riunioni interne, anche per le riunioni esterne verrà redatto un verbale con le stesse modalità descritte nella sezione relativa ai \hyperlink{par:verbaliInterni}{\textit{Verbali Interni}}. \\
Le linee guida per la redazione dei verbali esterni sono reperibili alla \textit{sezione~\ref{sec:Verbali}}.

\paragraph{Strumenti}
\begin{itemize}
	\item \textbf{Discord:} impiegato per la comunicazione sincrona e i meeting interni del team;
	\item \textbf{Element:} utilizzato per contattare l'azienda \textit{proponente}\textsubscript{\textit{G}} per richieste di chiarimenti, informazioni e per esporre dubbi;
	\item \textbf{Google Meet:} utilizzato per i \textit{SAL}\textsubscript{\textit{G}} e per i meeting con l'azienda \textit{proponente}\textsubscript{\textit{G}};
	\item \textbf{Telegram:} utilizzato per la comunicazione asincrona con il team;
	\item \textbf{GMail:} utilizzato come \textit{servizio}\textsubscript{\textit{G}} di posta elettronica.
\end{itemize}
\vspace{0.1cm}

\paragraph{Metriche}
\begin{table}[H]
	\centering
	\begin{tabular}{|C{3cm}|C{4cm}|C{3cm}|}
	\hline
	\textbf{Metrica} & \textbf{Nome} & \textbf{Riferimento} \\
	\hline \hline
	M4BV & Budeget Variance (BV) &  \hyperlink{item:M4BV}{M4BV} \\
	M6SV & Schedule Variance (SV) &  \hyperlink{item:M6SV}{M6SV} \\
	M12VR & Variazione dei Requisiti (VR) &  \hyperlink{item:M12VR}{M12VR} \\
	M21IF & Implementazione delle Funzionalità (IF) & \hyperlink{item:M21IF}{M21IF} \\ 
	\hline
	\end{tabular}
	\caption{Metriche relative alla gestione dei processi}
\end{table}
