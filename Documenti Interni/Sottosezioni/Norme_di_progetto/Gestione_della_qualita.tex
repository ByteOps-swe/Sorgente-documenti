\subsection{Gestione della qualità}
\subsubsection{Introduzione}
Le \textit{attività}\textsubscript{\textit{G}} correlate al processo di gestione della qualità mirano a garantire la qualità del flusso operativo adottato dal \textit{fornitore}\textsubscript{\textit{G}} sui prodotti sviluppati, al fine di soddisfare pienamente le aspettative del cliente e del \textit{proponente}\textsubscript{\textit{G}}, nonché di rispettare i requisiti di qualità specificati.

Questo include la definizione degli obiettivi di qualità, l'identificazione delle metriche e dei criteri di qualità, la pianificazione e l'esecuzione delle \textit{attività}\textsubscript{\textit{G}} di controllo della qualità, e la verifica della qualità attraverso revisioni, ispezioni e \textit{test}\textsubscript{\textit{G}}. L'obiettivo è garantire che il prodotto \textit{software}\textsubscript{\textit{G}} sia conforme alle aspettative degli utenti e ai requisiti del progetto.

\vspace{0.2cm}

La gestione della qualità rappresenta quindi un approccio \textit{olistico}\textsubscript{\textit{G}} che abbraccia l'intero ciclo di vita del \textit{software}\textsubscript{\textit{G}}, dal concepimento all'implementazione e oltre, con l'obiettivo di assicurare che il prodotto finale rispetti gli \textit{standard}\textsubscript{\textit{G}} di qualità predefiniti e consenta un continuo miglioramento dei \textit{processi}\textsubscript{\textit{G}}.
\pagebreak

\subsubsection{Attività}
Le \textit{attività}\textsubscript{\textit{G}} che il team si impegna a svolgere per assicurare qualità dei \textit{processi}\textsubscript{\textit{G}} e di conseguenza dei prodotti sono:
\begin{enumerate}
    \item \textbf{Definizione degli Standard di qualità:}
        la gestione della qualità inizia con la definizione chiara degli \textit{standard}\textsubscript{\textit{G}} di qualità che il \textit{software}\textsubscript{\textit{G}} dovrebbe raggiungere. Questi \textit{standard}\textsubscript{\textit{G}} possono includere requisiti funzionali e non funzionali;

    \item \textbf{Pianificazione della qualità:}
        viene sviluppato un piano di qualità che identifica \textit{attività}\textsubscript{\textit{G}} specifiche, risorse e tempistiche per garantire la qualità del prodotto durante l'intero ciclo di vita del progetto;

    \item \textbf{Assicurazione della qualità:}
        la fase di assicurazione della qualità coinvolge \textit{attività}\textsubscript{\textit{G}} continue di monitoraggio e valutazione per garantire che i \textit{processi}\textsubscript{\textit{G}} siano conformi agli \textit{standard}\textsubscript{\textit{G}} di qualità stabiliti;

    \item \textbf{Controllo della qualità:}
        il controllo della qualità include l'esecuzione di \textit{test}\textsubscript{\textit{G}} e verifiche per garantire che il prodotto soddisfi gli \textit{standard}\textsubscript{\textit{G}} di qualità e risponda alle specifiche richieste;

    \item \textbf{Gestione delle modifiche:}
        un \textit{sistema}\textsubscript{\textit{G}} di gestione delle modifiche è implementato per gestire e controllare le modifiche al prodotto. Questo assicura che ogni modifica venga valutata in termini di impatto sulla qualità complessiva del prodotto;

    \item \textbf{Miglioramento continuo e correzione:}
        la gestione della qualità promuove il miglioramento continuo dei \textit{processi}\textsubscript{\textit{G}}. Attraverso la raccolta di feedback, l'analisi delle prestazioni e l'implementazione di \textit{best practice}\textsubscript{\textit{G}}, il team cerca costantemente di ottimizzare la qualità dei \textit{processi}\textsubscript{\textit{G}} e dei prodotti;

    \item \textbf{Coinvolgimento degli stakeholder:}
        gli \textit{stakeholder}\textsubscript{\textit{G}}, inclusi clienti e utenti finali, sono coinvolti nel processo di gestione della qualità. I loro feedback sono preziosi per garantire che il prodotto risponda alle esigenze e alle aspettative;

    \item \textbf{Formazione e competenza del team:}
        la formazione continua del team è essenziale per mantenere elevate competenze e conoscenze. Un team ben addestrato è in grado di produrre un prodotto di alta qualità.
\end{enumerate}
\pagebreak

\subsubsection{Piano di Qualifica}

Le \textit{attività}\textsubscript{\textit{G}} di pianificazione, assicurazione e controllo della qualità sono ampiamente trattate nel documento \textit{Piano di Qualifica}, il quale riveste un ruolo essenziale nel contesto del processo di "gestione della qualità", offrendo un contributo significativo per assicurare che il prodotto soddisfi gli \textit{standard}\textsubscript{\textit{G}} di qualità richiesti. In tale documento vengono definite in dettaglio le specifiche di qualità relative al prodotto, identificando le azioni di controllo necessarie per garantire il rispetto di tali specifiche e assegnando le responsabilità pertinenti per l'esecuzione di tali azioni.

\subsubsection{PDCA}
Nell'ambito dell'\textit{attività}\textsubscript{\textit{G}} di miglioramento continuo e correzione, si è scelto di adottare il ciclo PDCA, conosciuto anche come ciclo di Deming. \\
Il ciclo PDCA è una metodologia iterativa che consente il controllo e il miglioramento continuo dei \textit{processi}\textsubscript{\textit{G}} e dei prodotti. Affinché si possa ottenere un miglioramento effettivo, è cruciale attuare scrupolosamente le seguenti 4 fasi:

\begin{itemize}
    \item \textbf{Plan} \\
    Consiste nello svolgere le \textit{attività}\textsubscript{\textit{G}} di pianificazione necessarie per stabilire quali \textit{processi}\textsubscript{\textit{G}} debbano essere avviati e in quale sequenza, al fine di raggiungere obiettivi specifici;
    \item \textbf{Do} \\
    Consiste nell'effettiva esecuzione di quanto pianificato, raccogliendo dati e misurando i risultati durante lo svolgimento delle \textit{attività}\textsubscript{\textit{G}};
    \item \textbf{Check} \\
    Consiste nell'analisi e nell'interpretazione dei dati raccolti durante l'esecuzione (Do). Questi dati vengono valutati utilizzando metriche prestabilite e confrontati con gli obiettivi previsti nella fase di Plan.
    \item \textbf{Act} \\
    Si consolida quanto di positivo è stato rilevato nella fase precedente (Check) e si implementano le strategie correttive necessarie per migliorare gli aspetti che non hanno raggiunto i risultati attesi, analizzandone approfonditamente le cause. In questo modo, il ciclo PDCA viene continuamente perfezionato e ogni iterazione successiva aggiunge valore al processo.
\end{itemize}
\pagebreak

\subsubsection{Strumenti}
Gli strumenti impiegati per la gestione della qualità sono rappresentati dalle metriche.

\subsubsection{Struttura e identificazione metriche}
\begin{itemize}
    \item \textbf{Codice:} \\
    identificativo della metrica nel formato:
        \begin{center}
            \textbf{M [numero] [abbreviazione]}
        \end{center}
        dove:
        \begin{itemize}
            \item \texttt{M}: sta per metrica;
            \item \texttt{[numero]}: numero progressivo univoco per ogni metrica;
            \item \texttt{[abbreviazione]}: abbreviazione composta dalle iniziali del nome della metrica.
        \end{itemize}
    \item \textbf{Nome:} specifica il nome della metrica;
    \item \textbf{Descrizione:} breve descrizione della funzionalità della metrica adottata;
    \item \textbf{Scopo:} il motivo per cui è importante tale misura al fine del progetto.
\end{itemize}
    \vspace{0.2cm}
Eventualmente anche:
\begin{itemize}
    \item \textbf{Formula:} come viene calcolata la metrica;
    \item \textbf{Strumento:} lo strumento che viene usato per calcolare la metrica.
\end{itemize}

\subsubsection{Criteri di accettazione}
Per ciascuna metrica nel documento \textit{Piano di Qualifica} vengono definiti in formato tabellare:
\begin{itemize}
    \item \textbf{Valore accettabile:} valore che la metrica deve raggiungere per essere considerata soddisfacente o confrome agli \textit{standard}\textsubscript{\textit{G}} stabiliti;
    \item \textbf{Valore preferibile:} valore ideale che dovrebbe essere assunto dalla metrica.
\end{itemize}

\subsubsection{Metriche}
\begin{table}[H]
    \centering
    \begin{tabular}{|C{3cm}|C{4cm}|C{3cm}|}
    \hline
    \textbf{Metrica} & \textbf{Nome} & \textbf{Riferimento} \\
    \hline \hline
    M1PMS & Percentuale di Metriche Soddisfatte (PMS) &  \hyperlink{item:M1PMS}{M1PMS} \\
    M24DE & Densità degli Errori (DE) &  \hyperlink{item:M24DE}{M24DE} \\
    \hline
    \end{tabular}
    \caption{Metriche relative alla gestione della qualità}
\end{table}

