\documentclass{article}
\usepackage[utf8]{inputenc}
\usepackage[absolute]{textpos}
\usepackage[default]{raleway}
\usepackage{titlesec, comment, tabularx, makecell, listings, array, setspace, geometry, graphicx, xcolor, xparse, fancyvrb, relsize, fancyhdr, booktabs, hyperref}
\usepackage{colortbl}
\usepackage{float}
%\geometry{a4paper, left=2cm, right=2cm, top=2cm, bottom=2.5cm}
\renewcommand{\headrulewidth}{0pt}

% Definisci uno stile per i comandi git
\definecolor{light-gray}{gray}{0.92}

\lstdefinestyle{code}{
    frame=single,
    framesep=1mm,
    rulecolor=\color{light-gray},
    backgroundcolor=\color{light-gray},
    basicstyle=\ttfamily,
}

% ----------------------------- Definizione tabella ---------------------------

\newcolumntype{C}[1]{>{\centering\arraybackslash}m{#1}}

%\setcellgapes{2ex} % Imposta l'altezza dell'header (2ex)


% ------------------------------Metadati indice --------------------------------
\title{\textbf{\fontsize{28}{6}\selectfont Indice}}
\author{\fontsize{14}{6}\selectfont ByteOps}
\date{Dicembre 22, 2023}


% -----------------------------Creazione footer --------------------------------

\pagestyle{fancy}
\fancyhf{}
\renewcommand{\footrulewidth}{0.4pt}
\lfoot{
    \parbox[c]{2cm}{\includegraphics[width=2cm]{../../Images/logo.png}}
    \textcolor[RGB]{120, 120, 120}{$\cdot$ Verbale Interno}
}
\rfoot{\thepage}

% --------------------------Modifica formato hyperlinks ------------------------

\hypersetup{
    colorlinks=true,
    linkcolor=black,
    filecolor=black,      
    pdftitle={Verbale Interno 22/12/2023},  %inserisci data verbale
    pdfpagemode=FullScreen,
}

% ------------------------------- Valore sotto-paragrafi indice --------------------------------------

\setcounter{secnumdepth}{4}
\setcounter{tocdepth}{4}

\titleformat{\section}
{\normalfont\huge\bfseries}{\thesection}{0.2cm}{}
\titlespacing*{\paragraph}{0pt}{0.5cm}{0.1cm}

\titleformat{\subsection}
{\normalfont\Large\bfseries}{\thesubsection}{0.2cm}{}
\titlespacing*{\paragraph}{0pt}{0.5cm}{0.1cm}

\titleformat{\subsubsection}
{\normalfont\large\bfseries}{\thesubsubsection}{0.2cm}{}
\titlespacing*{\paragraph}{0pt}{0.5cm}{0.1cm}

\titleformat{\paragraph}
{\normalfont\normalsize\bfseries}{\theparagraph}{0.2cm}{}
\titlespacing*{\paragraph}{0pt}{0.5cm}{0.1cm}

% ------------------------------- Front Page ---------------------------------------

\begin{document}

% --------------------------Aggiunta firma finale ------------------------
\begin{textblock*}{\textwidth}(0.85\textwidth, 1.16\textheight)
    Il responsabile: N. Preto
\end{textblock*}
% ------------------------------------------------------------------------

\pagestyle{fancy}
\begin{center}
\includegraphics[width = 0.7\textwidth]{../../Images/logo.png} \\
\vspace{0.2cm}
\textcolor[RGB]{60, 60, 60}{\textit{ByteOps.swe@gmail.com}} \\
\vspace{1cm}
\fontsize{16}{6}\selectfont Verbale Interno $\cdot$ Data: 22/12/2023 \\
\vspace{0.5cm}
\end{center}

\section*{Informazioni documento}
\def\arraystretch{1.2}
\begin{tabular}{>{\raggedleft\arraybackslash}p{0.2\textwidth}|>{\raggedright\arraybackslash}p{0.6\textwidth}c}
\hline
\addlinespace
\textbf{Luogo} & Discord \vspace{10pt} \\
\textbf{Orario} & 14:30 - 15:30 \vspace{10pt} \\
\textbf{Redattore} & F. Pozza \vspace{10pt} \\
\textbf{Verificatore} & N. Preto \vspace{10pt} \\
\textbf{Amministratore} & E. Hysa \vspace{10pt} \\
\textbf{Destinatari} & T. Vardanega \\ & R. Cardin \vspace{10pt} \\
\textbf{Partecipanti} & A. Barutta \\ & E. Hysa \\ & R. Smanio \\ & D. Diotto \\ & F. Pozza \\ & L. Skenderi \\ & N. Preto \vspace{10pt} \\
\end{tabular}
\pagebreak 

% ------------------------- Changelog ----------------------------

\section*{Registro delle modifiche}

\begin{tabular}{|C{2.5cm}|C{2.5cm}|C{2.5cm}|C{2.5cm}|C{2.5cm}|}
    \hline
    \textbf{Versione} & \textbf{Data} & \textbf{Autore} & \textbf{Verificatore} & \textbf{Dettaglio} \\
    \hline \hline
    0.0.1 & 28/12/2023 & F. Pozza & N. Preto & Redazione Verbale. \\
    \hline
\end{tabular}
\pagebreak

% ------------------------- Generazione automatica indice ----------------------
\setstretch{1.5}
\maketitle
\thispagestyle{fancy}
\tableofcontents
\setstretch{1.2}
\pagebreak

% ------------------------ INIZIO DOCUMENTO ----------------------
\flushleft

\section{Revisione del periodo precedente}
Non è stata ritenuta opportuna una revisione del periodo precedente poiché questa è stata svolta durante il precedente \textit{SAL}\textsubscript{\textit{G}} datato 21/12/2023, i cui dettagli sono documentati nel relativo verbale esterno. 

\section{Ordine del giorno}
    \subsection{Revisione documenti da consegnare per la revisione RTB}
    Si sono analizzate alcune questioni emerse nel corso della revisione finale dei documenti prodotti. In particolare, si è rilevato che è necessario apportare leggere modifiche al file "Norme di Progetto" al fine standardizzare i titoli delle sezioni che, per ogni \textit{attività}\textsubscript{\textit{G}}, mirano allo stesso obiettivo. Questo garantirà una struttura omogenea all'intero documento.

    \subsection{Power-point dedicato alla presentazione per la revisione RTB}
    È stata condotta un'analisi dettagliata per identificare i contenuti da integrare nelle dispense destinate alla presentazione associata alla revisione \textit{RTB}\textsubscript{\textit{G}}. In linea generale, la presentazione comprenderà elementi chiave riguardanti gli obiettivi del progetto, le motivazioni alla base delle scelte tecnologiche e la presentazione del Proof of Concept (PoC) sviluppato.

    \subsection{Sviluppo pagina web per maggiore accessibilità alle varie repositories}
    È stato deciso di sviluppare una pagina web attraverso le \textit{API}\textsubscript{\textit{G}} di GitHub al fine di semplificare e rendere più immediato l'accesso ai documenti e agli altri componenti del progetto. Questa \textit{piattaforma}\textsubscript{\textit{G}} sarà un punto centrale per accedere facilmente a tutti gli artefatti correlati al progetto, consentendo una navigazione agevole e intuitiva attraverso i vari documenti e risorse disponibili.

    \subsection{Miglioramento POC}
    Nonostante il PoC abbia ricevuto l'approvazione da parte dell’azienda \textit{proponente}\textsubscript{\textit{G}} e sia stato dichiarato pronto per la revisione \textit{RTB}\textsubscript{\textit{G}}, si è discusso sull'implementazione di alcune features rilevate durante l'ultimo \textit{SAL}\textsubscript{\textit{G}}. Queste richieste includono diverse aggiunte alla \textit{dashboard}\textsubscript{\textit{G}}, come l'introduzione di \textit{widget}\textsubscript{\textit{G}} per visualizzare le misurazioni in formato tabellare, la realizzazione di un grafico per il confronto delle misurazioni di un sottoinsieme specifico di sensori di una data tipologia e la possibile \textit{integrazione}\textsubscript{\textit{G}} di un \textit{widget}\textsubscript{\textit{G}} per mostrare la media delle misurazioni relative a una categoria di sensori.

    \subsection{Rotazione ruoli}
    \begin{table}[H]
        \centering
        \begin{tabular}{|c|c|} 
            \hline
            \textbf{Ruolo} & \textbf{Nome Cognome} \\
            \hline \hline
            Responsabile (Re) & N. Preto \\ 
            \hline
            Amministratore (Am) & E. Hysa \\ 
            \hline
            Analista (An) & \makecell{F. Pozza\\D. Diotto} \\
            \hline
            Verificatore (Ve) & \makecell{N. Preto\\L. Skenderi\\E. Hysa} \\
            \hline
            Programmatore (Pr) & \makecell{A. Barutta\\R. Smanio} \\
            \hline
        \end{tabular}
        \caption{Tabella che rappresenta i membri per ogni ruolo}
    \end{table}

\section{Attività da svolgere}
    \begin{center}
        \begin{tabular}{|C{7cm}|C{1,5cm}|C{3cm}|}
            \hline
            \textbf{Titolo} & \textbf{\# Issue} & \textbf{Verificatore} \\
            \hline\hline
            Creare pagina web repositories & 73 & L. Skenderi \\
            Inserire nella \textit{dashboard}\textsubscript{\textit{G}} \textit{widget}\textsubscript{\textit{G}} media misurazioni & 75 & N. Preto \\
            Inserire variabili per confrontare le misurazioni di più sensori in un singolo grafico & 76 & L. Skenderi \\
            Inserire \textit{widget}\textsubscript{\textit{G}} delle misurazioni in formato tabellare  & 77 & N. Preto \\
            \hline
        \end{tabular}
    \end{center}

\end{document}