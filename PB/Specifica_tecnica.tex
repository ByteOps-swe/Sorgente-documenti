\documentclass{article}
\usepackage[utf8]{inputenc}
\usepackage[default]{raleway}
\usepackage[table]{xcolor}
\usepackage[utf8]{inputenc}
\usepackage{booktabs} 
\usepackage{ragged2e}
\usepackage{bera}% optional: just to have a nice mono-spaced font
\usepackage{subcaption}
\usepackage{pgf-pie, multirow, longtable, tikz, titlesec, comment, tabularx, makecell, float, listings, array, setspace, geometry, graphicx, xcolor, xparse, fancyvrb, relsize, fancyhdr, booktabs, hyperref, eurosym, todonotes}
%\geometry{a4paper, left=2cm, right=2cm, top=2cm, bottom=2.5cm}

\renewcommand{\headrulewidth}{0pt}

\colorlet{punct}{red!60!black}
\definecolor{background}{HTML}{EEEEEE}
\definecolor{delim}{RGB}{20,105,176}
\colorlet{numb}{magenta!60!black}

\lstdefinelanguage{json}{
    basicstyle=\normalfont\ttfamily,
    numbers=left,
    numberstyle=\scriptsize,
    stepnumber=1,
    numbersep=8pt,
    showstringspaces=false,
    breaklines=true,
    frame=lines,
    backgroundcolor=\color{background},
    literate=
     *{0}{{{\color{numb}0}}}{1}
      {1}{{{\color{numb}1}}}{1}
      {2}{{{\color{numb}2}}}{1}
      {3}{{{\color{numb}3}}}{1}
      {4}{{{\color{numb}4}}}{1}
      {5}{{{\color{numb}5}}}{1}
      {6}{{{\color{numb}6}}}{1}
      {7}{{{\color{numb}7}}}{1}
      {8}{{{\color{numb}8}}}{1}
      {9}{{{\color{numb}9}}}{1}
      {:}{{{\color{punct}{:}}}}{1}
      {,}{{{\color{punct}{,}}}}{1}
      {\{}{{{\color{delim}{\{}}}}{1}
      {\}}{{{\color{delim}{\}}}}}{1}
      {[}{{{\color{delim}{[}}}}{1}
      {]}{{{\color{delim}{]}}}}{1},
}



\definecolor{linkcolor}{rgb}{1,0,1}

% Definisci uno stile per i comandi git
\definecolor{light-gray}{gray}{0.92}

\lstdefinestyle{code}{
    frame=single,
    framesep=1mm,
    rulecolor=\color{light-gray},
    backgroundcolor=\color{light-gray},
    basicstyle=\ttfamily,
}


% ----------------------------- Definizione tabella ---------------------------

\newcolumntype{C}[1]{>{\centering\arraybackslash}m{#1}}
\newcolumntype{L}[1]{>{\RaggedRight\arraybackslash}m{#1}}

% ------------------------------Metadati indice -------------------------------

\titleformat{\paragraph}[hang]{\normalfont\normalsize\bfseries}{\theparagraph}{1em}{}
\titlespacing*{\paragraph}{0pt}{0.6cm plus 0.1cm minus 0.05cm}{0.15cm plus 0.1cm minus 0.02cm}

% ------------------------------Metadati indice -------------------------------
\title{\textbf{\fontsize{30}{6}\selectfont Indice}}
\author{\fontsize{14}{6}\selectfont ByteOps}
\date{}

% -----------------------------Creazione footer --------------------------------
\pagestyle{fancy} 
\fancyhf{}
\renewcommand{\footrulewidth}{0.4pt} 
\lfoot{ 
    \parbox[c]{2cm}{\includegraphics[width=2cm]{../Images/logo.png}}
    \textcolor[RGB]{120, 120, 120}{$\cdot$ Specifica tecnica}
}
\rfoot{\thepage}
 
% --------------------------Modifica formato hyperlinks ------------------------
\hypersetup{
    colorlinks=true,
    linkcolor=black,
    filecolor=black,      
    pdftitle={Specifica tecnica},
    pdfpagemode=FullScreen,
}

\definecolor{responsabile}{RGB}{0,102,204}
\definecolor{amministratore}{RGB}{0,204,102}
\definecolor{analista}{RGB}{255,165,0}
\definecolor{progettista}{RGB}{255,0,0}
\definecolor{programmatore}{RGB}{128,0,128}
\definecolor{verificatore}{RGB}{255,192,203}

\definecolor{responsabile}{RGB}{0,102,204}
\definecolor{amministratore}{RGB}{0,204,102}
\definecolor{analista}{RGB}{255,165,0}
\definecolor{progettista}{RGB}{255,0,0}
\definecolor{programmatore}{RGB}{128,0,128}
\definecolor{verificatore}{RGB}{255,192,203}

\lstset{
    language=SQL,
    basicstyle=\color{white}\ttfamily,
    backgroundcolor=\color{black},
    keywordstyle=\color{blue},
    commentstyle=\color{gray},
    stringstyle=\color{green},
    numbers=left,
    numberstyle=\small,
    numbersep=5pt,
    tabsize=4,
    showspaces=false,
    showstringspaces=false,
    breaklines=true
}

\begin{document}
\pagestyle{fancy}
\begin{center}
    \includegraphics[width = 0.7\textwidth]{../Images/logo.png} \\
    \vspace{0.2cm}
    \textcolor[RGB]{60, 60, 60}{\textit{ByteOps.swe@gmail.com}} \\
    \vspace{1cm}
    \fontsize{16}{6}\selectfont Specifica tecnica \\
    \vspace{0.5cm}
\end{center}

\section*{Informazioni documento}
\def\arraystretch{1.2}
\begin{tabular}{>{\raggedleft\arraybackslash}p{0.2\textwidth}|>{\raggedright\arraybackslash}p{0.6\textwidth}c}
    \hline
    \addlinespace
    \textbf{Redattori}    & A. Barutta\\ & R. Smanio\\ & L. Skenderi\\ & F. Pozza\\ & D. Diotto \\ & N. Preto \vspace{10pt} \\
    \textbf{Verificatori} & E. Hysa\\ & A. Barutta\\ & D. Diotto\\ & L. Skenderi\\ & R. Smanio\\ & F. Pozza \vspace{10pt} \\
    \textbf{Destinatari}  & ByteOps\\ & T. Vardanega   \\ & R. Cardin \vspace{10pt} \\
\end{tabular}
\pagebreak

% ------------------------- Changelog ----------------------------
\begin{tabular}{|C{1.5cm}|C{2.1cm}|C{2cm}|C{2cm}|C{4.5cm}|}
    \hline 
    \textbf{Versione} & \textbf{Data} & \textbf{Autore} & \textbf{Verificatore} & \textbf{Dettaglio}            \\
    \hline
    \label{Git_Action_Version} 0.0.1 & 01/02/2024 & D. Diotto & L.Skenderi & Impostazione sezioni \\ 
    \hline 
\end{tabular}

\pagebreak

% ------------------------- Generazione automatica indice ----------------------
\setstretch{1.5}
\maketitle
\thispagestyle{fancy}
\tableofcontents
\setcounter{tocdepth}{4}
\listoffigures % Indice delle figure
\setstretch{1.2}
\pagebreak

% ---------------------------- Inizio documento -------------------------------

\flushleft

\section{Introduzione}
\subsection{Scopo del Manuale}
Il presente manuale è concepito per fornire un supporto completo agli utenti Autorità Locale nell'utilizzo efficace del software, consentendo loro di sfruttare appieno tutte le sue funzionalità al fine di garantire un'esperienza ottimale. \\
Poiché l'installazione del software è gestita da personale tecnico specializzato, questo manuale non include istruzioni dettagliate per l'installazione, ma si concentra piuttosto sui passaggi necessari per utilizzare il software una volta installato correttamente.
\subsection{Scopo del Prodotto}
L'obiettivo del progetto è quello di creare un'applicazione web per il monitoraggio di una "Smart City", consentendo un controllo completo sul suo stato di salute. Ciò permetterà di prendere decisioni rapide ed efficaci, oltre ad analizzare gli effetti delle azioni intraprese.\\
La piattaforma è in grado di fornire informazioni chiare e in tempo reale sullo stato della città tramite una dashboard Grafana, che mette a disposizione tutti gli strumenti necessari per l'analisi delle misurazioni provenienti dai sensori. \\
Come detto in precedenza, questa piattaforma è destinata alle autorità cittadine desiderose di ottenere una visione globale della situazione urbana, fornendo informazioni chiare e in tempo reale sullo stato della città.

\subsection{Accesso alla piattaforma}
La piattaforma è presentata come una web-application accessibile esclusivamente agli utenti autorizzati. L'accesso al servizio avviene tramite un browser web, senza richiedere l'installazione di alcun software aggiuntivo sul dispositivo dell'utente. Al fine di garantire la massima sicurezza e riservatezza dei dati, l'accesso è limitato esclusivamente agli utenti in possesso del link e delle credenziali di accesso, le quali vengono fornite dal team amministrativo o da personale autorizzato. Una volta ottenuto il link e le credenziali, gli utenti possono accedere alla web-application da qualsiasi dispositivo connesso a Internet, garantendo un'esperienza di utilizzo flessibile e accessibile ovunque si trovi
\subsection{Glossario}
Per evitare possibili ambiguità che potrebbero sorgere durante la lettura dei documenti,
alcuni termini utilizzati sono stati inseriti nel documento \textit{Glossario v 2.0.0 }. \\
Sarà possibile individuare il riferimento al Glossario per mezzo
di una G a pedice del termine considerato ambiguo.
\subsection{Riferimenti}
\subsubsection{Riferimenti informativi}
    \begin{itemize}
        \item \href {https://www.math.unipd.it/~tullio/IS-1/2023/Progetto/C6.pdf} {Capitolato d'appalto C6 - InnovaCity}
        \item \href{https://www.math.unipd.it/~tullio/IS-1/2023/Dispense/T4.pdf} {Slide del corso di Ingegneria del Software - Gestione di progetto}
        \item \href{https://www.math.unipd.it/~tullio/IS-1/2023/Dispense/T2.pdf} {Slide del corso di Ingegneria del Software - Ciclo di vita del software}
    \end{itemize}

\subsubsection{Riferimenti normativi}
    \begin{itemize}
    \item Norme di progetto
    \item \href {https://www.math.unipd.it/~tullio/IS-1/2023/Dispense/PD2.pdf} {Regolamento del progetto didattico}
    \end{itemize}

\section{Tecnologie}
In questa sezione sono definiti gli strumenti e le tecnologie impiegati per lo sviluppo e l'implementazione del software relativo al progetto InnovaCity.

Si procederà quindi con la descrizione delle tecnologie e dei linguaggi di programmazione utilizzati, delle librerie e dei framework necessari, nonché delle infrastrutture richieste.

L'obiettivo principale è garantire che il software sia sviluppato utilizzando le tecnologie più appropriate in termini di efficienza, sicurezza e affidabilità.

\subsection{Docker}
Per lo sviluppo, il testing e il rilascio del prodotto sono stati utilizzati container Docker in modo tale da garantire ambienti consistenti e riproducibili.

\subsubsection{Ambienti}
\begin{itemize}
  \item \textbf{Ambiente di sviluppo:}
    \begin{itemize}
      \item È l'ambiente dove i software developer scrivono, testano e modificano il codice sorgente;
      \item Può includere strumenti di debug e monitoraggio per facilitare lo sviluppo e la correzione di errori;
      \item Non è accessibile agli utenti finali.
    \end{itemize}
    \item \textbf{Ambiente di test:}
    \begin{itemize}
      \item Simula l'ambiente di produzione;
      \item Viene utilizzato per testare il software in modo completo e realistico prima del rilascio in produzione;
      \item I test vengono eseguiti automaticamente tramite Github workflow oppure manualmente in locale tramite profilo di testing docker e includono test di unità, integrazione, sistema (sicurezza, carico e prestazioni).
    \end{itemize}
    \item \textbf{Ambiente di produzione:}
    \begin{itemize}
      \item È l'ambiente dove il software viene rilasciato per poter essere utilizzato dagli utenti finali;
      \item Deve essere stabile, sicuro e performante per garantire un'esperienza utente ottimale;
      \item Le modifiche al software in produzione sono controllate rigorosamente per minimizzare i rischi di errori o downtime.
    \end{itemize}
\end{itemize}

\subsubsection{Docker images}

Di seguito sono elencate le immagini Docker utilizzate:

\begin{itemize}

  \item \textbf{Simulators - Python} 
    \begin{itemize}
      \item \textbf{Image:} Python: 3.9;
      \item \textbf{Riferimento:} \url{https://hub.docker.com/layers/library/python/3.9/images/sha256-1023bd4c5e0e6b7f4f612b034627826d91ec78ae0439313450ec30c0ad60c908?context=explore}~(consultato: 19/03/2024);
      \item \textbf{Ambiente:}
        \begin{itemize}
          \item Develop;
          \item Production.
        \end{itemize}
    \end{itemize}

  \item \textbf{Broker - Apache Kafka} 
    \begin{itemize}
      \item \textbf{Image:} confluentinc/cp-kafka: 7.6.0;
      \item \textbf{Riferimento:} \url{https://hub.docker.com/layers/confluentinc/cp-kafka/7.6.0/images/sha256-8fc15a671986983b83beecae14e013a91adcd3999f687de8b6b8153fd47e8f67?context=explore}~(consultato: 19/03/2024);
      \item \textbf{Ambiente:}
        \begin{itemize}
          \item Develop;
          \item Production;
          \item Testing;
        \end{itemize}
    \end{itemize}

  \item \textbf{Zookeeper} 
    \begin{itemize}
      \item \textbf{Image:} confluentinc/cp-zookeeper: 7.6.0;
      \item \textbf{Riferimento:} \url{https://hub.docker.com/layers/confluentinc/cp-zookeeper/7.6.0/images/sha256-6a0822643ceb4725db4f24bf2d228eee39bb5ade88f586449d87263cbc81bc97?context=explore}~(consultato: 19/03/2024);
      \item \textbf{Ambiente:}
        \begin{itemize}
          \item Develop;
          \item Production;
          \item Testing;
        \end{itemize}
    \end{itemize}

  \item \textbf{Apache Kafka UI} 
    \begin{itemize}
      \item \textbf{Image:} provectuslabs/kafka-ui: 53a6553765a806eda9905c43bfcfe09da6812035;
      \item \textbf{Riferimento:} \url{https://hub.docker.com/layers/provectuslabs/kafka-ui/53a6553765a806eda9905c43bfcfe09da6812035/images/sha256-633606ca07677d1c4b9405c5df1b6f0087aa75b36528a17eed142d06f65d0881?context=explore}~(consultato: 19/03/2024);
      \item \textbf{Ambiente:}
        \begin{itemize}
          \item Develop.
        \end{itemize}
    \end{itemize}

  \item \textbf{Schema registry} 
    \begin{itemize}
      \item \textbf{Image:} confluentinc/cp-schema-registry: 7.6.0;
      \item \textbf{Riferimento:} \url{https://hub.docker.com/layers/confluentinc/cp-schema-registry/7.6.0/images/sha256-7cea5369377b52823d3101dd22073a235a501256f6f140c66d2111224803af0b?context=explore}~(consultato: 19/03/2024);
      \item \textbf{Ambiente:}
        \begin{itemize}
          \item Develop;
          \item Production;
          \item Testing;
        \end{itemize}
    \end{itemize}

  \item \textbf{Schema registry UI} 
    \begin{itemize}
      \item \textbf{Image:} landoop/schema-registry-ui: latest;
      \item \textbf{Riferimento:} \url{https://hub.docker.com/layers/landoop/schema-registry-ui/latest/images/sha256-c8b7baf7c53224eaa066937410adae388384e3f7c6f26296ba6a98cc5880f866?context=explore}~(consultato: 19/03/2024);
      \item \textbf{Ambiente:}
        \begin{itemize}
          \item Develop.
        \end{itemize}
    \end{itemize}

  \item \textbf{Faust processing - Python} 
    \begin{itemize}
      \item \textbf{Image:} Python: 3.9;
      \item \textbf{Riferimento:} \url{https://hub.docker.com/layers/library/python/3.9/images/sha256-1023bd4c5e0e6b7f4f612b034627826d91ec78ae0439313450ec30c0ad60c908?context=explore}~(consultato: 19/03/2024);
      \item \textbf{Ambiente:}
        \begin{itemize}
          \item Develop;
          \item Production;
          \item Testing;
        \end{itemize}
    \end{itemize}

  \item \textbf{ClickHouse} 
    \begin{itemize}
      \item \textbf{Image:} clickhouse/clickhouse-server: 24.2.1.2248;
      \item \textbf{Riferimento:} \url{https://hub.docker.com/layers/clickhouse/clickhouse-server/24.2.1.2248/images/sha256-a5921f08bc3ab230e20db5970698b300279b29a353620e62729325fa8d1dc601?context=explore}~(consultato: 19/03/2024);
      \item \textbf{Ambiente:}
        \begin{itemize}
          \item Develop;
          \item Production;
          \item Testing;
        \end{itemize}
    \end{itemize}

  \item \textbf{Grafana} 
    \begin{itemize}
      \item \textbf{Image:} grafana/grafana-oss: 10.4.0;
      \item \textbf{Riferimento:} \url{https://hub.docker.com/layers/grafana/grafana-oss/10.4.0/images/sha256-c7ae30e06ee76656f4faf37df1f0d0dfb6941a706b66800a7b289a304d31d771?context=explore}~(consultato: 19/03/2024);
      \item \textbf{Ambiente:}
        \begin{itemize}
          \item Develop;
          \item Production.
        \end{itemize}
    \end{itemize}
\end{itemize}

\subsection{Linguaggi e formato dati}
\subsection{Python}

\subsubsection{Librerie o framework}
\paragraph{Confluent kafka}

\subsubsection{Utilizzo nel progetto}

\subsubsection{SQL (Structured Query Language)}
Linguaggio standard per la gestione e la manipolazione dei
database che lo supportano

\paragraph{Utilizzo nel progetto}
Gestione e interrogazione database Clickhouse.

\subsubsection{JSON (JavaScript Object Notation)}
JSON è un formato di scrittura leggibile dalle persone e facilmente interpretabile dai computer. È utilizzato principalmente per lo scambio di dati strutturati attraverso le reti, come Internet. \todo{non mi piace questa parte}

Il formato JSON si basa su due strutture di dati principali:

\begin{itemize}
  \item \textbf{Oggetti}: Rappresentati da coppie chiave-valore racchiuse tra parentesi graffe \{ \}, dove la chiave è una stringa e il valore può essere un altro oggetto, un array, una stringa, un numero, un booleano o \texttt{null}.
  \item \textbf{Array}: Una raccolta ordinata di valori, racchiusi tra parentesi quadre [ ], in cui ogni elemento può essere un oggetto, un array, una stringa, un numero, un booleano o \texttt{null}.
\end{itemize}

JSON è ampiamente impiegato in diversi contesti, tra cui lo sviluppo web, le API di servizi web e lo scambio di dati tra applicazioni, grazie alla sua sintassi semplice e chiara per la rappresentazione dei dati. La sua struttura basata su testo e la facilità di lettura lo rendono ideale per facilitare l'interazione tra sistemi eterogenei.

\paragraph{Utilizzo nel progetto}
\begin{itemize}
  \item Formato dei messaggi trasmessi dai simulatori dei sensori al broker Kafka;
  \item Configurazione dashboard Grafana.
\end{itemize}

\subsubsection{YAML (YAML Ain't Markup Language)}
Formato di serializzazione leggibile dall'uomo utilizzato per rappresentare dati strutturati in modo chiaro e semplice.

\paragraph{Utilizzo nel progetto}
\begin{itemize}
    \item Configurazione Docker Compose;
    \item Configurazione pipeline Github workflow per Countinuous Integration;
    \item Configurazione provisioning Grafana e politiche di notifica allerte.
\end{itemize}

\subsection{Database e servizi}
\subsubsection{Apache Kafka}
Apache Kafka è una piattaforma open-source di streaming distribuito sviluppata dall'Apache Software Foundation. Progettata per gestire flussi di dati in tempo reale in modo scalabile e affidabile, è ampiamente utilizzata nel data streaming e nell'integrazione dei dati nelle moderne applicazioni.

\paragraph{Versione}
La versione utilizzata è: 3.7.0
\paragraph{Documentazione}
\href{https://kafka.apache.org/20/documentation.html}{https://kafka.apache.org/20/documentation.html}

\paragraph{Funzionalità e vantaggi di Apache Kafka}
Le principali funzionalità e vantaggi di Apache Kafka includono:

\begin{itemize}
  \item \textbf{Pub-Sub Messaging:} Kafka utilizza un modello di messaggistica publish-subscribe, dove i produttori di dati inviano messaggi ad uno o più topic e i consumatori possono sottoscriversi a tali topic per ricevere i messaggi;
  
  \item \textbf{Disaccoppiamento Produttore - Consumatore:} questo principio si realizza grazie al fatto che i Produttori e i Consumatori non necessitano di essere consapevoli l'uno dell'altro o di interagire direttamente. Invece, essi comunicano attraverso il broker Kafka, che svolge il ruolo di intermediario per la trasmissione dei messaggi. Ciò consente una maggiore scalabilità e flessibilità nell'architettura del sistema, facilitando la gestione e il mantenimento delle applicazioni;
  
  \item \textbf{Architettura Distribuita:} Kafka è progettato per essere distribuito su un cluster di nodi, consentendo una scalabilità orizzontale per gestire grandi volumi di dati e carichi di lavoro. Questo approccio distribuito offre resilienza e alta disponibilità, garantendo che il sistema possa crescere in modo flessibile con l'aumentare delle richieste;
  
  \item \textbf{Persistenza e Affidabilità:} Kafka offre la possibilità di definire politiche specifiche per la conservazione dei dati, garantendo la durabilità dei messaggi. Questo non solo assicura la disponibilità dei dati anche in caso di eventuali interruzioni del servizio, ma consente anche ai consumatori di recuperare i messaggi dopo tali anomalie, garantendo un alto livello di affidabilità nel sistema.
  
  \item \textbf{Alta Disponibilità:} Kafka assicura un'elevata disponibilità e tolleranza ai guasti grazie alla sua architettura distribuita e al meccanismo di replica dei dati. Anche in caso di malfunzionamenti dei nodi o delle componenti, i cluster di Kafka mantengono la loro operatività, garantendo la continuità del servizio.
  
  \item \textbf{Elaborazione degli Stream:} Kafka supporta anche l'elaborazione degli stream di dati in tempo reale tramite API come Kafka Streams e Kafka Connect, consentendo agli sviluppatori di scrivere applicazioni per l'analisi e l'elaborazione dei dati in tempo reale.
\end{itemize}

\paragraph{Casi d'uso di Apache Kafka}

Apache Kafka è utilizzato in una vasta gamma di casi d'uso, tra cui:

\begin{itemize}
  \item \textbf{Data Integration:} Kafka viene utilizzato per integrare dati provenienti da diverse fonti e sistemi, consentendo lo scambio di dati in tempo reale tra applicazioni e sistemi eterogenei.
  
  \item \textbf{Streaming di Eventi:} Molte applicazioni moderne, come le applicazioni IoT (Internet of Things) e le applicazioni di monitoraggio in tempo reale, utilizzano Kafka per lo streaming di eventi in tempo reale e l'analisi dei dati.
  
  \item \textbf{Analisi dei Log:} Kafka è spesso utilizzato per l'analisi dei log di sistema e applicativi in tempo reale, consentendo il monitoraggio delle prestazioni, la rilevazione degli errori e l'analisi dei pattern di utilizzo.
  
  \item \textbf{Elaborazione di Big Data:} Kafka è integrato con tecnologie di big data come Apache Hadoop e Apache Spark, consentendo l'elaborazione di grandi volumi di dati in tempo reale.
  
  \item \textbf{Messaggistica Real-time:} Kafka è ampiamente utilizzato per la messaggistica real-time in applicazioni di social media, e-commerce e finanziarie, dove la velocità e l'affidabilità della messaggistica sono cruciali.
\end{itemize}

\paragraph{Utilizzo nel progetto}
\textit{Kafka} funge da intermediario dei messaggi, ricevendo i dati dai produttori e rendendoli disponibili ai consumatori. Nel contesto del progetto, i dati provenienti dalle simulazioni di sensori vengono inviati a \textit{Kafka} come messaggi in formato \textit{JSON}.

\paragraph*{Consumatori di dati:}
\begin{itemize}
  \item \textbf{\textit{ClickHouse:}} \textit{Kafka} invia \todo{è Kafka che li invia o i consumatori che se li prendono da Kafka?} i dati ai consumatori, inclusi i database come \textit{ClickHouse}, dove i dati vengono salvati per l'analisi e l'archiviazione a lungo termine.
  \item \textbf{\textit{Faust:}} per soddisfare il requisito opzionale del calcolo del punteggio di salute, \textit{Kafka} rende disponibili i dati in tempo reale a un'applicazione di Faust\todo{è corretto applicazione di Faust?}. Quest'ultima elabora i dati utilizzando una funzione di aggregazione per calcolare il punteggio e quindi mette a disposizione il risultato in una coda dedicata di Kafka per i servizi interessati.
\end{itemize}

In breve, \textit{Kafka} funge da ponte tra i produttori di dati (simulazioni di sensori) e i consumatori di dati (\textit{ClickHouse} o altri servizi futuri). Gestisce il flusso dei dati in tempo reale e garantisce che i dati siano disponibili per l'elaborazione e la visualizzazione in modo efficiente e scalabile.
\subsubsection{Schema Registry}
Schema Registry è un componente importante nell'ecosistema di Apache Kafka, progettato per la gestione e la convalida degli schemi dei dati utilizzati all'interno di un sistema di messaggistica distribuita.
\paragraph{Versione}
Versione utilizzata: 7.6.0
\paragraph{Documentazione}
\url{https://docs.confluent.io/platform/current/schema-registry/index.html}

\paragraph{Funzionalità e Vantaggi di Schema Registry}
Le funzionalità principali di Schema Registry includono:
\begin{itemize}
    \item \textbf{Gestione centralizzata degli schemi}: Fornisce un repository centralizzato per la gestione degli schemi dei dati.
    Contribuisce alla governance dei dati garantendo la qualità, la conformità agli standard e la tracciabilità dei dati;
    \item \textbf{Convalida degli schemi}: Assicura la validità e la compatibilità degli schemi dei dati;
    \item \textbf{Serializzazione e deserializzazione}: Supporta la serializzazione e la deserializzazione dei dati basati sugli schemi su reti distribuite.
\end{itemize}

\paragraph{Utilizzo nel progetto}
Nell'ambito del progetto didattico schema registry permette di validare i messaggi nell'ambito del topic kakfa di appartenenza definendo un contratto che i produttori, ovvero i sensori, dovranno rispettare nell'invio delle misurazioni.
\input{Sottosezioni/Specifica_tecnica/Zookeper.tex}
\subsection{ClickHouse} \label{sec:clickHouse}
ClickHouse è un sistema di gestione di database (DBMS) di tipo column-oriented, progettato principalmente per l'analisi di grandi volumi di dati in tempo reale. È un progetto open-source sviluppato da Yandex, un motore di ricerca russo, ed è stato creato per rispondere alle esigenze di elaborazione analitica ad alte prestazioni.
\subsubsection{Versione}
La versione utilizzata è: 24.1.5.6
\subsubsection{Link download}
\href{https://clickhouse.com/}{https://clickhouse.com/}

\subsubsection*{Funzionalità e Vantaggi di ClickHouse}
\begin{itemize}
    \item \textbf{ Modello di dati column-oriented:} a differenza dei tradizionali DBMS che memorizzano i dati in modo row-oriented, dove le righe complete sono memorizzate in sequenza, ClickHouse memorizza i dati in modo column-oriented. Questo significa che i dati di ogni colonna sono memorizzati insieme, permettendo una maggiore compressione e velocità di query per le analisi che coinvolgono molte colonne;
    \item \textbf{Architettura Distribuita e scalabilità:} ClickHouse è progettato per funzionare in un ambiente distribuito, consentendo la scalabilità orizzontale per gestire grandi carichi di lavoro;
    \item \textbf{Compressione dei Dati:} utilizza algoritmi efficienti per ridurre lo spazio di archiviazione richiesto per i dati, riducendo i costi di archiviazione;
    \item \textbf{Alte Prestazioni:} ottimizzato per eseguire query analitiche su grandi volumi di dati in tempo reale, garantendo tempi di risposta bassi anche con carichi di lavoro elevati.
    \item \textbf{Supporto per SQL:} supporta un sottoinsieme del linguaggio SQL, consentendo agli sviluppatori di scrivere query complesse per l'analisi dei dati;
    \item \textbf{Integrazione con Strumenti di Business Intelligence (BI):} può essere integrato con strumenti di BI popolari come Tableau, Power BI, Qlik, Grafana per la visualizzazione e l'analisi dei dati.
\end{itemize}


\subsubsection*{Casi d'Uso di ClickHouse}
ClickHouse è adatto per una vasta gamma di casi d'uso, tra cui:
\begin{itemize}
    \item \textbf{Analisi dei Log:} clickHouse può essere utilizzato per analizzare i log di grandi dimensioni generati da server, applicazioni web e dispositivi IoT;
    \item \textbf{Analisi dei Dati in Tempo Reale:} ClickHouse è ideale per l'analisi dei dati in tempo reale, consentendo agli utenti di eseguire query complesse su flussi di dati in continua evoluzione;
    \item \textbf{Reporting e Dashboard:} ClickHouse può essere utilizzato per generare report e dashboard interattivi per monitorare le prestazioni del business e identificare tendenze.
\end{itemize}



\subsection{Grafana}
Grafana è una piattaforma open-source per la visualizzazione e l'analisi dei dati, utilizzata per creare dashboard interattive e grafici da fonti di dati eterogenee. 
\subsubsection{Versione}
La versione utilizzata è: x.x.x
\subsubsection{Link download}
\href{https://clickhouse.com/}{https://clickhouse.com/}

\subsubsection{Funzionalità e Vantaggi di Grafana}
\begin{itemize}
    \item \textbf{Dashboard interattive}: Creazione di dashboard personalizzate e interattive per visualizzare dati provenienti da diverse fonti in un'unica interfaccia.
    
    \item \textbf{Connessione a sorgenti di dati eterogenee}: Supporto per una vasta gamma di sorgenti di dati, inclusi database, servizi cloud, sistemi di monitoraggio, API e altro ancora.
    
    \item \textbf{Ampia varietà di visualizzazioni}: Selezione di pannelli e visualizzazioni, tra cui grafici a linea, a barre, a torta, termometri, mappe geografiche e altro ancora, per adattarsi alle esigenze specifiche di visualizzazione dei dati.
    
    \item \textbf{Query e aggregazioni flessibili}: Esecuzione di query flessibili e aggregazione dei dati in modi personalizzati per ottenere insight approfonditi dai dati.
    
    \item \textbf{Notifiche e allarmi}: Impostazione di avvisi in base a criteri predefiniti, come soglie di performance, e ricezione di notifiche tramite diversi canali, tra cui email, Slack e molti altri.
    
    \item \textbf{Gestione degli accessi e dei permessi}: Controllo degli accessi e dei permessi degli utenti in modo granulare, gestendo chi può visualizzare, modificare o creare dashboard e pannelli.
    
    \item \textbf{Integrazione con altre applicazioni e strumenti}: Integrazione con una vasta gamma di applicazioni e strumenti, tra cui sistemi di log management, strumenti di monitoraggio delle prestazioni, sistemi di allerta e altro ancora.
    
   \end{itemize}
\subsubsection{Casi d'Uso di Grafana}
\begin{itemize}
    \item \textbf{Monitoraggio delle prestazioni}: Monitoraggio in tempo reale delle metriche di sistema come CPU, memoria e rete per identificare e risolvere rapidamente problemi di prestazioni.
    
    \item \textbf{Analisi dei log}: Analisi e visualizzazione dei log delle applicazioni e dell'infrastruttura per individuare pattern e risolvere problemi operativi.
    
    \item \textbf{Monitoraggio dell'infrastruttura}: Monitoraggio dello stato e delle prestazioni di server, servizi cloud, database e altri componenti IT per garantire un funzionamento ottimale dell'infrastruttura.
    
    \item \textbf{DevOps e CI/CD}: Monitoraggio dei processi di sviluppo, test e distribuzione del software per migliorare la collaborazione e l'efficienza del team.
    
    \item \textbf{Monitoraggio di dispositivi IoT}: Monitoraggio dei dispositivi IoT per raccogliere e visualizzare dati di sensori e dispositivi connessi, consentendo una gestione efficiente degli ambienti IoT.
\end{itemize}



\section{Architettura}

\subsection{Configurazione Database}
Si è optato per l'utilizzo di ClickHouse per il salvataggio dei dati, le motivazioni sono descritte nella sezione \ref{sec:clickHouse}. In particolare, per ogni sensore dei quali si desidera memorizzare i dati, viene creata una tabella che acquisisce i dati dal relativo topic Kafka.
Le tipologie di sensori cui misurazioni si vogliono trattare nel progetto sono:
\begin{itemize}
    \item Sensori di temperatura;
    \item Sensori di umidità;
    \item Sensori di rilevamento polveri sottili; 
    \item Sensori stato riempimento isole ecologiche;
    \item Sensori di stato occupazione colonnine di ricarica;
    \item Sensori di guasti elettrici;
    \item Sensori del livello dell'acqua.
\end{itemize}

La configurazione del database ClickHouse è stata cruciale nella progettazione, poiché un'adeguata ottimizzazione consente di garantire prestazioni ottimali per un sistema orientato al tempo reale e in grado di gestire analisi su enormi volumi di dati.


\subsubsection{Funzionalità Clickhouse utilizzate}
\paragraph{Materialized Views}
Le Materialized Views in ClickHouse sono un meccanismo potente per migliorare le prestazioni delle query e semplificare l'accesso ai dati. Funzionano mantenendo una copia fisica dei risultati di una query di selezione, che viene quindi memorizzata su disco. Questa copia è aggiornata periodicamente in base ai dati sottostanti.

\paragraph*{Utilizzi Principali delle Materialized Views}
\begin{itemize}
    \item \textbf{Calcolo aggregazioni e popolamento tabelle}:Spesso le delle materialized Views sono state utilizzate per calcolare aggregazioni su dati e quindi popolare altre tabelle con i risultati aggregati. Ad esempio, nel caso specifico in cui una Materialized View calcola la media delle temperature per ogni sensore ogni secondo, i risultati di questa vista possono essere utilizzati per popolare una tabella principale contenente i dati di temperatura aggregati, aggiornando i valori di temperatura medi per ogni sensore ogni secondo;
    \item \textbf{Ottimizzazione delle Prestazioni}: memorizzando i risultati di una query complessa, le Materialized Views consentono di eseguire rapidamente le Query successive senza dover ricalcolare i dati ogni volta. Ciò è particolarmente utile in applicazioni che richiedono interrogazioni frequenti su grandi volumi di dati;
    \item \textbf{Decomposizione delle \textit{Query} Complesse}: le Materialized Views consentono di decomporre query complesse in passaggi più semplici e riutilizzabili, migliorando la leggibilità del codice e semplificando lo sviluppo e la manutenzione delle query.
\end{itemize}



\paragraph{MergeTree}\label{sec:MergeTree}
Link alla documentazione: \href{https://clickhouse.com/docs/en/engines/table-engines/mergetree-family/mergetree#mergetree}{ClickHouse - MergeTree}.\newline
MergeTree è uno dei motori di tabella più potenti e utilizzati in ClickHouse, noto per la sua capacità di gestire e memorizzare grandi volumi di dati in modo efficiente. È una scelta ideale per applicazioni che richiedono l'archiviazione e l'analisi di dati cronologicamente ordinati, come i dati di log o di monitoraggio. L'architettura di MergeTree organizza i dati in parti, ciascuna contenente una serie di punti dati ordinati cronologicamente. Questa organizzazione ottimizzata consente di eseguire rapidamente le query che richiedono l'accesso a dati specifici all'interno di un intervallo di tempo definito, garantendo prestazioni elevate anche su grandi dataset. Oltre alla gestione efficiente dei dati, MergeTree supporta funzionalità avanzate come la compressione dei dati e la gestione automatica delle partizioni. Queste caratteristiche consentono di ottimizzare ulteriormente le prestazioni e la gestione complessiva dei dati, rendendo MergeTree una scelta affidabile per una vasta gamma di scenari di utilizzo in ClickHouse.



\paragraph{Time To Live in ClickHouse} \label{sec:RollupTTL}
Link alla documentazione: \href{https://clickhouse.com/docs/en/guides/developer/ttl#implementing-a-rollup}{ClickHouse - Implementing a Rollup}\newline
In ClickHouse, la funzionalità TTL (Time To Live) è un elemento chiave per gestire grandi volumi di dati in modo efficiente e garantire la pulizia automatica di informazioni obsolete o non più rilevanti. \\
Quando si specifica il motore Rollup per definire una tabella in ClickHouse, si abilita la creazione di tabelle che supportano il TTL. Questo consente di impostare un periodo temporale dopo il quale i dati saranno eliminati automaticamente dalla tabella. La struttura a Rollup organizza i dati in parti, ciascuna contenente una serie di punti dati ordinati cronologicamente. Il TTL può essere configurato per ciascuna parte dei dati, offrendo un controllo preciso sulla conservazione delle informazioni nel tempo. Questa flessibilità è particolarmente utile per applicazioni che richiedono la conservazione di dati storici per un periodo limitato, come ad esempio i dati di log o di monitoraggio. \newline
Un esempio di come potrebbe può venire utilizzato il motore Rollup per il TTL in ClickHouse è il seguente:
\begin{verbatim}
    TTL toDateTime(timestamp) + INTERVAL 1 MONTH
\end{verbatim}
L'uso del TTL di tipo Rollup in questo contesto è cruciale per garantire che la tabella rimanga efficiente e gestibile nel tempo, eliminando automaticamente i dati più vecchi e non più necessari dopo un periodo di tempo specificato. Questo aiuta a ottimizzare le prestazioni complessive del sistema e a gestire in modo efficiente i grandi volumi di dati accumulati nel tempo.


\paragraph{Partition}\label{sec:Partition}
Link alla documentazione: \href{https://clickhouse.com/docs/en/engines/table-engines/mergetree-family/mergetree#partition-by}{ClickHouse - Partitioning}.\\
Le partizioni sono una funzionalità fondamentale di ClickHouse che consente di organizzare in modo efficiente e gestire grandi volumi di dati. Questa caratteristica permette di suddividere i dati in gruppi logici in base a criteri specifici, come il valore di una colonna o un intervallo di tempo. Grazie a questa organizzazione ottimizzata, le query che richiedono l'accesso a dati specifici all'interno di una partizione possono essere eseguite rapidamente, garantendo prestazioni elevate anche su dataset di grandi dimensioni.\\
L'utilizzo delle partizioni nel nostro contesto viene giustificato dall'utilizzo di un TTL (Time To Live), infatti l'utilizzo combinato di queste due funzionalità consente:
\begin{itemize}
    \item Una gestione efficace dei dati nel tempo;
    \item Migliori prestazioni del sistema;
    \item Una semplificazione nella manutenzione del database.
\end{itemize}
Il partizionamento basato sul timestamp è una pratica comune in ClickHouse, poiché consente di organizzare i dati in partizioni in base al periodo temporale, ad esempio mensilmente. Questo approccio ottimizza l'archiviazione e facilita l'analisi dei dati di serie temporali, come le temperature o i log di eventi. Grazie a questa struttura, le query che coinvolgono dati all'interno di specifici intervalli temporali diventano più efficienti, consentendo un accesso rapido e una migliore analisi dei dati.




    
\paragraph{Projection}\label{sec:projections}
Link alla documentazione: \href{https://clickhouse.com/docs/en/sql-reference/statements/alter/projection}{https://clickhouse.com/docs/en/sql-reference/statements/alter/projection}\newline
Le proiezioni memorizzano i dati in un formato che ottimizza l'esecuzione delle \textit{Query}, questa caratteristica è utile per:

\begin{itemize}
    \item Eseguire \textit{Query} su una colonna che non fa parte della chiave primaria;
    \item Pre-aggregare colonne, riducendo sia i calcoli che l'I/O.
\end{itemize}

Puoi definire una o più proiezioni per una tabella e durante l'analisi della \textit{Query} la proiezione con meno dati da esaminare sarà selezionata da ClickHouse senza modificare la \textit{Query} fornita dall'utente.

\paragraph*{Utilizzo dello spazio su disco}
\textbf{Attenzione:} le proiezioni creeranno internamente una nuova tabella nascosta, ciò significa che saranno necessari più I/O e spazio su disco. Ad esempio, se la proiezione ha definito una chiave primaria diversa, tutti i dati dalla tabella originale verranno duplicati.

\subsubsection{Integrazione tramite Kafka Engine in ClickHouse}\label{sec:kafka_engine}
ClickHouse supporta l'integrazione con Kafka tramite Kafka Engine, permettendo la lettura dei dati da un topic Kafka e il loro salvataggio in una tabella ClickHouse. Tale funzionalità riveste un'importanza notevole per applicazioni che richiedono l'elaborazione in tempo reale di dati provenienti da fonti esterne, una necessità frequente nel contesto del monitoraggio urbano. L'integrazione con Kafka consente l'acquisizione e la memorizzazione efficiente dei dati, garantendo prestazioni elevate anche su grandi volumi di dati.\\
Kafka Engine è progettato per il recupero di dati una sola volta. Ciò significa che una volta che i dati vengono interrogati da una tabella Kafka, vengono considerati consumati dalla coda. Pertanto, non si dovrebbero mai selezionare dati direttamente da una tabella di Kafka Engine, ma utilizzare invece una vista materializzata. Una vista materializzata viene attivata una volta che i dati sono disponibili in una tabella di Kafka Engine. Automaticamente sposta i dati da una tabella Kafka a una tabella di tipo MergeTree o Distributed. Quindi, sono necessarie almeno 3 tabelle:
\begin{itemize}
  \item La tabella di origine del motore Kafka;
  \item La tabella di destinazione (famiglia MergeTree o distribuita);
  \item Vista materializzata per spostare i dati;
\end{itemize}
\begin{figure}[H]
  \centering
  \includegraphics[width=.7\textwidth]{../Images/SpecificaTecnica/kafka_engine_architecture.png}
  \caption{Architettura di Kafka Engine in ClickHouse}
  \label{fig:sensorKafka}
\end{figure}

\subsubsection{Trasferimento dati tramite Materialized View}
Una materialized view funge da ponte tra la fonte dei dati (Kafka Engine) e la destinazione dei dati (MergeTree). Quando nuovi dati vengono scritti nella tabella Kafka Engine, la materialized view viene attivata automaticamente.\\
La materialized view esegue una query sulla tabella Kafka Engine per selezionare i dati più recenti. Una volta selezionati, questi dati vengono inseriti nella tabella di destinazione (ad esempio, una tabella MergeTree). Questo processo avviene in modo automatico e immediato, senza bisogno di intervento manuale.\\
In pratica, la materialized view si assicura che la tabella di destinazione sia sempre aggiornata con i dati più recenti presenti nella tabella Kafka Engine. Questo offre numerosi vantaggi:
\begin{itemize}
  \item \textbf{Automatizzazione del processo}: Non è necessario eseguire manualmente operazioni di trasferimento dati da una tabella all'altra. La materialized view si occupa di tutto in modo automatico;
  \item \textbf{Efficienza}: Il trasferimento dei dati avviene in tempo reale, garantendo che la tabella di destinazione sia sempre allineata con la fonte dei dati senza ritardi;
  \item \textbf{Ottimizzazione delle risorse}: Il processo di trasferimento dei dati è gestito in modo efficiente, utilizzando al meglio le risorse disponibili e garantendo prestazioni elevate.
\end{itemize}
Nel contesto specifico, le materialized view sono responsabili di eseguire controlli sui dati, come ad esempio la verifica della loro correttezza ed affidabilità nel contesto di utilizzo, prima di inserirli nella tabella di destinazione. Questo processo assicura che i dati siano sempre affidabili e pronti per l'analisi, senza la necessità di ulteriori operazioni di pulizia o preparazione.\\
Per esempio, nel caso dei dati di umidità raccolti da sensori in un'area urbana, la materialized view potrebbe eseguire controlli per assicurarsi che i valori rientrino all'interno di un intervallo plausibile e che non ci siano discrepanze improbabili. Ciò garantirebbe che i dati di umidità inseriti nella tabella di destinazione siano accurati e affidabili per l'analisi meteorologica o ambientale.


\subsubsection{Tabella di origine di Kafka Engine per un sensore generico}
Le tabelle del database impiegate per registrare le misurazioni di ciascuna tipologia di sensore presentano una configurazione sostanzialmente simile, differenziandosi principalmente per il tipo di dato della colonna relativa alla misurazione e per il \textit{topic} di riferimento utilizzato per ottenere le misurazioni.
Nello specifico per ogni sensore si avrà la seguente tabella Clickhouse:
\begin{figure}[H]
    \centering
    \includegraphics[width=.6\textwidth]{../Images/SpecificaTecnica/sensorType_kafka.PNG}
    \caption{Tabella sensore generico per il reperimento da kafka - ClickHouse}
    \label{fig:sensorKafka}
  \end{figure}

    La tabella è configurata con il motore di storage \textit{Kafka}, il che significa che i dati verranno letti da un \textit{topic Kafka}. 

    I campi sono:
    \begin{itemize}
        \item \textbf{ID\_sensore}: un campo di tipo \textit{String} che identifica univocamente il sensore che ha effettuato la misurazione;
        \item \textbf{cella}: un campo di tipo \textit{String} che rappresenta la cella della città in cui è stata effettuata la misurazione;
        \item \textbf{value}: un campo di tipo variabile a seconda del tipo di misurazione che contiene il valore della temperatura;
        \item \textbf{timestamp}: campo di tipo \textit{DATETIME64} che rappresenta il timestamp della misurazione della temperatura;
        \item \textbf{latitude}: un campo di tipo \textit{Float64} che rappresenta la latitudine del luogo dove è stata effettuata la misurazione;
        \item \textbf{longitude}: un campo di tipo \textit{Float64} che rappresenta la longitudine del luogo dove è stata effettuata la misurazione.
    \end{itemize}

    Mentre i parametri esposti racchiusi da parentesi graffe variano per ogni tipolgia di sensore correlato alla misurazione e sono:
    \begin{itemize}
        \item \textbf{tipologiaSensore}: viene sostituito con la tipologia del sensore che effettua le misurazioni salvate nella tabella; (ex. temperatures)
        \item \textbf{TipoDatoMisurazione}: viene sostituito con il tipo del dato che rappresenta la misurazione (ex. Float32, UInt8);
        \item \textbf{IndirizzoServerKafka}: specifica l'indirizzo del server Kafka.
        Nel nostro caso il server Kafka è in esecuzione su un container \textit{Docker} raggiungibile tramite l'indirizzo:
         \textit{'kafka:9092'};
        \item \textbf{topicTipologiaSensore}: specifica il nome del topic Kafka da cui leggere i dati (ex.temperature);
        \item \textbf{ConsumerGroupKafka}: specifica il nome del consumer group Kafka che verrà utilizzato per leggere i messaggi dal topic \textit{Kafka} denominato 'temperature'.
        Un consumer group in \textit{Kafka} è un gruppo di consumatori che lavorano insieme per consumare i messaggi da uno o più topic. Ogni messaggio inviato a un \textit{topic Kafka} può essere consumato da uno dei consumatori nel gruppo. I consumer all'interno di uno stesso gruppo condividono l'elaborazione dei messaggi all'interno dei topic: ogni messaggio viene elaborato da uno e un solo consumatore all'interno del gruppo. Nel nostro caso sarà sempre '\textit{CG\_Clickhouse\_1}' per indicare il servizio di salvataggio \textit{Clickhouse}.
        \item \textbf{FormatoDatiTopicKafka}: specifica il formato dei dati nel \textit{topic Kafka}. Nel nostro caso, i dati sono nel formato JSONEachRow, che è un formato di serializzazione JSON di \textit{ClickHouse} che consente di scrivere o leggere record JSON separati da una riga. Quindi avremo che <<FormatoDatiTopicKafka>> = JSONEachRow.
        \item \textbf{KafkaSkipBrokenMessages}: specifica il numero di errori da tollerare durante il parsing dei messaggi, configurato a livello di tabella, rappresenta la quantità massima di errori accettabili che il sistema può gestire durante il processo di analisi dei messaggi. Questo parametro consente di regolare il livello di tolleranza agli errori a livello di tabella, offrendo la possibilità di controllare quanto il sistema debba essere flessibile nell'interpretazione dei dati.
    \end{itemize}

    
\subsubsection{Misurazioni temperatura} \label{sec:tab_temperatures}
Di seguito viene fornita la configurazione riguardante il salvataggio delle misurazioni di temperatura:
\paragraph{Tabella: temperatures\_kafka}
\begin{figure}[H]
    \centering
    \includegraphics[width=1\textwidth]{../Images/SpecificaTecnica/temperatures_kafka.PNG}
    \caption{Tabella temperatures\_kafka - ClickHouse}
    \label{fig:temperaturesKafka}
  \end{figure}

Il dato della misurazione è di tipo Float32, l'equivalente di float nel linguaggio \textit{C}.
Il topic kafka per ottenere i dati è: \textit{temperature}.

Considerando la possibilità di ricevere molteplici misurazioni dei dati di temperatura all'interno di un singolo secondo di tempo, si procede alla creazione della seguente tabella e alla materialized view correlata, il cui obiettivo è aggregare le misurazioni di temperatura per ridurle ad una singola misurazione per secondo.

\paragraph{Tabella: temperatures}
\begin{figure}[H]
    \centering
    \includegraphics[width=1\textwidth]{../Images/SpecificaTecnica/temperatures.PNG}
    \caption{Tabella temperatures - ClickHouse}
    \label{fig:temperatures}
  \end{figure}

   
    
    \paragraph{Projections per misurazioni di temperatura} \label{sec:temp_projections}
    Durante la fase di progettazione, è stata posta particolare attenzione all'utilizzo delle tabelle appena descritte e alle richieste che verranno formulate su di esse. È emerso, considerando il requisito di suddividere la città in una serie di celle e specificare la cella di origine della misurazione,che la filtrazione delle misurazioni per celle diventerà una richiesta effettuata con frequenza al database.
    Si è giunti quindi all'utilizzo delle \textit{PROJECTIONS}, descritte nella sezione \ref{sec:projections}.
    \vspace{0,3cm}
    \begin{lstlisting}[caption={Esempio di proiezione e materializzazione in una tabella}, captionpos=b]
      --Projection per tabella temperatures
      ALTER TABLE innovacity.temperatures ADD PROJECTION tmp_sensor_cell_projection (SELECT * ORDER BY cella);
      ALTER TABLE innovacity.temperatures MATERIALIZE PROJECTION tmp_sensor_cell_projection;
  \end{lstlisting}
    \vspace{0,3cm}
    La proiezione ci permetterà di filtrare per \textit{cella} e \textit{timestamp} rapidamente, anche se nella tabella originale queste non sOno definite come \textit{PRIMARY\_KEY}.


    \paragraph{Analisi benefici delle Projections}\label{sec:temp_projections_benefici}
    L'aggiunta delle \textit{PROJECTIONS} ha portato risultati di estremo rilievo di seguito esposti.
    Prendendo una \textit{Query} tipo svolta per l'analisi da \textit{Grafana}:
    
    \begin{lstlisting}[caption={Query tipica - Grafana}, captionpos=b]
      SELECT ID_sensore, avgMerge(value) AS value, timestamp
      FROM innovacity.temperatures
      WHERE (cella IN ('Arcella')) AND ((timestamp >= toDateTime64(1708338633507 / 1000, 3)) AND (timestamp <= toDateTime64(1708338933507 / 1000, 3) + INTERVAL 1 DAY))
      GROUP BY timestamp, ID_sensore
      HAVING (value >= -100) AND (value <= 100)

      --Query id: 48635435-9b35-4727-b580-9e33a9db92d4
    \end{lstlisting}

    \begin{figure}[H]
        \centering
        \includegraphics[width=1\textwidth]{../Images/SpecificaTecnica/ProjectionQuery.jpg}
        \caption{Query tipica - Grafana}
        \label{fig:ProjectionsQuery}
      \end{figure}
    senza l'utilizzo delle \textit{PROJECTIONS} il risultato è:
    \begin{figure}[H]
        \centering
        \includegraphics[width=0.9\textwidth]{../Images/SpecificaTecnica/SenzaProectionResult.jpg}
        \caption{Query tipica risultato senza projections}
        \label{fig:ProjectionsQueryWthout}
      \end{figure}
      ovvero sono state processate per ottenere il risultato della \textit{Query} \textbf{16,38 migliaia} di righe.

      Invece in seguito all'aggiunta delle \textit{PROJECTIONS}:
      \begin{figure}[H]
        \centering
        \includegraphics[width=0.9\textwidth]{../Images/SpecificaTecnica/ConProjectionRisultato.jpg}
        \caption{Query tipica risultato con projections}
        \label{fig:ProjectionsQueryWith}
      \end{figure}   
  ovvero sono state processate per ottenere il risultato della \textit{Query} \textbf{8,19 migliaia} di righe, circa la metà rispetto al risultato precedente consentendoci di apprezzare il miglioramento.
Inoltre tramite una \textit{Query} speciale è possibile visualizzare che la \textit{PROJECTIONS} è stata effettivamente utilizzata per ottenere il risultato della \textit{Query} in esame.
\begin{figure}[H]
    \centering
    \includegraphics[width=1\textwidth]{../Images/SpecificaTecnica/ProjectionUsedByClickHouse.jpg}
    \caption{Uso della Projection}
    \label{fig:ProjectionsUsed}
\end{figure}

Prendendo in esempio un altra \textit{Query} fatta dall'applicativo dove viene effettuata la media globale di \textbf{170 mila }misurazioni di temperatura si possono apprezzare i benifiici dell'utilizzo delle \textit{PROJECTIONS} e alla fine dell'immagine anche il suo effettivo utilizzo per il calcolo del risultato.
Con l'utilizzo della \textit{PROJECTIONS} abbiamo:
\paragraph{Tabella: humidity\_kafka}
\begin{figure}[H]
    \centering
    \includegraphics[width=1\textwidth]{../Images/SpecificaTecnica/query2ProjectionsWith.jpg}
    \caption{Query esempio Projection 2 - ClickHouse}
    \label{fig:with2proj}
  \end{figure}
Ovvero il totale di righe processate per ottenere il risultato è di \textbf{49,95 migliaia} con \textbf{0,07 secondi} di tempo utilizzati.
Si puo notare invece la differenza delle righe processate una volta rimossa la \textit{PROJECTIONS}:
\begin{figure}[H]
    \centering
    \includegraphics[width=1\textwidth]{../Images/SpecificaTecnica/query2ProjectionsWithout.jpg}
    \caption{Query esempio senza Projection 2 - ClickHouse}
    \label{fig:without2proj}
  \end{figure}

 Il totale di righe processate per ottenere il risultato è ora di \textbf{170,09 migliaia}, ovvero la totalità delle righe presenti nella tabella, con \textbf{0,09 secondi} di tempo utilizzati.

\subsubsection{Misurazioni umidità}
Le considerazioni relative al salvataggio delle misurazioni di umidità coincidono con quelle espresse nella sezione \ref{sec:tab_temperatures} riguardo alle misurazioni di temperatura.
\paragraph{Tabella: humidity\_kafka}
\begin{figure}[H]
    \centering
    \includegraphics[width=1\textwidth]{../Images/SpecificaTecnica/temperatures.PNG}
    \caption{Tabella humidity\_kafka - ClickHouse}
    \label{fig:umidities_kafka}
  \end{figure}
Il dato della misurazione è di tipo Float32, l’equivalente di float nel linguaggio C. Il topic
kafka per ottenere i dati è: \textbf{humidity}.
\paragraph{Tabella: umidities}
\begin{figure}[H]
    \centering
    \includegraphics[width=1\textwidth]{../Images/SpecificaTecnica/temperatures.PNG}
    \caption{Tabella humidity - ClickHouse}
    \label{fig:umidities}
  \end{figure}

\paragraph{Projections per misurazioni di umidità} 
Date le stesse considerazioni fatte nella sezione: \ref{sec:temp_projections} anche per le misurazioni di umidità si è deciso di adottare le \textit{PROJECTION}.
I risultati dei benifiici perfettamente riconducibili a quelli per le misurazioni di temperatura alla sezione: \ref{sec:temp_projections_benefici}.

Di seguito vengono fornite le configurazioni delle proiezioni sulle tabelle delle misurazioni di umidità:

\begin{lstlisting}
    --Projection per tabella humidity
    ALTER TABLE innovacity.humidity ADD PROJECTION umd_sensor_cell_projection (SELECT * ORDER BY cella);
    ALTER TABLE innovacity.humidity MATERIALIZE PROJECTION umd_sensor_cell_projection;
\end{lstlisting}


\subsubsection{Misurazioni polveri sottili}Le considerazioni relative al salvataggio delle misurazioni di polveri sottili coincidono con quelle espresse nella sezione \ref{sec:tab_temperatures} riguardo alle misurazioni di temperatura.







\end{document}
