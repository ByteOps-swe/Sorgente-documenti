\section{Architettura di deployment}
L'architettura di deployment, detta anche "architettura di rilascio", rappresenta la struttura e la configurazione di un sistema software in fase di esecuzione. Essa definisce come i componenti software, i dati e le risorse di rete sono distribuiti e interconnessi nell'ambiente di produzione.

\subsection{Architettura a microservizi}
La decisione di adottare un'architettura a microservizi è stata motivata dalla necessità di creare una struttura modulare e scalabile. L'applicazione è stata suddivisa in una suite di microservizi, ciascuno dei quali può essere sviluppato, modificato, deployato e scalato indipendentemente dagli altri. Docker rappresenta uno standard de facto per sistemi composti da microservizi, offrendo un ambiente uniforme e semplice da gestire.

L'architettura a microservizi si rivela una scelta solida nell'ambito dell'IoT, poiché si prevede che le diverse parti del sistema evolveranno in maniera indipendente nel tempo e poiché la scalabilità e l'isolamento dei guasti sono un aspetto critico. Questa scelta garantisce maggiore flessibilità e prestazioni ottimizzate mediante un utilizzo accurato delle risorse disponibili.

Infatti, nel contesto dell'architettura a microservizi, si presenta l'opportunità di assegnare risorse specifiche a ciascun servizio, il che permette loro di scalare in modo differenziato in base alle necessità. Questo è particolarmente vantaggioso in scenari in cui i servizi potrebbero essere sottoscritti a specifici topic e argomenti all'interno di un sistema di streaming come Kafka.

L'allocazione di risorse individualizzate consente ai servizi di adattarsi dinamicamente alla loro attività e al volume di dati con cui devono interagire. Ad esempio, un servizio che riceve un alto flusso di dati da un particolare topic potrebbe richiedere una maggiore capacità di calcolo e di memorizzazione rispetto a un altro servizio che gestisce un carico meno intenso.

Inoltre, questa flessibilità nell'allocazione delle risorse consente di ottimizzare l'efficienza complessiva del sistema, garantendo che le risorse siano allocate in modo proporzionale alla richiesta effettiva dei servizi. Ciò contribuisce a migliorare le prestazioni complessive del sistema e a garantire una gestione ottimale delle risorse disponibili.
\subsection{Il container deployment}
Docker è un software open source sviluppato in Go che facilita il deployment di sistemi software all'interno di container. Questi container contengono l'applicativo stesso e tutte le sue dipendenze, consentendo un'esecuzione flessibile in qualsiasi ambiente. Docker offre un'infrastruttura di deployment leggera e portatile, che consente di distribuire facilmente applicazioni in ambienti di sviluppo, test e produzione.

Per implementare l'intero stack tecnologico e i layer di elaborazione dati in streaming, è stato configurato un ambiente Docker a microservizi che simula la divisione e la distribuzione dei layer e dei componenti. In particolare, sono stati creati container per:
\begin{itemize}
    \item \textbf{Data feed}:
    \begin{itemize}
        \item \textbf{Simulators}: Esegue i \textbf{simulatori di sensori} per la raccolta dei dati.
        \item  Non espone porte all’esterno.
    \end{itemize} 
    \item \textbf{Streaming layer}:
    \begin{itemize}
        \item Esegue \textbf{Apache Kafka} per la gestione del flusso di dati in tempo reale.
        \item Accessibile agli altri container tramite l'indirizzo \textit{\textbf{kafka:9092}}.
        \item Permette l'invio e il recupero di messaggi attraverso librerie e framework appositi.
    \end{itemize} 
    \item \textbf{Processing layer}:
    \begin{itemize}
        \item Esegue l'app \textbf{Faust} per il processing e il calcolo del punteggio di salute.
        \item Non espone porte all'esterno.
    \end{itemize}
    \item \textbf{Storage layer}:
    \begin{itemize}
        \item Esegue \textbf{Clickhouse} per lo storage delle misurazioni.
        \item La banca dati è accessibile agli altri container tramite l'indirizzo \textit{\textbf{clickhouse:8123}}.
    \end{itemize}
    \item \textbf{Data Visualization Layer}:
    \begin{itemize}
        \item Esegue \textbf{Grafana} come interfaccia utente per la visualizzazione dei dati.
        \item Espone la porta 3000 all'esterno per permettere l'accesso al servizio di dashboarding.
    \end{itemize}
\end{itemize}
Questa struttura permette una distribuzione modulare e scalabile del sistema, semplificando la gestione e la manutenzione dei componenti e consentendo una rapida scalabilità in risposta alle esigenze emergenti. Grazie all'uso di Docker, si garantisce coerenza e riproducibilità dell'ambiente di esecuzione, semplificando il deployment e garantendo maggiore affidabilità nell'ambiente di produzione nonchè la possibilità di attribuire le risorse necessarie ad ogni servizio in modo mirato.

\subsection{Comunicazione tra i componenti}
Nel contesto di Docker, la comunicazione tra i container avviene tramite la rete Docker interna, che è una rete virtuale creata automaticamente da Docker per i servizi all'interno di uno stesso file Compose o di un ambiente Docker. Questa rete consente ai container di comunicare tra loro utilizzando i nomi dei servizi come hostnames.

Quando un container viene avviato, Docker assegna un hostname basato sul nome del servizio definito nel file Compose. Ad esempio, nel file Compose fornito, il servizio Kafka ha il nome "kafka" e il servizio ClickHouse ha il nome "clickhouse". Questi nomi sono utilizzati all'interno dei container stessi per identificare gli altri servizi. Quando un container desidera comunicare con un altro container sulla stessa rete Docker, può semplicemente utilizzare il nome del servizio come hostname.

Inoltre, Docker offre una funzionalità chiamata "Discovery", che consente ai container di scoprire automaticamente gli altri container sulla stessa rete Docker senza dover conoscere esplicitamente i loro indirizzi IP. Questo semplifica la configurazione e la gestione della comunicazione tra i container.

È anche possibile specificare dipendenze tra i servizi utilizzando l'attributo "depends\_on" nel file Compose. Questo assicura che un servizio venga avviato solo dopo che i servizi di cui dipende sono stati avviati e sono nella condizione desiderata.

Infine, per i servizi che espongono porte, come Kafka, ClickHouse e Grafana, è possibile mappare le porte del container su porte del sistema host utilizzando l'attributo "ports" nel file Compose. Questo consente ad altri componenti esterni al Docker network di comunicare con i container attraverso le porte esposte.

Quando un container invia dati a un altro container tramite la rete Docker, i dati vengono incapsulati in pacchetti di rete utilizzando il protocollo TCP/IP. Questi pacchetti vengono quindi instradati attraverso la rete Docker, che si occupa di consegnarli al destinatario corretto.

In sintesi, Docker fornisce un'infrastruttura di rete integrata che gestisce la comunicazione tra i container all'interno dello stesso ambiente di deployment, semplificando la configurazione e la gestione della comunicazione tra i diversi componenti del sistema.
\begin{itemize}
    \item \textbf{Comunicazione data feed Layer -> Streaming Layer}: Si utilizza libreria \textit{Confluent Kafka} per Python che offre un'API efficiente e flessibile per inviare dati dai simulatori dei sensori a specifici topic Kafka.
    \item \textbf{Comunicazione Processing Layer -> Streaming Layer}: Si utilizza Faust come API di alto livello per Kafka. Faust offre un'API intuitiva e ben documentata per consumare dati da topic con flussi di dati in tempo reale.
    Ulteriori informazioni in: \ref{sec:faust}
    \item \textbf{Comunicazione Storage Layer -> Streaming Layer}: Per ottenere in tempo reale i dati dai topic kafka viene utilizzato l'engine kafka di Clickhouse. Ulteriori informazioni in: \ref{sec:kafka_engine}
    \item \textbf{Comunicazione Data Visualization Layer-> Storage Layer}: Grafana si connette a Clickhouse per ottenere i dati da visualizzare in tempo reale tramite lo specifico plugin ClickHouse che permette l'utilizzo di un database Clickhouse come \textit{data source}.
    Il plugin nasconde alcuni dettagli di implementazione sottostanti, come la gestione della connessione,protocolli di comunicazione e l'esecuzione delle query. Ulteriori informazioni in: \ref{sec:click_plugin}
\end{itemize}
    





    \subsubsection{Dipendenze tra i servizi}
\begin{itemize}
    \item \textbf{Clickhouse -> kafka}: Il servizio ClickHouse dipende dal servizio Kafka. Questo assicura che il servizio ClickHouse venga avviato solo dopo che il servizio Kafka è stato avviato e è nella condizione desiderata.

    \item \textbf{Simulators -> kafka}: Il servizio Simulators dipende dal servizio Kafka. Questo assicura che il servizio Simulators venga avviato solo dopo che il servizio Kafka è stato avviato e è nella condizione desiderata.
    
    \item \textbf{ Faust app -> kafka}: Il servizio Processor dipende dal servizio Kafka. Questo assicura che il servizio Processor venga avviato solo dopo che il servizio Kafka è stato avviato e è nella condizione desiderata.
\end{itemize}


