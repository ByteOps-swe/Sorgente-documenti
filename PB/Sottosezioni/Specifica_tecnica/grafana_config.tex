\subsection{Grafana}
Grafana è un software open source per la visualizzazione e l'analisi dei dati. È progettato per funzionare con vari database di serie temporali, tra cui Clickhouse. Grafana offre un'interfaccia utente intuitiva e flessibile che consente di creare e condividere dashboard personalizzate per monitorare i dati in tempo reale. È ampiamente utilizzato per monitorare sistemi e applicazioni, nonché per analizzare e visualizzare dati in tempo reale.

\subsection{Dashboards}
Per soddisfare tutti i requisiti definiti in \textit{Analisi dei requisiti v2.0.0} sono state create due Dashboard:
\begin{itemize}
    \item \textbf{Dashboard Principale:} Questa dashboard fornisce una visualizzazione chiara e intuitiva delle misurazioni provenienti da tutti i sensori, di tutte le tipologie, distribuiti nell'area urbana. La dashboard include una mappa interattiva della città che mostra la posizione geografica di ciascun sensore e la relativa ultima misurazione. Inoltre, viene presentato il punteggio di salute della città o di celle specifiche.
    \item \textbf{Dashboard dedicata:} Mostra le misurazioni di una specifica tipologia di sensore selezionata dall'utente in modo più dettagliato e permette di effettuare le attività di filtraggio e aggregazione definite in \textit{Analisi dei requisiti v2.0.0}.
\end{itemize}




\subsubsection{ClickHouse data source plugin} \label{sec:click_plugin}
\paragraph{Documentazione:} \href{https://grafana.com/grafana/plugins/grafana-clickhouse-datasource/}{https://grafana.com/grafana/plugins/grafana-clickhouse-datasource/}

Questo plugin di grafana consente di connettersi a un'istanza di ClickHouse e di visualizzare i dati in tempo reale. È possibile eseguire query SQL personalizzate e visualizzare i risultati in forma di grafici, tabelle e pannelli personalizzati. Il plugin offre anche funzionalità di aggregazione e di calcolo dei dati, consentendo di analizzare e visualizzare i dati in modo flessibile e personalizzato.

\paragraph{Data sources configuration}
La configurazione del data source avviene tramite file \textit{yaml} che deve essere presente in \textit{"grafana/provisioning/datasources"}.
Il protocollo di trasporto utilizzato è TLS ma puo essere modificato nel file appena citato grazie al parametro di configurazione: "protocol".

\paragraph{Macro utilizzate}\label{sec:macros}
Per semplificare la sintassi e consentire operazioni dinamiche, come i filtri dell'intervallo di date, le queries al database Clickhouse possono contenere macro.
Quelle utilizzate sono:
\begin{itemize}
    \item \textbf{\$\_\_timeFilter(columnName)}: Permette di effettuare il filtro temporale alla query per ottenere le sole misurazioni all'interno dell'intervallo di tempo selezionato dall'utente.
    \item  \textbf{\$\_\_timeInterval(columnName)}: Permette di modificare il raggruppamento temporale delle misurazioni in automatico sulla base dell'ampiezza dell'intervallo temporale selezionato dall'utente.
    In questo modo è possibile avere una visione ottimizzate delle misurazioni.
\end{itemize}

\subsubsection{Variabili Grafana}
\paragraph*{Documentazione:} \href{https://grafana.com/docs/grafana/latest/dashboards/variables/}{https://grafana.com/docs/grafana/latest/dashboards/variables/}


 Le variabili in Grafana sono un potente strumento per rendere le dashboard dinamiche e interattive. Permettono di filtrare i dati visualizzati in base a valori scelti dall'utente, rendendo la dashboard più versatile e adattabile a differenti esigenze.
\paragraph*{Utilizzo delle variabili nella dashboard principale:}
Nella dashboard principale, le variabili sono:
\begin{itemize}
    \item \textbf{variabile \$cella}: per mostrare solo le misurazioni provenienti da determinate celle della città, 
    \item \textbf{variabili \$<TipoSensore>\_sensors\_id}: per mostrare le misurazioni di determinati sensori di un certo tipo.
\end{itemize}
Queste variabili all'interno delle query al database permettono il filtraggio delle misurazioni sulla base di quanto selezionato dall'utente.
Un esempio di query per la visualizzazione delle misurazioni time-series di temperatura è:
\begin{lstlisting}[style=code]
    SELECT    ID_sensore, avg(value) as value,
              $__timeInterval(timestamp) as timestamp
    FROM    innovacity.temperatures 
    WHERE    $__timeFilter(timestamp) AND cella IN ($Cella) AND ID_sensore in (${tmp_sensors_id})
    GROUP BY ID_sensore, timestamp;
\end{lstlisting}

La query esposta mostra anche l'utilizzo delle macro esposte in: \ref{sec:macros}
\paragraph*{Utilizzo delle variabili nella dashboard dedicata:}
Nella dashboard dedicata alla visualizzazione spefica delle misurazioni di una sola tipologia di sensori, le variabili sono:
\begin{itemize}
    \item \textbf{variabile \$cella}: per mostrare solo le misurazioni provenienti da determinate celle della città, 
    \item \textbf{variabili \$<TipoSensore>\_sensors\_id}: per mostrare le misurazioni di determinati sensori di un certo tipo.
    \item \textbf{variabili \$tabella}: per selezionare la tipoligia di sensore di cui si vuole visualizzare la dashboard dedicata e quindi la tabella del database da cui ricavare i dati.
    \item \textbf{\$aggregazione}: variabile per selezionare l'intervallo temporale di aggregazione delle misurazioni
    (Automatico, Secondo,Minuto,Ora,Giorno,Mese,Nessuno).
    Nel caso della selezione della modalità "Automatico" si utilizzza l'intervallo temporale di aggregazione più opportuno sulla base dell'ampiezza dell'intervallo temporale selezionato dall'utente.
    \item \textbf{\$Max\_value}: variabile ad input numerico per filtrare le misurazioni con valore al di sotto di quello indicato.
    \item \textbf{\$Min\_value}: variabile ad input numerico per filtrare le misurazioni con valore al di sopra di quello indicato.
\end{itemize}
\paragraph{Variable Panel plugin}
\textbf{Documentazione:} \href{https://volkovlabs.io/plugins/volkovlabs-variable-panel/}{https://volkovlabs.io/plugins/volkovlabs-variable-panel/}
Il plugin permette di creare dei pannelli grafana che possono essere posizionati ovunque nella Dashboard e che consentono di selezionare i valori delle variabili.
Inoltre permette la visualizzazione ad albero delle variabili utile nel nostro caso dove i sensori sono contenuti all'interno di celle della città.

\subsubsection{Grafana alerts}
\textbf{Documentazione}\href{https://grafana.com/docs/grafana/latest/alerting/}{https://grafana.com/docs/grafana/latest/alerting/}


Grafana offre un sistema di alerting completo per monitorare i dati e inviare notifiche quando si verificano determinate condizioni. Le notifiche possono essere inviate tramite diversi canali, tra cui email, Slack, Telegram e Discord.

\paragraph{Alert Rule}
Per poter configurare un alert è necessario creare una regola di alert. La regola di alert viene impostata tramite query al data source e fa scattare l'alert quando la query restituisce un risultato che soddisfa le condizioni impostate.
Gli alert impostati sono per:
\begin{itemize}
    \item Quando un sensore di temperatura registra una temperatura superiore ai 40°C;
    \item Quando un sensore di polveri sottili supera i 50 microgrammi al metro cubo;
    \item Quando un sensore di guasti elettrici rileva un guasto.
\end{itemize}

Gli alert attraversano 3 stati:
\begin{itemize}
    \item \textbf{Pending:} indica che un alert è stato attivato, ma la sua valutazione non è ancora stata completata.
                            Quando si è in questo stato è perchè il valore della query di alert è stato valutato e risulta essere vero ma la configurazione della regola di allerta ha impostato che l'allerta deve essere attiva per un certo periodo di tempo prima di essere considerata vera e quindi inviare la notifica.
\item \textbf{Firing:}  indica che un alert è stato attivato e la sua valutazione ha confermato che la condizione di alert è soddisfatta per il periodo impostato nella regola e quindi viene inviata la notifica ai canali impostati.
\item \textbf{OK:} indica che un alert è stato disattivato e la sua valutazione ha confermato che la condizione di alert non è più soddisfatta.

Le regole di allerta sono impostabili tramite l'interfaccia grafica di Grafana e vengono esportate in formato \textit{yaml} ed inserite in "/provisioning/alerting".
                            
\end{itemize}

\paragraph{Configurazione il canale di notifica}
Per configurare i canali di notifica è necessario andare in "Alerting" e selezionare "Notification channels" dall'interfaccia grafica di Grafana.

Per il progetto è stato scelto Discord come unico canale di notifica.
Per configurare il canale di notifica è necessario:
\begin{itemize}
    \item Seleziona Discord come canale di notifica.
    \item Inserisci il webhook URL del tuo canale Discord.
    \item Personalizza il messaggio di notifica.
\end{itemize}

Anche le impostazioni di configurazione del canale di notifica sono esportabili in formato \textit{yaml} e vengono inserite in "/provisioning/alerting".

\paragraph{Notification policies}
Le norme di notifica negli Alert di Grafana sono un modo potente per gestire l'invio degli alert a diversi canali di notifica.

Per una spiegazione dettagliata della configurazione si rimanda alla documentazione ufficiale di Grafana: \href{https://grafana.com/docs/grafana/latest/alerting/alerting-rules/create-notification-policy/}{https://grafana.com/docs/grafana/latest/alerting/alerting-rules/create-notification-policy/}

Anche le impostazioni delle notification policies sono esportabili in formato \textit{yaml} e vengono inserite in "/provisioning/alerting".

\subsubsection{Altri plugin utilizzati}
\paragraph{Orchestra Cities Map plugin}
\textbf{Documentazione:} \href{https://grafana.com/grafana/plugins/orchestracities-map-panel/}{https://grafana.com/grafana/plugins/orchestracities-map-panel/}

Il plugin Orchestra Cities Map per Grafana estende il pannello Geomap di Grafana con diverse funzionalità avanzate per la visualizzazione di dati geolocalizzati su mappe:

Funzionalità principali:
\begin{itemize}
    \item \textbf{Supporto per GeoJSON}: Permette di visualizzare dati geoJSON su mappe, come shapefile di città, regioni o stati.
    \item \textbf{Icone personalizzate}: Puoi utilizzare icone personalizzate per rappresentare diversi tipi di dati sui punti mappa.
    \item \textbf{Popup informativi}: Mostra popup con informazioni dettagliate quando si clicca su un punto mappa.
    \item \textbf{Strati multipli}: Permette di creare più strati sovrapposti per visualizzare diversi set di dati sulla stessa mappa.
    \item \textbf{Filtraggio e ricerca}: Puoi filtrare i punti mappa in base a diversi criteri, come proprietà dei dati o valori delle metriche.
    \item \textbf{Colorazione dei punti}: Puoi colorare i punti mappa in base a valori di metriche o ad altri criteri.
    \item \textbf{Legende personalizzate}: Puoi creare legende personalizzate per spiegare il significato dei colori e delle icone utilizzati nella mappa.
\end{itemize}

Viene utilizzato per poter visualizzare in modo diverse le icone dei diversi tipi di sensori dislocati nella città oltre che l'ultima misurazione effettuata ovvero lo stato attuale del sensore.