\subsection{Apache Kafka}

Apache Kafka è una piattaforma open-source di streaming distribuito sviluppata da Apache Software Foundation. È progettata per la gestione di flussi di dati in tempo reale in modo scalabile, affidabile e efficiente. Kafka è utilizzato ampiamente nell'ambito del data streaming e del data integration in molte applicazioni moderne.
\subsubsection{Versione}
La versione utilizzata è: 3.6.1
\subsubsection{Link download}
\href{https://kafka.apache.org/downloads}{https://kafka.apache.org/downloads}

\subsubsection*{Funzionalità e Vantaggi di Apache Kafka}
Le principali funzionalità e vantaggi di Apache Kafka includono:

\begin{itemize}
  \item \textbf{Pub-Sub Messaging:} Kafka utilizza un modello di messaggistica publish-subscribe, dove i produttori di dati inviano messaggi a un topic e i consumatori possono sottoscriversi a tali topic per ricevere i messaggi.
  
  \item \textbf{Architettura Distribuita:} Kafka è progettato per essere distribuito su un cluster di nodi, consentendo una scalabilità orizzontale per gestire grandi volumi di dati e carichi di lavoro.
  
  \item \textbf{Persistenza e Affidabilità:} Kafka conserva i dati in modo persistente su disco, garantendo la durabilità dei messaggi anche in caso di guasti hardware o arresti anomali. Questo assicura anche un alto livello di affidabilità.
  
  \item \textbf{Alta Disponibilità:} Grazie alla sua architettura distribuita, Kafka offre alta disponibilità e tolleranza ai guasti, consentendo ai cluster di continuare a funzionare anche in presenza di nodi o componenti falliti.
  
  \item \textbf{Elaborazione degli Stream:} Kafka supporta anche l'elaborazione degli stream di dati in tempo reale tramite API come Kafka Streams e Kafka Connect, consentendo agli sviluppatori di scrivere applicazioni per l'analisi e l'elaborazione dei dati in tempo reale.
\end{itemize}

\subsubsection*{Casi d'uso di Apache Kafka}

Apache Kafka è utilizzato in una vasta gamma di casi d'uso, tra cui:

\begin{itemize}
  \item \textbf{Data Integration:} Kafka viene utilizzato per integrare dati provenienti da diverse fonti e sistemi, consentendo lo scambio di dati in tempo reale tra applicazioni e sistemi eterogenei.
  
  \item \textbf{Streaming di Eventi:} Molte applicazioni moderne, come le applicazioni IoT (Internet of Things) e le applicazioni di monitoraggio in tempo reale, utilizzano Kafka per il streaming di eventi in tempo reale e l'analisi dei dati.
  
  \item \textbf{Analisi dei Log:} Kafka è spesso utilizzato per l'analisi dei log di sistema e applicativi in tempo reale, consentendo il monitoraggio delle prestazioni, la rilevazione degli errori e l'analisi dei pattern di utilizzo.
  
  \item \textbf{Elaborazione di Big Data:} Kafka è integrato con tecnologie di big data come Apache Hadoop e Apache Spark, consentendo l'elaborazione di grandi volumi di dati in tempo reale.
  
  \item \textbf{Messaggistica Real-time:} Kafka è ampiamente utilizzato per la messaggistica real-time in applicazioni di social media, e-commerce e finanziarie, dove la velocità e l'affidabilità della messaggistica sono cruciali.
\end{itemize}