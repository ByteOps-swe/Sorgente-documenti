\subsection{Apache Kafka}

Apache Kafka è una piattaforma open-source di streaming distribuito sviluppata da Apache Software Foundation. È progettata per la gestione di flussi di dati in tempo reale in modo scalabile, affidabile e efficiente. Kafka è utilizzato ampiamente nell'ambito del data streaming e del data integration in molte applicazioni moderne.
\subsubsection{Versione}
La versione utilizzata è: 3.6.1
\subsubsection{Link download}
\href{https://kafka.apache.org/downloads}{https://kafka.apache.org/downloads}

\subsubsection{Funzionalità e Vantaggi di Apache Kafka}
Le principali funzionalità e vantaggi di Apache Kafka includono:

\begin{itemize}
  \item \textbf{Pub-Sub Messaging:} kafka utilizza un modello di messaggistica publish-subscribe, dove i produttori di dati inviano messaggi a un topic e i consumatori possono sottoscriversi a tali topic per ricevere i messaggi;
  \item \textbf{Disaccoppiamento Produttore - Consumatore:}il disaccoppiamento avviene perché Produttori e Consumatori non devono essere a conoscenza l'uno dell'altro o interagire direttamente. Invece, interagiscono attraverso il broker Kafka, che funge da intermediario per la comunicazione;
  \item \textbf{Architettura Distribuita:} kafka è progettato per essere distribuito su un cluster di nodi, consentendo una scalabilità orizzontale per gestire grandi volumi di dati e carichi di lavoro;
  
  \item \textbf{Persistenza e Affidabilità:} Kafka conserva i dati in modo persistente su disco, garantendo la durabilità dei messaggi anche in caso di guasti hardware o arresti anomali. Questo assicura anche un alto livello di affidabilità;
  
  \item \textbf{Alta Disponibilità:} grazie alla sua architettura distribuita, Kafka offre alta disponibilità e tolleranza ai guasti, consentendo ai cluster di continuare a funzionare anche in presenza di nodi o componenti falliti;
  
  \item \textbf{Elaborazione degli Stream:} kafka supporta anche l'elaborazione degli stream di dati in tempo reale tramite API come Kafka Streams e Kafka Connect, consentendo agli sviluppatori di scrivere applicazioni per l'analisi e l'elaborazione dei dati in tempo reale.
\end{itemize}

\subsubsection{Casi d'uso di Apache Kafka}

Apache Kafka è utilizzato in una vasta gamma di casi d'uso, tra cui:

\begin{itemize}
  \item \textbf{Data Integration:} Kafka viene utilizzato per integrare dati provenienti da diverse fonti e sistemi, consentendo lo scambio di dati in tempo reale tra applicazioni e sistemi eterogenei.
  
  \item \textbf{Streaming di Eventi:} Molte applicazioni moderne, come le applicazioni IoT (Internet of Things) e le applicazioni di monitoraggio in tempo reale, utilizzano Kafka per il streaming di eventi in tempo reale e l'analisi dei dati.
  
  \item \textbf{Analisi dei Log:} Kafka è spesso utilizzato per l'analisi dei log di sistema e applicativi in tempo reale, consentendo il monitoraggio delle prestazioni, la rilevazione degli errori e l'analisi dei pattern di utilizzo.
  
  \item \textbf{Elaborazione di Big Data:} Kafka è integrato con tecnologie di big data come Apache Hadoop e Apache Spark, consentendo l'elaborazione di grandi volumi di dati in tempo reale.
  
  \item \textbf{Messaggistica Real-time:} Kafka è ampiamente utilizzato per la messaggistica real-time in applicazioni di social media, e-commerce e finanziarie, dove la velocità e l'affidabilità della messaggistica sono cruciali.
\end{itemize}

\subsubsection{Utilizzo nel progetto}
\textit{Kafka} funge da intermediario dei messaggi che riceve i dati dai produttori di dati e li invia ai consumatori. Nel contesto di questo progetto, i dati provenienti dalle simulazioni di sensori vengono inviati a \textit{Kafka} come messaggi in formato \textit{JSON}.

\paragraph*{Consumatori di dati:}
\begin{itemize}
  \item \textbf{\textit{ClickHouse:}} \textit{Kafka} invia i dati ai consumatori, inclusi i database come \textit{ClickHouse}, dove i dati vengono salvati per l'analisi e l'archiviazione a lungo termine.
  \item \textbf{Script di processing:} In futuro, per rispettare il requisito opzionale del calcolo del punteggio di salute, si potrebbe utilizzare \textit{kafka} per rendere disponibili i dati in tempo reale ad un script di processing il quale calcoli il punteggio attraverso una funzione di aggregazione complessa e renda  disponbile il risultato in una coda dedicata \textit{kafka} ai servizi interessati.
\end{itemize}

In breve, \textit{Kafka} funge da ponte tra i produttori di dati (simulazioni di sensori) e i consumatori di dati (\textit{ClickHouse} o altri servizi futuri). Gestisce il flusso dei dati in tempo reale e garantisce che i dati siano disponibili per l'elaborazione e la visualizzazione in modo efficiente e scalabile.