\subsubsection{Python}
Linguaggio di programmazione ad alto livello, interpretato e multi-paradigma.
\paragraph{Versione}
La versione utilizzata è: 3.9

\paragraph{Librerie o framework}

\begin{itemize}
    \item \textbf{Confluent Kafka}
    \begin{itemize}
        \item \textbf{Documentazione:} \href{https://developer.confluent.io/get-started/python/}{https://developer.confluent.io/get-started/python/}
        \item \textbf{Versione:} 2.3.0
        \item Libreria Python che fornisce un insieme completo di strumenti per agevolare la produzione e il consumo di messaggi da Apache Kafka.
    \end{itemize}
    
    \item \textbf{Faust}
    \begin{itemize}
        \item \textbf{Documentazione:} \href{https://faust.readthedocs.io/en/latest/}{https://faust.readthedocs.io/en/latest/}
        \item \textbf{Versione:} 1.10.4
        \item Framework Python per la creazione di applicazioni di streaming in tempo reale, con un'enfasi particolare sull'elaborazione di eventi e dati in tempo reale. Fornisce un'API dichiarativa e funzionale per definire i flussi di dati e le trasformazioni, consentendo agli sviluppatori di scrivere facilmente applicazioni scalabili e affidabili per il trattamento di grandi volumi di dati in tempo reale. Faust si integra nativamente con Apache Kafka e offre funzionalità avanzate come il bilanciamento del carico, la gestione dello stato, la gestione delle query, e la tolleranza ai guasti, rendendolo una scelta potente per lo sviluppo di sistemi di streaming complessi e robusti.
    \end{itemize}
    
    \item \textbf{Pytest}
    \begin{itemize}
        \item \textbf{Documentazione:} \href{https://docs.pytest.org/en/7.1.x/contents.html}{https://docs.pytest.org/en/7.1.x/contents.html}
        \item \textbf{Versione:} 8.0.2
        \item Framework di testing per Python, noto per la sua semplicità e potenza. Consente agli sviluppatori di scrivere test chiari e concisi utilizzando una sintassi intuitiva e flessibile. Pytest supporta una vasta gamma di funzionalità, tra cui test di unità, integrazione e accettazione, parametrizzazione dei test e gestione delle fixture. Da citare anche l'utilizzo di \textit{Pytest-asyncio} per testare codice asincrono e \textit{Pytest-cov} per la copertura del codice.
    \end{itemize}
    
    \item \textbf{Pylint}
    \begin{itemize}
        \item \textbf{Documentazione:} \href{https://pylint.readthedocs.io/en/stable/}{https://pylint.readthedocs.io/en/stable/}
        \item \textbf{Versione:} 3.1.0
        \item Strumento di analisi statica per il linguaggio di programmazione Python. Esamina il codice sorgente per individuare potenziali errori, conformità alle linee guida di stile e altre possibili problematiche. Pylint fornisce un punteggio di qualità del codice e suggerimenti per migliorare la leggibilità, la manutenibilità e la correttezza del codice Python.
    \end{itemize}
    
    \item \textbf{Clickhouse-connect}
    \begin{itemize}
        \item \textbf{Documentazione:} \href{https://clickhouse.com/docs/en/integrations/python}{https://clickhouse.com/docs/en/integrations/python}
        \item \textbf{Versione:} 1.10.4
        \item ClickHouse Connect è un modulo Python che viene utilizzato nei test. Fornisce un'API semplice e intuitiva per eseguire query su ClickHouse direttamente da codice Python
    \end{itemize}
\end{itemize}

\paragraph{Utilizzo nel progetto}
\begin{itemize}
    \item Creazione di simulazioni dei sensori e dei microcontrollori, incluse le logiche di scrittura e invio dei dati registrati.
    \item Modello per il calcolo del punteggio di salute della città;
    \item Testing.
    \end{itemize}